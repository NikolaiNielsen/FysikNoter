\documentclass[a4paper,12pt]{article}
\usepackage[utf8x]{inputenc}
\usepackage[danish]{babel}
\usepackage[T1]{fontenc}
\usepackage[pdftex]{graphicx}
\usepackage{amsmath, amssymb, textcomp, stmaryrd, wasysym, mathtools, url}
\usepackage[amsmath,thmmarks]{ntheorem}
\usepackage[danish=quotes]{csquotes}
\newcommand{\abs}[1]{\lvert#1\rvert}
\newcommand{\lrarrow}{\Leftrightarrow}
\newcommand{\gra}{\text{\textdegree}}
\newcommand{\rarrow}{\Rightarrow}
\newcommand{\larrow}{\Leftarrow}
\newcommand{\udarrow}{\Updownarrow}
\newcommand{\bsm}{\begin{smallmatrix}}
\newcommand{\esm}{\end{smallmatrix}}
\newcommand{\shiftline}{\newline \newline}
\newcommand{\lbl}{\left(}
\newcommand{\lbr}{\right)}
\newcommand{\HRule}{\rule{\linewidth}{0.5mm}} %regel indført i forbindelse med forsiden
\newcommand{\D}{\Delta}
\newcommand{\half}{\frac{1}{2}}
\newcommand{\PD}{\partial}
\newcommand{\Epsilon}{\varepsilon}
\newcommand{\real}{\mathbb{R}}
\newcommand{\gradient}{\nabla}
\newcommand{\Bf}[1]{\mathbf{#1}}
\bibliographystyle{dk-plain} %vælger bibiografistil
%\def\thesection{\Roman{section}} %benytter romertal istedet for arabiske tal i sektionerne
%\def\thesubsection{\Roman{subsection}} %benytter romertal til undersektionerne
\theoremstyle{break}
\newtheorem{thm}{Sætning}
\newtheorem{defn}{Definition}
\newtheorem{korol}{Korollar}
\newtheorem{konk}{Konklusion}
\newtheorem{regel}{Regneregler}
\theoremstyle{nonumberbreak}
\theoremsymbol{\ensuremath{\square}}
\theoremseparator{.}
\newtheorem{proof}{Bevis}
\allowdisplaybreaks[1]
\usepackage{pdfpages}
\pagestyle{headings}
\usepackage{hyperref}
\hypersetup{
	colorlinks,
	citecolor=black,
	filecolor=black,
	linkcolor=black,
	urlcolor=blue,
	breaklinks=true
}

\title{Formelsamling til MatIntro kurset på Københavns Universitet}
\author{af Michael Flemming Hansen  \\ Version 1.0}

\begin{document}
\maketitle
\tableofcontents
\newpage
\begin{abstract}
Denne formelsamling er skrevet i forbindelse med re-eksamen i MatIntro forløbet. Jeg har forsøgt at medtage de sætninger og metoder jeg finder mest nyttige i forbindelse med hurtige opslag til forskellige opgaver. Jeg har ikke medtaget udregnede eksempler, da man i denne forbindelse i forvejen nok bør kigge i sin bog, i mit tilfælde \emph{Kalkulus af Tom Lindström} og \emph{Funktioner af flere variable af Tore A. Kro}. Jeg har lagt særlig vægt på at udspecificere regnemetoderne lidt i de store analytiske arter, \textit{Differentialregning, integration og differentialligninger}. Men metoder til Taylor serier er også medtaget. Til sidst i arket er en række udregnede differential koefficienter og stamfunktioner og et ark med udregnede værdier for sinus og cosinus, samt enhedscirklen. 

Mvh\newline
Michael F. Hansen\newline
København 2012\newline
\end{abstract}

%\newpage
\section{Funktioner af en variabel}
\subsection{Komplekse tal}
\subsubsection{Regneregler for komplekse tal}
\begin{regel}[Komplekse tal]
Hvis $z = a+ib$ og $w = c+id$ er to komplekse tal, så gælder der
\begin{align}
 z + w &= (a + c) + i(b + d) \\
 z - w &= (a - c) + i(b - d) \\
 z \cdot w &= (ac - bd) + i(ad-bc)\\
 \frac{z}{w} &= \frac{a + ib}{c + id} = \frac{ac + bd}{c^2+d^2} + i\frac{bc - ad}{c^2 + d^2}\; ; (w \neq 0) \\
 z^{-1} &= \frac{a}{a^2 + b^2} - i\frac{b}{a^2 + b^2}\; ; (z \neq 0)
\end{align}
\end{regel}

\subsubsection{Regneregler for konjugation}
\begin{regel}[Konjugation]
Hvis $z$ og $w$ er komplekse tal, så gælder der
\begin{align}
 \bar{z} + \bar{w} &= \overline{z + w} \\
 \bar{z} - \bar{w} &= \overline{z - w} \\
 \bar{z}\bar{w} &= \overline{zw} \\
 \frac{\bar{z}}{\bar{w}} &= \overline{\left(\frac{z}{w}\right)}\; ; (w \neq 0)
\end{align}
\end{regel}
\subsubsection{Polarform}
Hvis en vektor kan beskrives med længderne $a$ og $b$ som hhv. ligger langs x aksen og y aksen, vil man også kunne beskrive vektoren ud fra vinklen $\theta$ imellem x-aksen og længden af vektoren $r$. Det giver
\[
 a = r\cos{\theta},\; b = r \sin{\theta}
\]
Ved en vektor $(a,b)$ som et komplekst tal $z = a + ib$, får man
\[
 z = r \cos{\theta} + ir \sin{\theta}
\]
For præcist at finde de tre koordinater, benytter man
\[
 r = \sqrt{a^2 + b^2},\; \cos{\theta} = \frac{a}{r},\; \sin{\theta} = \frac{r}{b}
\]

$r$ kaldes normalt for \emph{modulus} til $z$ og $\theta$ kaldes for argumentet. For dem gælder der yderligere at
\begin{thm}
 Hvis $z_1 = a_1 + ib_1$ og $z_2 = a_2 + ib_2$ er to komplekse tal og $z = (a_1a_2 - b_1b_2) + i(a_1b_2 + a_2b_1)$ er produktet. Da vil der med $z_1$ med modulus $r_1$ og argument $\theta_1$ og $z_2$ med modulus $r_2$ og argument $\theta_2$ gælde at $z$ har modulus $r_1r_2$ og argument $\theta_1 + \theta_2$.
\end{thm}

\subsubsection{Trekantsuligheden for komplekse tal}
\[
  \abs{z + w} \leq \abs{z} + \abs{w}
\]

\subsubsection{Komplekse eksponentialer}
Hvis $z = a + ib$ er et komplekst tal
\[
 e^z = e^a(\cos{b} + i\sin{b}
\]
hvor $e^a$ er modulus og $b$ er argumentet.

For alle komplekse tal $z,w$ gælder
\[
e^{z+w} = e^z \cdot e^w
\]

\begin{thm}[De Moivres formel]
For alle naturlige tal $n$ gælder der at
\[
(\cos{\theta} + i\sin{\theta})^n = \cos{n\theta} + i\sin{n\theta}
\]
\end{thm}

Der gælder i øvrigt at
\begin{align*}
  \cos{(n\theta)} &= \cos^n{\theta} - \begin{pmatrix} n \\ 2 \end{pmatrix} \cos^{n-2}{\theta} \sin^2{\theta} + \begin{pmatrix} n \\ 4 \end{pmatrix} \cos^{n-4}{\theta} \sin^4{\theta} - \ldots \\
  \sin{(n\theta)} &= n\cos{n-1}{\theta} \sin{\theta} - \begin{pmatrix} n \\ 3 \end{pmatrix} \cos^{n-3}{\theta} \sin^3{\theta} + \begin{pmatrix} n \\ 5 \end{pmatrix} \cos^{n-5}{\theta} \sin^5{\theta}  \ldots
\end{align*}
hvor $n$ er et naturligt talt.

\begin{thm}
Hvis $z = re^{i\theta}$ er et komplekst tal forskellig fra $0$ og $n$ er et naturligt tal, vil $z$ have $n$ rødder $w_0, w_1, w_2,\ldots ,w_{n-1}$ og er givet ved
\[
w_k = r^{1/n}e^{i(\theta + 2k\pi)/n} = r^{1/n}\left(\cos{\frac{\theta + 2k\pi}{n}} + i\sin{\frac{\theta + 2k\pi}{n}}\right)
\]
\end{thm}

\subsubsection{Komplekse andengradsligninger}
\begin{thm}
 Antag at a,b og c er komplekse tal, $a \neq 0$. Da vil en andensgradsligning på formen $az^2 + bz + c = 0$ have løsningen
\[
 z = \frac{-b \pm \sqrt{b^2 - 4ac}}{2a}
\] 
\end{thm}

\begin{thm}
Lad $ax^2 + bx + c = 0$ være en andensgradsligning med reelle koefficienter.
\begin{enumerate}
 \item Hvis $b^2 > 4ac$ har ligningen to relle rødder \[ z = \frac{-b \pm \sqrt{b^2 - 4ac}}{2a} \]
 \item Hvis $b^2 = 4ac$ har ligningen en reel rod \[ z = \frac{-b}{2a} \]
 \item Hvis $b^2 < 4ac$ har ligningen to komplekse rødder \[ z = \frac{-b \pm i\sqrt{4ac-b^2}}{2a} \] som er hinandens konjugerede.
\end{enumerate}
\end{thm}

\subsection{Følger og konvergens}
\begin{defn}
Følgen ${a_n}$ konvergerer mod et tal $a$, såfremt der for ethvert reelt tal $\varepsilon > 0$ kan findes et tal $N \in \mathbb{N}$ så $\abs{a_n - a} < \varepsilon$ for alle $n \geq N$. Vi skriver dette således
\[
 \lim_{n \rightarrow \infty}{a_n} = a
\]
En følge som konvergerer mod et tal kaldes \emph{konvergent}, mens en følge som ikke konvergerer kaldes \emph{divergent}.
\end{defn}

\subsubsection{Regneregler for grænseværdier}
\begin{regel}[Grænseværdier]
Der gælder følgende regneregler for konvergente følger ${a_n}$ og ${b_n}$, hvor $\lim_{n \rightarrow \infty}{a_n} = A$ og $\lim_{n \rightarrow \infty}{b_n} = B$.
\begin{enumerate}
 \item $\lim_{n \rightarrow \infty}{(a_n + b_n)} = A + B$
 \item $\lim_{n \rightarrow \infty}{(a_n - b_n)} = A - B$; (specielt gælder der at $\lim_{n \rightarrow \infty}{(-b_n)} = -B$)
 \item $\lim_{n \rightarrow \infty}{(a_n \cdot b_n)} = A \cdot B$
 \item Hvis $B \neq 0$, er $\lim_{n \rightarrow \infty}{\left(\frac{a_n}{b_n}\right)} = \frac{A}{B}$.
\end{enumerate}
\end{regel}

\begin{thm}[Konvergens sætningen]
En monoton, begrænset følge er altid konvergent
\end{thm}

\subsection{Kontinuitet}
\begin{defn}
 En funktion $f$ er \emph{kontinuert} i et punkt $a \in D_f$ hvis der for ethvert $\Epsilon > 0$ kan findes et $\delta > 0$ således at når $x \in D_f$ og $\abs{x - a} < \delta$, så er $\abs{f(x) - f(a)} < \varepsilon$.
\end{defn}

\begin{thm}
 Antag at $f$ og $g$ er kontinuerte i punktet $a$. Da vil funktionerne $f+g$, $f-g$ og $f\cdot g$ også være kontinuerte i $a$. Hvis $g(a) \neq 0$ vil $\frac{f}{g}$ også være kontinuert i $a$
\end{thm}

\begin{thm}
 Antag at $g$ er kontinuert i punktet $a$ og $f$ i punktet $g(a)$. Da er den sammensatte funktion $h(x) = f[g(x)]$ også kontinuert i punktet $a$.
\end{thm}

\begin{defn}
 En funktion $f:D_f \rightarrow \mathbb{R}$ kaldes \emph{kontinuert} såfremt $f$ er kontinuert i alle punkter $a \in D_f$.
\end{defn}

\subsubsection{Skæringssætningen}
\begin{thm}[Skæringssætningen]
Antag at $f:[a,b] \rightarrow \mathbb{R}$ er en kontinuert funktion hvor $f(a)$ og $f(b)$ har modsatte fortegn. Da findes der et tal $c \in (a,b)$ således at $f(c) = 0$.
\end{thm}

\begin{korol}
 Antag at $g:[a,b]\rightarrow \mathbb{R}$ og $h:[a,b]\rightarrow \mathbb{R}$ er to kontinuerte funktioner, sådan at $g(a) < h(a)$ og $g(b) > h(b)$ Da vil der findes et $c \in (a,b)$ sådan at $g(c) = h(c)$.
\end{korol}

\subsubsection{Ekstremalværdisætningen}
\begin{defn}
En funktion $f:A\rightarrow \mathbb{R}$ er begrænset såfremt der findes et reelt tal $M$ sådan at $\abs{f(x)} \leq M$ for alle $x \in A$.
\end{defn}

\begin{thm}
En kontinuert funktion $f:[a,b] \rightarrow \mathbb{R}$ som er defineret i et lukket begrænset interval, er altid begrænset.
\end{thm}

\begin{defn}
 En punkt $a$ er et maksimum for funktionen $f:D_f \rightarrow \mathbb{R}$ hvis $f(a) \geq f(x)$ for alle $x \in D_f$. Vi kalder $a$ for et minimum hvis $f(a) \leq f(x)$ for alle $x \in D_f$. Disse punkter kaldes tilsammen for ekstremalpunkter.
\end{defn}

\begin{thm}[Ekstremalværdisætningen]
Lad $f:[a,b] \rightarrow \mathbb{R}$ være en kontinuert funktion defineret på et lukket begrænset interval. Da har $f$ både maksimums og minimums-punkter.
\end{thm}

\subsection{Grænseværdier}
De generelle regneregler for grænseværdier er siger at for en grænseværdierne $\lim_{x\rightarrow a}{f(x)} = F$ og $\lim_{x\rightarrow a}{g(x)} = G$, gælder der at
\begin{regel}[Grænseværdier]
\begin{align*}
 (i)	&\; \lim_{x\rightarrow a}{[f(x) + g(x)]} = F + G \\
 (ii)	&\; \lim_{x\rightarrow a}{[f(x) - g(x)]} = F - G \\ 
 (iii)	&\; \lim_{x\rightarrow a}{f(x) \cdot g(x)} = F \cdot G \\ 
 (iv)	&\; \lim_{x\rightarrow a}{\frac{f(x)}{g(x)}} = \frac{F}{G};\; \text{forudsat}G \neq 0
\end{align*}
\end{regel}

\begin{defn}
 Vi siger at $f(x)$ går mod b når x nærmer sig a ovenfra, såfremt der for enhver $\Epsilon > 0$ findes et $\delta > 0$ således at $\abs{f(x) - b} < \Epsilon$ for alle $x$ sådan at $a < x < a + \delta$. Vi siger at
\[
\lim_{x\rightarrow a^+}{f(x)} = b
\]
Ligeledes siger vi at $f(x)$ går mod b når x nærmer sig a nedefra, såfremt der for enhver $\Epsilon > 0$ findes et $\delta > 0$ således at $\abs{f(x) -b} < \Epsilon$ for alle x, sådan at $a - \delta < x < a$ og vi siger
\[
 \lim_{x\rightarrow a^-}{f(x)} = b
\]
\end{defn}

\begin{defn}
 Vi siger at $f(x)$ går mod b som grænseværdi, når x går mod $\infty$ hvis der for ethvert $\Epsilon > 0$ findes et $N \in \mathbb{R}$ sådan at $\abs{f(x)-b} < \Epsilon$ for alle $x \leq N$. Vi skriver $\lim_{x\rightarrow \infty}{f(x)} = b$. 
 
Ligeledes gælder der at når $f(x)$ nærmer sig b som grænseværdi når x går mod $-\infty$ hvor der for enhver $\Epsilon > 0$ findes et $N \in \mathbb{R}$ sådan at $\abs{f(x)-b} < \Epsilon$ når $x \leq N$. Vi skriver $\lim_{x \rightarrow \infty}{f(x)} = b$.
\end{defn}
De sædvanlige regneregler gælder også når $a = \infty$ og $a = -\infty$.

\begin{defn}
Vi siger at $f(x)$ går mod $\infty$ når x nærmer sig $a \in \mathbb{R}$ hvis der for ethvert $N \in \mathbb{R}$ findes et $\delta > 0$ sådan at $f(x) > N$ når $0 < \abs{x-a} < \delta$. Vi skriver at
\[
 \lim_{x \rightarrow a}{f(x)} = \infty
\]

Tilsvarende gælder der at når $f(x)$ går mod $-\infty$ såfremt der for ethvert $N \in \mathbb{R}$ findes et $\delta$ sådan at $f(x) < N$ når $0 < \abs{x-a} < \delta$. Vi skriver at
\[
 \lim_{x \rightarrow a}{f(x)} = -\infty
\]
Tilsvarende vis defineres hvad der menes med de ensidige grænser $\lim_{x \rightarrow a^+}{f(x)} = -\infty$ og $\lim_{x \rightarrow a^-}{f(x)} = -\infty$.
\end{defn}

\begin{defn}
 Vi siger at $f(x)$ går mod $\infty$ når x går mod $\infty$ hvis der for ethvert $N \in \mathbb{R}$ findes et $M \in \mathbb{R}$ sådan at $f(x) \geq N$ når $x \geq M$. Dette skrives 
\[
 \lim_{x \rightarrow \infty}{f(x)} = \infty
\]
\end{defn}

\subsection{Differentiation}
\subsubsection{Generelle regneregler}
\begin{defn}
Antag at $f$ er defineret i en omegn om punktet $a$. Der gælder 
\[
 \lim_{x \rightarrow a}{\frac{f(x) - f(a)}{x -a}}
\]
og vi har at
\[
f'(a) = \lim_{x \rightarrow a}{\frac{f(x) - f(a)}{x -a}}
\]
\end{defn}

\begin{regel}[Differentiation]
\begin{align*}
 (i)&	\; (cf)'(a) = c \cdot f'(a) \\
 (ii)&	\; (f + g)'(a) = f'(a) + g'(a) \\ 
 (iii)&	\; (f - g)'(a) = f'(a) - g'(a) \\ 
 (iv)&	\; (f \cdot g)'(a) = f'(a)g(a) + f(a)g'(a) \\
 (v)&	\; \left(\frac{f}{g}\right)'(a) = \frac{f'(a)g(a) - f(a)g'(a)}{g(a)^2} 
\end{align*}
\end{regel}

\begin{regel}[Kædereglen]
Kædereglen gælder for differentation af sammensatte funktioner og udtrykkes ved
\[
 h'(a) = f'[g(a)]g'(a)
\]
\end{regel}

\subsubsection{Middelværdisætningen}
\begin{thm}[Rolles' sætning]
 Antag at funktionen $f:[a,b] \rightarrow \mathbb{R}$ er kontinuert og at den er differentiabel i alle indre punkter $x \in (a,b)$. Antag yderligere at $f(a) = f(b)$. Da findes et punkt $c \in (a,b)$ således at $f'(c) = 0$.
\end{thm}
 
\begin{thm}[Middelværdisætningen]
Antag at funktionen $f:[a,b] \rightarrow \mathbb{R}$ er kontinuert og at den er differentiabel i alle indre punkter $x \in (a,b)$. Da findes et punkt $c \in (a,b)$ således at 
\[
f'(c) = \frac{f(b) - f(a)}{b-a}
\]
\end{thm}

\begin{korol}
Ved en kontinuerlig funktion $f$ i intervallet $[a,b]$ gælder der at hvis $f'(x) \geq 0$ for alle $x \in (a,b)$ er $f$ voksende i $[a,b]$. Hvis $f'(x) \leq 0$ for alle $x \in (a,b)$ er $f$ aftagende i $[a,b]$.

Hvis der i stedet er strenge uligheder, hvor $f'(x) > 0$ er $f$ strengt voksende og ved $f'(x) < 0$ er $f$ strengt aftagende.
\end{korol}

\subsubsection{L'Hôpitals regel}
\begin{thm}[Cauchys middelværdisætning]
Ved to kontinuerte differentiabele funktioner $f,g:[a,b]\rightarrow \mathbb{R}$ findes der et $c \in (a,b)$ således at
\[
 \frac{f(b) - f(a)}{g(b) - g(a)} = \frac{f'(c)}{g'(c)}
\]
\end{thm}

\begin{regel}[L'Hôpitals regel for 0/0 udtryk]
Antag at $\lim_{x\rightarrow a}{f(x)} = \lim_{x \rightarrow a}{g(x)} = 0$ og at grænseværdien
\[
 \lim_{x \rightarrow a}{\frac{f'(x)}{g'(x)}}
\]
eksisterer og at kan være lig $\infty$ eller $-\infty$. Da eksisterer en grænseværdi $\lim_{x \rightarrow a}{f(x)/g(x)}$ og 
\[
 \lim_{x \rightarrow a}{\frac{f(x)}{g(x)}} = \lim_{x \rightarrow a}{\frac{f'(x)}{g'(x)}}
\]
\end{regel}

\begin{regel}[L'Hôpitals regel for $\infty$/$\infty$ udtryk]
Hvis $\lim_{x \rightarrow a}{f(x)} = \lim_{x \rightarrow a}{g(x)} = \infty$ og grænseværdier $\lim_{x \rightarrow a}{\frac{f'(x)}{g'(x)}}$ eksisterer, siges det at grænseværdier $\lim_{x \rightarrow a}{f(x)/g(x)}$ og
\[
 \lim_{x \rightarrow a}{\frac{f(x)}{g(x)}} = \lim_{x \rightarrow a}{\frac{f'(x)}{g'(x)}}
\]
\end{regel}

\subsubsection{Vækst af potenser, logaritmer og eksponentialfunktioner}
\begin{thm}
\begin{align*}
 a)&\;	\lim_{x \rightarrow \infty}{\frac{(\ln{x})^a}{x^b}} = 0\;\; \text{for alle } a,b > 0 \\
 b)&\;	\lim_{x \rightarrow \infty}{\frac{x^b}{e^{ax}}} = 0 \;\; \text{for alle } a,b > 0 
\end{align*}
\end{thm}

\begin{thm}
For alle $a,b > 0$ er $\lim_{x \rightarrow 0^+}{x^2\abs{\ln{x}}^b} = 0$
\end{thm}

\subsubsection{Undersøgelser af grafer}
Her gives der en kort beskrivelse af de sætninger som kan være nyttige til undersøgelse af grafer.
\begin{defn}
Et punkt $a \in \real$ kaldes for et lokalt minimum for en funktion såfremt der findes et tal $\delta > 0$ sådan at $f(a) \geq f(x)$ for alle $x \in D_f \cap (a-\delta,a+\delta)$. Vi kalder $a$ et maksimum hvis $f(a) \geq f(x)$ for alle $x \in D_f$. 

Tilsvarende siger vi at $a \in \real$ er et lokalt minimum for $f$ såfremt der findes et tal $\delta > 0$ sådan at $f(a) \leq f(x)$ for alle $x \in D_f \cap (a-\delta,a+\delta)$ og vi kalder $a$ for et minimum såfremt $f(a) \leq f(x)$ for alle $x \in D_f$.
\end{defn}

\begin{thm}
Antag at funktionen $f : [a,b] \rightarrow \real$ har et lokalt minimum eller maksimum i $c$. Da gælder der enten at
\begin{enumerate}
\item c er et af endepunkterne a og b eller
\item $f'(x) = 0$ eller
\item f er ikke differentiabel i c
\end{enumerate}
\end{thm}

\begin{thm}
Om en kontinuert (i $c$) differentiabel (i en omegn af $(c-\delta,c+\delta)$) funktion $f$ gælder der

Hvis $f'(x) \geq 0$ for alle $x \in (c - \delta, c)$ og $f'(x) \leq 0$ for alle $x \in (c, c + \delta)$ er $c$ et lokalt maksimum. 

Hvis $f'(x) \leq 0$ for alle $x \in (c - \delta,c)$ og $f'(x) \geq 0$ for alle $x \in (c,c + \delta)$ er $c$ er lokalt minimum. 

Hvis $f'$ har samme fortegn i begge intervaller $(c-\delta,c)$ og $(c,c+\delta)$ er $c$ hverken lokalt minumum eller lokalt maksimum.
\end{thm}

\subsection{Integration}
\subsubsection{Generelle regneregler}
\begin{thm}
Antag at $f:[a,b] \rightarrow \real$ er begrænset og at $c \in (a,b)$. Da gælder
\[
\int_a^b{f(x)\,dx} = \int_a^c{f(x)\,dx} + \int_c^b{f(x)\,dx}
\]
\end{thm}

\begin{thm}[Analysens fundamentalsætning]
Antag at $f : [a,b] \rightarrow \real$ er en kontinuert funktion. Da er $f$ integrabel på ethvert interval $[a,x]$ hvor $a \leq x \leq b$ og funktionen
\[
 F(x) = \int_a^x{f(t)\,dt}
\]
er stamfuntionen til $f$ på $[a,b]$.
\end{thm}

\begin{korol}
Ved samme funktion gælder der at
\[
\int_a^b{f(x)\,fx} = K(b) - K(a)
\]
\end{korol}

\begin{regel}[Integration]
\begin{enumerate}
 \item $\int{af(x)\,dx} = a \int{f(x)\,fx}$
 \item $\int{[f(x) + g(x)]\,dx} = \int{f(x)\,dx} + \int{g(x)\,dx}$
 \item $\int{[f(x) - g(x)]\,dx} = \int{f(x)\,dx} - \int{g(x)\,dx}$
 \item Hvis $\int{f(x)\,dx} = F(x) + C$ og $a \neq 0$, så er $\int{f(ax)\,dx]} = \frac{F(ax)}{a}+C$
\end{enumerate}
\end{regel}

\subsubsection{Partiel integration}
Ved partiel integration integrerer man efter følgende formel
\[
\int{u(x)v'(x)\,dx} = u(x)v(x) - \int{u'(x)v(x)\,dx}
\]

\subsection{Integration med substitution}
Denne regel opskrives generelt
\begin{thm}
Hvis $g$ er differentiabel, $f$ er kontinuert og $F$ er en stamfunktion til $f$, så er
\[
 \int{f[g(x)]g'(x)\,dx} = F[g(x)] + C
\]
I praksis betyder dette at man vælger at substituere en del af integralet ud med $u = g(x)$ og finder da $du = g'(x)\,dx$. Dette skal vælges således at integralet er lettest at løse.
\end{thm}

\subsection{Differentialligninger}
\subsubsection{Førsteordens, lineære differentialligninger}
\begin{thm}
En førsteordens lineær differentialligning er givet ved det generelle udtryk
\[
 y' + f(x)y = g(x)
\]
og har den generelle løsning
\[
 y = e^{-F(x)}\left(\int{e^{F(x)}g(x)\,dx}+C\right)
\]
\end{thm}
Denne kan også skrives på den mere generelle form
\begin{thm}
Ved en differentialligning givet ved
\[
 y' + f(x)y = g(x)
\]
findes der netop én løsning, givet ved
\[
 y(x) = e^{-\int_c^x{f(t)dt}}\left(\int_c^x{g(t)e^{\int_c^t{f(s)ds}}\,dt}+d\right)
\]
\end{thm}

\subsubsection{Separable differentialligninger}
Ligninger som kan løses med seperation af variable, kan skrives på formen
\[
 p(x)y' = q(y)
\]
og løses ved at samle ens variable på hver side af lighedstegnet, således at udtrykket bliver
\[
q(y)y' = p(x)
\]
og man kan nu løse med almindelige regneregler for differentiation og integration.

\subsubsection{Andenordens, homogene ligninger med konstante koefficienter}
En andensordens lineær differentialligning er givet ved udtrykket
\[
y'' + p(x)y' + q(x)y = h(x)
\]
Af denne findes også undertypen 
\[
y'' + py' + qy = 0
\]
som er en homogen differentialligning.

Til en sådan differentialligning hører en karakteristisk andengradsligning med på formen
\[
r^2 + pr + q = 0
\]
Løser man denne får man tre tilfælde.

\begin{thm}[To reelle rødder]
Antag at den karakteristiske ligning til 
\[
y'' + py' + qy = 0
\]
har to reelle rødder. Da er alle løsninger til differentialligningen på formen
\[
y = Ce^{r_1x}+De^{r_2x}
\]
\end{thm}
Den specifikke løsning til differentialligningen findes hvis man kender startbetingelserne $y(c) = d$ og $y'(c) = e$

For at bestemme konstanterne $C$ og $D$ opstiller man ligningerne $y(c) = Ce^{r_1c}+De^{r_2c}$ og $y'(c) = Cr_1e^{r_1c}+Dr_2e^{r_2c}$. Man skal nu løse ligningssystemet således at $y(c) = d$ og $y'(c) = e$ er opfyldt. Tallene $c, d$ og $e$ er de startbetingelser man får oplyst.

\begin{thm}[Én reel rod]
Hvis den karakteristiske ligning til $y'' + py' + qy = 0$ har en reel rod, så findes den generelle løsning til differentialligningen på formen
\[
y = Ce^{r_1x} + Dxe^{r_1x}
\]
\end{thm}
Metoden til at finde konstanterne $C$ og $D$, og dermed den specifikke løsning, findes på samme måde som ved de to reelle rødder.

\begin{thm}[To komplekse rødder]
Hvis den karakteristiske ligning til $y'' + py' + qy = 0$ har to komplekse rødder $r_1 = a + ib$ og $r_2 = a - ib$, så er alle løsningerne til differentialligningen på formen
\[
 y = e^{ax}(C\cos{bx} + D\sin{bx})
\]
\end{thm}
Og ved kendskab til startbetingelserne $y(c) = d$ og $y'(c) = e$ kan man som i de foregående tilfælde, finde den specifikke løsning til differentiallingningen. Metoden og tankegangen er den samme.

I visse tilfælde kan det være en fordel at omskrive løsningen på \emph{faseform}
\[
 y = Ae^{ax}\sin{(bx + \phi)}
\]
hvor $A = \sqrt{C^2 + D^2}$ kaldes for amplituden og fasen $\phi$ er bestemt ved
\[
\sin{\phi} = \frac{C}{\sqrt{C^2 + D^2}}
\]
og
\[
\cos{\phi} = \frac{D}{\sqrt{C^2 + D^2}}
\]

\subsubsection{Andenordens inhomogene ligninger}
Andensordens inhomogene differentialligninger findes på den generelle form
\[
y'' + py' + qy = f(x)
\]
og løsningen findes generelt ved
\begin{thm}
Antag at $y_p$ er en løsning til den inhomogene ligning
\[
y'' + py' + qy = f(x)
\]
Da er de andre løsninger givet ved
\[
y = y_p + y_h
\]
hvor $y_h$ er en vilkårlig løsning til den homogene ligning $y'' + py' + qy = 0$.

$y_p$ kaldes for den partikulære løsning og $y_h$ kaldes den homogene løsning.
\end{thm}

\subsubsection{Ukendte koefficienters metode}
For de ukendte koefficienters metode gælder der generelt tre regler. Generelt søger man at gætte sig til en løsning $f(x) = y_p$ og derefter indsætte denne i differentialligningen således at man får
\[
ay_p'' + by_p' + cy_p = 0
\]
og løser denne. Tallene $(a,b$ og $c)$ er de eventuelle konstanter der måtte være i den oprindelige differentialligning.

\begin{regel}[Regel 1]
Når $f(x)$ er et polynomium, gætter man på en løsning som også er et polynomium. Som regel vil $y_p$ være af samme grad som $f(x)$, men i visse tilfælde må man gå en eller flere grader op.
\end{regel}

\begin{regel}[Regel 2]
Hvis $f(x) = a^xP(x)$ hvor $a$ er et positivt tal og $P(x)$ er et polynomium, gætter man på en løsning af formen $y_p = a^xQ(x)$ hvor $Q$ er et polynomium. Som regel vil $Q(x)$ have samme grad som $P(x)$, men i visse tilfælde må man også her gå en eller flere grader op.
\end{regel}

\begin{regel}[Regel 3]
Hvis $f(x) = a^x(A \cos{bx} + B \sin{bx})$ forsøger man at finde en løsning på formen $y_p = a^x(C \cos{bx} + D \sin{bx})$. Hvis $y = a^x \cos{bx}$ eller ækvivalente, som $y = a^x \sin{bx}$ er en løsning af den homogene ligning, må man forsøge med $y_p = x\cdot a^x(C \cos{bx} + D \sin{bx})$ i stedet.
\end{regel}

Generelt er det en god ide at løse den homogene ligning først, især i forbindelse med de to sidste regler, da man her vil vide om den valgte løsning til den partikulære er den samme som den homogene.

\subsubsection{Variation af parametre}
For at have en metode til at løse andenordens inhomogene differentialligninger benytter man metoden \emph{variation af parametre}. For at gøre brug af denne metode må man først finde den såkaldte Wronski determinant. Denne findes ved at finde den homogene løsning til differentialligningen $y'' + py' + qy = f(x)$. Hvis det eksempelvis er ligningen
\[
y'' - y = \sin{x}
\]
kan man finde en homogen løsning på formen
\[
y = Ce^x + De^{-x}
\]
og her vil man så have $y_1(x) = e^x$ og $y_2(x) = e^{-x}$. Man opstiller da
\[
W(y_1,y_2) = y_1y_2' - y_1'y_2
\]
Og fra disse opstiller man da integralerne
\[
c(x) = -\int{\frac{y_2(x)f(x)}{W(y_1,y_2)}dx} 
\]
og
\[
d(x) = \int{\frac{y_1(x)f(x)}{W(y_1,y_2)}dx}
\]
og man kan da opstille den fuldstændige løsning til differentialligningen som
\[
y(x) = c(x)y_1(x) + d(x)y_2(x)
\]

\subsection{Taylor serier}
For at finde et Taylor polynomium, benytter man generelt metoden
\begin{defn}[Taylor polynomium]
Hvis f er en $n$ gange differentiabel funktion i punktet $a$, kan man finde polynomiet
\[
T_nf(x) = \sum_{k=0}^n{\frac{f^{(k)}(a)}{k!}(x-a)^k}
\]
\end{defn}
og man vil generelt have et polynomiun lig
\[
h(x) = f(a) + f'(a)(x-a)+\frac{f''(a)}{2}(x-a)^2 + \frac{f'''(a)}{6}(x-a)^3+\cdots+\frac{f^{(n)}(a)}{n!}(x-a)^n
\]

\subsubsection{Taylors formel med restled}
\begin{thm}[Taylors formel med restled]
Antag at $f$ og dens $n+1$ første afledte er kontinuert i intervallet $[a,b]$. Da er
\[
f(b) = T_nf(b) + \frac{1}{n!}\int_a^b{f^{(n+1)}(t)(n-t)^n\,dt}
\]
Taylors formel med restled, hvor $T_nf(b)$ er det oprindelige Taylor polynomium.
\end{thm}

\begin{korol}
Hvis man lader $M$ være et tal således at $\abs{f^{(n+1)}(t)}\leq M$ for alle t imellem a og x, får man
\[
\abs{R_nf(x)}\leq \frac{M}{(n+1)!}\abs{x-a}^{n+1}
\]
\end{korol}

\begin{thm}[Lagranges restledsformel]
Antag at funktionen $f$ og dens $n+1$ første afledte er kontinuert i intervallet $[a,x]$. Da er
\[
R_nf(x) = \frac{f^{(n+1)}(c)}{(n+1)!}(x-a)^{n+1}
\]
for et tal c i det åbne interval mellem a og x.
\end{thm}
og vi vil da få
\[
f(x) = f(a) + f'(a)(x-a) + \frac{f''(a)}{2}(x-a)^2 + \cdots + \frac{f^{(n)}(a)}{n!}(x-a)^n + \frac{f^{(n+1)}(c)}{(n+1)!}(x-a)^{n+1}
\]

\section{Funktioner af flere variable}
\subsection{Koordinatsystemer}
Udover de almindelige kartesiske koordinater, kan man også benytte
\paragraph*{Polære koordinater}
Som er givet ved
\begin{align*}
x &= r\cos{\theta} \\
y &= r\sin{\theta} \\
\end{align*}

\paragraph*{Cylinderkoordinater}
Som er givet ved
\begin{align*}
x &= r\cos{\theta} \\
y &= r\sin{\theta} \\
z &= z
\end{align*}

\paragraph*{Kuglekoordinater}
Som er givet ved
\begin{align*}
x &= \rho \cos{\theta} \sin{\phi} \\
y &= \rho \sin{\theta} \sin{\phi} \\
z &= \rho \cos{\phi}
\end{align*}

\subsection{Topologiske begreber}
\subsubsection{Åbenhed og lukkethed}
\begin{defn}
En mængde $A \subseteq \real^n$ siges at være åben hvis $A$ er lig det indre af $A$.
\end{defn}
I praksis kan en mængde siges at være åben hvis den er afgrænset af $\leq$ eller $\geq$.

\begin{defn}
En mængde $A \subseteq \real^n$ siges at være lukket hvis $A = \overline{A}$
\end{defn}
Man kan også benytte sætningen om indre punkter
\begin{defn}
Lad $A \subseteq \real^n$ og $\Bf{a} \in \real^n$. Da er $\Bf{a}$ et indre punkt for $A$ hvis og kun hvis der findes et $\delta > 0$ således at alle $\Bf{x}$ som opfylder 
\[
\abs{\abs{\Bf{x}-\Bf{a}}}>\delta
\]
\end{defn}
I praksis siges en mængde at være lukket, hvis den er afgrænset af $<$ eller $>$.

\subsubsection{Begrænsning}
\begin{defn}
Lad $A \subseteq \real^n$. VI siger at $A$ er begrænset hvis der findes et positivt reelt tal, $R$, således at der for alle $\Bf{a} \in A$ gælder
\[
 \abs{\abs{\Bf{a}}} > R
\]
\end{defn}
I praksis siges en mængde at være begrænset såfremt man kan finde et udtryk, som repræsenterer en cirkel, som kan afgrænse mængden.

\subsection{Grænseværdier for flere variable}
\begin{defn}
Antag at $f:A\rightarrow \real$ er defineret på en delmængde $A \subseteq \real^n$ og at $ \Bf{a} $ er et akkumulationspunkt for $ A $. Vi siger at det reelle tal $ L $ er grænseværdien for $ f(\Bf{x}) $, når $ \Bf{x} $ går mod $ \Bf{a} $, såfremt der for ethvert tal $\Epsilon > 0$ findes et tal $\delta > 0$, således at der for alle $\Bf{x} \in A$, som opfylder uligheden $0 < \abs{\abs{\Bf{x}-\Bf{a}}} < \delta$, gælder at $\abs{f(\Bf{x})-L} < \Epsilon$. Vi skriver da
\[
L = \lim_{\Bf{x}\rightarrow \Bf{a}}{f(\Bf{x})}
\]
\end{defn}
Denne måde er ret besværlig at regne med og derfor indføres der en række regneregler, man benytter
\begin{regel}
\begin{align*}
1.&\; \lim_{\Bf{x}\rightarrow \Bf{a}}{(f(\Bf{x}) + g(\Bf{x}))} = \lim_{\Bf{x}\rightarrow \Bf{a}}{f(\Bf{x})} + \lim_{\Bf{x}\rightarrow \Bf{a}}{g(\Bf{x})} \\
2.&\; \lim_{\Bf{x}\rightarrow \Bf{a}}{(f(\Bf{x}) - g(\Bf{x}))} = \lim_{\Bf{x}\rightarrow \Bf{a}}{f(\Bf{x})} - \lim_{\Bf{x}\rightarrow \Bf{a}}{g(\Bf{x})} \\
3.&\; \lim_{\Bf{x}\rightarrow \Bf{a}}{)f(\Bf{x}) \cdot g(\Bf{x}))} = \lim_{\Bf{x}\rightarrow \Bf{a}}{f(\Bf{x})} \cdot \lim_{\Bf{x}\rightarrow \Bf{a}}{g(\Bf{x})} \\
4.&\; \lim_{\Bf{x}\rightarrow \Bf{a}}{\frac{f(\Bf{x})}{g(\Bf{x})}} = \frac{\lim_{\Bf{x}\rightarrow \Bf{a}}{f(\Bf{x})}}{\lim_{\Bf{x}\rightarrow \Bf{a}}{g(\Bf{x})}}\;\text{forudsat at }\lim_{\Bf{x}\rightarrow \Bf{a}}{g(\Bf{x})} \neq 0 \\
5.&\; \lim_{\Bf{x}\rightarrow \Bf{a}}{h(f(\Bf{x}))} = h(L)\;\text{såfremt }\lim_{\Bf{x}\rightarrow \Bf{a}}{f(\Bf{x})} = L \text{ og h er kontinuert}
\end{align*}
\end{regel}

\subsection{Kontinuitet}
\begin{defn}
Funktionen f er kontinuert i punktet $\Bf{a} \in D_f$ såfremt $ \Bf{a} $ er et isoleret punkt for $D_f$ eller
\[
\lim_{\Bf{x}\rightarrow \Bf{a}}{f(\Bf{x})} = f(\Bf{a})
\]
Vi siger at f er kontinuert, såfremt den er kontinuert i alle punkter i sin definitionsmængde.
\end{defn}

\begin{thm}
Lad funktionerne $ f $ og $ g $ være kontinuerte i $ \Bf{a} $. Da er funktionerne $f+g$,$f-g$,$f\cdot g$ (forudsat at $g(\Bf{x}) \neq 0$) og $\frac{f}{g}$ også kontinuerte.
\end{thm}

\begin{thm}
Hvis $f_1,\ldots,f_n$ er funktioner af $ m $ variable  som alle er kontinuerte i $\Bf{a} \in \real^m$ og $ g $ er en funktion af $ n $ variable, der er kontinuerte i punktet $(f_1(\Bf{a}),\ldots,f_n(\Bf{a}))$, så er sammensætningen $g \circ f$
\[
x \mapsto g(f_1(\Bf{x}),\ldots,f_n(\Bf{x}))
\]
også kontinuert i $ \Bf{a} $.
\end{thm}

\subsection{Ekstremalværdisætningen}
\begin{defn}
Et punkt i $ \Bf{a} $ i definitionsmængden for $ f $ kaldes et størsteværdipunkt for funktionen, såfremt $f(\Bf{a}) \geq f(\Bf{x})$ for alle $ \Bf{x} $ i definitionsmængden for $ f $. 

Tilsvarende kaldes $ \Bf{a} $ et mindsteværdipunkt for $ f $ såfremt $f(\Bf{a}) \leq f(\Bf{x})$ for alle $ \Bf{x} $ i definitionsmængden for $ f $.
\end{defn}

\begin{thm}[Ekstremalværdisætningen]
Lad $ A $ være en lukket begrænset mængde og antag at $f:A \rightarrow \real$ er kontinuert. Da har både $ f $ en største og mindsteværdi.
\end{thm}

\subsection{Differentiation}
\subsubsection{Retningsafledede}
\begin{defn}
For en kontinuert funktion, hvor $ \Bf{a} $ er et indre punkt i $ A $ og $r \in \real$ kan tænkes som en vektor, gælder der at den retningsafledede kan findes ved
\[
f'(\Bf{a};\Bf{r})=\lim_{h\rightarrow 0}{\frac{f(\Bf{a}+h\Bf{r})-f(\Bf{a})}{h}}
\]
\[
f'(\Bf{a};\Bf{r})=\nabla f(\Bf{a})\cdot\Bf{r}
\]
\end{defn}

\subsubsection{Partielt afledede}
\begin{defn}
Hvis $ f $ er en funktion af $ n $ variable og $ \Bf{a} $ er et indre punkt i $ A $. Da vil den $ i $-te partielt afledede $\frac{\PD f}{\PD x_i}(\Bf{a})$ er den retningsafledede af $ f $ i retning af den $ i $-te enhedsvektor $\Bf{e_i}$. Det vil sige
\[
 \frac{\PD f}{\PD x_i}(\Bf{a}) = f'(\Bf{a},\Bf{e_i})
\]
\end{defn}

Denne definition kan i praksis være svær at bruge. Teknikken bag partielt afledede er dog ret simpel. Hvis man ønsker at differentiere funktionen $f(x,y) = 2x^2 + 4x+2y + 3x$, differentierer man en variabel og de andre holdes konstant. Hvor man differentierer $ x $
\[
\frac{\PD f}{\PD x} = 4x + 7
\]
og ved $ y $
\[
\frac{\PD f}{\PD y} = 2
\]

\subsubsection{$C^1$-funktioner}
\begin{defn}
En funktion $ f $ defineret på en åben mængde $ A $ er $C^1$ såfremt de partielt afledede eksisterer på $ A $ og er kontinuerte.
\end{defn}
Denne sætning kan bruges til at afgøre hvorvidt en funktion af flere variable har pæne egenskaber mht. differentiation.

\subsubsection{Gradient}
En særlig form for partielt afledt kaldes for gradienten og er en vektor. Den benyttes særligt i elektrodynamikken, hvor samtlige af Maxwells berømte ligninger har gradienten som faktor. Den defineres ved
\begin{defn}
Gradienten for en funktion $ f $ af $ n $ variable i et indre punkt $ \Bf{a} $ i definitionsmængden er vektoren
\[
\gradient f(\Bf{a}) = \left(\frac{\PD f}{\PD x_1}(\Bf{a}),\ldots,\frac{\PD f}{\PD x_n}(\Bf{a})\right)
\]
\end{defn}
Eksempelvis vil gradienten til en funktion $f(x,y,z)$ være givet ved
\[
\gradient f(x,y,z) = \left( \frac{\PD f}{\PD x},\frac{\PD f}{\PD y},\frac{\PD f}{\PD z} \right)
\]

\subsubsection{Tangentplan til grafen}
Tangentplanen til en graf kan findes ud fra sætningen
\begin{thm}
 Antag at $ f $ er $C^1$ funktion og $ \Bf{a} $ et indre punkt i $D_f$. Da har $ f $ tangenthyperplan i $ \Bf{a} $ lig grafen for den affine funktion $ h $ givet ved
\[ h(\Bf{x}) = f(\Bf{a}) + \gradient f(\Bf{a}) \cdot (\Bf{x}-\Bf{a}) \]
\end{thm}

\subsubsection{Kædereglen}
En vigtig regneregel indenfor differentiation er kædereglen. Denne kan også benyttes til flere variable og er for en funktion $k(x) = f(g(x),h(x))$ givet ved
\[
\frac{dk}{dx} = \frac{\PD f}{\PD u} \cdot \frac{dg}{dx} + \frac{\PD f}{\PD v} \cdot \frac{dh}{dx}
\]
Denne kan naturligvis udvides til at omfatte flere variable og differentiationer. Derfor navnet.

I praksis betyder reglen at man udregner de partielt afledte som normalt. Man substituerer derefter således at man har $u = f(x)$ og $v = h(x)$.

Skal man udregne en partielt afledt til funktionerne $x(t) = r(t)\cos{\theta(t)}$ og $y(t) = r(t)\sin{\theta(t)}$ og signalstyrken er givet ved
\[
S(t) = f(x(t),y(t))
\]
hvor $x(t)$ og $y(t)$ er som ovenfor. Udregning med kædereglen er da
\[
\frac{dS}{dt} = \frac{\PD f}{\PD x}\frac{dx}{dt} + \frac{\PD f}{\PD y}\frac{dy}{dt}
\]

Den præcise definition af kædereglen er givet ved sætningen
\begin{thm}
Hvis $f(u_1,\ldots,u_m)$ være en reel $C^1$-funktion af m variable defineret på en åben mængde og lad $g(x_1,\ldots,x_n),\ldots,g_m(x_1,\ldots,x_n)$ være m reelle $C^1$ funktioner af n variable med åbne definitionmængder. Da er den sammensatte funktion
\[
h(x_1,\ldots,x_n) = f(g_1(x_1,\ldots,x_n),\ldots,g_m(x_1,\ldots,x_n))
\]
en $C^1$ funktion og hvis h er defineret i a og b $=(g_1(a),\ldots,g_m(a))$ så er
\[
\frac{\PD h}{\PD x_i}(\Bf{a}) = \frac{\PD f}{\PD u_1}(\Bf{b}) \cdot \frac{\PD g_1}{x_i}(\Bf{a}) + \frac{\PD f}{\PD u_2}(\Bf{b}) \frac{\PD f}{\PD x_i}(\Bf{a})+\cdots+\frac{\PD f}{\PD u_m}(\Bf{b})\cdot \frac{\PD g_m}{\PD x_i}(\Bf{a})
\]
\end{thm}

\subsubsection{Niveau kurver og niveau flader}
Niveaukurvens tangentlinje for en funktion af to variable er givet ved definitionen
\begin{defn}[Niveaukurvens tangentlinje]
Lad $ f $ være en $C^1$ funktion i to variable, lad (a,b) være et indre punkt i $D_f$ og lad $c = f(a,b)$. Såfremt $\gradient f(a,b) \neq 0$ definerer vi tangentlinjen for niveaukurven $f(x,y) = c$ i $(a,b)$ til at være givet ved
\[
\gradient f(a,b)\cdot ((x,y)-(a,b)) = 0
\]
\end{defn}

Niveauhyperfladens tangenthyperplan er givet ved følgende definition
\begin{defn}[Niveauhyperfladens tangenthyperplan]
Lad $ f $ være en $C^1$ funktion i $ n $ variable, lad $ \Bf{a} $ være et indre punkt i $D_d$ og lad $c = f(\Bf{a})$. Såfremt $\gradient f(\Bf{a}) \neq 0$ defineres tangenthyperplanen for niveauhyperfladen $f(\Bf{x}) = c$ i $ \Bf{a} $ til
\[
 \gradient f(\Bf{a}) \cdot (\Bf{x}-\Bf{a}) = 0
\]
\end{defn}

\subsubsection{Højere ordens afledede}
Den generelle notation for højere ordens afledede er
\[
\frac{\PD^r f}{\PD x_{i_r},\ldots,\PD x_{i_2} \PD x_{i_1}}
\]

En ofte benyttet regnemetode til løsning af højere ordens afledede, er Hessematricen

Hvis man har gradienten givet ved
\[
\gradient f(x,y) = \left(\frac{\PD f}{\PD x}(x,y),\frac{\PD f}{\PD y}(x,y) \right)
\]
vil Hessematricen være den tilsvarende for højere ordens afledede og være givet ved
\[
Hf(x,y) = \begin{pmatrix} \frac{\PD^2 f}{\PD x^2}(x,y) & \frac{\PD^2 f}{\PD x \PD y}(x,y) \\
 						  \frac{\PD^2 f}{\PD y \PD x}(x,y) & \frac{\PD^2 f}{\PD y^2}(x,y) \end{pmatrix}
\]

\subsubsection{Symmetri blandt partielt afledede}
For en funktion $f(x,y)$ vil de anden ordens afledede være givet ved
\[
\frac{\PD^2 f}{\PD x^2}, \frac{\PD^2 f}{\PD y^2}, \frac{\PD^2 f}{\PD x \PD y}, \frac{\PD f}{\PD y \PD x}
\]
Særligt for de to sidste gælder der at
\[
\frac{\PD^2 f}{\PD x \PD y} = \frac{\PD^2 f}{\PD y \PD x}
\]
Og dette gør sig gældende for alle disse typer af højere ordens afledte. Der gælder også at
\[
\frac{\PD^4 f}{\PD x \PD y \PD z \PD x} = \frac{\PD^4 f}{\PD z \PD y \PD x \PD x}
\]
og vi har
\begin{thm}
Hvis $f(x_1,\ldots,x_n)$ være en $C^2$ funktion af n variable defineret på en åben mængde $A \subseteq \real^n$ og lad $ \Bf{a} $ være et punkt i definitionsmængden. Da er
\[
\frac{\PD^2 f}{\PD x_k \PD x_i}(\Bf{a}) = \frac{\PD^2 f}{\PD x_i \PD x_j}(\Bf{a})
\]

Dette gør sig også gældende for alle funktioner $C^r$ såfremt de blot indeholder lige mange differentiationer med hensyn til hver variabel.
\end{thm}

\subsection{Største og mindsteværdier}
\begin{thm}
Lad $f:A \rightarrow \real$ være en funktion af n variable. Antag at f har et lokalt maksimum eller minimum i $ \Bf{a} $. Da er en af følgende tre betingelser opfyldt
\begin{enumerate}
\item $ \Bf{a} $ er et randpunkt for A eller
\item gradienten $\gradient f$ eksisterer ikke i $ \Bf{a} $ eller
\item $\gradient f(\Bf{a}) = 0$
\end{enumerate}
\end{thm}

Eksempelvis for funktionen $f(x,y) = 3xy-3x+9y$ som har de partielt afledede $\frac{\PD f}{\PD x} = 3y-4$ og $\frac{\PD f}{\PD y} = 3x+9$. Da vil et maksimums eller minimumspunkt findes hvor begge partielt afledede giver nul. Dette giver ligningerne
\begin{align*}
 3y-3 &= 0 \\
 3x+9 &= 0 
\end{align*}
og dette giver løsningerne $x = -3$ og $y = 1$. Dette betyder at det eneste mulige maksimums eller minimumspunkt for f er $(-3,1)$.

\subsubsection{ABC-kriteriet}
\begin{thm}
Lad $f(x,y)$ være en $C^2$ funktion og antag at $(a,b)$ er et stationært punkt for f. Lad
\[
A = \frac{\PD^2 f}{\PD x^2}(a,b), B=\frac{\PD^2 f}{\PD x \PD y}(a,b), C=\frac{\PD^2 f}{\PD y^2}(a,b)
\]
og lad $D = AC - B^2$. Da gælder der at
\begin{enumerate}
\item Hvis $D < 0$, så er $(a,b)$ et saddelpunkt.
\item Hvis $D > 0$ og $A > 0$, så er $(a,b)$ et lokalt minimum.
\item Hvis $D > 0$ og $A < 0$, så er $(a,b)$ et lokalt maksimum.
\end{enumerate}
\end{thm}

Tallet $D$ kan også findes som determinanten til Hessematricen
\[
D = \det{(H(f(a,b))} = \begin{vmatrix} \frac{\PD^2 f}{\PD x^2}(a,b) & \frac{\PD^2 f}{\PD x \PD y}(a,b) \\
 					   \frac{\PD^2 f}{\PD y \PD x}(a,b) & \frac{\PD^2 f}{\PD y^2}(a,b) \end{vmatrix} =
 					   \begin{vmatrix} A & B \\ B & C \end{vmatrix} = AC - B^2
\]

\subsubsection{Lagranges multiplikationsmetode}
For at udregne maks/min problemer med bi-betingelser, kan man benytte to metoder, som her vil være beskrevet med to eksempler fra bogen. I det første eksempel benytter jeg ikke Lagranges metode.

Den generelle problemstilling. Vi skal producere cylinderformede dåser med top og bund og vi skal nu undersøge hvor stor en dåse vi kan lave, givet vi gerne vil have så meget dåse for pengene. Givet at dåsernes volumen er V, hvordan skal højden h og diameteren d vælges.

Vi ser at overflade arealet er givet ved
\[
A(h,d) = \frac{\pi}{2}d^2+\pi hd
\]
Bibetingelsen angående volumet giver
\[
V = \frac{\pi h d^2}{4}
\]

Opgaven er derfor at finde mindsteværdien for $A(h,d)$ på mængden
\[
\left\{(h,d) \in \real^2 | V = \frac{\pi h d^2}{4} \right\}
\]
En måde at gøre dette på er ved at løse ligningen $V = \frac{\pi h d^2}{4}$ med hensyn til $h$ og sætte dette ind i $A$.

\paragraph*{Uden Lagrange}
\[
h = \frac{4V}{\pi d^2}
\]
Dette giver at
\[
A(d) = \frac{\pi}{2}d^2 + \frac{4V}{d}
\]
Vi differentierer dette og får
\[
A'(d) = \pi d - \frac{4V}{d^2}
\]
Løser vi $A'(d) = 0$ får vi at $d = \sqrt[3]{\frac{4V}{\pi}}$. Dette må være et minimum og den tilsvarende værdi for h er $h = \sqrt[3]{\frac{4V}{\pi}}$.

Hvis vi udfører samme beregning ved brug af Lagranges multiplikationsmetode, får vi at

\paragraph*{Med Lagrange}
\[
A(h,d) = \frac{\pi}{2}d^2 + \pi hd
\]
og
\[
V = \frac{\pi hd^2}{4}
\]
og vi får derfor
\[
g(h,d) = \frac{\pi hd^2}{4}
\]
Vi ser at både $A$ og $g$ er $C^1$ funktioner. Vi udregner derfor
\[
\gradient A(h,d) = (\pi d, \pi d + \pi h) \text{ og } \gradient g(h,d) = \left(\frac{\pi}{4}d^2,\frac{\pi}{2}hd \right)
\]
Disse vektorer er parallelle hvis der findes et tal $\lambda$ således at
\begin{align*}
 \pi d &= \lambda \frac{\pi}{4}d^2 \\
 \pi d + \pi h &= \lambda \frac{\pi}{2}hd
\end{align*}
Og bibetingelsen
\[
V = \frac{\pi hd^2}{4}
\]
For at finde mindsteværdien forsøger vi nu at løse ligningssættet som består af disse tre ligninger. 
\[
4 = \lambda d
\]
og
\[
d + h = \frac{1}{2}h\lambda d
\]
Nu kan vi erstatte $\lambda d$ på den højde side med 4. Dette giver
\[
d + h = \frac{1}{2}h\cdot 4 = 2h
\]
Dette giver at $d = h$. Sætter vi dette ind i bibetingelsen, får vi
\[
\frac{\pi}{4}d^3 = V
\]
som giver at $h = d = \sqrt[3]{\frac{4V}{\pi}}$ og dette stemmer med det resultat vi fik i første udregning.

\subsection{Integraler af flere variable}
\subsubsection{Itereret integrale}
\begin{thm}[Itereret integrale]
Lad f være en kontinuert funktion defineret på et simpelt domæne,

1: Hvis f er en funktion af to variable og D er af formen $D = {(x,y)\,|\,a\leq x \leq b,\; u(x) \leq y \leq o(x)}$, gælder
\[
\int_D{f(x,y)\,dA} = \int_{x=a}^{x=b}{\left(\int_{y=u(x)}^{y=o(x)}{f(x,y)\,dy}\right)dx}
\]

2: Hvis f er en funktion af to variable og D er af formen $D = {(x,y)\,|\,c \leq y \leq d,\;v(y) \leq x \leq h(y)}$, gælder
\[
\int_D{f(x,y)\,dA} = \int_{y=c}^{y=d}{\left(\int_{x=v(y)}^{x=h(y)}{f(x,y)\,dx}\right)dy}
\]

3: Hvis f er en funktion af tre variable og R er af formen $R = {(x,y,z)\,|\,a \leq x \leq b,\,u(x) \leq y \leq o(x),\,b(x,y) \leq z \leq t(x,y)}$ og $t,b:{(x,y)\,|\,a \leq x \leq b,\,u(x) \leq y \leq o8x)}$, gælder
\[
\int_R{f(x,y,z)\,dV} = \int_{x=a}^{x=b}{\left(\int_{y=u(x)}^{y=o(x}{\left(\int_{z=b(x,y)}^{z=t(x,y)}{f(x,y,z)\,dz}\right)dy}\right)dx}
\]
\end{thm}

Vi udregner eksempelvis integralet af funktionen $f(x,y) = xy$ over mængden $D = {(x,y) \in \real^2\,+\,0 \leq x \leq 1,\, x \leq y \leq 2x}$.
\[
 \int_D{xy\;dA} = \int_0^1{\left(\int_x^{2x}{xy\,dy}\right)dx} = \int_0^1{x\frac{1}{2}((2x)^2)-x^2)dx} = \int_0^1{\frac{3}{2}x^3\,dx} = \frac{3}{8}
\]

\subsubsection{Transformationssætningen}
Transformationssætningen er udtrykt ved
\begin{thm}[Transformationssætningen]
\[
\int_{T(D)}{f} = \int_D{(f\circ T)J(T)}
\]
hvor $J(T(x))$ er arealet for $n=2$ og rumfanget for $n=3$, af den figur der udspændes af gradientvektorerne $\gradient T_1(x),\ldots,\gradient T_n(x)$
\end{thm}
Man beregner $J(T(x))$ ud fra formlerne for arealet for et parallellogram udspændt af to vektorer $(a,b)$ i planen og rumfanget af et parallelepipedum udspændt af tre vektorer $(a,b,c)$ i rummet. Formlerne kan udtrykket ved
\[
\text{Areal}(a,b) = \det{\begin{vmatrix} a_1 & a_2 \\ b_1 & b_2 \end{vmatrix}}
\]
\[
\text{Rumfang}(a,b,c) = \det{\begin{vmatrix} a_1 & a_2 & a_3 \\ b_1 & b_2 & b_3 \\ c_1 & c_2 & c_3 \end{vmatrix}}
\]

Udsagnet i transformationssætningen kan skrive som
\[
\int_{T([a,b])}{f(y)dy} = \int_{[a,b]}{f(T(x))|T'(x)|dx}
\]

\subsubsection{Polære koordinater}
\begin{defn}[Polære koordinater i planen]
Polære koordinater i planen er givet ved transformationen
\[
T(r,\theta) = (x(r,\theta)),y(r,\theta)) = (r\cos{\theta},r\sin{\theta})
\]
\end{defn}
Den er injektiv på mængden af $(r,\theta)$ der opfylder
\[
0 < r,\;0 < \theta \leq 2\pi
\]
Vi finder dermed
\[
J(T)(r,\theta) = \det{\begin{vmatrix} \cos{\theta} & -r\sin{\theta} \\ \sin{\theta} & r\cos{\theta} \end{vmatrix} = r}
\]

\subsubsection{Sfæriske koordinater i rummet}
\begin{defn}[Sfæriske (kugle) koordinater i rummet]
Sfæriske koordinater i rummet er givet ved funktionen
\[
T(\rho, \theta, \phi) = x((\rho, \theta, \phi),y((\rho, \theta, \phi)z(\rho, \theta, \phi))
\]
hvor 
\[
x(\rho, \theta, \phi) = \rho \sin{\phi}\cos{\theta}, y(\rho, \theta, \phi) = \rho \sin{phi}\sin{\theta}, z(\rho, \theta, \phi) = \rho \cos{\phi}
\]
\end{defn}
Funktionen er injektiv på mængden af $(\rho, \theta, \phi)$ der opfylder
\[
0 < \rho,\; 0 < \theta \leq 2\pi,\; 0 < \phi < \pi
\]
Og vi finder
\[
J(T)(\rho, \theta, \phi) = \det{\begin{vmatrix} \sin{\phi}\cos{\theta} & -\rho \sin{\phi}\cos{\theta} & \rho \cos{\phi}\cos{\theta} \\ \sin{\phi}\sin{\theta} & \rho \sin{\phi}\cos{\theta} & \rho \cos{\phi}\sin{\theta} \\ \cos{\phi} & 0 & -\rho \sin{\phi} \end{vmatrix}} = \rho^2 \sin{\phi}
\]

Vi vil udregne integralet af funktionen $f(x,y) = x$ over mængden $D'$ i første kvadrant omgrænset af x-aksen, linien $x = y$ og cirklen $x^2 + y^2 = 4$. Dette kan skrives i polære koordinater som
\[
 {(r,\theta)\,|\, r \in [0,2],\, 0 \leq \theta \leq \pi/4}
\]
Dette betyder at f er udtrykt i polære koordinater som $r \cos{\theta}$.

Da får vi integralet
\[
\int_{T(D)}{f(x,y)dA} = \int_D{r \cos{\theta}r\,dA}
\]
Ved at omskrive til et itereret integrale får vi
\[
\int_{r=0}^{r=2}{\int_{\theta=0}^{\theta=\pi/4}{r^2\cos{\theta}\,d\theta dr}} = \int_{r=0}^{r=2}{r^2\,dr}\int_{\theta=0}^{\theta=\pi/4}{\cos{\theta}\,d\theta} = 4\frac{\sqrt{2}}{3}
\]

Vi vil ligeledes benytte metoderne for de sfæriske koordinater til at udregne rumfanget af en kugle med radius R i rummet. Kuglens koordinater er givet ved
\[
B = {((\rho, \theta, \phi)\,|\, 0 \leq \rho \leq R,\; 0\leq \theta \leq 2\pi,\; 0 \leq \phi \leq \pi}
\]
Vi finder derfor rumfanget af kuglen til
\[
\int_{T(B)}{1\;dV} = \int_{\rho=0}^{\rho=R}{\int_{\theta=0}^{\theta=2\pi}{\int_{\phi=0}^{\phi=\pi}{\rho^2\sin{\phi}\;d\phi d\theta d\rho}}} =  \frac{1}{3}4\pi R^3
\]

\newpage
\section{Bilag}

\subsection{Differentialkvotienter og stamfunktioner}
Her præsenteres en række typisk brugte differentialkvotienter og stamfunktioner
\begin{table}[h!]
 \centering
\caption{Ofte brugte differentialkvotienter og stamfunktioner}
\begin{tabular}{|c|c|c|} 
\hline
$f(x)$ 					& $f'(x)$ 									& $F(x)$ 				\\ \hline \hline
$k$						&$0$										&$k$	 				\\ \hline
$kx$					&$k$										&$kx$	 				\\ \hline
$x^n$					&$nx^{n-1}$ 								&$\frac{1}{n+1}x^{n+1}$	\\ \hline
$\frac{1}{x} = x^{-1}$	&$-\frac{1}{x^2} = -x^{-2}$ 				&$\ln{x}$				\\ \hline
$\sqrt{x} = x^{1/2}$	&$\frac{1}{2\sqrt{x}}=\frac{1}{2}x^{-1/2}$	&$\frac{2}{3}x^{3/2}$	\\ \hline
$a^x$					&$a^x\ln(a)$								&$\frac{1}{\ln{a}}a^x$	\\ \hline
$\sin{x}$				&$\cos{x}$									&$-\cos{x}$				\\ \hline
$\cos{x}$				&$-\sin{x}$									&$\sin{x}$				\\ \hline
$\tan{x}$				&$\frac{1}{(\cos{(x))}^2}=1+\tan^2{x}$		&$\ln{\cos{x}}$			\\ \hline
$\ln{x}$				&$\frac{1}{x}$								&$x\ln{x}-x$			\\ \hline
$e^x$					&$e^x$										&$e^x$					\\ \hline
$e^{kx}$				&$ke^{kx}$									&$\frac{1}{k}e^{kx}$	\\ \hline
$x^a$					&$\frac{x^a}{x}a$							&$\frac{1}{a+1}x^{a+1}$ \\ \hline
$\cos^2{x}$				&$-2\cos{x}\sin{x}$							&$\frac{1}{2}(x+\cos{x}\sin{x})$ \\ \hline
$\sin^2{x}$				&$2\cos{x}\sin{x}$							&$\frac{1}{2}(x-\sin{x}\cos{x})$ \\ \hline
$\tan^2{x}$				&$2\tan{x}(1+\tan^2{x})$					&$\tan{x}-x$					\\ \hline
$\sin^{-1}{x}$			&$\frac{1}{\sqrt{1-x^2}}$					&$x\sin^{-1}{x}+\sqrt{1-x^2}$ \\ \hline
$\cos^{-1}{x}$			&$-\frac{1}{\sqrt{1-x^2}}$					&$x\cos^{-1}{x}+\sqrt{1-x^2}$ \\ \hline
$\tan^{-1}{x}$			&$\frac{1}{1+x^2}$							&$x\tan^{-1}{x}-\frac{1}{2}\ln(1+x^2)$ \\ \hline
$\cos{x}\sin{x}$		&$-\sin^2{x}+\cos^2{x}$						&$-\frac{1}{2}\cos^2{x}$ \\ \hline
$\cos^2{x}\cos^2{x}$	&$-4\cos^3{x}\sin{x}$						&$\frac{1}{4}\cos^3{x}\sin{x}+\frac{3}{8}\cos{x}\sin{x}+\frac{3}{8}x$	\\ \hline
$\sin^2{x}\sin^2{x}$	&$4\sin^3{x}\cos{x}$						&$-\frac{5}{8}\cos{x}\sin{x}+\frac{1}{4}\cos^3{x}\sin{x}+\frac{3}{8}x$ \\ \hline
$\sin^2{x}\cos^2{x}$	&$2\sin{x}\cos{x}(2\cos^2{x}-1)$			&$-\frac{1}{4}\cos^3{x}\sin{x}+\frac{1}{8}\cos{x}\sin{x}+\frac{1}{8}x$ \\ \hline
\end{tabular}
\label{stamfunktioner}
\end{table}

\subsection{Sinus, Cosinus og Enhedscirklen}
Enhedscirklen på figur \ref{enhed}
\begin{figure}[ht!]
 \centering
% \includegraphics[width=10 cm,keepaspectratio=true]{enhedscirkel.png}
 % enhedscirkel.png: 500x500 pixel, 72dpi, 17.64x17.64 cm, bb=0 0 500 500
 \caption{Enhedscirklen.}
 \label{enhed}
\end{figure}

\subsection{Værdier for sinus og cosinus}
Her er skrevet en række almindelige værdier for cosinus og sinus. \newline
\begin{tabular}{| c | c | c | c |}
 \hline
 Vinkel  & Radian 	 	& Sinus  			& 	Cosinus \\ \hline
$0\gra$	 & $0$	 		&0      			& 	 1 	\\ \hline
$180\gra$&$\pi$			&0				&	-1	\\ \hline
$15\gra$& $\frac{\pi}{12}$ 	&$\frac{\sqrt{6}-\sqrt{2}}{4}$	& $\frac{\sqrt{6}+\sqrt{2}}{4}$	\\ \hline
$165\gra$&$\frac{11\pi}{12}$	&$\frac{\sqrt{6}-\sqrt{2}}{4}$	& $-\frac{\sqrt{6}+\sqrt{2}}{4}$ \\ \hline
$30\gra$&$\frac{\pi}{6}$	&$\frac{1}{2}$			& $\frac{\sqrt{3}}{2}$ \\ \hline
$150\gra$&$\frac{5\pi}{6}$	&$\frac{1}{2}$			& $-\frac{\sqrt{3}}{2}$	\\ \hline
$45\gra$&$\frac{\pi}{4}$	&$\frac{\sqrt{2}}{2}$		& $\frac{\sqrt{2}}{2}$ \\ \hline
$135\gra$&$\frac{3\pi}{4}$	&$\frac{\sqrt{2}}{2}$		& $-\frac{\sqrt{2}}{2}$ \\ \hline
$60\gra$&$\frac{\pi}{3}$	&$\frac{\sqrt{3}}{2}$		& $\frac{1}{2}$		\\ \hline
$120\gra$&$\frac{2\pi}{3}$	&$\frac{\sqrt{3}}{2}$		& $-\frac{1}{2}$	\\ \hline
$75\gra$&$\frac{5\pi}{12}$	&$\frac{\sqrt{6}+\sqrt{2}}{4}$	& $\frac{\sqrt{6}-\sqrt{2}}{4}$ \\ \hline
$105\gra$&$\frac{7\pi}{12}$	&$\frac{\sqrt{6}+\sqrt{2}}{4}$  & $-\frac{\sqrt{6}-\sqrt{2}}{4}$ \\ \hline
$90\gra$&$\frac{\pi}{2}$	&1				& 0				\\ \hline
$270\gra$&$\frac{3\pi}{2}$	&-1				& 0				\\ \hline
\end{tabular}

\subsection{Ofte benyttede geometriske formler}
\subsubsection{Arealer og perimetre}
\paragraph*{Cirkel}
\[ A = \pi r^2 \]
Perimeter af en cirkel
\[ P = 2\pi r \]

\paragraph*{Cirkeludsnit}
\[ A = \frac{1}{2} r^2 \theta \]
hvor $\theta$ er i radianer.

\paragraph*{Cirkelring}
\[ A = \pi (R^2 - r^2) \]
hvor $R$ er radius af den ydre ring og $r$ er radius af den indre.

\paragraph*{Cirkelafsnit/cirkelkant}
\[ A = \frac{1}{2} r^2 (\theta - \sin{\theta}) \]
hvor $r$ er radius af de to linjer der udgør afsnittet og $\theta$ er vinklen imellem dem.

\paragraph*{Ellipse}
\[ A = \pi ab \]

\paragraph*{Parabel}
\[ A = \frac{2}{3}ab \]

\paragraph*{Rektangel}
\[ A = ab \]

\paragraph*{Trekant}
\[ A = \frac{1}{2}hb \]

\paragraph*{Parallellogram}
\[ A = hb = ab\sin{\theta} \]

\paragraph*{Trapez}
\[ A = \frac{1}{2}h(a+b) \]

\subsubsection{Volumener og overflader}
\paragraph*{Kugle}
\[ V = \frac{4}{3}\pi r^3 \]
\[ O = 4 \pi r^2 \]

\paragraph*{Kugletop med radius $r$ og højde $h$}
\[ V = \frac{1}{3}\pi h^2 (3r-h) \]
\[ O = 2 \pi rh \]

\paragraph*{Retvinklet cylinder}
\[ V = \pi r^2 h \]
Krum overflade
\[ O = 2\pi rh \]

\paragraph*{Ret kegle}
\[ V = \frac{1}{3}\pi r^2 h \]
Krum overflade
\[ O = \pi r \sqrt{r^2 + h^2} = \pi r a \]
hvor $a$ er den skrå yderside af keglen.

\paragraph*{Keglestub}
\[ V = \frac{1}{3}\pi h (a^2 + a + b + b^2) \]
Krum overflade
\[ O = \pi (a+b) \sqrt{h^2+(b-a)^2} = \pi (a+b)c \]
hvor $c$ er den krumme længde

\paragraph*{Tresidet pyramide}
\[ V = \frac{1}{3}Ah \]
Hvor $A$ er grundfladearealet.

\paragraph*{Firesidet pyramide}
\[ V = \frac{1}{3}Ah \]
Hvor $A$ er grundfladearealet.

\end{document}