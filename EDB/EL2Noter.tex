% Nikolai Nielsens "Fysiske Fag" preamble
\documentclass[a4paper,10pt]{article} 	% A4 papir, 10pt størrelse

\usepackage{Nikolai} 					% Min hjemmelavede pakke
\usepackage{xr}
\renewcommand{\konj}{^*}
\newtheorem{theorem}{Theorem}[section]
\usepackage{makeidx}
\makeindex

% Margen
\usepackage[margin=1in]{geometry}

% Max antal kolonner i en matrix. Default er 10
%\setcounter{MaxMatrixCols}{20}

% Hvor dybt skal kapitler labeles?
%\setcounter{secnumdepth}{4}	
%\setcounter{tocdepth}{4}

% Til de første fire side i Griffiths
\newcounter{VektorListe}

% Hvilket nummer skal der startes med i sections? (n-1)
%\setcounter{section}{0}	

% Til nummerering af ligninger. Så der står (afsnit.ligning) og ikke bare (ligning)
\numberwithin{equation}{section}

% Header
%\usepackage{fancyhdr}
%\head{Nikolai Plambech Nielsen, 21-06-95\\Dato:. Klasse 5.C - De Fysiske Fag}
%\pagestyle{fancy}

%Titel
\title{Noter til EM2 på KU (Elektromagnetisme 2)}
\author{af Nikolai Plambech Nielsen, LPK331. Version 1.0}

\begin{document}

	\selectlanguage{danish}
	
	\maketitle
	\tableofcontents
	
	\section*{Ting der mangler}
	Jeg mangler umiddelbart følgende
	\begin{itemize}
		\item Introduktion
		\item Stikordsregister fra side 17 og efter
		\item Teoremopstillinger til Optiklovene
		\item Eksempler til, 10 og 11, plus teoremopstilling
		\begin{itemize}
			\item Eksempel 10.3
			\item Eksempel 10.4
			\item Bevis for toghalløjsa (side 431)
		\end{itemize}
		\item teoremopstilling til G 7 
		\item Evt. illustrationer
		\begin{itemize}
			\item Illustration af knudepunkter og masker
			\item Figur 9.15
			\item Figur 10.6
			\item Figur 11.2
		\end{itemize}
	\end{itemize}
	\newpage
	\subfile{intro}
	
	\part{Kredsløbsregning}
	\subfile{grundel}
	\newpage
	
	\part{Elektrodynamik og bølger}
	\subfile{maxwell}
	\newpage
	\subfile{konserves}
	\newpage
	\subfile{boelger}
	\newpage
	\subfile{potentialer}
	\newpage
	\subfile{radiation}
	\newpage
	
	\part{Appendiks}
	\appendix
	\subfile{kompleks}
	\newpage
	\subfile{fourier}
	\newpage
	\subfile{tabeller}
	\newpage
	\subfile{udregninger}
	\newpage
	\printindex

\end{document}

