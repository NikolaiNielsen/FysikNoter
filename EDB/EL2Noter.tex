% Nikolai Nielsens "Fysiske Fag" preamble
\documentclass[a4paper,10pt]{article} 	% A4 papir, 10pt størrelse

\usepackage{Nikolai} 					% Min hjemmelavede pakke
\usepackage{xr}
\renewcommand{\konj}{^*}
\newtheorem{theorem}{Theorem}[section]
\usepackage{makeidx}
\makeindex
\usepackage[dvipsnames]{xcolor}
\newtheorem{optiklov}{Lov}

% Margen
\usepackage[margin=1in]{geometry}

% Max antal kolonner i en matrix. Default er 10
%\setcounter{MaxMatrixCols}{20}

% Hvor dybt skal kapitler labeles?
%\setcounter{secnumdepth}{4}	
%\setcounter{tocdepth}{4}

% Til de første fire side i Griffiths
\newcounter{VektorListe}

% Hvilket nummer skal der startes med i sections? (n-1)
%\setcounter{section}{0}	

% Til nummerering af ligninger. Så der står (afsnit.ligning) og ikke bare (ligning)
\numberwithin{equation}{section}
\date{}

% Header
%\usepackage{fancyhdr}
%\head{Nikolai Plambech Nielsen, 21-06-95\\Dato:. Klasse 5.C - De Fysiske Fag}
%\pagestyle{fancy}

%Titel
\title{Noter til EM2 på KU (Elektrodynamik og Bølger)}
\author{af Nikolai Plambech Nielsen, LPK331. Version 1.0}

\begin{document}

	\selectlanguage{danish}
	
	\maketitle
	\tableofcontents
	
	\section*{Introduktion}
	Velkommen til mit notesæt til Elektrodynamik og Bølger (EL2). I kurset bliver Introduction to Electrodynamics (D. J. Griffiths), samt diverse notesæt (Anders Sørensen \& Per Hedegaard). Jeg har brugt 3. udgave af Electrodynamics, så hvis der står formelnumre eller lignende, så er de måske anderledes end din udgave. Dette notesæt er ikke helt færdigt, men al pensum er beskrevet. Nedenunder ses en liste over ting, der mangler i notesættet. Jeg nåede dem ikke, for der var simpelthen ikke tid nok til det inden eksamen. Så hvis én af jer læsere har lyst til at være herre for seje, så kan I lige hjælpe lidt, og ændre i \LaTeX-dokumentet. Dette ville være rigtig fedt. Hvis ikke for jer selv, så for den næste årgangs russere. Glædelig regning!
	
	\subsection*{Ting der mangler}
	Jeg mangler umiddelbart følgende
	\begin{itemize}
		\item Stikordsregister fra kapitel 3 og efter.
		\item Bevis for toghalløjsa (side 431).
		\item Udregninger til Kirchhoffs anden lov på kompleks form.
		\item Referencer til formelnumre/eksempelnumre i bøgerne.
	\end{itemize}
	\newpage
	\subfile{intro}
	
	\part{Kredsløbsregning}
	\subfile{grundel}
	\newpage
	
	\part{Elektrodynamik og bølger}
	\subfile{maxwell}
	\newpage
	\subfile{konserves}
	\newpage
	\subfile{boelger}
	\newpage
	\subfile{potentialer}
	\newpage
	\subfile{radiation}
	\newpage
	
	\part{Appendiks}
	\appendix
	\subfile{kompleks}
	\newpage
	\subfile{fourier}
	\newpage
	\subfile{tabeller}
	\newpage
	\subfile{udregninger}
	\newpage
	\printindex

\end{document}

