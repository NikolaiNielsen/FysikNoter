% Nikolai Nielsens "Fysiske Fag" preamble
\documentclass[a4paper,10pt]{article} 	% A4 papir, 10pt størrelse

\usepackage{Nikolai} 					% Min hjemmelavede pakke
\usepackage{xr}
\renewcommand{\konj}{^*}

% Margen
\usepackage[margin=1in]{geometry}

% Max antal kolonner i en matrix. Default er 10
%\setcounter{MaxMatrixCols}{20}

% Hvor dybt skal kapitler labeles?
%\setcounter{secnumdepth}{0}	

% Hvilket nummer skal der startes med i sections? (n-1)
\setcounter{section}{0}	

% Til nummerering af ligninger. Så der står (afsnit.ligning) og ikke bare (ligning)
\numberwithin{equation}{section}

% Header
%\usepackage{fancyhdr}
%\head{Nikolai Plambech Nielsen, 21-06-95\\Dato:. Klasse 5.C - De Fysiske Fag}
%\pagestyle{fancy}

%Titel
\title{Noter til EM2 på KU (Elektromagnetisme 2)}
\author{af Nikolai Plambech Nielsen, LPK331. Version 1.0}

\begin{document}

	\selectlanguage{danish}
	
	\maketitle
	\tableofcontents
	
	\section*{Forslag til andet layout}
	Jeg overvejer at opdele noterne i 3/4 forskellige dele:
	\begin{itemize}
		\item Kredsløb
		\item Bølger og anden elektrodynamik
		\item Andet (komplekse tal / Fourier)
		\item evt bilag/udregninger
	\end{itemize}
	Det kræver en lille omrokering af afsnittene, og kommer til at indebære enten ">chapters"< eller ">parts"< af dokumentet. Jeg ved ikke helt, hvilken jeg bruger endnu, og det er mest af alt for min egen skyld, jeg skriver dette ned. I behøver ikke lægge mærke til det. (med mindre, selvfølgelig, I har nogle vildt gode forslag). 
	
	\newpage
	
	\subfile{kompleks}
	\newpage
	\subfile{grundel}
	\newpage
	\subfile{fourier}
	\newpage
	\subfile{maxwell}
	\newpage
	\subfile{konserves}
	\newpage
	\subfile{boelger}
	
	\newpage
	\appendix
	\subfile{udregninger}


\end{document}

