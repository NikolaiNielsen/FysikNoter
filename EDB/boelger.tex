\documentclass[EL2Noter.tex]{subfiles}

\begin{document}
	\section{Bølger (Griffiths kapitel 9, noter kapitel 4)}
	\subsection{Bølger i 1 dimension}
	Bølger er svære at definere, men en start er (direkte oversat fra kapitlet): \textit{En forstyrrelse af et kontinuert medium, der udbredes med en bestemt, fast form, og konstant hastighed}. Der er dog flere forskellige ">variationer"< her: Hvis der er absorbering i mediet, så vil bølgens størrelse formindskes, som den udbredes. Hvis der er spredning vil bølger med forskellige frekvenser bevæge sig med forskellige hastigheder. Hvis det er i to eller tre dimensioner, vil amplituden af bølgen mindskes, som bølgen spredes ud i rummet (eller planen). Og til sidst findes stående bølger, der slet ikke udbreder sig.
	
	Det nemmeste er dog endimensionale bølger med fast form og konstant hastighed. Disse kan beskrives ved enhver funktion, der afhænger af variablen $ z-vt $. Hvor $ z $ er bevægelsesretningen og $ v $ er udbredelseshastigheden. Minusset er til for at sørge for udbredelsen af bølgen, langs bevægelsesretningen. Det kan virke spøjst, at det er et minus, og ikke et plus, der giver bevægelse i positiv retning. Man skulle jo tro, det var omvendt (plus giver positiv, ikke?). Men i mit hoved virker det kun spøjst, hvis man kigger på origo, og forventer at se fremtiden. Hvis man derimod står et sted langs $ z $-aksen (for bekvemmelighedens skyld), og skal se, hvordan man ser ud, så kigger man \textit{tilbage} langs $ z $-aksen for at se, hvordan bølgen så ud på et tidligere tidspunkt. Grundet bølgens faste form, vil dette så også være den form, man selv vil have, når bølgen når det punkt man befinder sig ved.
	
	Det viser sig at dette egentlig ikke er den eneste type funktion, der kan beskrive bølgebevægelse. Disse er defineret ved funktioner, der løser \textit{bølgeligningen}:
	\begin{equation}
		\diff{^2 f}{z^2} = \frac{1}{v^2} \diff{^2 f}{t^2}
	\end{equation}
	hvor $ v $ igen er udbredelseshastigheden. For bølger på en streng, er $ v = \sqrt{T/\mu} $, hvor $ T $ er snorkraften i strengen og $ \mu $ er strengens masse per længdeenhed.
	
	Bølgeligningen tillader alle funktioner af formen $ g(z-vt) $, men idet den involverer $ v^2 $, ses det også, at den tillader funktioner af formen $ h(z+vt) $, der er lignende bølger, men som bevæger sig i \textit{negativ} retning. Ydermere, idet det er en \textit{lineær} differentialligning, så er superpositionen af to løsninger også en løsning til den oprindelige ligning. Dermed er den mest generelle løsning (af disse former):
	\begin{equation}
		f(z,t) = g(z-vt)+h(z+vt)
	\end{equation}
	
	\subsubsection{Sinusbølger}
	Den mest kendte funktion af den førnævnte type er sinusbølgen (eller cosinus. Den bruges mere). Det er primært denne, vi beskæftiger os med i kurset. Dennes generelle form er
	\begin{equation}
		f(z,t) = A \cos[k(z-vt) + \delta]
	\end{equation}
	hvor $ A $ er bølgens \textit{amplitude}, $ k $ er \textit{bølgetallet}, $ v $ er \textit{udbredelseshastigheden} og $ \delta $ er \textit{faseforskydningen}. Af disse, er $ A $, $ v $ og $ k $ \textbf{altid positive}, mens $ \delta $ kan være enten positiv eller negativ (eller selvfølgelig 0). Generelt er størrelsen $ \delta/k $ den afstand, hvormed bølgen er forsinket, i forhold til origo (dette fås ved at sætte $ t = 0$ og sætte det resterende udtryk lig 0). Ud over disse introducerede størrelser, er der nogle flere, der relaterer sig til $ k $ og $ v $. Disse er \textit{bølgelængden} $ \lambda $, \textit{perioden} $ T $ (ikke at forveksle med snorkraften!), \textit{frekvensen} $ \nu $ (græsk bogstav ">nu"<) og \textit{vinkelfrekvensen} $ \omega $. Nedenunder ses en række småformler for, hvordan disse størrelser relaterer til hinanden:
	\begin{align}
		\lambda &= \frac{2 \pi}{k} = \frac{v}{\nu} = Tv = \frac{2\pi v}{\omega}, \\
		T &= \frac{2 \pi}{k v} = \frac{1}{\nu} = \frac{\lambda}{v} = \frac{2 \pi}{\omega}, \\
%		v & = \frac{\lambda}{T} = \frac{\omega}{k} = \frac{2 \pi}{k T}, \\
		\nu &= \frac{1}{T} = \frac{kv}{2 \pi} = \frac{v}{\lambda} = \frac{\omega}{2\pi}, \\
		\omega &= 2 \pi \nu = \frac{2 \pi}{T} = kv = \frac{2\pi v}{\lambda}.
	\end{align}
	lige som $ v $ og $ k $, er disse størrelser \textbf{altid positive}! Normalt skrives sinusbølger med $ \omega $ i stedet for $ v $, så man slipper for parentesen inde i argumentet af cosinus:
	\begin{equation}
		f(z,t) = A \cos (kz-\omega t + \delta)
	\end{equation}
	En bølge der bevæger sig den anden vej kan beskrives ved blot at skifte fortegn på $ k $ (dette er fordi cosinus er en \textit{lige} funktion):
	\begin{equation}
		f(z,t) = A \cos (-kz-\omega t + \delta)
	\end{equation}
	Bølger af disse typer kan også skrives på kompleks form (som før i kurset):
	\begin{equation}
		\T{f} (z,t) = \T{A} e^{i(kz-\omega t)}, \ \T{A} = Ae^{i\delta}, \quad f(z,t) = \Re\bb{\T{f}(z,t)}
	\end{equation}
	
	
	\subsubsection{Randbetingelser: reflektion og transmission}
	En stor del af det, der sker med bølgerne på en streng, afhænger af, hvordan denne er fastgjort i enden. Generelt vil der i enden ske to ting: Der vil opstå en \textbf{reflekteret bølge} og en \textbf{transmitteret bølge}. Hvis der ses på en streng med $ \mu_1 $, bundet til en streng med $ \mu_2 $, i punktet $ z=0 $, vil $ T $ være konstant for strengene. Da vil den indkommende bølge være
	\begin{equation}
		\T{f}_I(z,t) = \T{A}_I e^{i(k_1 z-\omega t)}, \quad (z<0),
	\end{equation}
	den reflekterede bølge vil være
	\begin{equation}
		\T{f}_R(z,t) = \T{A}_R e^{i(-k_1 z-\omega t)}, \quad (z<0),
	\end{equation}
	og den transmitterede bølge vil være
	\begin{equation}
		\T{f}_T(z,t) = \T{A}_T e^{i(k_2 z-\omega t)}, \quad (z>0),
	\end{equation}
	hvor alle bølger svinger med samme frekvens $ \omega $, og har gjort det i al uendelighed. Med andre ord, så gælder disse ligninger \textbf{kun}, hvis det er uendeligt lange bølger. En bølgepuls kan ikke beskrives på denne måde, idet den ikke har en veldefineret frekvens. Den skal opbygges af en uendelig serie af sinusbølger (Fouriertransformation), men såfremt bølgen har oscilleret i lang tid, vil den langt dominerende frekvens være den ønskede. Dermed kan disse ligninger \textit{tilnærmelsesvist} bruges i dette tilfælde.
	
	Idet de to strenge har forskellige masser, vil hastigheden, bølgelængden og bølgetallet være forskellige:
	\begin{equation}
		\frac{\lambda_1}{\lambda_2} = \frac{k_2}{k_1} = \frac{v_1}{v_2}.
	\end{equation}
	Den samlede bølge vil da være summen af disse tre bølger (eller rettere, den indkomne og reflekterede for $ z<0 $ og den transmitterede for $ z>0 $). I selve knuden må de være kontinuerte (ellers er de jo ikke bundet sammen!), så $ f(0^+,t) = f(0^-,t) $, og hvis knuden har negligibel masse, vil den rumligt afledte også være kontinuert i punktet 0: $ f'(0^+,t) = f'(0^-,t) $. Dette gælder også for den komplekse bølgeform. Dermed kan to ligninger for amplituderne opstilles:
	\begin{equation}
		\T{A}_I + \T{A}_R = \T{A}_T, \quad k_1(\T{A}_I + \T{A}_R) = k_2 \T{A}_T,
	\end{equation}
	Og dermed:
	\begin{align*}
		\T{A}_R &= \frac{k_1-k_2}{k_1+k_2} \T{A}_I = \frac{v_2-v_1}{v_2+v_1} \T{A}_I, \\
		\T{A}_T &= \frac{2 k_1}{k_1+k_2} \T{A}_I = \frac{2 v_2}{v_2+v_1} \T{A}_I.
	\end{align*}
	og de reelle amplituder er
	\begin{equation}
		A_R e^{i\delta_R} = \frac{v_2-v_1}{v_2+v_1} A_I e^{i\delta_I}, \quad 	A_T e^{i\delta_T} = \frac{2 v_2}{v_2+v_1} A_I e^{i\delta_I}.
	\end{equation}
	Hvis den første streng er \textit{lettere} end den anden, så $ \mu_1 < \mu_2 $ og $ v_1 > v_2 $ vil alle tre bølger være i fase, og de reflekterede og transmitterede bølger vil begge have amplituder, der er \textit{mindre} end den indgående bølges amplitude.
	
	Hvis den første streng er \textit{tungere} end den anden, så $ \mu_1 > \mu_2 $ og $ v_1 < v_2 $ vil den reflekterede bølge være 180\Deg ude af fase, hvilket ses matematisk ved, at den komplekse amplitude bliver negativ i størrelse, netop fordi $ v_1 < v_2 $. Men fordi amplituden ikke må være negativ, så trækkes et $ -1 = e^{i\pi} $ ud fra størrelsen, og lægges til fasen. Dermed er de reelle amplituder:
	\begin{align}
		\mu_1 < \mu_2, \ v_1 > v_2, \quad \Rightarrow \quad A_R &= \frac{v_2-v_1}{v_2+v_1} A_I, \quad A_T = \frac{2 v_2}{v_2 + v_1} A_I,\\
		\mu_1 > \mu_2, \ v_1 < v_2, \quad \Rightarrow \quad A_R &= \frac{v_1-v_2}{v_2+v_1} A_I, \quad A_T = \frac{2 v_2}{v_2 + v_1} A_I.
	\end{align}
	I grænsen hvor $ \mu_2 = \infty$ (hvor den sidder fast i en væg), fås $ v_2 = 0 $ og:
	\begin{equation}
		\mu_2 = \infty, \quad \Rightarrow \quad A_R = A_I, \quad A_T = 0
	\end{equation}
	
	
	\subsubsection{Polarisation}
	Der findes i grunden to forskellige typer bølger, når man ser på ">hvordan"< de svinger. Enten kan de svinge parallelt med bevægelsesretningen (lydbølger, eksempelvis, der er trykbølger i luften), ellers kan de svinge vinkelret på bevægelsesretningen (bølger på en streng, eller elektromagnetiske bølger). Disse kaldes for henholdsvis longitudinale og transverse bølger. I dette kursus arbejdes der kun med transverse bølger, idet dette indebærer elektromagnetiske bølger. Hvis en bølge udbreder sig i $ z $-retningen, kan den svinge i to, lineært uafhængige retninger: lodret og vandret ($ x $ og $ y $. Hvad, der er hvad, kan man egentlig selv bestemme, så længe man vælger et højrehåndskoordinatsystem til at beskrive dem med), eller en superposition af disse retninger. Dette kaldes for bølgen \textit{polarisation} (ikke at forvirre med polariserede materialer, der er insulatorer i et elektrisk felt, oh no!) Hvis en bølge er lodret polariseret fås
	\begin{equation}
		\T{f}_v(z,t) = \T{A} e^{i(kz-\omega t)} \Vx.
	\end{equation}
	Er den vandret polariseret fås
	\begin{equation}
		\T{f}_h(z,t) = \T{A} e^{i(kz-\omega t)} \Vy.
	\end{equation}
	hvis den svinger i en hvilken som helst anden retning (i $ xy $-planen) er den givet ved
	\begin{equation}
		\T{f}(z,t) = \T{A} e^{i(kz-\omega t)} \U{n}
	\end{equation}
	hvor $ \U{n} $ kaldes for \textit{polarisationsvektoren}. Idet den er vinkelret på bevægelsesretningen gælder det også at $ \U{n} \D \Vz = 0 $. Udtrykt ved polarisationsvinklen $ \theta $ (0\Deg ved $ \Vx $ og 90\Deg ved $ \Vy $) er polarisationsvektoren givet ved
	\begin{equation}
		\U{n} = \cos \theta \Vx + \sin \theta \Vy,
	\end{equation}
	hvilket giver, at en tilfældig polariseret bølge kan skrives som en superposition af to lodret og vandret polariserede bølger, ved hjælp af polarisationsvinklen:
	\begin{equation}
		\T{f}(z,t) = (\T{A} \cos \theta) e^{i(kz-\omega t)} \Vx + (\T{A} \sin \theta) e^{i(kz-\omega t)} \Vy
	\end{equation}
	
	
	
	
	
	\subsection{Elektromagnetiske bølger i vakuum og bølger i 3 dimensioner}
	I vakuum reduceres Maxwells ligninger til (\textbf{Layout fixes senere})
	\begin{align}
		(\text{i})	\quad	&\diverg{E} = 0,	(\text{iii}) 	\quad &\curl{E} = -\diff{\V{B}}{t}, \\
		(\text{ii})	\quad	&\diverg{B} = 0,  	(\text{iv}) 	\quad &\curl{B} = \mu_0 \epsilon_0 \diff{E}{t}, 
	\end{align}
	der er et sæt koblede, \textit{lineære}, førsteordens differentialligninger. For at afkoble dem tages curlen af Faradays (iii) og Ampères (iv) lov. Hermed fås
	\begin{equation}
		\grad^2 \V{E} = \mu_0 \epsilon_0 \diff{^2 \V{E}}{t^2}, \quad \grad^2 \V{B} = \mu_0 \epsilon_0 \diff{^2 \V{B}}{t^2}
	\end{equation}
	der nu er afkoblede, \textit{lineære}, andenordens differentialligninger. De er faktisk et eksempel på den tredimensionale bølgeligning:
	\begin{equation}
		\grad^2 f = \frac{1}{v^2} \diff{^2 f}{t^2}
	\end{equation}
	med udbredelseshastigheden $ v = 1/\sqrt{\mu_0 \epsilon_0} = c $. 
	
	
	\subsubsection{Monokromatiske planbølger}
	Som ved bølger i én dimension, begrænser vi os her til sinusbølger med en enkelt frekvens $ \omega $. Idet forskellige frekvenser af lysbølger svarer til forskellige farver (såfremt vi er i det synlige område), kaldes disse for \textit{monokromatiske} bølger (enkeltfarvede, fra latin). En liste over det elektromagnetiske spektrum ses tabuleret herunder \textbf{indsæt tabel}.
	
	Ydermere begrænser vi os lige til bølger der kun bevæger sig i én retning ($ z $ for nemhedens skyld), og som dermed ikke afhænger af andre koordinater, vinkelret på bevægelsesretningen ($ x $ og $ y $ i det simple tilfælde). Disse kaldes for \textbf{planbølger} fordi det kan ses som planer med samme fase (altså altid bølgedal/bølgetop eller lignende), der står vinkelret på bevægelsesretningen. Disse har formen
	\begin{equation}
		\tv{E} (z,t) = \tv{E}_0 e^{i(kz-\omega t)}, \quad \tv{B} (z,t) = \tv{B}_0 e^{i(kz-\omega t)},
	\end{equation}
	hvor $ \tv{E}_0 $ og $ \tv{B}_0 $ er de komplekse amplituder, og de fysiske felter selvfølgelig er den reelle del af disse komplekse felter.
	
	Disse ligninger overholder de respektive felters bølgeligning, men idet de også er nødt til at overholde Maxwells ligninger for at være elektriske/magnetiske felter, kommer der yderligere restriktioner på, der relaterer felterne til hinanden.
	
	For det første skal divergensen af begge felter være 0. Dette giver at \textbf{elektromagnetiske bølger er transverse}. Og Faradays lov giver følgende resultat, efter lidt hyggelig differentiation og gode ideer:
	\begin{equation}
		\tv{B}_0 = \frac{k}{\omega} (\Vz \times \tv{E}_0).
	\end{equation}	
	(Ampères lov giver helt samme resultat). Dette vil altså sige at $ \V{E} $ og $ \V{B} $ er \textbf{i fase}, og \textbf{vinkelrette på hinanden \textit{og} bevægelsesretningen}. Den reelle amplitude af $ B $-bølgen er
	\begin{equation}
		B_0 = \frac{k}{\omega} = \frac{1}{c} E_0
	\end{equation}
	
	For at generalisere dette resultat, skal \textbf{bølgevektoren} $ \V{k}=k \U{k} $ indføres. Denne har størrelsen $ k $ (bølgetallet) og peger i bevægelsesretningen. Da er $ \V{k} \D \V{r} $ generaliseringen af $ kz $ og generaliseringen af de monokromatiske planbølger er
	\begin{align}
		\tv{E}(\V{r},t) &= \T{E}_0 e^{i(\V{k}\D \V{r} - \omega t)} \U{n},\\
		\tv{B}(\V{r},t) &= \frac{1}{c} \T{E}_0 e^{i(\V{k}\D \V{r} - \omega t)} (\U{k} \times \U{n}) = \frac{1}{c} \U{k} \times \tv{E}_0,
	\end{align}
	hvor $ \U{n} $ som før er polarisationsvektoren. Og fordi $ \V{E} $ er transvers er $ \U{n} \D \U{k} = 0$. Man kan så spørge sig selv om, hvorfor $ \U{n} $ defineres ud fra $ \V{E} $-bølgen: dette er tradition, idet det er E-felter, der accelerer partikler, og $ \U{n} $ beskriver den retning de ladede partikler vil begynde at svinge i, hvis de bliver ramt af elektromagnetiske bølger.
	
	De egentlige, reelle felter for monokromatiske planbølger er givet ved
	\begin{align}
		\V{E} (\V{r},t) &= E_0 \cos(\V{k}\D \V{r} - \omega t + \delta) \U{n}, \\
		\V{B} (\V{r},t) &= \frac{E_0}{c} \cos (\V{k}\D \V{r} - \omega t + \delta) (\U{k} \times \U{n}).
	\end{align}
	
	
	\subsubsection{Energi i elektromagnetiske bølger}
	Energien per volumenenhed i elektromagnetiske felter er
	\begin{equation}
		u = \frac{1}{2} \pp{\epsilon_0 E^2 + \frac{B^2}{\mu_0}}
	\end{equation}
	og idet $ B^2 = (E/c)^2 =\mu_0 \epsilon_0 E^2 $ indeholder felterne fra elektromagnetiske bølger lige meget elektrisk og magnetisk energi, dermed er $ u = \epsilon_0 E^2 $. Energifluxen af denne bølge er givet ved Poyntingvektoren, $ \V{S} = \mu_0^{-1} (\V{E} \times \V{B}) $. Hvis bølgen bevæger sig i $ z $-retningen, fås
	\begin{equation}
		\V{S} = c\epsilon_0 E^2 \U{z} = cu \U{z}, \quad E = E_0 \cos(kz-\omega t + \delta)
	\end{equation}
	Men idet bølgerne har så høj en frekvens, giver det ikke rigtig mening at opgive disse størrelser, idet de svinger så hurtigt. Derfor bruger man igen tidsligt midlede størrelser. Den tidsligt midlede størrelse af $ \cos^2 $ er $ 1/2 $, og den tidsligt midlede energi og Poyntingvektor er da
	\begin{equation}
		\langle u \rangle = \frac{1}{2} \epsilon_0 E_0^2, \quad \langle \V{S} \rangle = \frac{1}{2} c \epsilon_0 E_0^2 \U{z}
	\end{equation}
	Størrelsen af Poyntingvektoren kaldes også for intensiteten $ I $ (igen må man endelig ikke forveksle den med andre størrelser. Navnlig strøm):
	\begin{equation}
		I \equiv \langle S \rangle = \frac{1}{2} c \epsilon_0 E_0^2.
	\end{equation}
	
	
	
	
	\subsection{Elektromagnetiske bølger i materialer}
	I materialer, hvor der ingen fri ladning eller strøm er, og som er både er \textit{lineære} (så $ \V{D} = \epsilon \V{E} $ og $ \V{H} = \V{B}/\mu $, med $ \epsilon = \epsilon_r \epsilon_0 $ og ligeledes for $ \mu $) og \textit{homogene} (så $ \epsilon$ og $ \mu $ ikke varierer i rummet), bliver Maxwells ligninger (\textbf{Layout Layout Layout...})
	\begin{align}
		(\text{i})	\quad	&\diverg{E} = 0,	(\text{iii}) 	\quad &\curl{E} = -\diff{\V{B}}{t}, \\
		(\text{ii})	\quad	&\diverg{B} = 0,  	(\text{iv}) 	\quad &\curl{B} = \mu \epsilon \diff{E}{t}, 
	\end{align}
	Altså er der kun den forskel, at $ \epsilon_0 \to \epsilon $ og $ \mu_0 \to \mu $, hvilket egentlig er ret cool! Det vil sige at elektromagnetiske bølger bevæger sig gennem lineære, homogene medier med hastigheden
	\begin{equation}\label{key}
		v = \frac{1}{\sqrt{\epsilon\mu}} = \frac{c}{n}, \quad n \equiv \sqrt{\frac{\epsilon\mu}{\epsilon_0 \mu_0}}
	\end{equation}
	hvor $ n $ kaldes \textbf{refraktionsindekset}. Som oftest er $ \mu \approx \mu_0 $, så $ n \approx \sqrt{\epsilon_r} $, og da $ \epsilon_r $ (næsten) altid er større end 1, så er $ n $ det også, og lys bevæger sig dermed næsten altid langsommere i materialer, som i vakuum.
	
	Med disse ting på plads, kan alle ligninger for sidste afsnit direkte kopieres med $ \epsilon_0 \to \epsilon $, $ \mu_0 \to \mu $ og $ c \to v $. Dermed er energidensiteten
	\begin{equation}\label{key}
		u = \frac{1}{2} \pp{\epsilon E^2+ \frac{1}{\mu} B^2}
	\end{equation}
	og Poyntingvektoren
	\begin{equation}\label{key}
		\V{S} = \frac{1}{\mu} (\V{E} \times \V{B})
	\end{equation}
	og amplituderne af felterne er
	\begin{equation}\label{key}
		B_0 = \frac{1}{v} E_0
	\end{equation}
	Og da er intensiteten
	\begin{equation}\label{key}
		I = \frac{1}{2} \epsilon v E_0^2
	\end{equation}
	
	
	\subsubsection{Vinkelret refleksion og transmission}
	Som i det endimensionale tilfælde, opstår der refleksion og transmission, når elektromagnetiske bølger når randen mellem to materialer. I dette tilfælde skal de dog overholde nogle lidt andre randbetingelser. Der er 4 randbetingelser, hvoraf de to første har med vinkelrette komponenter at gøre (altså i bevægelsesretningen) og de to sidste har med parallelle komponenter at gøre (parallel på randen, og dermed vinkelret på bevægelsen). Da elektromagnetiske bølger er transverse, er det kun de sidste to, der er interessante. Disse lyder
	\begin{equation}\label{key}
		\V{E}_1 ^{\parallel} = \V{E}_2 ^{\parallel}, \quad \frac{1}{\mu_1} \V{B}_1^{\parallel} = \frac{1}{\mu_2} \V{B}_2^{\parallel}.
	\end{equation}
	I det simple tilfælde for bølger, der bevæger sig i $ z $-retningen, med randen i $ xy $-planen ($ z=0 $), er de indkomne bølger
	\begin{equation}\label{key}
		\tv{E}_I(z,t) = \T{E}_{0_I} e^{i(k_1 z- \omega t)} \U{x}, \quad \tv{B}_I(z,t) = \frac{1}{v_1} \T{E}_{0_I} e^{i (k_1 z - \omega t)}\U{y},
	\end{equation}
	de reflekterede bølger
	\begin{equation}\label{key}
		\tv{E}_R(z,t) = \T{E}_{0_R} e^{i(-k_1 z- \omega t)} \U{x}, \quad \tv{B}_R(z,t) = -\frac{1}{v_1} \T{E}_{0_R} e^{i (-k_1 z - \omega t)}\U{y},
	\end{equation}
	og de transmitterede bølger
	\begin{equation}\label{key}
		\tv{E}_T(z,t) = \T{E}_{0_T} e^{i(k_2 z- \omega t)} \U{x}, \quad \tv{B}_T(z,t) = \frac{1}{v_2} \T{E}_{0_T} e^{i (k_2 z - \omega t)}\U{y},
	\end{equation}
	hvor minusset i den reflekterede magnetiske bølger kommer af, at bølgen (og dermed også Poyntingvektoren) peger i den anden retning, og de skal udgøre et højrehåndskoordinatsystem. I randbetingelserne er $ \V{E}_1 = \tv{E}_I+\tv{E}_R $ og $ \V{E}_2 = \tv{E}_T $. De to randbetingelser giver henholdsvis:
	\begin{equation}\label{key}
		\T{E}_{0_I} + \T{E}_{0_R} = \T{E}_{0_T}, \qquad \frac{1}{\mu_1} \pp{\frac{1}{v_1}\T{E}_{0_I} - \frac{1}{v_1} \T{E}_{0_R}} = \frac{1}{\mu_2} \frac{1}{v_2} \T{E}_{0_T}
	\end{equation}
	hvor den anden betingelse kan skrives som
	\begin{equation}\label{key}
		\T{E}_{0_I} - \T{E}_{0_R} = \beta\T{E}_{0_T}, \quad \beta \equiv \frac{\mu_1 v_1}{\mu_2 v_2} = \frac{\mu_1 n_2}{\mu_2 n_1}
	\end{equation}
	Dette giver amplituderne
	\begin{equation}\label{key}
		\T{E}_{0_R} = \frac{1-\beta}{1+\beta} \T{E}_{0_I}, \quad \T{E}_{0_T} = \frac{2}{1+\beta} \T{E}_{0_I},
	\end{equation}
	og hvis $ \mu_1 = \mu_2 = \mu_0 $, hvilket ofte er tilfældet, er $ \beta = v_1/v_2 $, og resultaterne er \textit{identiske} med de endimensionale tilfælde:
	\begin{equation}\label{key}
		\T{E}_{0_R} = \frac{v_2-v_1}{v_2+v_1} \T{E}_{0_I}, \quad \T{E}_{0_T} = \frac{2 v_2}{v_2+v_1} \T{E}_{0_I}.
	\end{equation}
	I dette tilfælde er den reflekterede bølge \textbf{i fase}, hvis $ v_2 > v_1 $ og 180$\degree$ \textbf{ude af fase}, hvis $ v_2 < v_1 $. De reelle amplituder er
	\begin{equation}\label{key}
		E_{0_R} = \vv{\frac{v_2-v_1}{v_2+v_1}} E_{0_I}= \vv{\frac{n_1-n_2}{n_1+n_2}} E_{0_I}, \quad E_{0_T} = \frac{2v_2}{v_2+v_1} E_{0_I} = \frac{2n_1}{n_1+n_2} E_{0_I}.
	\end{equation}
	
	Intensiteten af lyset er $ I = \epsilon v E_0^2 / 2 $, og hvis igen $ \mu_1 = \mu_2 = \mu_0 $ fås at brøkdelen af energi, som reflekteres er:
	\begin{equation}\label{key}
		R \equiv \frac{I_R}{I_I} = \pp{\frac{E_{0_R}}{E_{0_I}}}^2 = \pp{\frac{n_1-n_2}{n_1+n_2}}^2,
	\end{equation}
	og brøkdelen der transmitteres er
	\begin{equation}\label{key}
		T \equiv \frac{I_T}{I_I} = \frac{\epsilon_2 v_2}{\epsilon_1 v_1} \pp{\frac{E_{0_T}}{E_{0_I}}}^2 = \frac{4 n_1 n_2}{(n_1+n_2)^2},
	\end{equation}
	hvor disse kaldes for henholdsvis \textbf{refleksionskoefficienten} og \textbf{transmissionskoefficienten}. Det ses også at $ R+T = 1 $, hvilket stemmer overens med energibevarelse. Et eksempel på størrelserne er for overgangen mellem luft ($ n_1 = 1 $) og glas ($ n_2 = 1.5 $). Da er $ R=0.04 $ og $ T=0.96 $, hvilket giver god mening, da glas er gennemsigtigt.
	
	
	
	\subsubsection{Ikkevinkelret refleksion og transmission (bølger polariseret i indfaldsplanen)}
	Nu kommer det sjove! Nu er det en bølge, der kommer mod randen mellem to materialer (for simpelhedens skyld er randen stadig i $ xy $-planen, hvor $ z=0 $). Her er bølgerne
	\begin{align}\label{key}
		\tv{E}_I (\V{r},t) &= \tv{E}_{0_I} e^{i (\V{k}_I \D \V{r} - \omega t)}, \quad 
		\tv{B}_I (\V{r},t) = \frac{1}{v_1} (\U{k}_I \times \tv{E}_I), \\
		\tv{E}_R (\V{r},t) &= \tv{E}_{0_R} e^{i (\V{k}_R \D \V{r} - \omega t)}, \quad 
		\tv{B}_R (\V{r},t) = \frac{1}{v_1} (\U{k}_I \times \tv{E}_I), \\
		\tv{E}_T (\V{r},t) &= \tv{E}_{0_T} e^{i (\V{k}_T \D \V{r} - \omega t)}, \quad 
		\tv{B}_T (\V{r},t) = \frac{1}{v_2} (\U{k}_T \times \tv{E}_T).
	\end{align}
	Hvor den indkomne bølge kommer med en eller anden vinkel $ \omega_I $, i forhold til vinkelret på overgangen mellem de to materialer. Alle bølgerne svinger med samme frekvens $ \omega $, bestemt af bølgernes kilde, der i sin tid producerede dem. Da $ \omega = kv $ fås
	\begin{equation}\label{key}
		\omega = k_I v_1 = k_R v_1 = k_T v_2, \quad k_I = k_R = \frac{v_2}{v_1} k_T = \frac{n_1}{n_2} k_T.
	\end{equation}
	Felterne $ \tv{E}_I +\tv{E}_R $ og $ \tv{B}_I + \tv{B}_R $ i det første materiale skal nu sammenføres med $ \tv{E}_T $ og $ \tv{B}_T $ i det andet materiale, i overensstemmelse med de samme randbetingelser som før.
	
	Først kigges der dog kun på, at de skal være kontinuerte: Idet alle felterne har en ens struktur, hvor alle rumlige og tidslige størrelser står i eksponenten, og at de skal sammenføres i alle punkter i $ xy $-planen, til alle tider, giver at eksponenterne nødvendigvis må være ens. De tidslige er altid ens, da de svinger med samme frekvens. Dette giver bare:
	\begin{equation}\label{key}
		\V{k}_I \D \V{r} = \V{k}_R \D \V{r} = \V{k}_T \D \V{r}, \quad \text{for}\ z=0.
	\end{equation}
	Dette kan kun lade sig høre, hvis alle vektorerne $ \V{k}_j $ ligger i samme plan. Dette giver den første lov: (\textbf{Lav en eller anden form for fremhævelse})
	
	De indkomne, reflekterede og transmitterede bølger danner et plan, kaldet \textbf{indfaldsplanen}, der også inkluderer normalvektoren til overfladen.
	
	Dette giver også at
	\begin{equation}\label{key}
		k_I \sin \theta_I = k_R \sin \theta_R = k_T \sin \theta_T,
	\end{equation}
	og da $ k_I = k_R $ fås den anden lov, også kaldet refleksionsloven:
	
	Den indkomne bølges vinkel er lig den reflekterede bølges vinkel:
	\begin{equation}\label{key}
		\theta_I = \theta_R.
	\end{equation}
	
	Og for den transmitterede bølges vinkel fås den tredje lov, kaldet for refraktionsloven eller Snell's lov:
	\begin{equation}\label{key}
		\frac{\sin \theta_T}{\sin \theta_I} = \frac{n_1}{n_2}.
	\end{equation}
	
	Disse tre love er den geometriske optiks fundamentale love. 
	
	Nu skal de 4 randbetingelser fra Maxwells ligninger behandles. Hvis det antages, at bølgerne polariseres i indfaldsplanen (altså at $ \U{n} $ ligger i samme plan som alle $ \V{k}_j $), kan man med en dejlig gang algebra få et par ligninger ud, der giver den komplekse amplitude for de reflekterede og transmitterede bølger:
	\begin{equation}\label{key}
		\T{E}_{0_R} = \frac{\alpha-\beta}{\alpha+\beta} \T{E}_{0_I}, \quad \T{E}_{0_T} = \frac{2}{\alpha+\beta} \T{E}_{0_I}, \quad \alpha \equiv \frac{\cos \theta_T}{\cos \omega_I}, \quad \beta \equiv \frac{\mu_1 v_1}{\mu_2 v_2} = \frac{\mu_1 n_2}{\mu_2 n_1}
	\end{equation}
	Disse ligninger kaldes \textbf{Fresnel's ligninger} for polarisering i indfaldsplanen. Hvis $ \alpha > \beta $ er den reflekterede bølge \textbf{i fase} med de to andre, men hvis $ \alpha < \beta $ er den reflekterede bølge 180\Deg \textbf{ude af fase}. Størrelsen $ \alpha $ kan også udtrykkes kun ved refraktionsindeksene og indfaldsvinklen $ \theta_I $:
	\begin{equation}\label{key}
		\alpha =\frac{\cos \theta_T}{\cos \theta_I} = \frac{\sqrt{1-[(n_1/n_2) \sin \theta_I]^2}}{\cos \theta_I}
	\end{equation}
	hvor det ses, at ved $ \theta_I = 0 $ er $ \alpha = 1 $ og de vinkelrette amplituder igen opnås. Ved $ \theta_I = 90\degree $ divergerer $ \alpha $ og hele bølgen reflekteres. Der er også en vinkel mellem disse to, hvor $ \alpha = \beta $, hvor den reflekterede bølge helt udslukkes. Denne kaldes for \textbf{Brewster's vinkel}, $ \theta_B $, og er givet ved
	\begin{equation}\label{key}
		\sin^2 \theta_B = \frac{1-\beta^2}{(n_1/n_2)^2 - \beta^2}.
	\end{equation}
	Typisk er $ \mu_1 \approx \mu_2 $, og $ \beta \approx n_2/n_1 $, hvormed $ \sin^2 \theta_B \approx \beta^2 /(1+\beta^2) $. Dermed fås
	\begin{equation}\label{key}
		\tan \theta_B \approx \frac{n_2}{n_1}
	\end{equation}
	
	
	Effekten per enhedsareal, der rammer overgangen mellem de to materialer er prikproduktet mellem Poyntingvektorerne og fladenormalen: $ \V{S} \D \U{k} $. For de tre bølger er disse:
	\begin{align}\label{key}
		I_I &= \frac{1}{2} \epsilon_1 v_1 E^2_{0_I} \cos \theta_I, \\
		I_R &= \frac{1}{2} \epsilon_1 v_1 E^2_{0_R} \cos \theta_R, \\
		I_T &= \frac{1}{2} \epsilon_2 v_2 E^2_{0_T} \cos \theta_T.
	\end{align}
	Refleksions- og refraktionskoefficienterne for bølger polariseret i indfaldsplanen er da
	\begin{align}
		R &\equiv \frac{I_R}{I_I} = \pp{\frac{E_{0_R}}{E_{0_I}}}^2 = \pp{\frac{\alpha-\beta}{\alpha+\beta}}^2, \\
		T &\equiv \frac{I_T}{I_I} = \frac{\epsilon_2 v_2 \cos \theta_T}{\epsilon_1 v_1 \cos \theta_I} \pp{\frac{E_{0_T}}{E_{0_I}}}^2 = \alpha \beta \pp{\frac{2}{\alpha + \beta}}^2.
	\end{align}
	Og det ses igen, at summen af disse giver 1, i overenstemmelse med energibevarelse.
	
	
	
	
	
	
	\subsection{Absorbering og spredning, elektromagnetiske bølger i ledere}
	I tilfældet, hvor de elektromagnetiske bølger bevæger sig gennem ledere, er $ \V{J}_f $ og $ \rho_f $ ikke nødvendigvis 0. I følge Ohms lov er $ \V{J}_f = \sigma \V{E} $ og Maxwells ligninger siger
	\begin{align}
		\diverg{E} = \frac{1}{\epsilon_0} \rho_f, \quad &\curl{E} = -\diff{\V{B}}{t}, \\
		\diverg{B} = 0, \quad &\curl{B} = \mu \sigma \V{E} + \mu \epsilon \diff{\V{E}}{t}.
	\end{align}
	Kontinuitetsligningen, sammen med Ohms lov og Faradays lov, siger
	\begin{equation}\label{key}
		\grad \D \V{J}_f = -\diff{\rho_f}{t}, \quad \diff{\rho_f}{t} = -\sigma (\diverg{E}) = -\frac{\sigma}{\epsilon} \rho_f.
	\end{equation}
	Og i homogene, lineære ledere giver dette
	\begin{equation}\label{key}
		\rho_f(t) = e^{-\sigma t / \epsilon} \rho_f(0),
	\end{equation}
	der er et eksponentielt henfald, med karakteristisk tid $ \tau = \epsilon/\sigma $. Dette vil sige, at hvis du smider noget fri ladning på en leder, så vil det bevæge sig ud i enderne af lederen. $ \tau $ siger, hvor lang tid dette tager. For en perfekt leder er $ \sigma = \infty $ og $ \tau = 0 $. For en \textit{god} leder er $ \tau $ meget mindre end andre relevante tider (eksempelvis i oscillerende systemer er $ \tau \ll 1/\omega $) og for en \textit{dårlig} leder er $ \tau $ meget større end disse tider (oscillerende systemer: $ \tau \gg 1/\omega $).
	
	Men nok om det, resten af dette afsnit forudsætter at \textbf{al fri ladning er forsvundet}, så $ \rho_f = 0 $. Da bliver Faradays lov $ \diverg{E} = 0 $. Og den eneste forskel der er fra elektromagnetiske bølger i lineære medier, er det ekstra led i Ampères lov. Hvis vi igen tager curlen af Faradays og Ampères lov, kan disse ligninger afkobles til en modificeret bølgeligning, hvor der inkluderes en førsteordensafledet:
	\begin{equation}\label{key}
		\grad^2 \V{E} = \mu \epsilon \diff{^2 \V{E}}{t^2} + \mu \sigma \diff{\V{E}}{t}
	\end{equation}
	og identisk for $ \V{B} $. Denne ligning tillader stadig løsninger, der er planbølger. Dog er de modificerede, med komplekse bølgetal:
	\begin{equation}\label{key}
		\tv{E}(z,t) = \tv{E}_0 e^{i(\T{k}z-\omega t)}, \quad \tv{B}(z,t) = \tv{B}_0 e^{i(\T{k}z-\omega t)}, \quad \T{k}^2 = \mu \epsilon \omega^2 + i \mu \sigma \omega,
	\end{equation}
	hvor bølgetallet fås ved at tage kvadratroden.
	\begin{equation}\label{key}
		\T{k} = k+i\kappa, \quad k \equiv \omega \sqrt{\frac{\epsilon \mu}{2}} \bb{\sqrt{1+\pp{\frac{\sigma}{\epsilon \omega}}^2}+1}^{1/2}, \quad \kappa \equiv \omega \sqrt{\frac{\epsilon \mu}{2}} \bb{\sqrt{1+\pp{\frac{\sigma}{\epsilon \omega}}^2}-1}^{1/2}
	\end{equation}
	den imaginære del $ \kappa $ af det komplekse bølgetal resulterer i en eksponentiel svækkelse af bølgen (som funktion af rummet):
	\begin{equation}\label{key}
		\tv{E}(z,t) = \tv{E}_0 e^{-\kappa z} e^{i(kz-\omega t)}, \quad \tv{B}(z,t) = \tv{B}_0 e^{-\kappa z} e^{i(kz-\omega t)}.
	\end{equation}
	Den afstand det tager for bølgen at svækkes med en faktor $ e \approx 1/3 $ kaldes for \textbf{hudtykkelsen} (eller noget i den dur), og er givet ved $ d \equiv \kappa\inverse $. Dette er et mål for, hvor langt ind i lederen som bølgerne trænger. Den reelle del $ k $ af det komplekse bølgetal bestemmer bølgelængden, udbredelseshastigheden og refraktionsindekset, som normalt:
	\begin{equation}\label{key}
		\lambda = \frac{2\pi}{k}, \quad v = \frac{\omega}{k}, \quad n = \frac{ck}{\omega}.
	\end{equation}
	
	Som før kommer der dog ekstra begrænsninger på bølgernes komplekse amplituder, der bestemmer den reelle amplitude, fase og polarisation, som følge af Maxwells ligninger. Gauss' lov siger at der ikke kan være noget udsving i bevægelsesretningen, og bølgerne er igen transverse. Da kan vi lige så godt orientere vores felter, så det elektriske felt er polariseret langs $ x $-aksen: $ \tv{E} = \T{E}\U{x} $. Da giver Faradays lov (og Ampères lov) at
	\begin{equation}\label{key}
		\tv{B}(z,t) = \frac{\T{k}}{\omega} \T{E}_0 e^{-\kappa z} e^{i(kz-\omega t)} \U{y}.
	\end{equation}
	De to felter er igen indbyrdes vinkelrette med bevægelsesretningen. De reelle amplituder og faser kan fås ved at skrive det komplekse bølgetal på polær form:
	\begin{equation}\label{key}
		\T{k} = K e^{i\phi}, \quad K \equiv |\T{k}| = \sqrt{k^2+\kappa^2} = \omega \sqrt{\epsilon \mu \sqrt{1+\pp{\frac{\sigma}{\epsilon \omega}}^2}}, \quad \phi \equiv \tan\inverse(\kappa/k).
	\end{equation}
	Da er de komplekse amplituder relaterede ved
	\begin{equation}\label{key}
		B_0 e^{i \delta_B} = \frac{K e^{i\phi}}{\omega} E_0 e^{i\delta_E}.
	\end{equation}
	Dermed er de to felter ikke længere i fase! faktisk fås $ \delta_B-\delta_E = \phi $, hvilket giver at B-felter ">halter bagefter"< E-feltet. Forholdet mellem de reelle amplituder er givet ved
	\begin{equation}\label{key}
		\frac{B_0}{E_0} = \frac{K}{\omega} = \sqrt{\epsilon \mu \sqrt{1+\pp{\frac{\sigma}{\epsilon \omega}}^2}}.
	\end{equation}
	De reelle elektriske og magnetiske felter er da
	\begin{equation}\label{key}
		\V{E}(z,t) = E_0 e^{-\kappa z} \cos (kz-\omega t+ \delta_E) \U{x}, \quad \V{B}(z,t) = B_0 e^{-\kappa z} \cos (kz-\omega t + \delta_B + \phi) \U{y}.
	\end{equation}
	
	\subsubsection{Reflektion ved en ledende overflade}
	Randbetignelserne, der blev brugt ved overgangen mellem to dielektriske materialer kan ikke bruges her. I stedet skal ligningerne fra slutningen af \textbf{afsnit 4} bruges:
	\begin{table}[H]
		\centering
		\begin{tabular}{*{2}{r >{$\displaystyle} l <{$} }}
			\vspace{0.3cm}
			(i) & \epsilon_1 E_1^{\perp} - \epsilon_2 E_2^{\perp} = \sigma_f, & (iii) & \V{E}_1^{\parallel} - \V{E}_2^{\parallel} = 0, \\
			(ii) & B_1^{\perp} - B_2^{\perp} = 0, & (iv) & \frac{1}{\mu_1} \V{B}_1^{\parallel} - \frac{1}{\mu_2} \V{B}_2^{\parallel} = \V{K}_f \times \U{n}.
		\end{tabular}
	\end{table}
	hvor $ \sigma_f $ er den frie overfladeladning (ikke at forveksle med ledeevnen!), $ \V{K}_f $ den frie overfladestrøm og $ \U{n} $ en enhedsvektor der peger fra medium 2 ind i 1 (altså ">modsatrettet"< bevægelsen af bølgen. Ikke at forveksle med polarisationsvektoren!) . I Ohmske materialer (materialer, hvor Ohms lov gælder), kan der ikke være nogen overfladestrøm, så $ \V{K}_f= 0 $. 
	
	Vi antager igen, at overgangen er i $ xy $-planen, med første medium et lineært dielektrisk materiale (luft, eksempelvis) og det andet medium en leder. En monokromatrisk planbølge bevæger sig i $ z $-retningen, fra medium 1 mod medium 2. For nemhedens skyld er denne også polariseret i $ x $-retningen, så de forskellige bølger er
	
	\begin{table}[H]
		\centering
		\begin{tabular}{*{2}{ >{$\displaystyle} l <{$}}}
			\vspace{0.3cm}
			\tv{E}_I(z,t) = \T{E}_{0_I} e^{i(k_1 z-\omega t)} \U{x}, &
			\tv{B}_I(z,t) = \frac{1}{v_1} \T{E}_{0_I} e^{i (k_1 z - \omega t)} \U{y}, \\
			\tv{E}_R(z,t) = \T{E}_{0_R} e^{i(-k_1 z-\omega t)} \U{x}, &
			\tv{B}_R(z,t) = -\frac{1}{v_1} \T{E}_{0_I} e^{i (-k_1 z - \omega t)} \U{y}, \\
			\tv{E}_T(z,t) = \T{E}_{0_T} e^{i(\T{k}_2 z-\omega t)} \U{x}, & 
			\tv{B}_T(z,t) = \frac{\T{k}_2}{\omega} \T{E}_{0_T} e^{i (\T{k}_2 z - \omega t)} \U{y}.
		\end{tabular}
	\end{table}
	hvor den reflekterede bølge bevæger sig i den negative $ z $-retning, og den transmitterede bølge er svækket som den udbreder sig i lederen. Randbetingelse (i) siger nu at $ \sigma_f = 0$, og fordi $ \tv{B} $ er transvers, er (ii) automatisk overholdt. (iii) og (iv) (med $ \V{K}_f = 0$) giver
	\begin{equation}\label{key}
		\T{E}_{0_I} + \T{E}_{0_R} = \T{E}_{0_T}, \quad 	\T{E}_{0_I}-T{E}_{0_R} = \T{\beta} \T{E}_{0_T}, \ \T{\beta} = \frac{\mu_1 v_1}{\mu_2 \omega} \T{k}_2
	\end{equation}
	Det følger da at
	\begin{equation}\label{key}
		\T{E}_{0_R} = \frac{1-\T{\beta}}{1+\T{\beta}} \T{E}_{0_I}, \quad \T{E}_{0_T} = \frac{2}{1+\T{\beta}}\T{E}_{0_I}.
	\end{equation}
	Disse ligner da meget amplituderne fra refleksion og transmission mellem to dielektriske materialer. Forskellen er bare at $ \T{\beta} $ nu er kompleks i stedet for reel. I en perfekt leder, med $ \sigma = \infty $, giver $ k_2 = \infty $ og $ |\T{\beta}| = \infty $, og dermed
	\begin{equation}\label{key}
		\T{E}_{0_R} = -\T{E}_{0_I}, \quad \T{E}_{0_T} = 0.
	\end{equation}
	Bølgen er da totalt reflekteret og oplever en faseforskydning på 180\Deg. Dette betyder at gode ledere udgør gode spejle. For eksempel er \textbf{hudtykkelsen} af sølv i omegnen af 100 Å, så der skal kun bruges en meget lille mængde sølv til at lave et godt spejl.
	
	
	
	
	\section{Mere om bølger (Noter kap 4)}
	\subsection{Stående bølger}
	Stående bølger, er bølger der ikke bevæger sig, men som derimod ">står stille"< og oscillerer. Dette kendes for eksempel fra en guitar, og er kendetegnet ved, at strengen sidder fast i enderne:
	\begin{equation}\label{eq:bounds}
		f(0,t) = f(L,t) = 0,
	\end{equation}
	hvor $ L $ er strengens længde. Ligesom før arbejder vi med komplekse bølgefunktioner, så $ f(z,t) = \Re[\T{f}(z,t)] $, og som er opbygget af sinusbølger. Den mest generelle bølge af denne type er
	\begin{equation}\label{key}
		\T{f}(z,t) = \pp{\T{A}_- e^{-ikz}+\T{A}_+ e^{ikz}} e^{-i\omega t},
	\end{equation}
	der er superpositionen af to bølger, der bevæger sig i hver sin retning, med $ v = \omega/k $. Ved at indføre randbetingelserne fra \eqref{eq:bounds}, fås
	\begin{equation}\label{key}
		f(0,t)=0 \Rightarrow \T{A}_- = -\T{A}_+, \quad f(L,t)=0 \Rightarrow \sin kL=0 \Rightarrow kL=n\pi, \ n \in \Set{N}.
	\end{equation}
	Og skrevet som frekvenser ses det, at det kun er meget bestemte frekvenser, der er tilladt, som følge af disse randbetingelser:
	\begin{equation}\label{key}
		\omega_n = n \frac{v\pi}{L}, \quad n\in \Set{N}
	\end{equation}
	Hvor $ \omega_1 = v\pi/L $ kaldes for strengens \textit{grundfrekvens} og $ \omega_n $ kaldes for strengens $ n-1 $'te overtone (\textbf{Ret endelig, hvis det er forkert terminologi}). Når bølgen svinger med disse frekvenser, danner den en stående bølge, der kan skrives som
	\begin{equation}\label{key}
		f_n(z,t) = A \sin(k_n z) \cos (\omega_n t), \quad k_n = \omega_n/v.
	\end{equation}
	For $ n>1 $ er der steder på strengen, hvor $ f(z_0,t) = 0 $, til alle tider $ t $ (ud over enderne, selvfølgelig). Disse kaldes for \textit{noder} eller \textit{knudepunkter} (enderne er \textit{trivielle noder}), mens der andre steder på strengen, hvor $ \sin(k_n z_0) =\pm 1$. Her er oscillationerne størst, og punkterne kaldes for \textit{antinoder}. Disse steder optræder med regelmæssig afstand, og fra Mek2 ved vi, at afstanden mellem to nærmeste noder er $ \lambda/2 $ (UP, 13. udgave, side 492). Dette er også afstanden mellem en antinode og dennes nærmeste. Afstanden mellem en node og dennes tætteste antinode er halvdelen af afstanden mellem to noder, altså $ \lambda/4 $.
	
	\subsection{Bølger i 3 dimensioner}
	I dette afsnit behandles bølgeligningen i 3 dimensioner, samt to forskellige løsninger til denne: planbølger og sfæriske bølger. Idet planbølger allerede er beskrevet, vil jeg kun beskrive sfæriske bølger i dette afsnit.
	
	Bølgeligningen i 3 dimensioner er, som bekendt:
	\begin{equation}\label{key}
		\grad^2 f = \frac{1}{v^2} \diff{^2 f}{t^2}
	\end{equation}
	hvor $ \grad^2 $ er Laplaceoperatoren (summen af de dobbelte rumafledte). Sfæriske bølger har dette navn fordi bølgefronten bevæger sig ud fra udbredelsespunktet som en sfære med aftagende amplitude. 
	
	Den todimensionale ækvivalent er bølgerne i vand, der kommer fra en sten smidt ned i pølen. Her udbredes bølgerne fra nedslagspunktet, med mindre amplitude som afstanden stiger. Et 2D plot af dette kan ses herunder:
	\begin{figure}[H]
		\centering
		\includegraphics[width=12cm]{img/2dWave.png}
		\label{fig:2dWave}
		\caption{Et plot af $ 2 \sin(2r)/r $ for $ x,y = [-4\pi,4\pi] $. MATLAB-koden kan hentes fra min GitHub: https://github.com/NikolaiNielsen/FysikNoter/blob/master/EDB/sphericalWave.m}
	\end{figure}
	Her aftager bølgen som $ 2/r $, og hvis denne funktion blev lagt ind over bølgen, vil man tydelig kunne se, at alle bølgetoppene rører funktionen $ 2/r $. Dette bliver lagt som en øvelse til læseren, da det bliver ualmindelig grimt, med mindre man har interaktivitet (eller gider at gøre rigtig meget ved det, som gennemsigtighed og sådan noget). Det kan også visualiseres nemmere i én dimension, hvor koden til sådan et plot også ligger i filen på GitHub.
	
	Men nok om kedelige én og to dimensioner. Vi har jo tre at gøre godt med! Den simpleste sfæriske bølge er givet ved
	\begin{equation}\label{key}
		\T{f}(\V{r},t) = \T{A} \frac{e^{i(kr-\omega t)}}{r},
	\end{equation}
	der har en diskontinuitet i origo, hvor $ r = 0 $. Hvis der ses på bølgen et punkt langt væk fra origo, for eksempel $ z_0 \gg x,y $, og $ z=z_0+z' $ med $ z_0 \gg z' $. Her bliver $ r $:
	\begin{equation}\label{key}
		r = \sqrt{(z_0+z')^2+x^2+y^2} = (z_0+z') \sqrt{1+ \frac{x^2+y^2}{(z_0+z')^2}} \approx z_0+z' + \frac{x^2+y^2}{2z_0} \approx z_0
	\end{equation}
	hvor det er udnyttet at $ \sqrt{1+x} \approx 1+\frac{1}{2} x $, for små $ x $ (Taylorudvikling af $ \sqrt{1+x} $ omkring 0, til første orden). Denne approksimation kan dog kun bruges uden for eksponenten, da eksponenter er dumme med approksimationer. Eksempelvis er $ kz_0 \gg kz' $, men $ kz' $ kan sagtens være stort, og dermed have en stor indflydelse på eksponenten. Dette er dog ikke tilfældet for $ k(x^2+y^2)/z_0 $, der er småt nok til at ignorere. Med alt dette fås
	\begin{equation}\label{key}
		\T{f} (\V{r},t) = \T{A}' e^{i(kz'-\omega t)}, \quad \T{A}' = \T{A} \frac{e^{ikz_0}}{z_0},
	\end{equation}
	der jo er en planbølge! Dette giver matematisk grundlag for, at når vi ser på en kugle meget tæt på dennes overflade, så ligner den et plan (eller omvendt, hvis kuglen bare er meget stor, og vi dermed er langt væg fra centrum. Det der betyder noget, er bare forholdet mellem afstanden af dig til overfladen og radius af kuglen).
	
	\subsection{Interferens}
	Idet bølgeligningen er lineær, er enhver sum af løsninger til ligningen også en løsning til ligningen. Det er netop dette vi udnytter i studiet af stående bølger, hvor det er to modsatrettede bølger, med samme frekvens amplitude og bølgetal, der producerer fænomenet. Dette er et eksempel på \textit{interferens}. Der findes to typer: \textbf{konstruktiv} og \textbf{destruktiv}. Konstruktiv interferens er steder, hvor bølgerne er i fase, og der dermed er bølgetop/bølgedal for begge bølger. Dette resulterer i den dobbelte amplitude for den samlede bølge. Dette er jo præcis de steder på en streng, hvor der er antinoder! Destruktiv interferens er der, hvor bølgerne er præcis en halv bølgelængde ude af fase. Her mødes bølgetop med bølgedal, og resultatet er en præcist udslukket bølge: noderne på en streng!
	
	Der findes to ">steder"< hvor interferens kan opstå: i rummet og i tiden.
	\subsubsection{Rumlig interferens og smarte tricks}
	I stående bølger er det rumlig interferens der sker. Navnet kommer af, at interferensen sker i bestemte steder i rummet, og til alle tider (noderne og antinoderne er faste punkter i rummet, og interferensen sker så længe strengen vibrerer). Men stående bølger på en streng er uden tvivl det nemmeste tilfælde.
	
	I notesættet behandler de eksemplet med to højttalere i $ x= \pm L/2 $, og de kigger på området omkring $ z_0 $ for $ z_0 \gg x, y, L $. Jeg vil ikke gå i dybden med udregningerne, men blot nævne nogle af de smarte tricks de bruger, så I selv kan få alle de der ">gode ideer"< man skal bruge, til at løse opgaverne.
	
	For det første, skal to eksponentialfunktioner lægges sammen, hvilket er et helvede fordi det er meget nemmere at gange dem sammen. Og hvis der ikke lige er en fælles faktor, man \textit{lige} kan trække ud, så er det endnu mere træls. Men frygt ej! Man kan \textit{altid} finde en fælles faktor. Man skal blot indse følgende:
	\begin{equation}\label{key}
		\frac{a_1+a_2}{2}+\frac{a_1-a_2}{2} = a_1, \quad \frac{a_1+a_2}{2}-\frac{a_1-a_2}{2} = a_2.
	\end{equation}
	Regn selv gerne efter. Hvis vi nu antager at de to bølger har formen $ e^{ia_1} $ og $ e^{ia_2} $ hvor $ a_1 $ og $ a_2 $ afhænger af rum og tid, så er summen af disse:
	\begin{equation}\label{key}
		e^{ia_1} + e^{ia_2} = e^{i(a_1/2 + a_2/2)} \pp{e^{i(a_1/2-a_2/2)} + e^{-i(a_1/2-a_2/2)}} =  2 e^{i(a_1+a_2)/2} \cos\pp{\frac{a_1+a_2}{2}}.
	\end{equation}
	Se nu, hvor pænt det blev! Da er den nye amplitude $ 2 \cos((a_1+a_2)/2) $. Dette resultat kan direkte overføres til \textit{alle} opgaver af denne type. Dette trick bliver brugt på side 81 i notesættet.
	
	Det andet trick de bruger, er approksimationer. Jeg har benævnt den stort set lige før, i afsnittet om sfæriske bølger, men det er så smart, så det er værd at nævne en anden gang. Taylorudviklingen af funktionen $ a\sqrt{1+x} $ for små $ x $ (altså omkring punktet $ x=0 $) kan bruges til at komme af med drilske kvadratrødder, så man nemt kan lægge to af disse sammen:
	\begin{equation}\label{key}
		a \sqrt{1+x} \approx a\pp{1+\frac{1}{2} x}, \quad x \approx 0
	\end{equation}
	Dette er en Taylorudvikling til første orden ($ x $ står i første potens). Dette trick bliver brugt både på side 79 og 82 i notesættet, når der snakkes om sfæriske bølger og interferens.
	
	
	\subsubsection{Tidslig interferens}
\end{document}