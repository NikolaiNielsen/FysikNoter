\documentclass[EL2Noter.tex]{subfiles}

\begin{document}
	\section{Elektrodynamik (Griffiths kapitel 7)}
	
	\subsection{Elektromagnetisk induktion} \index{Induktion}
	
	\subsubsection{Faradays lov} \index{Faradays lov}
	En gang i atten-hundrede-og-grønlangkål udførte Michael Faraday 3 eksperimenter, der foregik nogenlunde således:
	\begin{enumerate}
		\item Han trak en ledning gennem et magnetisk felt, og der løb en strøm.
		\item Han bevægede det magnetiske felt i modsat retning, og igen løb der en strøm.
		\item Med begge dele stående stille ændrede han styrken af det magnetiske felt, og igen igen løb der en strøm.
	\end{enumerate} 
	Det første giver ret god mening - det er jo emf som vi kender det. Men i det andet bevæger ledningen sig jo ikke, så der er jo ingen magnetisk kraft! Det giver selvfølgelig god mening for os, at det er den \textit{relative} bevægelse af magnetfeltet, i forhold til ledningen, der skaber emf'en, men på hans tid kendte man ikke til speciel relativitetsteori, så det var da noget spøjst! I det sidste forsøg er der endda ingen bevægelse, men igen er der emf.
	
	Faradays forklaring på det andet forsøg var, at \textbf{et ændrende magnetfelt inducerer et elektrisk felt}, og at det var dette elektriske felt, der stod for den opståede emf. Fra fluxreglen fås
	\begin{equation*}
		\emf = \oint \V{E} \D \ud \V{l} = - \diff[\ud]{\Phi}{t} = - \diff[\ud]{}{t} \int \V{B} \D \U{n} \ud a = - \int \diff{\V{B}}{t} \D \U{n} \ud a
	\end{equation*}
	Resultatet er
	\begin{equation}
		\oint \V{E} \D \ud \V{l} = - \int \diff{\V{B}}{t} \D \U{n} \ud a
	\end{equation}
	Dette er \textbf{Faradays lov} i integralform. Ved Stokes sætning kan dette omdannes til differentialform:
	\begin{equation}
		\curl{E} = - \diff{\V{B}}{t}
	\end{equation}
	I det tredje eksperiment er det igen et ændrende magnetfelt - men denne gang er det styrken og ikke en bevægelse relativ til magnetfeltet, der giver anledning til en emf. Igen er det fluxreglen på spil
	\begin{equation}
		\emf = - \diff[\ud]{\Phi}{t}
	\end{equation}
	Det vil sige at \textbf{når den magnetiske flux gennem et kredsløb ændres, lige meget hvad årsagen er, opstår en emf}.
	For at hjælpe med at holde styr på ens fortegn i sine udregninger med Faradays lov (hvilket kan være lidt af en pain in the ass) findes \textbf{Lenz Lov}, der siger at \textbf{Naturen hader en ændring i flux}. Det vil sige at den inducerede strøm vil prøve at lave et magnetisk felt, således at ændringen i flux slukkes. I praksis vil det sige at hvis en løkke oplever et magnetfelt, der stiger i styrke, vil den inducerede strøm løbe i en retning, således at der skabes et modsvarende B-felt, således at ændringen i flux bliver så lille som muligt. Og hvis der slukkes for magnetfeltet vil den inducerede strøm i ledningen lave et magnetfelt \textit{i samme retning} som det slukkede magnetfelt, igen således at ændringen i flux bliver så lille som mulig.
	
	
	
	
	\subsubsection{Det inducerede E-felt}
	Til udregningen af det magnetisk inducerede E-felt, kan parallellen mellem E-feltet og B-feltet bruges. Rotationerne er nemlig
	\begin{equation}
		\curl{E} = -\diff{\V{B}}{t}, \quad \curl{B} = \mu_0 \V{J}
	\end{equation}
	Og hvis det er et rent Faraday-felt (altså magnetisk induceret), så er divergensen 0:
	\begin{equation}
		\diverg{E} = 0, \quad \diverg{B} = 0 
	\end{equation}
	Dermed kan alle tricks forbundet med Ampères lov til udregning af B-felter også bruges i disse tilfælde. Her er $ \mu_0 I_{\text{enc}} $ dog udskiftet med $ -\uuud \Phi/\uuud t $:
	\begin{equation}
		\oint \V{E} \D \ud \V{l} = -\diff[\ud]{\Phi}{t}
	\end{equation}
	Der er dog et lille problem, at Ampères lov jo kun dur i magnetostatik, mens her er E-feltet skyldt af et ændrende magnetfelt. Så denne metode er faktisk kun en approksimation, men med mindre der er ekstremt hurtige fluktuationer, så er det intet stort problem. Selv i tilfældet af en ledning, der bliver klippet over af en saks, er det statisk nok til, at disse metoder dur. Det er normalt set kun når der ses på elektromagnetiske bølger og stråling, at man skal være påpasselig.
	
	Situationer hvor det magnetostatiske metoder kan bruges, uden det rent faktisk er magnetostatik, kaldes for \textbf{quasistatik} (lige som tilfældene fra termodynamik, hvor en process kan regnes som tilnærmelsesvist adiabatisk, hvis den er quasistatisk).
	 
	
	\subsubsection{Induktans} \index{Induktans}
	Hvis man har to løkker af ledning, der begge er i hvile, og man lader en jævn strøm $ I_1 $ gennem den ene, vil den danne et magnetfelt, noget af hvilken, der giver anledning til en flux gennem den anden ledning: $ \Phi_2 $. Fra Biot-Savart-loven ses det, at det magnetiske felt er proportionalt med strømmen:
	\begin{equation*}
		\V{B}_1 = \frac{\mu_0}{4 \pi} I_1 \oint \frac{\ud \V{l}_1 \times \usr}{\sr^2}
	\end{equation*}
	Dermed er fluxen gennem den anden løkke også proportional med strømmen:
	\begin{equation}
		\Phi_2 = \int \V{B}_1 \D \U{n} \ud a_2 = M_{21} I_1
	\end{equation}
	hvor $ M_{21} $ er proportionalitetskonstanten, kaldet den ">gensidige induktans"<\index{Induktans!Gensidig induktans} for de to løkker. En formel for denne kan findes ved at skrive fluxen ved vektorpotentialet, og så bruge Stokes sætning. Da fås
	\begin{equation}
		M_{21} = \frac{\mu_0}{4 \pi} \oint \oint \frac{\ud \V{l}_1 \D \ud \V{l}_2}{\sr}
	\end{equation}
	Denne kaldes for \textbf{Neumannformlen} \index{Induktans!Neumannformlen} og er et dobbelt kurveintegral. Først langs den ene, så langs den anden. Den er ikke særlig praktisk, men den viser dog to ting:
	\begin{enumerate}
		\item $ M_{21} $ er en ren geometrisk størrelse
		\item Hvis der byttes om på de to integralers roller i formlen er resultatet uændret. Det vil sige at $ M_{21}=M_{12} $, så vi smider bare tallene væk, så formlen for fluxen bliver
		\begin{equation*}
			\Phi_2 = MI_1
		\end{equation*}
	\end{enumerate}
	Dette betyder altså, at lige meget hvilken udformning eller position de to løkker har, så vil fluxen gennem løkke 2, når der løber en strøm $ I $ gennem løkke 1, være den samme, som fluxen gennem løkke 1, hvis strømmen $ I $ løb gennem løkke 2. Eller:
	\begin{equation*}
		\Phi_2 = MI = \Phi_1
	\end{equation*}
	Hvis strømmen gennem den første løkke varieres (quasistatisk), vil fluxen gennem løkke 2 også ændres, så der induceres en emf i løkken:
	\begin{equation}
		\emf_2 = -\diff[\ud]{\Phi_2}{t} = -M \diff[\ud]{I_1}{t}
	\end{equation}
	Det vil sige, at hvis strømmen gennem løkke 1 ændres, vil der løbe en strøm gennem løkke 2, også selvom der ingen ledninger er mellem dem. Det sker faktisk også, at der induceres en emf i løkke 1, når strømmen ændres (i løkke 1). Her er fluxen igen proportionel med strømmen
	\begin{equation}
		\Phi = LI
	\end{equation}
	hvor $ L $ kaldes \textbf{selvinduktansen} \index{Selvinduktans|see {Induktans}} (eller bare \textbf{induktansen}) af løkken. Den måles i \textbf{henries} (H), der er et volt-sekund per ampere. Induktansen er, lige som $ M $, ren geometrisk, og den er, lige som kapacitansen $ C $, en udelukkende positiv størrelse. Den inducerede emf er givet ved
	\begin{equation}
		\emf = -L \diff[\ud]{I}{t}
	\end{equation}
	Denne emf kaldes også for \textbf{modspænding} (back emf), da den har en retning, der modstrider ændringen i strøm. Der skal altså kæmpes mod denne modspænding, når strømmen skal ændres. Dermed svarer $ L $ lidt til en ">elektrisk masse"<, idet en større induktans gør at et kredsløb er sværere at drive, ligesom en større masse, gør det sværere at accelerere et objekt.
	
	\subsection{Maxwells ligninger}\index{Maxwells ligninger}
	Der er fundet 4 love, der specificerer divergensen og rotationen af elektriske og magnetiske felter. De er som følger:
	
	\begin{table}[H]
		\centering
		\index{Maxwells ligning!Integralform}
		\begin{tabular}{r >{$\displaystyle} l <{$} p{1cm} >{$\displaystyle} l <{$}}
			\vspace{0.3cm}
			& \text{Differentialform} & & \text{Integralform} \\
			\vspace{0.3cm}
			(i)		& \diverg{E} = \frac{1}{\epsilon_0} \rho 					& & \oint_{\mathcal{S}} \V{E} \D \U{n} \ud a = \frac{1}{\epsilon_0} Q_{\text{enc}} \\
			\vspace{0.3cm}
			(ii)	& \diverg{B} = 0 											& & \oint_{\mathcal{S}} \V{B} \D \U{n} \ud a = 0 \\
			\vspace{0.3cm}
			(iii)	& \curl{E} = - \diff{\V{B}}{t} 								& & \oint_{\mathcal{P}} \V{E} \D \ud \V{l} = - \int_{\mathcal{S}} \diff{\V{B}}{t} \D \U{n} \ud a \\
			(iv) 	& \curl{B} = \mu_0 \V{J} + \mu_0 \epsilon_0 \diff{\V{E}}{t}	& &\oint_{\mathcal{P}} \V{B} \D \ud \V{l} = \mu_0 I_{\text{enc}}+\mu_0 \epsilon_0 \int_{\mathcal{S}} \diff{\V{E}}{t} \D \U{n} \ud a
		\end{tabular}
	\end{table}
	\noindent
	Hvor overfladeintegralerne i de to første er over enhver overflade $ \mathcal{S} $, mens overfladeintegralerne i de to sidste, er over enhver overflade $ \mathcal{S} $ med $ \mathcal{P} $ som grænse. 
	
	
	
	\subsubsection{Maxwells ligninger i lineære medier}
	I stof der har tendens til at blive enten elektrisk eller magnetisk polariseret, og som dermed oplever enten en mængde bundne ladninger eller strømme, er det oftest smartere at arbejde med en version af Maxwells ligninger, der kun refererer til de frie ladninger i materialet.
	
	Maxwells love er I dette tilfælde
	\begin{table}[H]
		\index{Maxwells ligninger!I lineære medier}
		\centering
		\begin{tabular}{r >{$\displaystyle} l <{$} p{1cm} >{$\displaystyle} l <{$}}
			\vspace{0.3cm}
			& \text{Differentialform} & & \text{Integralform} \\
			\vspace{0.3cm}
			(i)		& \diverg{E} = \frac{1}{\epsilon} \rho_f & & \oint_{\mathcal{S}} \V{E} \D \U{n} \ud a = \frac{1}{\epsilon} Q_{f_{\text{enc}}} \\
			\vspace{0.3cm}
			(ii)	& \diverg{B} = 0 											& & \oint_{\mathcal{S}} \V{B} \D \U{n} \ud a = 0 \\
			\vspace{0.3cm}
			(iii)	& \curl{E} = - \diff{\V{B}}{t} 								& & \oint_{\mathcal{P}} \V{E} \D \ud \V{l} = - \int_{\mathcal{S}} \diff{\V{B}}{t} \D \U{n} \ud a \\
			(iv) 	& \curl{B} = \mu \V{J}_f + \mu \epsilon \diff{\V{E}}{t}	& &\oint_{\mathcal{P}} \V{B} \D \ud \V{l} = \mu I_{f_{\text{enc}}}+\mu \epsilon \int_{\mathcal{S}} \diff{\V{E}}{t} \D \U{n} \ud a
		\end{tabular}
	\end{table}
	\noindent
	Hvor overfladeintegralerne i de to første, som før, er over enhver overflade $ \mathcal{S} $, mens overfladeintegralerne i de to sidste, er over enhver overflade $ \mathcal{S} $ med $ \mathcal{P} $ som grænse. 

\end{document}