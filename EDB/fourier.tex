\documentclass[EL2Noter.tex]{subfiles} % HUSK FOR FANDEN AT REDIGERE DENNE LINJE

% Hvis ikke dokumenterne (hoved & under) er i samme mappe, skal den relative stig bruges.



\begin{document}
	
	\section{Fouriertransformation og serier (Noter kap 3, egentlig EMM)}
	Idet jeg allerede har skrevet om Fouriertransformation og Fourierserier (MatF1 Noter), gider jeg ikke gøre det igen. Jeg har til gengæld inkluderet en lettere modificeret udgave af afsnittene. Jeg har slettet afsnittet om den reelle Fourierserie, for med den komplekse skal man kun udregne et enkelt integral. Jeg har også inkluderet et lille skriv om, hvordan man omskriver til cos/sin funktioner fra den komplekse fourierserie.
	
	Fourierserier er en måde at opskrive enhver, periodisk\footnote{De behøver ikke at være periodiske, faktisk, de kan nemlig gøres periodiske.} funktion (givet nogle betingelser), som en uendelig række af harmoniske komponenter. De førnævnte betingelser går under navnet \textbf{Dirichletbetingelserne}, og er som følger:
	\begin{enumerate}
		\item Funktionen skal være periodisk (hvis den er defineret over et endeligt interval, kan den gøres periodisk)
		\item Den skal være enkelt-værdiet og kontinuert (der må gerne være et endeligt antal af endelige diskontinuiteter)
		\item Der skal være et endeligt antal ekstrema inden for én periode $ T $.
		\item Integralet over én periode af $ |f(x)| $ skal konvergere
	\end{enumerate}
	Er alle disse betingelser opfyldt, kan funktionen skrives ved:
	\begin{equation}
	f(x) = \sum_{n = -\infty}^{\infty} \T{c}_n \exp \pp{-in\omega t}, \quad \omega = \frac{2\pi}{T}
	\end{equation}
	hvor
	\begin{equation}
	\T{c}_n = \frac{1}{T} \int_{t_0}^{t_0+T} f(x) e^{in\omega t} \ud t
	\end{equation}
	kaldes for Fourierkoefficienten, og $ t_0 $ er arbitrært, men sættes ofte til $ 0 $ eller $ -T/2 $. Fourierkoefficienterne er da et integrale over én periode (med et smart valg af startpunkt). Læg også mærke til, at jeg har udskiftet $ \omega_n $ med $ n \omega $, idet jeg synes dette er mindre forvirrende.
	
	For funktioner defineret på et interval, der bliver lavet periodiske, er perioden af Fourierserien ikke altid den samme som intervallet, men et helt multiplum af denne ($ 2T $, $ 3T $, etc).
	
	Ved diskontinuiteter konvergerer Fourierserien mod gennemsnittet af de to grænseværdier. Der opstår også \textbf{Gibbs fænomenet} $ \delta $, som er et ekstra ekstremum med amplitude $ \delta $ i begge grænser af diskontinuiteten. Fænomenet rykker tættere og tættere på grænserne, jo flere led der medtages, men denne forsvinder aldri. Et eksempel er følgende firkantbølge, approksimeret ved 125 led af Fourierserien.
	
	\begin{figure}[H]
		\includegraphics[width=0.9\textwidth]{img/Gibbs.png}
		\caption{Illustration af Gibbs fænomenet. Her vises Fourierserien for en firkantbølge, med 125 led. Gibbs fænomenet ses ved de "lodrette" dele, hvor der overskydes en smule. \href{https://commons.wikimedia.org/w/index.php?curid=10212831}{Illustration fra Wikipedia, public domain}.}
		\label{fig:Gibbs}
	\end{figure}
	
	\subsubsection*{Integration og differentiation}
	Ved (ledvis) integration af en Fourierserie konvergerer denne mod stamfunktionen af $ f(x) $, med forskel i en integrationskonstant, der skal findes.
	
	Ved (ledvis) differentiation af en Fourierserie konvergerer denne mod $ f'(x) $, såfremt $ f'(x) $ opfylder Dirichletbetingelserne.
	
	
	
	\subsection{Dirac \texorpdfstring{$\delta$}{delta}-funktion}
	Dirac $ \delta $-funktionen er en funktion defineret ved:
	\begin{equation}
	\int_{a}^{c} f(t) \delta(t-b) \ud t = f(b)
	\end{equation}
	hvor $ a \leq b \leq c $. Ellers er integralet 0. Ydermere er $ \int \delta(t-a) \ud t = 1 $, hvis $ a $ ligger inden for integralgrænserne.
	Deltafunktionen kan også defineres som
	\begin{equation}
	\delta(t) = 0, t \neq 0
	\end{equation}
	Hvor funktionsværdien, når $ t=0 $, er "uendelig". Nogle andre egenskaber ved denne funktion er:
	\begin{equation}
	\delta(-t) = \delta (t), \quad \delta(at) = \frac{1}{|a|} \delta(t), \quad t\delta(t) = 0
	\end{equation}
	Dens stamfunktion er Heavyside-funktionen, der er defineret ved $ H(t) = 1 $ til $ t>0 $ og $ H(t) = 0$ for $ t<0 $. Dermed er $ H'(t) = \delta(t) $. $ \delta $-funktionen kan også beskrives ved integraler:
	\begin{equation}
	\delta(t-u) = \frac{1}{2 \pi} \int_{-\infty}^{\infty} e^{i \omega (t-u)} \ud\omega, \quad \delta(\V{r}) = \frac{1}{(2\pi)^3} \int e^{i \Vk \D \V{r}} \ud^3 \Vk
	\end{equation}
	
	
	\subsection{Fouriertransformation}
	Fouriertransformationer er en generalisering af Fourierserier, hvor perioden tages til at være uendelig, og der dermed ikke er noget krav om periodicitet. Det eneste krav er faktisk følgende:
	\begin{equation}
		\int_{-\infty}^{\infty} |f(t)| \ud t \neq \infty
	\end{equation}
	altså at det uendelige integrale konvergerer. Selve transformationen er en lineær integraltransformation $ f(t) \rightarrow \tilde{f} (\omega) = \mathcal{F}[f(t)] $ defineret ved
	\begin{equation}
		\tilde{f}(\omega) = \frac{1}{\sqrt{2\pi}} \int_{-\infty}^{\infty} f(t) e^{-i\omega t} \ud t
	\end{equation}
	Og dens inverse
	\begin{equation}
		f(t) = \frac{1}{\sqrt{2\pi}}\int_{-\infty}^{\infty} \tilde{f}(\omega)e^{i\omega t} \ud\omega
	\end{equation}
	Her ses ligheden med den komplekse Fourierserie, men i stedet for en diskret sum af harmoniske svingninger, fra negativ til positiv uendeligt, har vi her en kontinuert funktion, der beskriver summen af disse harmoniske svingninger - netop et integral.
	
	
	\subsubsection{Foldning}
	Et emne der, som så, ikke omhandler Fouriertransformation, men som gøres nemmere ved brug af denne, er foldning. Målinger af fysiske fænomener medbringer ofte en eller anden form for støj grundet måleapparatet. Hvis det, der måles gives navnet $ h(z) $, mens det fysiske fænomen er givet ved $ f(x) $, så fås følgende sammenhæng mellem dem:
	\begin{equation}
		h(z) = \int_{-\infty}^{\infty} f(x) g(z-x) \ud x \label{eq:fold}
	\end{equation}
	hvor $ g(z-x) $ er støjen introduceret af måleapparatet. Dette kan også bruges til at introducere "falske" måleinstrumenter, som for eksempel filtre, for at frafiltrere støj. Ligningen \eqref{eq:fold} kaldes for \textbf{foldningen} af $ f $ og $ g $, og benævnes også $ f\ast g $. Foldning er både \textbf{kommutativt} ($ f\ast g = g \ast f $), \textbf{associativt} og \textbf{distributivt}.
	
	Måden dette relaterer til Fouriertransformation, er ved at foldningen nemt kan skilles ad i frekvensdomænet, givet \textbf{foldeteoremet}, der siger:
	\begin{equation}
		\mathcal{F}[h(z)] = \tilde{h}(k) = \sqrt{2\pi} \tilde{f}(k) \tilde{g}(k)
	\end{equation}
	Altså bliver det grimme, uegentlige integral til et simpelt produkt i frekvensdomænet. Ligeledes gælder der omvendt:
	\begin{equation}
		\mathcal{F}[f(x)g(x)] = \frac{1}{\sqrt{2\pi}} \tilde{f}(k) \ast \tilde{g}(k)
	\end{equation}
	
	
	\subsubsection{Nyttige egenskaber ved Fouriertransformation}
	For Fouriertransformationer gælder \textbf{Parseval's teorem} også:
	\begin{equation}
		\int_{-\infty}^{\infty} |f(t)|^2 \ud t = \int_{-\infty}^{\infty} |\tilde{f}(\omega)|^2 \ud\omega
	\end{equation}
	Der er også følgende transformationer:
	\begin{table}[H]
		\begin{tabular}{ll}
			Funktion				& Fouriertransformation                                               \\
			\hline	\rule[-2ex]{0pt}{4.5ex} 
			$ f(at) $           	& $ a\inverse \tilde{f}\pp{\frac{\omega}{a}} $                        \\
			\rule[-2ex]{0pt}{4.5ex}
			$ f(t-b) $          	& $ e^{-ib\omega} \tilde{f}\pp{\omega} $                              \\
			\rule[-2ex]{0pt}{5.5ex}
			$ e^{\alpha t}f(t) $	& $ \tilde{f}\pp{\omega+i\alpha} $                                    \\
			\rule[-2ex]{0pt}{4.5ex}
			$ f'(t) $               & $ i \omega \tilde{f}\pp{\omega} $                                   \\
			\rule[-2ex]{0pt}{4.5ex}
			$ f''(t) $              & $ -\omega^2 \tilde{f}\pp{\omega} $                                  \\
			\rule[-2ex]{0pt}{4.5ex}
			$ f^{(n)}(t) $          & $ (i\omega)^n\tilde{f}\pp{\omega} $                                 \\
			\rule[-2ex]{0pt}{4.5ex}
			$ \int^{t} f(u) \ud u $ & $ (i\omega)\inverse \tilde{f}\pp{\omega} + 2\pi c \delta(\omega) $  \\
			\rule[-2ex]{0pt}{4.5ex}
			$ f(t)\ast g(t) $       & $\sqrt{2\pi} \tilde{f}\pp{\omega}\tilde{g}(\omega) $                \\
			\rule[-2ex]{0pt}{4.5ex}
			$ f(t)g(t) $            & $ \frac{1}{\sqrt{2\pi}}\tilde{f}\pp{\omega}\ast \tilde{g}(\omega) $
		\end{tabular}
	\end{table}

	
	
\end{document}

