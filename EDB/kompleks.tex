\documentclass[EL2Noter.tex]{subfiles}


\begin{document}
	\section{Komplekse Tal (Note kap 1)}
	Meget kort om komplekse tal:
	
	De er en udvidelse af de reelle tal. Her består hvert komplekst tal af 2 led, hvor det ene er ganget med tallet $ i $. Dette er et meget specielt talt, idet det har egenskaben:
	\begin{equation}
		i^2 = -1
	\end{equation}
	Et generelt komplekst tal $ z $ skrives da:
	\begin{equation}
		z = x+iy
	\end{equation}
	hvor $ x $ kaldes den \textit{reelle} del og $ y $ kaldes den \textit{imaginære} del. Disse skrives også henholdsvis
	\begin{equation}
		\Re z = x, \quad \Im z = y
	\end{equation}
	Man definerer også den \textit{kompleks konjugerede} $ z^*$ ved
	\begin{equation}
		z\konj = x-iy
	\end{equation}
	Så man flipper bare fortegnet på den imaginære del. Den reelle og imaginære del kan udregnes ved hjælp af den konjugerede
	\begin{equation}
		\Re z = \frac{z+z^*}{2}, \quad \Im z = \frac{z-z^*}{2i}
	\end{equation}
	Man definerer også modulus (størrelse) og argument (vinkel/retning) af komplekse tal:
	\begin{equation}
		|z| = \sqrt{z*z} = \sqrt{x^2+y^2}, \quad \arg z = \theta
	\end{equation}
	hvor $ \theta $ kan findes ved trigonometri:
	\begin{equation}
		\tan \theta = \frac{y}{x}, \quad \cos \theta = \frac{x}{|z|}, \quad \sin \theta = \frac{y}{|z|}
	\end{equation}
	Men pas på med at bruge $ \tan $ til at finde argumentet, idet den ikke altid giver det rigtige svar. Så brug helst $ \cos $ eller $ \sin $ (jeg har ikke en god forklaring på det, lige nu, så det må vente til et senere tidspunkt.)
	
	Hvis man er nødt til at dividere komplekse tal på kartesisk form (eller har et $ i $ i nævneren og gerne vil af med det), så er der et smart trick: forlæng brøken med nævnerens konjugerede. Dette giver nemlig:
	\begin{equation}
		z = \frac{a+ib}{c+id} = \frac{a+ib}{c+id}\frac{c-id}{c-id} = \frac{(a+ib)(c-id)}{c^2+d^2}
	\end{equation}
	Eller mere generelt:
	\begin{equation}
		z = \frac{w_1}{w_2} = \frac{w_2^* w_1}{|w_2|}
	\end{equation}
	
	
	
	\subsection{Den komplekse eksponentialfunktion}
	For imaginære tal ($ z = i\theta $) er eksponentialfunktionen defineret ved
	\begin{equation}
		e^{i\theta} = \cos \theta + i \sin \theta
	\end{equation}
	Størrelsen af denne kan findes på to måder:
	\begin{equation}
	 |e^{i\theta}| = \sqrt{cos^2 \theta + \sin^2 \theta} = 1, \ |e^{i\theta}| = \sqrt{(e^{i\theta})^*\, e^{i\theta}} = \sqrt{e^{-i\theta}e^{i\theta}} = 1
	\end{equation}
	Den komplekse eksponentialfunktion for et komplekst tal er da
	\begin{equation}
		e^{x+iy} = e^x \d e^{iy} = e^x (\cos y + i \sin y)
	\end{equation}
	
	Komplekse tal kan også skrives på en anden form, der gør brug af den komplekse eksponentialfunktion. Denne form kaldes for \textit{polær form}.
	\begin{equation}
		z = |z| e^{\i \theta}
	\end{equation}
	Denne måde at skrive komplekse tal på, har en stor fordel når det kommer til multiplikation og division. For i stedet for at skulle gange hvert led sammen, så er multiplikation bare at gange/dividere modulus, og lægge argumenter sammen (eller trække dem fra):
	\begin{equation}
		z_1\D z_2 = |z_1||z_2| e^{i(\theta_1+\theta_2)}, \quad \frac{z_1}{z_2} = \frac{|z_1|}{|z_2|} e^{i(\theta_1-\theta_2)}
	\end{equation}
	
	\subsection{Tidsudvikling (komplekse harmoniske svingninger)}
	Den komplekse eksponentialfunktion bruges ofte til at beskrive harmoniske svingninger. Eksempelvis ses en tidsligt varierende emf, givet ved $ \emf = \emf_0 \cos \omega t $. Dette kan skrives som den reelle del af den komplekse eksponentialfunktion:
	\begin{equation}
		\emf = \Re(\emf_0 e^{-i\omega t})
	\end{equation}
	Det negative fortegn i eksponenten er et arbitrært valg, men det er den konvention bøgerne, vi bruger, bruger.
	
	Generelt har eksponenten både en reel og imaginær del. Eksempelvis:
	\begin{equation}
		f(t) = e^{-(\gamma + i \omega) t} = e^{-\gamma t} e^{-i\omega t}
	\end{equation}
	hvor den reelle del af dette udtryk er en eksponentielt dæmpet harmonisk svingning (se Mek2/MatF2 pensum for grundigere gennemgang af denne, med udledning):
	\begin{equation}
		\Re(f(t)) = e^{-\gamma t} \cos \omega t
	\end{equation}
\end{document}