\documentclass[MatF2Noter.tex]{subfile}

\begin{document}
	\section{Binomialformler og binomialekspansioner}
	Fra bogen Kalkulus (kapitel 1.4 og kapitel 12.10) fås binomialformlen og binomialrækken/binomial-ekspansionen. \textbf{Binomialformlen} er som følger (for $ n\in \Set{N} $):
	\begin{equation}
		(a+b)^n = \sum_{k=0}^{n} {n \choose k} a^{n-k}b^k, \quad {n \choose k} = \frac{n!}{k!(n-k)!} = \frac{m(m-1)(m-2)\dots(m-k+1)}{k!}
	\end{equation}
	Læg mærke til, at der er $ n+1 $ led i denne sum, og at $ {n \choose k} $ har præcis $ k $ faktorer i både tæller og nævner. $ {n \choose k} $ kaldes for \textbf{binomialkoefficienterne} og disse kan findes ved hjælp af Pascals trekant, hvor de er summen af de to koefficienter lige oven for:
	\begin{equation}
		{n \choose k} + {n \choose k+1} = {n+1 \choose k+1}
	\end{equation}
	skriv endelig Pascals trekant ud, så dette giver mening. På engelsk kaldes koefficienterne også for ">choose functions"<, hvor $ {n \choose k} $ siges som ">$ n $ choose $ k $"<. Dette har med kombinatorik at gøre, og der står mere i kapitel 1.4. Det ses også, at $ {n \choose 0} = 1 $, da $ 0!=1 $ og dermed:
	\begin{equation}
		{n \choose 0} = \frac{n!}{0! (n-0)!} = \frac{n!}{n!} = 1
	\end{equation}
	
	Binomialkoefficienter kan udvides til at gælde for alle reelle tal, hvilket er smart for binomialekspansioner. For et reelt tal $ \alpha $ er $ {\alpha \choose k} $ givet ved:
	\begin{equation}
		{\alpha \choose k} = \frac{\alpha(\alpha-1)(\alpha-2)\dots(\alpha-k+1)}{k!}
	\end{equation}
	altså helt det samme, som for et naturligt tal. Forskellen er bare, at der her ikke kan bruges fakultet for $ \alpha $ (med mindre, selvfølgelig $ \alpha $ er et naturligt tal, men så er denne definition jo ret redundant, for helvede da!). Igen har $ {\alpha \choose k} $ $ k $ faktorer i både tæller og nævner, hvilket er ret smart, hvis man ikke er helt sikker på, hvor mange gange, man skal gange. Også her er $ {0 \choose k} = 1 $.
	
	\subsection{Binomialekspansion} \label{sec:binomial}
	Hvor dette bliver rigtig relevant for MatF2 er \textbf{binomialekspansioner}. Taylorrækken for $ (1+x)^{\alpha} $ omkring 0, er givet ved (se beviset for dette i kapitel 12.10, der er for mange udregninger til, at jeg gider skrive det her. Kort fortalt er faktorerne ud for den $ n $-teafledte lig tælleren i binomialkoefficienterne, så $ a_n $, faktoren ud for det $ n $-te led i rækken, er da givet ved binomialkoefficienten)
	\begin{align}
		(1+x)^{\alpha} &= \sum_{k=0}^{\infty} a_k x^k, \quad a_k = \frac{\alpha(\alpha-1)(\alpha-2)\dots(\alpha-k+1)}{k!} = {\alpha \choose k} \nonumber \\
		&= \sum_{k=0}^{\infty} {\alpha \choose k} x^k = 1 + \sum_{k=1}^{\infty} {\alpha \choose k} x^k
	\end{align}
	Det ses, at hvis $ \alpha \in \Set{N} $ er $ {\alpha \choose k} = 0 $ for $ k>\alpha $, og dermed er det en endelig serie, i stedet for en uendelig.
	
	Idet dette er Taylorrækken omkring $ x=0 $ gælder denne for små $ x $ (konvergensradius for reelle $ \alpha $ er 1, så den gælder faktisk for $ x<1 $, såfremt alle led tages med - altså hele den tællelige uendelighed af dem), og den er rigtig smart til at udregne residuer med (se afsnit \ref{sec:residuer}). 
\end{document}