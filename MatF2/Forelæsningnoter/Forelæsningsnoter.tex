% Nikolai Nielsens "Fysiske Fag" preamble
\documentclass[a4paper,10pt]{article} 	% A4 papir, 10pt størrelse

% Packages:
\usepackage[Danish]{babel}		% Dansk og engelsk
\usepackage[utf8]{inputenc}				% Vi bruger UTF-8 til Æ, Ø & Å
\usepackage{amsmath,amssymb,url,graphicx} % Math-ting, URLs og illustrationer
\usepackage{subfiles}
\usepackage{mathtools, gauss}
\usepackage{enumerate}

\usepackage{bm}					% BoldMath


% Forskellige pakker til illustrationer og figurer. Udkommenter hvis unødvendigt
\usepackage{wrapfig}		% Ombrydning af tekst
\usepackage{caption}
\usepackage{subcaption}
\usepackage{floatrow}		% Float-rækker

% Links i indholdsfortegnelsen. Udkommenter hvis unødvendig
\usepackage{hyperref}
\hypersetup{
	colorlinks,
	citecolor=black,
	filecolor=black,
	linkcolor=black,
	urlcolor=blue,
	breaklinks=true
}

% Commands:
\newcommand{\D}{\cdot}					% Gange
\newcommand{\V}[1]{\bm{#1}}				% Vektornotation
\newcommand{\Set}[1]{\mathbb{#1}}		% Sæt, fx reelle tal
\newcommand{\e}[1]{e^{#1}}				% e^x
\newcommand{\DU}[1]{\underline{\underline{#1}}} % dobbelt understregning
\newcommand{\Lim}[2]{\lim\limits_{#1 \rightarrow #2}}	% Nyttig grænseværdi
\newcommand{\Maple}[2]{\\\includegraphics[width=#2cm]{img/maple#1.png}} % Til Matematikopgaver
\newcommand{\diff}[3][\partial]{\frac{#1 #2}{#1 #3}}
\newcommand{\Bf}[1]{\mathbf{#1}} 		% til fed matematikskrift
\newcommand{\pp}[1]{\left( #1 \right)}	% Stor blød parentes
\newcommand{\bb}[1]{\left[ #1 \right]}	% Stor firkantet parentes
\newcommand{\kk}[1]{\left\{ #1 \right\}}% Stor tuborgklamme
\newcommand{\inverse}{^{-1}}			% hvis man tit skal skrive ^{-1}, ie $X\inverse$
\newcommand{\konj}[1]{\overline{#1}}	% Kompleks konjugeret


% 'upright d'. Gør at du kan afslutte integraler med f.eks. \int x \ud x
\DeclareMathOperator{\ud}{\,d\negthickspace\,}

% LinAlg
\usepackage{mathrsfs}
\usepackage{ushort}					% kort understregning
\usepackage{blkarray}				% blockarrays, til angivning af rækker/søjler i en matrix.
\DeclareMathOperator{\sign}{sign}
\DeclareMathOperator{\Pol}{Pol}
\DeclareMathOperator{\Span}{span}
\DeclareMathOperator{\rg}{rg}
\DeclareMathOperator{\id}{id}					% Identitetafbildning - LinAlg
\newcommand{\morf}[3]{#1\colon #2 \to #3} 		% \morf{f}{A}{B} = f\colon A\to B
\newcommand{\morffFF}[3][f]{#1 \colon \Set{F}^{#2} \to \Set{F}^{#3}} % Afbildning i Set(F) - LinAlg
\newcommand{\U}[1]{\ushort{#1}} 				% vektornotation - LinAlg
\newcommand{\mat}[1]{\ushortd{#1}}				% Matrixnotation - LinAlg
\newcommand{\trans}[2]{{}_{#1}\mathsf{T}_{#2}}
\newcommand{\bas}[1]{\mathscr{#1}}
\newcommand{\ele}[2]{{}_{#1}[#2]}
\newcommand{\transf}[3]{{}_{#1}[#2]_{#3}}

% MatF
% Standardenhedsvektorer
\newcommand{\Vi}{\V{i}}
\newcommand{\Vj}{\V{j}}
\newcommand{\Vk}{\V{k}}
\newcommand{\grad}{\nabla}
\newcommand{\ind}[2]{\langle #1 | #2 \rangle}
\newcommand{\norm}[1]{\lVert #1 \rVert}
\DeclareMathOperator{\erf}{erf}


% Margen
\usepackage[margin=1in]{geometry}

% Max antal kolonner i en matrix. Default er 10
%\setcounter{MaxMatrixCols}{20}

% Hvor dybt skal kapitler labeles?
%\setcounter{secnumdepth}{0}		

% Til nummerering af ligninger. Så der står (afsnit.ligning) og ikke bare (ligning)
\numberwithin{equation}{section}		

% Header
%\usepackage{fancyhdr}
%\lhead{Nikolai Plambech Nielsen, 21-06-95\\Dato:. Klasse 5.C - De Fysiske Fag}
%\pagestyle{fancy}

%Titel
\title{MatF2 forelæsningsnoter på KU (Matematik for Fysikere 2)}
\author{af Nikolai Plambech Nielsen, LPK331 \\ Version 1.0}


\begin{document}
	\selectlanguage{danish}
	%Udkommenter ovenstående, hvis du skriver på engelsk.
	
	\maketitle
	\tableofcontents
	
	\newpage
	\section{Tirsdag, 26/4, 9-10}
	Kort gennemgang af kurset:
	\begin{itemize}
		\item Komplekse variable (Kompelse funktioner, integraler, rækker afbildninger, potentialer og transformationer)
		\item Laplace-transformationer (og inverse)
		\item anvendelser, eksempler etc
		\item Kort afslutning om specielle funktioner (Bessel, Legendre, ...)
	\end{itemize}
	
	\paragraph{Eksamensform.}
	4 timers skriftlig eksamen. IKKE ITX-eksamen. Intern censur, alle tilladte hjælpemidler.
	
	\subsection{Komplekse variable}
	systematiske redskaber for at løse den harmoniske oscillator:
	\begin{align*}
	m \diff{^2 x}{t^2} + kx &= 0 \\
	\diff{^2 x}{t^2} + \omega_0^2 x &= 0, \quad \omega_0^2 = k/m \\
	 \pp{\diff{^2}{t^2} + \omega_0^2} x &= 0\\
	 \pp{\diff{}{t} +i \omega_0} \pp{\diff{}{t} - i \omega_0} x &= \text{Skriv ud!! }= 0 \\
	 \pp{\diff{}{t} +i \omega_0} &= \hat{A}_+, \quad \pp{\diff{}{t} - i \omega_0} = \hat{A}_- \\
	 \hat{A}_+ \hat{A}_- x(t) &= 0 = \hat{A}_- \hat{A}_+ x(t) \text{ Kummutation }
	\end{align*}
	Kummutation gør at:
	\begin{align}
		\hat{A}_- x(t) &= 0, \quad x= k_1 \exp(i\omega_0 t)\\
		\hat{A}_+ x(t) &= 0, \quad x= k_2 \exp(-i\omega_0 t)\\
		x(t) &= k_1 \exp(i\omega_0 t) + k_2 \exp(-i\omega_0 t)
	\end{align}
	Parallel til LinAlg. Svarer til løsning af karakteristisk polynomium for egenvektorer.
	
	\subsection{Komplekse tal}
	\begin{equation}
		z = x+iy, \quad (x,y) \in \Set{R}^2
	\end{equation}
	$ x $ kaldes reel del $ \Re z $, $ y $ kaldes imaginær del $ \Im z $. Skrives ofte i Argand-diagram (polær form), med norm ($ r = \sqrt{x^2+y^2} $) og argument ($ \theta = r \cos x = r \sin y $). \textbf{HUSK DIAGRAM}
	
	\begin{equation}
		z^* = x-iy, \quad |z|^2 = z \, z^* = x^2+y^2
	\end{equation}
	
	Komplekse tal multipliceres ved at multiplicerer længder, og addere længder:
	\begin{equation}
	z_1 = r_1 \exp(i\, \theta_1), \quad z_2 = r_2 \exp(i\, \theta_2), \quad z_1 z_2 = r_1 r_2 \exp (i\,(\theta_1+\theta_2))
	\end{equation}
	

	\subsection{Komplekse funktioner}
	Kontinuitet defineres på samme måde som for funktioner af én variabel:
	
	
	$ f(z) $ siges at være kontinuert i $ z_0 $m hvis for alle $ \epsilon>0 $ findes et $ \delta>0 $, således at $ |z-z_0| < \delta \Rightarrow |f(z) - f(z_0) | < \epsilon $. $ f $ er kontinuert i et domæne $ R $, hvis det er kontinuert i alle punkter $ z_0 $ i $ R $
	
	\subsection{Regnehold 4}
	Underviser: Tobias Thune, mail: TobiasThune@gmail.com
	
	
	
	
	
	
	
	\section{Torsdag, 28/4, 9-10}
	\subsection{Alternativ notation for kontinuitet}
	\begin{equation}
		\forall \ \epsilon >0 \  \exists \ \delta >0 : |z-z_0| < \delta \Rightarrow |f(z)-f(z_0)| < \epsilon
	\end{equation}
	"> for alle epsilon større end 0, eksisterer der et delta således at..."<
	\subsection{Differentiabilititet}
	En funktion $\morf{f}{\Set{C}}{\Set{C}} $ er differentiabelt i $ z $, hvis 
	\begin{equation}
		f'(z) = \lim\limits_{h\to 0} \frac{f(z+h)-f(z)}{h}
	\end{equation}
	eksisterer og er unik. Forskellen fra reelle funktioner er, at der er uendelige måder hvorpå $ h $ kan gå mod 0 (uendelig mængde rette linjer, der har forskellige hældninger).
	
	\subsection{Geometrisk fortolkning af kompleks differentiabilitet}
	For $ h>> 1 $ er
	\begin{equation}
		\Delta f = f(z+h) - f(z) \approx f(z) + h f'(z) -f(z) = h f'(z) = h |f'(z)| \exp(i \theta), \quad  \theta = \arg f'(z)
	\end{equation}
	Det svarer altså til en rotation af $ \theta $ og en forstrækning $ |f'(z)| $ af rummet (">twist and stretch"<)
	
	
	
	
	\section{Tirsdag, 10/5, 9-10}
	\subsection{Leibnez og Bernoulli}
	1712-1713, lille venlig kampstrid. Hvad er $ \ln (-x) $? imaginært, reelt eller udefineret?
	
	Leibnez argument var at det er ikke-eksisterende. logaritmen er ikke defineret for negative tal (da man kan få imaginære tal ved at tage $ 1/2 \ln (-1) =\ln i$). 
	
	Bernoullis argument var
	\begin{align*}
		\frac{dx}{x} &= \frac{-dx}{-x} = \frac{d(-x)}{-x} \\
		\int \frac{dx}{x} &= \int \frac{d(-x)}{-x}, \quad u = -x, \quad ln(x) = ln(u) = ln(-x)
	\end{align*} 
	Og det er et rungende nej, der skal en integrationskonstant, så $ \ln x = \ln (-x) + C $.
	
	Regneregler:
	\begin{align}
	-x=(-1)x = (e^{i\pi k}) x, k = \text{ ulige}\\
	ln(-x) = i \pi k + \ln x
	\end{align}
	
	\subsection{potensrækker}
	\begin{equation}
		S_n = \sum_{i=0}^{n} \rho^i
	\end{equation}
	\begin{align}
		S_n - \rho S_n = 1- \rho^{n+1} \\
		S_n = \frac{1-\rho^{n+1}}{1-\rho}
	\end{align}
	hvis $ |\rho|<1 $ er
	\begin{equation}
	\lim\limits_{n\to \infty} S_n = \frac{1}{1-\rho}
	\end{equation}
	
	For komplekse rækker
	\begin{equation}
		f(z) = \sum_{n=0}^{\infty} a_n z^n, \quad a_n,z \in \Set{C}
	\end{equation}
	
	Vi definerer en følge af partielle rækker
	\begin{equation}
		S_j = \sum_{n=0}^{j} a_n z^n
	\end{equation}
	hvis vi tager følgerne $ S_0, S_1, S_2,\dots $, så siges rækken at være konvergent, hvis $ \lim\limits_{j\to \infty} S_j = S $. Alternativt:
	\begin{equation}
	\forall \ \epsilon > 0\  \exists\  M \in \Set{N}\  \forall \ j > M : |S-S_j| < \epsilon
	\end{equation}
	
	Eksempel
	\begin{equation}
		S= 1+1-1+1-1+1-\dots  = 1/2, \quad 1-S = 1-(1-1+1-1+ \dots) = S \Leftrightarrow 2S = 1 \Leftrightarrow S= 1/2 = \sum_{n=0}^{\infty}  (-1)^n
	\end{equation}
	
	Kigger på partielle rækker
	\begin{align}
		S_0 = 1 \\
		S_1 = 0 \\
		S_2 = 1 \\
		S_3 = 0 \\
		S_4 = 1 
	\end{align}
	Konvergerer ikke, da $ \lim\limits_{j\to\infty} S_j = \varnothing $
	
	
	\subsection{Absolut konvergens}
	En række er absolut konvergent hvis
	\begin{equation}
	\sum_{n=0}^{\infty} |a_n z^n| < \infty
	\end{equation}
	For $ S = \sum_{n=0}^{\infty}  (-1)^n $ er $\sum_{n=0}^{\infty} |(-1)^n| = \infty$, da der summeres en uendelig mængde af 1.
	
	\textbf{Hvis $ f(z) = \sum_{n=0}^{\infty} a_n z^n $ er konvergent i $ z_0 $, vil rækken være absolut konvergent for alle $ |z| < |z_0| $}. $ \leftarrow $ Det virker ret vigtigt!
	
	Kort bevis:	
	\begin{equation}
		\sum |a_n z^n| = \sum |a_n| |z|^n = \sum |a_n| \frac{|z_0|^n}{|z_0|^n} |z|^n = \sum |a_n| |z_0|^n \frac{|z|^n}{|z_0|^n} 
	\end{equation}
	
	hvis $ \sum a_n z_0^n $ konvergerer opfylder alle led $ |a_n z_0|^n $. Dvs at der findes $ M>0 $ hvor $ |a_n z_0^n| $, og
	
	\begin{equation}
	\sum |a_n| |z_0|^n \frac{|z|^n}{|z_0|^n}  < \sum M \vert \frac{z}{z_0} \vert^n = M\sum \vert \frac{z}{z_0} \vert^n = M \sum \rho^n, \quad \rho = \vert \frac{z}{z_0}\vert < 1
	\end{equation}
	\begin{equation}
		M \sum \rho^n = \frac{M}{1-\rho} < \infty
	\end{equation}
	
	
	\subsection{Kvotientkriteriet}
	Række $ \sum c_n $. Hvis $ \lim\limits_{n\to \infty} |c_n|^{1/n} <1 $ så er rækken konvergent. 
	
	$ c_n = a_n z^n $, $ \lim\limits_{n\to \infty} |c_n|^{1/n} =\rho < 1$
	\begin{equation}
	\lim\limits_{n\to \infty} |a_n z^n|^{1/\infty} = \lim\limits_{n \to \infty} |a_n|^{1/z} |z| = \rho <1
	\end{equation}
	Rækken er konvergent, så længe
	\begin{equation}
		|z| < \frac{1}{\lim\limits_{n\to \infty} |a_n|^{1/n}}
	\end{equation}
	hvor højreside kaldes konvergentradius $ R $.
		
\end{document}

