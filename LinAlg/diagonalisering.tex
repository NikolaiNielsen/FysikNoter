\documentclass[LinAlgNoter.tex]{subfiles} % HUSK FOR FANDEN AT REDIGERE DENNE LINJE

% Hvis ikke dokumenterne (hoved & under) er i samme mappe, skal den relative stig bruges.

\begin{document}
	
	
	
	\section{Diagonalisering af matricer}
	
	
	
	\subsection{Diagonaliserbare matricer, egenværdier/vektorer}
	For et endeligdimensionelt vektorrum defineres diagonaliserbare matricer som:
	
	\subsubsection*{Diagonaliserbare matricer}
	En lineær afbildning $ \morf{f}{V}{V} $ kaldes \textit{diagonaliserbar}, hvis der findes en basis $ (\V{a}_1,\dots,\V{a}_n)$ således at $ f $ i denne basis repræsenteres som en diagonalmatrix. Hvis dette gælder, så er $ f(\V{a}_1)=\lambda_1 \V{a}_1,\dots,f(\V{a}_n)=\lambda_n \V{a}_n $, hvor $ \lambda_1,\dots,\lambda_n $ er elementerne i diagonalmatricen. $ f(\V{a}_j) $ er altså proportional med $ \V{a}_j $, med porportionalitetsfaktoren $ \lambda_j $.
	
	\subsubsection*{Egenvektorer og -værdier}
	En egenvektor er en vektor, der er forskellig fra 0, og proportionel med dens funktionsværdi: $ f(\V{x})=\lambda \V{x} $. Proportionalitetsfaktoren $ \lambda $ kaldes for den til egenvektoren $ \V{x} $ hørende \textbf{egenværdi}. Dette giver anledning til følgende sætning:
	
	\paragraph{Sætning 6.1.3} Den lineære afbildning $ \morf{f}{V}{V} $ er diagonaliserbar, netop hvis der findes en basis for $ V $ bestående af egenvektorer for $ f $.
	
	\subsubsection*{Egenrum}
	Et egenrum $ V_\lambda $ hørende til $ \lambda\in\Set{F} $ defineres som underrummet
	\begin{equation}
		V_\lambda = \{\V{x}\  |\ f(\V{x})=\lambda \V{x} \}= \{\mat{X}\in \Set{F}^n \ |\  (\mat{A}-\lambda \mat{E})\mat{X}=\mat{0}\}
	\end{equation}
	Og \textbf{egenværdimultipliciteten} eller den \textbf{geometriske multiplicitet} defineres som $ \Em \lambda = \dim V_\lambda $.
	
	Vi lader $ \morf{e}{V}{V} $ betegne den identiske afbildning $ \morf{e}{\V{x}}{\V{x}} $. Vi definerer da afbildningen $ \morf{f-\lambda e}{V}{V} $ som
	\begin{equation}
		(f-\lambda e)(\V{x}) = f(\V{x}) -\lambda e(\V{x}) = f(\V{x})-\lambda \V{x}
	\end{equation}
	Da gælder, per sætning 6.1.5
	\begin{enumerate}
		\item $ V_\lambda = \ker (f-\lambda e) $
		\item $ \Em \lambda + \rg(f-\lambda e)=\dim V= \Em \lambda + \rg(\mat{A}-\lambda \mat{E})=n$
	\end{enumerate}
	Og følgende er ækvivalente (sætning 6.1.6)
	\begin{enumerate}
		\item $ \lambda $ er egenværdi for $ f $
		\item $ V_\lambda \neq \{\V{o}\} $
		\item $ \Em \lambda > 0 $
		\item $ f-\lambda e $ er \textit{ikke} bijektiv
		\item $ \det(f-\lambda e)=0 $
	\end{enumerate}
	
	\subsubsection*{Karakteristiske polynomium}
	Det karakteristiske polynomium $ P_f $ hørende til $ \morf{f}{V}{V} $ defineres som
	\begin{equation}
		P_f(\lambda) = \det (f-\lambda e) = P_{\mat{A}}(\lambda) = \det(\mat{A}-\lambda \mat{E})
	\end{equation}
	\paragraph{Sætning 6.1.8} Hvor egenværdierne for $ f $ er $ \Set{F} $-rødderne i det karakteristiske polynomium (følger af sætning 6.1.6)
	\subsubsection*{Rodmultipliciteter}
	Ved rodmultipliciteter (eller \textbf{algebraisk multiplicitet} forstås tallet $ k $, hvor polynomiet er deleligt med $ (x-x_0)^k $ men ikke $ (x-x_0)^{k+1} $. Altså hvor mange gange et tal er en rod (dobbelt-rod, trippel-rod osv). Rodmultipliciteterne for hver rod $ x_0 $ betegnes $ \Rm x_0 = k $.
	
	\paragraph{Sætning 6.2.2} Lad $ \morf{f}{V}{V} $ være en lineær afbildning og $ V $ være et e.dim vektorrum. For hvert tal $ \lambda_0 \in \Set{F} $ gælder at $ \Em \lambda_0 \leq \Rm \lambda_0 $
	
	\paragraph{Sætning 6.2.3} Hvis summen af rodmultipliciteterne af $ \Set{F} $-rødderne i $ P_f $ for $ f $ er (skarpt) mindre end $ \dim V $, altså hvis der findes en egenværdi $ \lambda_0 $ med $ \Em \lambda_0 < \Rm \lambda_0 $, da er $ f $ \textbf{ikke} diagonaliserbar.
	
	\paragraph{Sætning 6.3.1} Hvis summen af rodmultipliciteterne for $ \Set{F} $-rødderne i $ P_f $ for $ f $ er $ n=\dim V $, og der gælder $ \Em \lambda_0 =\Rm \lambda_0$ for alle egenværdier $ \lambda_0 $, da er $ f $ diagonaliserbar. En diagonaliserende basis fås i så fald, ved at vælge en basis for hvert af egenrummene og sammenstille disse baser.
	
	\paragraph{Sætning 6.3.3} Hvis $ P_f $ for $ f $ har $ n=\dim V $ forskellige $ \Set{F} $-rødder $ \lambda_1,\dots,\lambda_n $, da er $ f $ diagonaliserbar. En diagonaliserende basis $ (\V{a}_1,\dots,\V{a}_n) $ fås i dette tilfælde ved, for hver rod $ \lambda_i $, at vælge en egenvektor $ \V{a}_i $ med egenværdien $ \lambda_i $.
	
	Bemærk at $ f $ godt kan være diagonaliserbar, selvom der ikke er $ n $ rødder (per sætning 6.3.1). Sætning 6.3.3 siger blot, at hvis der er $ n $ rødder, så er $ f $ med sikkerhed diagonaliserbar. 
	
	\subsection{Opskrift på at finde egenvektorer og egenværdier}
	\label{sub:egenvektorogvaerdi}
	\begin{enumerate}
		\item Beregn det karakteristiske polynomium $ P_{\mat{A}}(\lambda) = \det(\mat{A}-\lambda \mat{E}) $
		\item Find $ \Set{F} $-rødderne til $ P_{\mat{A}}(\lambda) $. Dette er \textit{egenværdierne}
		\item For hver egenværdi $ \lambda_0 $ find den fuldstændige løsning til ligningssystemet $ (\mat{A}-\lambda_0 \mat{E})\V{x}=\V{o} $.
	\end{enumerate}
	De fundne løsninger (fraregnet $ \V{o} $-løsningen) er præcist egenvektorerne hørende til egenværdien $ \lambda_0 $.
	
	\subsection{Opskrift på diagonalisering af en kvadratisk matrix}
	\label{sub:diagonalisering}
	\begin{enumerate}
		\item Bestem det karakteristiske polynomium $ P_{\mat{A}}(\lambda)=\det(\mat{A}-\lambda \mat{E}) $, og find dets rødder (egenværdier), samt de tilhørende rodmultipliciteter
		\item Bestem baser for hvert egenrum $ V_{\lambda_0} $ (egenvektorer) samt de tilhørende egenværdimultipliciteter (= antal basiselementer i basen).
		\item Matricen er diagonaliserbar \textit{hvis og kun hvis} $ \Rm \lambda_0 = \Em \lambda_0 $ for \textit{ALLE} $ \lambda_0 $. I så tilfælde kan vi sætte $ \mat{S}\inverse $ til at være den matrix der fremkommer ved at lade basiselementerne fra punkt 2 være søjler (altså en blokmatrix af basiselementer for egenrummene). Der gælder så at $ \mat{S} $ er koordinattransformationen $ \trans{\bas{B}}{\bas{E}} $ fra basis $ \bas{E} $ til $ \bas{B} $, hvor $ \bas{B} $ er basen bestående af egenvektorerne. Ydermere gælder at
		\begin{equation}
			\mat{S}\mat{A}\mat{S}\inverse
		\end{equation}
		er diagonal
	\end{enumerate}
	Punkt 1 og 2 i denne opskrift er egentlig også punkterne i forrige opskrift \ref{sub:egenvektorogvaerdi}. Opskrift \ref{sub:diagonalisering} leder efter \textit{kompleks} diagonalisering. Hvis man kun er interesseret i reel diagonalisering af en reel matrix kan man stoppe efter punkt 1, hvis en af rødderne er ikke-reel. Man kan også stoppe under punkt 2, hvis $ \Rm \lambda_0 \neq \Em \lambda_0 $, for bare ét $ \lambda_0 $. Da er matricen slet ikke diagonaliserbar.
	
	Med fordel kan følgende skema opsættes for diagonalisering af matricer:
	\begin{table}[H]
		\centering
		\begin{tabular}{|c||c|c|c|}
			\hline \rule[-2ex]{0pt}{5.5ex} Egenværdi & Rodmult. & Egenværdimult. & Basis \\ 
			\hline\hline \rule[-2ex]{0pt}{5.5ex} &  &  &  \\ 
			\hline 
		\end{tabular} 
	\end{table}
	Hvor der så indsættes en ekstra række for hver egenværdi.
	
	\subsection{Potensopløftning af matricer}
	Hvis $ \mat{A} $ er diagonaliserbar, og $ \mat{D}=\mat{S}\mat{A}\mat{S}\inverse $ er en diagonalmatrix, så gælder der at
	\begin{align}
		\mat{D}^k&=\mat{S}\mat{A}^k\mat{S}\inverse \\
		\mat{A}^k&=\mat{S}\inverse\mat{D}^k\mat{S}
	\end{align}
	Hvor det kan udnyttes, at $ \mat{D}^k $ er nem at udregne (se afsnit \ref{sub:regneregel})
	
	\subsection{Diagonalisering af reelle symmetriske matricer}
	En reel symmetrisk matrix (hvor $ \mat{A}=\mat{A}^t $, og den altså er $ n\times n $), i et Euklidisk (reelt, indre-produkt rum) vektorrum, er reelt diagonaliserbar.
	
	Følgende gælder for reelle symmetriske matricer og afbildninger
	\begin{itemize}
		\item En lineær afbildning $ \morf{f}{V}{V} $ kaldes symmetrisk, hvis $ f(\V{x})\D \V{y}=\V{x}\D f(\V{y}) $, for alle $ \V{x},\V{y}\in V $, og denne repræsenteres, i en ortonormalbasis, af en symmetrisk matrix
		\item Hvis $ \V{x} $ og $ \V{y} $ er egenvektorer til den samme \textit{symmetriske} lineære afbildning, hørende til forskellige egenværdier, er $ \V{x} $ og $ \V{y} $ ortogonale.
		\item Det karakteristiske polynomium for en symmetrisk matrix har (mindst) én reel rod.
		\item En reel symmetrisk $ n\times n $ matrix er \textit{reelt} diagonaliserbar, og der findes en ortogonal, reel $ n\times n$-matrix $ \mat{S} $, så $ \mat{D}=\mat{S}\mat{A}\mat{S}\inverse $ er en diagonalmatrix.
		\item Ligeledes: hvis $ \mat{A} $ er diagonaliserbar m.h.t en reel, ortogonal matrix $ \mat{S} $, så er $ \mat{A} $ symmetrisk. 
	\end{itemize}
	
	
\end{document}