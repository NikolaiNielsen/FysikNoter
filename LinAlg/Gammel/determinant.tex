\documentclass[LinAlgNoter.tex]{subfiles} % HUSK FOR FANDEN AT REDIGERE DENNE LINJE

% Hvis ikke dokumenterne (hoved & under) er i samme mappe, skal den relative stig bruges.



\begin{document}
	
	\section{Determinanter}
	En determinant er et tal, der knyttes til enhver kvadratisk matrix. Notationen er som følger. For en $ n\times n $-matrix $ \mat{A} $ er determinanten $ \det(\mat{A}) $ skrevet som:
	\begin{equation}
		\det(\mat{A}) = |\mat{A}|
	\end{equation}
	Determinanten for en $ n\times n $-matrix har $ n! $ led, hver med $ n $ faktorer.
	
	\subsection{Determinanter for $ 2\times2 $ og $ 3\times3 $ matricer}
	For en $ 2\times 2 $-matrix $ \mat{A} $ er determinanten
	
	\begin{align*}
		\mat{A}&= \begin{pmatrix}
		a_{11} & a_{12} \\
		a_{21} & a_{22}
		\end{pmatrix}, \quad \det(\mat{A})=\begin{vmatrix}
		a_{11} & a_{12} \\
		a_{21} & a_{22}
		\end{vmatrix} = a_{11} a_{22}-a_{12} a_{21} \\
	\end{align*}
	
	Og for $ 3\times 3 $-matricen $ \mat{A} $ er determinanten
	\begin{align*}
	\mat{A} &=\begin{pmatrix}
	a & b & c \\
	d & e & f \\
	g & g & i
	\end{pmatrix}, \quad \det(\mat{A})=|\mat{A}|=\begin{vmatrix}
	a & b & c \\
	d & e & f \\
	g & g & i
	\end{vmatrix} \\
	|\mat{A}|&= aei+bfg+cdh-gec-hfa-idb
	\end{align*}
	Denne kan også findes ved hjælp af pilereglen, hvor værdierne skrives op, og de to første søjler skrives igen, denne gang efter den sidste søjle.
	\begin{figure}[H]
		\centering
		\includegraphics[width=5cm]{img/pilereglen}
	\end{figure}
	De tre første pile lægges til, og de tre sidste pile trækkes da fra.
	
	
	\subsection{Determinant af $ n\times n $-matrix}
	Definitionen på en determinant for en $ n\times n $ matrix $ \mat{A} $ er
	\begin{equation}
		\det \mat{A}= \sum_{\sigma\in S_n} \sign(\sigma) a_{1\sigma(1)}a_{2\sigma(2)}\dots a_{n\sigma(n)}
	\end{equation}
	Denne bruges i praksis ikke, da den bruger permutationer, som ikke er en del af pensum.
	\subsubsection{Række/søjle-operationer}
	For en $ n\times n $-matrix $ \mat{A} $, hvor matricen $ \mat{B} $ fremkommer ved en række- eller søjleoperation på $ \mat{A} $ gælder følgende:
	\begin{itemize}
		\item Type M (multiplikation med skalar $ c $): $ \det \mat{B} =c \det \mat{A} $
		\item Type B (ombytning af to rækker/søjler): $ \det \mat{B}= - \det \mat{A} $
		\item Type S (addering af række/søjle, multipliceret med skalar $ c $, til en anden række/søjle): $ \det \mat{B} =\det \mat{A}$
	\end{itemize}
	
	
	\subsubsection{Udvikling af determinant}
	Udviklingen af en determinant for matricen $ \mat{A} $ er givet ved
	\begin{align*}
		\det \mat{A} &= a_{i1}A_{i1}+\dots+a_{in}A_{in}=(-1)^{i+1} a_{i1}\det \mat{A}_{i1}+\dots +(-1)^{i+n} a_{in}\det \mat{A}_{in} \\
		&= a_{1i}A_{1i}+\dots+a_{ni}A_{ni}=(-1)^{1+i} a_{1i}\det \mat{A}_{1i}+\dots +(-1)^{n+i} a_{ni}\det \mat{A}_{ni}
	\end{align*}
	Hvor $ 1\leq i \leq n $ og hvor $ \mat{A}_{ij} $ er den $ n-1\times n-1 $-matrix der fremkommer ved at slette den $ i $-te \textit{række} og den $ j $-te \textit{søjle}. $ A_{ij} $ kaldes for \textit{komplementet} til $ a_{ij} $ og er givet ved $ A_{ij} = (-1)^{i+j} \det \mat{A}_{ij} $.
	
	Denne metode kaldes for udviklingen af den $ i $-te række (første linje) eller $ i $-te søjle (anden linje)
	
	\subsubsection{Opskrift på beregning determinant}
	For at udregne determinanten $ \det \mat{A} $ bruger man i praksis følgende opskrift
	\begin{enumerate}
		\item Lav række/søjleoperationer på $ \det\mat{A} $, så der kommer flere 0'er i en række eller søjle (og hold øje med, om determinantens værdi ændres, hvis der for eksempel bruges operationer af typen M eller B)
		\item Uregn den nye determinant ved brug af udvikling af determinanten efter en række eller søjle, der har mange 0'er. 
	\end{enumerate}
	Man kan godt komme ud for at skulle bruge denne opskrift flere gange, før den endelige værdi er fundet.
	
	\subsection{Egenskaber ved determinanter}
	Determinanter har følgende egenskaber
	\begin{itemize}
		\item $ \det (\mat{A})=\det(\mat{A}^t) $
		\item For en trekantsmatrix (\textbf{øvre}, såvel som \textbf{nedre}, og \textbf{diagonalmatricer}) er $ \det(\mat{A}) $ lig produktet af diagonalelementerne.
		\item for to $ n \times n $-matricer $ \mat{A} $ og $ \mat{B} $ gælder: $ \det(\mat{A}\,\mat{B})=\det\mat{A} \det \mat{B}$
		\item En $ n \times n $ matrix er \textbf{regulær}, hvis $ \det\mat{A}\neq 0 $
		\item For en regulær matrix gælder: $ \det(\mat{A}\inverse) = \frac{1}{\det \mat{A}} $
	\end{itemize}
	
	
	\subsection{Invers matrix og determinant}
	For en $ n\times n $-matrix $ \mat{A} $ forstås \textbf{komplementet} til indgangen $ a_{ij} $ som
	\begin{equation}
		A_{ij} = \begin{blockarray}{cccccc}
		\begin{block}{|ccccc|c}
		a_{11}	& \dots	& 0			& \dots	& a_{1n}	& \\
		\vdots	& 		& \vdots	&		& \vdots	&\\
		0		& \dots & 1			& \dots	& 0 		& \leftarrow i\\
		\vdots	& 		& \vdots	&		& \vdots	&\\
		a_{n1}	& \dots	& 0			& \dots	& a_{nn}	&\\
		\end{block}
				&		& \uparrow	&		&			&\\
				&		& j			&		&			&\\		
		\end{blockarray}
	\end{equation}
	Denne kan også udregnes ved
	\begin{equation}
		A_{ij} = A_{ij} = (-1)^{i+j} \det \mat{A}_{ij}
	\end{equation}
	hvor $ \mat{A}_{ij} $ er den $ n-1\times n-1 $-matrix, der fremkommer ved at slette den $ i $-te række og den $ j $-te søjle i $ \mat{A} $.
	
	\textbf{Komplementmatricen} $ K(\mat{A}) $ er
	\begin{equation}
		K(\mat{A})=\begin{pmatrix}
		A_{11}	& \dots	& A_{1n} \\
		\vdots 	&		& \vdots \\
		A_{n1}	& \dots & A_{nn}
		\end{pmatrix}
	\end{equation}
	Og for denne gælder
	\begin{equation}
		\mat{A}K(\mat{A})^t=(\det\mat{A})\mat{E}
	\end{equation}
	For en regulær matrix $ \mat{A} $ gælder
	\begin{equation}
		\mat{A}\inverse[i,j] = \frac{A_{ji}}{\det\mat{A}}=\frac{(-1)^{i+j} \det \mat{A}_{ji}}{\det \mat{A}}
	\end{equation}
	Den inverse matrix er altså givet på formen:
	\begin{equation}
		\mat{A}\inverse = \frac{1}{\det \mat{A}}K(\mat{A})^t
	\end{equation}
\end{document}

