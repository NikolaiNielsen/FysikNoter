\documentclass[LinAlgNoter.tex]{subfiles} % HUSK FOR FANDEN AT REDIGERE DENNE LINJE

% Hvis ikke dokumenterne (hoved & under) er i samme mappe, skal den relative stig bruges.



\begin{document}
	
	\section{Vektorrum}
	\subsection{Definition}
	Et vektorrum er en mængde $ V $, hvori der er givet to operationer: addition af to elementer og multiplikation med en skalar. For $ \U{x},\U{y}\in V $ og $ \lambda \in \Set{F} $ gælder
	\begin{equation}
		\lambda \U{x} \in V, \qquad \U{x}+\U{y}\in V
	\end{equation}
	Ud fra disse fås (som ved regneregler for vektorer og matricer):
	\begin{enumerate}[{V}1:]
		\item $ (\U{x}+\U{y})+\U{z}=\U{x}+(\U{y}+\U{z}) $
		\item $ \U{x}+\U{o}=\U{x} $
		\item $ \U{x}+(-\U{x})=\U{o} $
		\item $ \U{x}+\U{y}=\U{y}+\U{x} $
		\item $ \lambda(\U{x}+\U{y})=\lambda\U{x}+\lambda\U{y} $
		\item $ (\lambda+\mu)\U{x}=\lambda\U{x}+\mu\U{x} $
		\item $ (\lambda\mu)\U{x}=\lambda(\mu\U{x}) $
		\item $ 1\U{x}=\U{x} $
	\end{enumerate}
	Vi definerer $ \Set{F}^0 = \{0\}$, altså at denne mængde kun indeholder nul-vektoren. Vi siger også at $ V $ er defineret over talrummet $ \Set{F} $. Talrummene $ \Set{F}^n, n\in \Set{N}_0 $ er vektorrum over $ \Set{F} $. Hvis $ \Set{F} =\Set{R} $ kaldes vektorrummet reelt, og hvis $ \Set{F}=\Set{C} $ kaldes vektorrummet komplekst. $ \Set{M}_{m,n}(\Set{F}) $ er mængden af $ m\times n $-matricer med indgange i $ \Set{F} $, og er ligeledes et vektorrum. $ \Pol(\Set{F}) $ er mængden af polynomier med én variabel og koefficienter i $ \Set{F} $. Altså polynomier på formen
	\begin{equation}
		p(x)=a_0+a_1x+a_2x^2+\dots+a_nx^n,\ x\in\Set{F}
	\end{equation}
	
	
	\subsection{Lineære afbildninger og isomorfi}
	Vi kalder $ f $ en lineær afbildning fra $ U $ til $ V $, hvis linearitetsbetingelserne er opfyldt:
	\begin{enumerate}[{L}1:]
		\item $ f(\lambda\U{x})=\lambda f(\U{x}) $ for alle $ \U{x}\in U, \, \lambda\in \Set{F} $
		\item $ f(\U{x}+\U{y})=f(\U{x})+f(\U{y}) $ for alle $ \U{x},\U{y} \in U $
	\end{enumerate}
	En lineær afbildning kaldes også for en \textit{homomorfi}, og der gælder, at hvis $ \morf{f}{W}{V} $ og $ \morf{g}{U}{W} $ er lineære afbildninger, da er $ \morf{f\circ g}{U}{V} $ også lineær.
	
	For en afbildning $ \morf{f}{\Set{F}^n}{V} $ givet ved
	\begin{equation}
		f\begin{pmatrix}
		x_1 \\ \vdots \\ v_n
		\end{pmatrix}= x_1 \U{a}_1 + \dots + x_n \U{a}_n
	\end{equation}
	hvor $ \U{a}_1,\dots,\U{a}_n $ er vektorer i $ V $, er denne lineær, og alle lineære afbildninger $ \morf{f}{\Set{F}^n}{V} $ har denne form. Yderligere gælder at $ \U{a}_j=f(\U{e}_j) $.
	
	En bijektiv, lineær afbildning $ \morf{f}{U}{V} $ kaldes for en \textbf{isomorfi} fra $ U $ til $ V $, og de to vektorrum $ U $ og $ V $ kaldes \textbf{isomorfe}, hvis der finde blot én isomorfi fra $ U $ til $ V $. For isomorfier gælder:
	
	\begin{itemize}
		\item Den identiske afbildning $ \morf{\id_U}{U}{U} $ er en isomorfi
		\item Hvis $ \morf{f}{U}{V} $ er en isomorfi gælder at $ \morf{f\inverse}{V}{U} $ er en isomorfi
		\item Hvis $ \morf{f}{W}{V} $ og $ \morf{g}{U}{W} $ er isomorfier, da er $ \morf{f\circ g}{U}{V} $ en isomorfi.
	\end{itemize}

	\subsection{Endeligdimensionalitet, basis, underrum, span, kerne}
	
	Et vektorrum kaldes \textbf{endeligdimensionalt}, hvis det er isomorft med et talrum $ \Set{F}^n, n\in \Set{N}_0 $. Dimensionen $ \dim V $ defineres da til at være $ n $. Og hvis $ \Set{F}^m $ er isomorft med $ \Set{F}^n $, da er $ m=n $.
	
	En \textbf{basis} for et vektorrum $ V $ er et ordnet sæt vektorer, $ \U{a}_1,\dots,\U{a}_n $, hvor hver vektor $\U{a}$ i $ V $ kan skrives som en linearkombination af basisvektorerne:
	\begin{equation}
		\U{a}=x_1 \U{a}_1+\dots + x_n \U{a}_n
	\end{equation}
	Basen betegnes $ \mathcal{A}=(\U{a}_1, \dots , \U{a}_n) $ og koordinaterne $ x_1,\dots,x_n $ for vektoren $ \U{a} $ \textit{med hensyn til $ \mathcal{A} $}, betegnes $ \mathcal{A}[\U{a}]=(x_1,\dots,x_n) $. Bemærk, at den $ j $'te basisvektors ($ \U{a}_j $) koordinater i basen $ \mathcal{A} $ er den $ j $'te standardenhedsvektor, altså $ \mathcal{A}[\U{a}_j]=\U{e}_j $.
	
	\begin{itemize}
		\item Den lineære afbildning $ \morf{f}{\Set{F}^n}{V} $ givet ved
		\begin{equation}
		f \begin{pmatrix} x_1 \\ \vdots \\ x_n \end{pmatrix} = x_1 \U{a}_1+\dots + x_n \U{a}_n
		\end{equation}
		er en isomorf, hvis $ \U{a}_1,\dots,\U{a}_n $ er en basis.
		\item Et vektorrum $ V \neq \{\U{o}\} $ har en basis, netop hvis det er endeligdimensionalt. I så fald indeholder enhver basis for $ V $ netop $ \dim V $ elementer.
		\item et sæt $ \U{a}_1,\dots,\U{a}_n $ af $ n $ vektorer i $ \Set{F}^n $ er en basis hvis $ n\times n $-matricen $ \mat{A}=(\U{a}_1 \ \dots \ \U{a}_n) $ er regulær
		\item For en lineær afbildning $ \morf{f}{U}{V} $ hvor $ \dim U = \dim V $ er følgende tre udsagn ækvivalente: $ f $  er surjektiv, injektiv og bijektiv.
	\end{itemize}
	
	En ikke-tom delmængde $ U \subseteq V $ kaldes for et \textbf{underrum}, hvis følgende to betingelser er opfyldt:
	\begin{enumerate}[{U}1:]
		\item Hvis $ \U{x}\in U, \ \lambda \in \Set{F} $ gælder $ \lambda \U{x} \in U $.
		\item Hvis $ \U{x},\U{y} \in U $ gælder $ \U{x}+\U{y} \in U $.
	\end{enumerate}
	Nulvektoren $ \U{o} $ er da altid i et underrum, og et underrum kaldes stabilt ved vektoroperationerne, hvis U1 og U2 er opfyldt. V1-V8 gælder da også for underrum, da disse i sig selv er vektorrum. Ligeledes kan der bestemmes dimension og baser af $ U $. $ V $ og $ \{\U{o}\} $ kaldes for \textit{trivielle} underrum af $ V $.
	
	For en ikke-tom delmængde $ M $ af $ V $ forstås \textbf{span} som mængden af alle linearkombinationer af vektorer i $ M $. Dette betegnes $ \Span M $. Hvis $ \Span M = M $ er $ M $ et underrum.
	
	For en lineær afbildning $ \morf{f}{U}{V} $ forstås \textbf{kernen} $ \ker f $ som mængden
	\begin{equation}
		\ker f = \{\U{x} \in U | f(\U{x}) = \U{o} \}
	\end{equation}
	Og ved \textbf{billedet} $ f(U) $ for $ f $ forstås mængden
	\begin{equation}
		f(U)=\{\U{y}\in V | \U{y} = f(\U{x}) \text{ for et } \U{x} \in U\}
	\end{equation}
		
	\begin{itemize}
		\item Kernen $ \ker f $ for den lineære afbildning $ \morf{f}{U}{V} $ er et underrum i $ U $.
		\item Billedet $ f(U) $ af $ U $ ved $ f $ er et underrum i $ V $.
		\item Den lineære afbildning $ f $ er injektiv når $ \ker f = {\U{o}} $.
		\item Hvis den lineære afbildning er givet ved matricen $ \mat{A} $, er billedet $ f(U)= \Span\{\U{a}_1,\dots,\U{a}_n\} $
	\end{itemize} 
		
	
	\subsection{Lineær (u)afhængighed}
	Et sæt vektorer $ \U{a}_1,\dots,\U{a}_n $ i $ V $ kaldes \textbf{lineært uafhængige}, hvis ligningen
	\begin{equation}
		x_1\U{a}_1+\dots+x_n\U{a}_n=\U{o}
	\end{equation}
	kun har løsningen $ (x_1,\dots,x_n)=(0,\dots,0) $. Ellers kaldes sættet \textbf{lineært afhængigt}
	\begin{itemize}
		\item En basis for $ V $, eller et underrum heraf, består af lineært uafhængige vektorer.
		\item Et sæt af vektorer $ \U{a}_1,\dots,\U{a}_n $ i $ V $ er lineært afhængige, hvis blot én vektor kan skrives som en linearkombination af de andre.
		\item $ \{\U{a}_1\} $ er lineært uafhængig, hvis $ \U{a}_1 \neq \U{o} $.
		\item $ \{\U{a}_1,\U{a}_2\} $ er lineært uafhængie hvis $ \U{a}_1 \neq \lambda \U{a}_2 $. 
		\item Et sæt af vektorer $ \U{a}_1,\dots,\U{a}_n $ i $ V $ er lineært uafhængige, hvis den lineære afbildning $ \morf{f}{\Set{F}^n}{V} $, givet ved linearkombinationen af $ \U{a}_1,\dots,\U{a}_n $, er injektiv.
		\item Et sæt af vektorer $ \U{a}_1,\dots,\U{a}_n $ i $ V $ er lineært uafhængige, hvis de udgør en basis for $ \Span\{\U{a}_1,\dots,\U{a}_n\} $
		\item for et endeligdimensionalt vektorrum $ V $, med et sæt lineært uafhængige vektorer $ \U{a}_1,\dots,\U{a}_n $. Da er $ n \leq \dim V $, og $ n=\dim V $ når $ \U{a}_1,\dots,\U{a}_n $ er en basis for $ V $.
		\item For et endeligdimensionalt vektorrum $ V $, med et sæt vektorer $ \U{a}_1,\dots,\U{a}_n $, der frembringer $ V $; er $ n \geq \dim V $ og hvis $ n= \dim V $ er $ \U{a}_1,\dots,\U{a}_n $ en basis for $ V $.
		\item Et sæt lineært uafhængige vektorer $ \U{a}_1,\dots,\U{a}_n $ i $ V $ og en yderligere vektor $ \U{a}_{n+1} $ er lineært afhængige, hvis $ \U{a}_{n+1} \in \Span\{\U{a}_1,\dots,\U{a}_n\}$.
	\end{itemize}
	
	Ved et \textbf{maksimalt lineært uafhængige} sæt vektorer forstås et sæt lineært uafhængige vektorer $ \U{a}_1,\dots,\U{a}_n \in M \subseteq V $, der ikke kan udvides til et større lineært uafhængigt sæt $ \U{a}_1,\dots,\U{a}_n,\U{a}_{n+1} \in M$.
	
	\begin{itemize}
		\item Ethvert lineært uafhængigt sæt vektorer i $ M $ (der er en delmængde af det endeligdimensionale vektorrum $ V $) kan udvides til et maksimalt lineært uafhængigt sæt af vektorer fra $ M $.
		\item Ethvert maksimalt lineært uafhængigt sæt vektorer fra $ M $ (delmængde af e.dim rum $ V $) udgør en basis for $ \Span M $.
		\item Et underrum $ U $ af et endeligdimensionalt vektorrum $ V $ er endeligdimensionalt, og $ \dim U \leq \dim V $.
		\item Et lineært uafhængigt sæt vektorer i det e.dim rum $ V $ kan udvides til en basis for $ V $.
	\end{itemize}
	
	\subsection{Udtyndings-/udvidelsesalgoritmen}
	Udtyndingsalgoritmen er en måde at reducere et sæt lineært afhængige vektorer til et sæt lineært uafhængige vektorer, mens udvidelsesalgoritmen er en metode til at udvide et sæt lineært uafhængige vektorer til et sæt maksimalt lineært uafhængige vektorer, og dermed en basis for et vektorrum. Givet et sæt vektorer $ \U{a}_1,\dots,\U{a}_n \in \Set{F}^m$, der enten er lineært uafhængige (udvidelsesalgoritmen) eller lineært afhængige (udtyndningsalgoritmen) er algoritmerne som følger:	
	
	\subsubsection*{Udtyndingsalgoritmen}
	\begin{enumerate}
		\item Lav matricen $ \mat{A}=(\U{a}_1 \ \dots \ \U{a}_n) $, der altså har de givne vektorer som søjler.
		\item Rækkereducer $ \mat{A} $ til en trappematrix $ \mat{B} $
		\item Vælg de søjler i $ \mat{A} $, hvor der er trinpositioner i $ \mat{B} $ (eksempelvis: trin i $ \mat{B} $, i søjle 1, 2 og 4 $ \rightarrow $ $ \U{a}_1, \U{a}_2,\U{a}_4 $ er lineært uafhængige, og udgør en basis for $ \Span\{\U{a}_1, \U{a}_2,\U{a}_4\} $).
	\end{enumerate}
	
	\subsubsection*{Udvidelsesalgoritmen}
	\begin{enumerate}
		\item Lav matricen $ \mat{A}= (\U{a}_1 \ \dots \ \U{a}_n \ \mat{E}) $, der har sættet af lineært uafhængige vektorer som de første $ n $ søjler, og $ m\times m $-enhedsmatricen $ \mat{E} $ som de næste søjler.
		\item Reducer $ \mat{A} $ til trappematricen $ \mat{B} $.
		\item Vælg de søjler i $ \mat{A} $, hvor der er trinpositioner i $ \mat{B} $. De valgte søjlevektorer ($ \U{a}_1,\dots,\U{a}_n $, samt et sæt af standardenhedsvektorerne) udgør da en basis for $ \Set{F}^m $
	\end{enumerate}
	
	
	
	\subsection{Rang og dimensionssætningen}
	For to endeligdimensionale vektorrum $ U $ og $ V $ forstås \textbf{rangen} $ \rg f $ af en lineær afbildning $ \morf{f}{U}{V} $ som dimensionen af billedrummet $ f(U) $:
	\begin{equation}
		\rg f = \dim f(U)
	\end{equation}
	Og ved rangen $ \rg \mat{A} $ af en $ m\times n $-matrix $ \mat{A} $ forstås rangen af den til $ \mat{A} $ hørende afbildning $ \morffFF{m}{n} $
	\begin{equation}
		\rg \mat{A} = \dim (\Span \{\U{a}_1 , \dots , \U{a}_n \})
	\end{equation}
	\subsubsection*{Dimensionssætningen}
	For en lineær afbildning $ \morf{f}{U}{V} $ gælder
	\begin{equation}
		\rg f + \dim (\ker f) = \dim U
	\end{equation}
	
	\subsubsection*{Egenskaber for rang}
	\begin{itemize}
		\item Rangen af $ m\times n $-matricen $ \mat{A}=(\U{a}_1 \ \dots \ \U{a}_n) $ er lig antallet af vektorer i et maksimalt lineært uafhængigt sæt af vektorer fra $ \{\U{a}_1 , \dots , \U{a}_n \} $
		\item Rangen af en trappematrix $ \mat{A} $ er lig antallet af trin i denne.
		\item Rangen af en matrix ændres ikke ved udførelse af række-/søjle-operationer
		\item Rangen af en matrix er den samme som rangen af dens transponerede matrix: $ \rg \mat{A}  = \rg \mat{A}^t$
	\end{itemize}
	
\end{document}