\documentclass[LinAlgNoter.tex]{subfiles} % HUSK FOR FANDEN AT REDIGERE DENNE LINJE

% Hvis ikke dokumenterne (hoved & under) er i samme mappe, skal den relative stig bruges.



\begin{document}
	
	\section{Introduktion, sprogbrug \& notation}
	
	
	\subsection{Talrum ($ \Set{R} $ \& $ \Set{C} $)}
	Disse noter omhandler lineær algebra, og derfor også $ n $-dimensionale rum. Mængden $ \Set{R} $ er de reelle tal, mens $ \Set{C} $ er de komplekse tal. Da reelle tal er et undersæt at komplekse tal (den imaginære del er bare lig 0), er $ \Set{R} \subset \Set{C} $. I denne bog, og dermed også disse noter, defineres mængden $ \Set{F} $ til enten at være de komplekse tal $ \Set{C} $ eller $ \Set{R} $. Dette betyder egentlig, at $ \Set{F} $ altid indeholder de reelle tal, men den bruges til at specificere, at vi kan både bruge reelle og komplekse værdier i en operation, mens $ \Set{C} $ bruges til at specificere, at vi med sikkerhed bruger komplekse værdier i denne operation. Mængden $ \Set{F} $ bruges da i definitioner og sætninger, for at specificere, at det er lige meget, hvilket af de to rum, man arbejder med.
	
	Yderligere betegner $ n $ i $ \Set{F}^n $ antallet af dimensioner, og altså antallet af værdier ethvert koordinatsæt indeholder. Eksempelvis er $ \Set{R}^2 $ realplanen, hvor der normalt er $ x $ og $ y $ koordinater, hvor $ \Set{C}^3 $ er det 3-dimensionale, komplekse rum.
	
	
	\subsection{Afbildninger}
	Ved en \textbf{afbildning} af en mængde $ X $, ind i en mængde $ Y $ (eller en \textbf{funktion} fra $ X $ til $ Y $) forstås en forskrift, hvorved der til hvert element $ x \in X $ knyttes et element $ y \in Y $. $ X $ og $ Y $ kan altså være enhver mængde, eksempelvis kendes funktioner af én variabel, hvor $ X=\Set{R} $ og $ Y\Set{R} $,	hvis funktionen er ubegrænset og defineret i hele $ \Set{R} $. Mængden $ X $ kaldes for 
	\textbf{definitionsmængden}, mens $ Y $ kaldes for \textbf{dispositionsmængden}.
	
	Normalt betegnes en afbildning (som en funktion) ved ét bogstav. Notationen er:
	
	\begin{equation*}
		\morf{f}{X}{Y}
	\end{equation*}
	
	Elementet tilsvarende $ x $ kaldes $ f(x) $ og kaldes \textbf{billedet} af $ x $ ved $ f $ eller den tilhørende \textbf{funktionsværdi}. At $ f(x) $ svarer til $ x $ skrives som
	\begin{equation*}
		x \to f(x)
	\end{equation*}
	Der kan også tages et billed af en hel delmængde af $ X $. Eksempelvis udgør billederne for alle $ x \in A \subseteq X $ en delmængde af $ Y $, der kaldes billedet af $ A $ ved $ f $, og betegnes $ f(A) $. Dette skrives:
	\begin{equation*}
		f(A)=\{f(x) \ | \ x\in A\}
	\end{equation*}
	Billedet af \textit{hele} mængden $ X $ (kaldet $ f(X) $) kaldes for \textbf{billedmængden} eller \textbf{værdimængden} for $ f $. Dette kendes også fra funktioner for én variabel, hvor værdimængden er mængden af alle $ y $-værdier, mens dispositionsmængden (som ikke altid er den samme), for det meste er hele $ \Set{R} $. Altså gælder:
	\begin{equation*}
		f(X) \subseteq Y
	\end{equation*}
	
	En afbildning $ \morf{f}{X}{Y} $ kaldes \textbf{surjektiv} eller en afbildning af $ X $ \textbf{på} $ Y $, hvis $ f(X)=Y $, altså at værdimængden er lig hele dispositionsmængden. Afbildningen kaldes \textbf{injektiv}, hvis der for vilkårlige valg af $ x $-værdier er forskellige funktionsværdier, altså at $ f(x_1)=f(x_2) $ kun hvis $ x_1 = x_2 $. Til sidst kaldes en afbildning \textbf{bijektiv}, hvis denne både er surjektiv og injektiv, og hvis der altså til ethvert billed $ y\in Y $ findes ét og kun ét $ x \in X $.
	
	Disse egenskaber kan også beskrives, hvis der for en afbildning $ \morf{f}{X}{Y} $ er givet et $ x\in X $ og $ y \in Y $, og lignignen $ f(x)=y $ betragtes. Da gælder det:
	\begin{enumerate}
		\item Ligningen har \textit{mindst} en løsning for hvert $ y $, hvis $ f $ er surjektiv
		\item Ligningen har \textit{højst} en løsning for hvert $ y $, hvis $ f $ er injektiv
		\item Ligningen har \textit{kun} en løsning for hvert $ y $, hvis $ f $ er bijektiv
	\end{enumerate}
	
	Til bijektive afbildninger $ \morf{f}{X}{Y} $ hører den \textbf{omvendte} eller \textbf{inverse} afbildning $ \morf{f\inverse}{Y}{X} $. Denne afbildning er også bijektiv, og der hører et $ x\in X $ til ethvert billed af $ y\in Y $: $ f\inverse(y)=x $. Yderligere gælder $ (f\inverse)\inverse=f $.
	
	\textbf{Sammensatte afbildninger} er som følger: lad $ X,Y,Z $ være mængder og lad $ \morf{f}{X}{Y} $ og $ \morf{g}{Y}{Z} $ være afbildninger. Deres sammensatte afbildning $ \morf{f \circ g}{X}{Z} $ knytter et element $ g(f(x)) \in Z $ til hvert element $ x \in X $. For sammensatte afbildninger gælder den associative regel: Lad $ X,Y,Z,W $ være mængder, og lad $ \morf{f}{X}{Y}, \ \morf{g}{Y}{Z}, \ \morf{h}{Z}{W} $. Da gælder:
	\begin{equation*}
		h \circ (g\circ f) = (h \circ g) \circ f = h\circ g \circ f
	\end{equation*}
	Yderligere gælder det, at hvis både $ f $ og $ g $ er enten surjektive, injektive eller bijektive, så er den sammensatte afbildning $ g\circ f $ også henholdsvis surjektiv, injektiv eller bijektiv. Og for bijektive, sammensatte afbildninger gælder at $ (g\circ f)\inverse = g\inverse \circ f\inverse $
	
	\textbf{Identiske afbildninger} er afbildninger $ \morf{\id_X}{X}{X} $ af en mængde $ X $ ind i sig selv, hvor billedet af $ x\in X $ er lig $ x \in X $. Altså $ \id_X(x)=x $. For enhver bijektiv afbildning $ \morf{f}{X}{Y} $ gælder at $ f\inverse \circ f = \id_X$ og $ f\circ f\inverse=\id_Y $.
	
	Hvis $ \morf{f}{X}{Y} $ og $ \morf{g}{Y}{X} $ er afbildninger så $ g\circ f=\id_X $ og $ f\circ g = \id_Y $, da er begge afbildninger bijektive, og $ f\inverse=g, \ g\inverse=f $.
	
	
	
	\subsection{Vektorer}
	Vektorer er en konstruktion i $ \Set{F}^n $, der har en størrelse og retning i $ n $ dimensioner. Eksempelvis kendes vektorer i planen som $ \Set{R}^2 $. Disse skrives normalt som:
	\begin{equation*}
		\vec{a}=\begin{pmatrix}
		x \\ y
		\end{pmatrix}=\Bf{a}
	\end{equation*}
	I denne bog, og i disse noter bruges dog en anden notation:
	\begin{equation*}
		\U{x}=\begin{pmatrix}
		x_1 \\ x_2
		\end{pmatrix}
	\end{equation*}
	Her noteres en vektor altså med en enkelt understregning, mens de enkelte værdier $ x_1 $ og $ x_2 $ viser hvilken dimension/række de tilhører. Dette muliggører nem udvidelse til mere end 3 dimensioner. Generelt er vektorformen for en vektor $ \U{x} $ i $ \Set{F}^n $
	\begin{equation*}
		\U{x}=\begin{pmatrix}
		x_1 \\ \vdots \\ x_n
		\end{pmatrix}
	\end{equation*}
	De mest almindelige operationer er skalering af vektorer, eller multiplicering med en skalar, og addition/subraktion. Skalering er at gange en vektor $ \U{x} $ i $ \Set{F}^n $ med en skalar (et tal i $ \Set{F} $):
	\begin{equation*}
		a\D \U{x}=\begin{pmatrix}
		ax_1 \\ \vdots \\ ax_n
		\end{pmatrix}
	\end{equation*}
	En negativ vektor $ -\U{x} $ er så en vektor ganget med $ -1 $:
	\begin{equation*}
	-\U{x}=(-1)\U{x}=\begin{pmatrix}
		-x_1 \\ \vdots \\ -x_n
	\end{pmatrix}
	\end{equation*}
	Addition defineres som:
	\begin{equation*}
		\U{x}+\U{y}=\begin{pmatrix}
		x_1+y_1 \\ \vdots \\ x_n+y_n
		\end{pmatrix}
	\end{equation*}
	Og subtraktion:
	\begin{equation*}
		\U{x}-\U{y}=\U{x}+(-\U{y})=\begin{pmatrix}
		x_1-y_1 \\ \vdots \\ x_n-y_n
		\end{pmatrix}
	\end{equation*}
	Et udtryk på formen $ a\U{x}+b\U{y} $ kaldes en \textbf{linearkombination} af $ \U{x} $ og $ \U{y} $. Til sidst, inden regnereglerne for vektorer opskrives, defineres nulvektoren $ \U{o} $ i $ \Set{F}^n $ som:
	\begin{equation*}
		\U{o}=\begin{pmatrix}
			0 \\ \vdots \\ 0
		\end{pmatrix}
	\end{equation*}
	(altså med bogstavet ``o'')
	
	
	
	\subsubsection{Regneregler}
	Regnereglerne for vektorer er således:
	\begin{enumerate}[{V}1:]
		\item $ (\U{x}+\U{y})+\U{z}=\U{x}+(\U{y}+\U{z}) $
		\item $ \U{x}+\U{o}=\U{x} $
		\item $ \U{x}+(-\U{x})=\U{o} $
		\item $ \U{x}+\U{y}=\U{y}+\U{x} $
		\item $ a(\U{x}+\U{y})=a\U{x}+a\U{y} $
		\item $ (a+b)\U{x}=a\U{x}+b\U{x} $
		\item $ (ab)\U{x}=a(b\U{x}) $
		\item $ 1\U{x}=\U{x} $
	\end{enumerate}
	\subsubsection*{Skalarprodukt}
	I $ \Set{F}^n $ defineres skalarproduktet $ \U{x}\D \U{y} $:
	\begin{equation*}
		\U{x}\D\U{y}=x_1\konj{y_1}+\dots+x_n\konj{y_n}=\sum_{j=1}^{n}x_j\konj{y_j}
	\end{equation*}
	Hvor $ \konj{y} $ er den kompleks konjugerede af $ y $ (altså $ y $ med omvendt imaginærværdi). Dette gør da, at hvis $ y \in \Set{R} $, er $ \konj{y} = y $, og dermed bliver skalarproduktet det velkendte $ x_1+y_1+\dots+x_n+y_n $. Regnereglerne for skalarprodukter er således:
	\begin{enumerate}[{S}1:]
		\item $ (\U{x}+\U{y})\D \U{z}=\U{x}\D \U{z}+\U{y}\D\U{z} $
		\item $ (a\U{x})\D\U{y}=a(\U{x}\D\U{y}) $
		\item $ \U{x}\D\U{y}=\konj{\U{x}\D\U{y}} $
		\item $ \U{x}\D\U{x}\geq 0 $
		\item $ \U{x}\D\U{x} = 0 \Rightarrow \U{x}=\U{o} $
	\end{enumerate}
	Længden af en vektor defineres som:
	\begin{equation*}
		|x|=\sqrt{x_1^2+\dots+x_n^2}=\sqrt{\U{x}\D\U{x}}
	\end{equation*}
	Det bemærkes at $ |a\U{x}|=|a||\U{x}| $. To vektorer siges at være \textbf{ortogonale}, hvis
	\begin{equation*}
		\U{x}\D\U{y}=0
	\end{equation*}
	
	\subsubsection{Krydsprodukt}
	Krydsproduktet af to vektorer i $ \Set{R}^3 $ er givet ved:
	\begin{equation}
		\U{a} \times \U{b} = \begin{vmatrix}
			\hat{x}	& \hat{y}	& \hat{z}	\\
			a_1		& a_2		& a_3		\\
			b_1		& b_2		& b_3
		\end{vmatrix}
	\end{equation}
	
	\subsection{Matricer}
	En \textbf{matrix} (en matrix, matricen, flere matricer, alle matricerne) er en tabel med talværdier, og har formen:
	\begin{equation}
		\mat{A}=\begin{pmatrix}
		a_{11} & a_{12} & \dots & a_{1n} \\
		a_{21} & a_{22} & \dots & a_{2n} \\
		\vdots & \vdots &  & \vdots \\
		a_{m1} & a_{m2} & \dots & a_{mn} \\
		\end{pmatrix}
	\end{equation}
	Dette er den \textbf{generelle} matrix, hver værdi $ a_{ij} $ kaldes en \textbf{indgang}, og ligger i $ \Set{F} $. Hvis $ \Set{F}=\Set{R} $ kaldes matricen reel, og hvis $ \Set{F}=\Set{C} $ kaldes matricen kompleks. Den 
	\textit{i}'te række i $ \mat{A} $ betegnes $ \mat{A}[i,*] $, og er
	\begin{equation*}
		(a_{i1}, \dots, a_{in})
	\end{equation*}
	Og den \textit{j}'te søjle i $ \mat{A} $ betegnes $ \mat{A}[*,j] $, og er
	\begin{equation*}
		\begin{pmatrix}
		a_{1j} \\ \vdots \\ a_{mj}
		\end{pmatrix}
	\end{equation*}
	Indgangen $ a_{ij} $, som står i den \textit{i}'te række og \textit{j}'te søjle, betegnes da $ \mat{A}[i,j] $. Matricen $ \mat{A} $ har $ m $ rækker og $ n $ søjler (men ikke nødvendigvis flere søjler end rækker. At disse betegnes $ m $ og $ n $ siger intet om deres indbyrdes størrelse). Matricen består da af $ mn $ indgange fra $ \Set{F} $, og kalder en sådan matrix for en $ m\times n $-matrix.
	
	Matricen $ \mat{A} $ kan også opfattes som bestående af $ n $ $ m $-dimensionelle vektorer, der skrives:
	\begin{equation}
		\U{a}_1=\begin{pmatrix}
			a_{11}\\\vdots\\a_{m1}
		\end{pmatrix},\cdots,\U{a}_n=\begin{pmatrix}
			a_{1n}\\\vdots\\a_{mn}
		\end{pmatrix}
	\end{equation}
	Disse vektorer kaldes for matricens \textbf{søjlevektorer}.
	
	En $ m\times 1 $-matrix kaldes en \textbf{søjlematrix} og har form som en vektor:
	\begin{equation*}
		\begin{pmatrix}
			a_{11} \\ \vdots \\ a_{m1}
		\end{pmatrix}
	\end{equation*}
	Og en $ 1 \times n $-matrix kaldes en \textbf{rækkematrix} og har formen
	\begin{equation*}
		\begin{pmatrix}
		a_{11} & \dots & a_{1n}
		\end{pmatrix}
	\end{equation*}
	Søjlematricer og vektorer bruges i forskellige sammenhænge (for eksempel bruges søjlematricer til matrixmultiplikation), men der er ingen stor forskel på dem, ud over sammenhæng og notation.
	
	En matrix hvor alle elementer er lig 0 kaldes en \textbf{nulmatrix} og betegnes $ \mat{0} $ eller $ \mat{0}_{m,n} $ (altså med tallet ``0'')
	
	Matricen $ \mat{A} $ kan også opskrives på en kortere måde, end ved en tabel:
	\begin{equation}
		\mat{A} = (a_{ij})_{1\leq i\leq m, \, 1\leq j \leq n},
	\end{equation}
	eller endnu kortere
	\begin{equation}
		\mat{A}=(a_{ij})_{m,n}.
	\end{equation}
	Og disse skrivemåder er altså ækvivalente med opskrivning ved en tabel.
	
	En $ n\times n $-matrix, der altså har lige mange rækker og søjler, kaldes en \textbf{kvadratisk matrix}, og alle indgange $ a_{ij} $ hvor $ i=j $ siges at stå i diagonalen, mens alle andre indgange (hvor $ i \neq j $) siges at stå uden for diagonalen. En kvadratisk matrix, hvor alle indgange ude for diagonalen er lig 0, kaldes for en \textbf{diagonalmatrix}.
	
	En kvadratisk matrix hvor $ a_{ij} =0 $ for alle $ i $ og $ j $ med $ i<j $ kaldes en \textbf{nedre trekantsmatrix}:
	\begin{equation}
		\begin{pmatrix}
			a_{11}	& 0			& \cdots	& 0 \\
			a_{21}	& a_{22}	& \ddots	& \vdots \\
			\vdots 	& \vdots	& \ddots	& 0 \\
			a_{n1}	& a_{n2}	& \cdots	& a_{nn}
		\end{pmatrix}
	\end{equation}
	De forskellige indgange hvor $ i<j $ kan sagtens være 0. Kravet er kun, at alle indgange over diagonalen er lig 0.
	
	Ligeledes kaldes en kvadratisk matrix, hvor alle indgange under diagonalen er lig 0 ($ a_{ij}=0 $ for alle $ i>j $), for en \textbf{øvre trekantsmatrix}:
	\begin{equation}
		\begin{pmatrix}
			a_{11}	& a_{12}	& \cdots	& a_{1n} \\
			0		& a_{22}	& \cdots	& a_{2n} \\
			\vdots 	& \ddots	& \ddots	& \vdots \\
			0		& \cdots	& 0			& a_{nn}
		\end{pmatrix}
	\end{equation}
	Fra en $ m\times n $-matrix $ \mat{A} $ og en $ m \times p $ matrix $ \mat{B} $ kan der dannes en $ m \times (n+p) $-matrix $ \mat{C} $ ved at opskrive $ \mat{B} $'s søjler efter $ \mat{A} $'s søjler. Matricen $ \mat{C} $ kaldes for en \textbf{blokmatrix} og skrives:
	\begin{equation}
		\mat{C}=\begin{pmatrix}
		\mat{A} & \mat{B}
		\end{pmatrix}
	\end{equation}
	Det samme kan gøres for flere matricer. For eksempel kan $ \mat{A} $ dannes ud fra sine søjlematricer:
	\begin{equation*}
		\mat{A}= \begin{pmatrix}
		\mat{A}_1 & \cdots & \mat{A}_n
		\end{pmatrix}
	\end{equation*}
	Det samme kan gøres med søjlevektorer.
	
	
	\subsubsection{Enhedsmatricer og elementære matricer}
	En diagonalmatrix hvor alle indgange på diagonalen er lig 1 kaldes for en \textbf{enhedsmatrix}. Disse betegnes enten $ \mat{E} $ eller $ \mat{E}_{n,n} $, hvis antallet af rækker/søjler ønskes specificeret. Hvis der ikke specificeres antal rækker/søjler følger enhedsmatricens størrelse af kontekst.
	
	Enhedsmatricen kan også ses opbygget af \textbf{standard enhedssøjlematricer-ne} eller \textbf{standard enhedsvektorerne}:
	\begin{align}
		\mat{E}_1 &= \begin{pmatrix}
		1 \\ \vdots \\ 0
		\end{pmatrix}, \cdots,\mat{E}_n = \begin{pmatrix}
		0 \\ \vdots \\ 1
		\end{pmatrix}\\
		\U{e}_j &= \begin{pmatrix}
		1 \\ 0 \\ \vdots \\ \vdots \\ 0
		\end{pmatrix}, \cdots, \U{e}_j=\begin{pmatrix}
		0 \\ \vdots \\ 1 \\ \vdots \\ 0
		\end{pmatrix} \leftarrow j, \cdots, \U{e}_n=\begin{pmatrix}
		0 \\ \vdots \\ \vdots \\ 0 \\ 1
		\end{pmatrix}
	\end{align}
	En matrix hvor alle indgange (eller elementer) pånær plads ($ j,k $), der er lig 1, kaldes for en \textbf{elementærmatrix}. Denne ser sådan ud:
	\begin{equation}
		\mat{I}_{j,k}=\begin{pmatrix}
		0 		& \cdots	& 0			& \cdots	& 0 \\
		\vdots 	& \ddots	& \vdots	& \ddots	& \vdots \\
		0		& \cdots	& 1 		& \cdots	& 0 \\
		\vdots	& \ddots	& \vdots 	& \ddots	& \vdots \\
		0 		& \cdots	& 0 		& \cdots	& 0 \\
		\end{pmatrix}
	\end{equation}
	hvor 1 altså er i række $ j $ og søjle $ k $. Elementærmatricens størrelse defineres normalt ikke, andet end med kontekst (hvis der arbejdes med en $ m\times n $-matrix $ \mat{A} $, så har dens elementærmatricer også størrelsen $ m\times n $).
	
	En vilkårlig $ m\times n $-matrix $ \mat{A} $ kan opbygges som en sum af skalerede elementærmatricer: (se regnereglerne for matricer, i afsnittet Matrixalgebra)
	
	\begin{equation}
		\mat{A}=\sum_{j,k}a_{jk}\mat{I}_{j,k}
	\end{equation}
	
	Og for to elementærmatricer, der kan multipliceres (igen, se regneregler for matricer) gælder følgende:
	\begin{equation*}
		\mat{I}_{j,k}\mat{I}_{m,n}=\delta_{km}\mat{I}_{j,n}, \text{ hvor }\delta_{km}=\begin{cases}
		1 & k=m\\
		0 & \text{ellers}
		\end{cases}
	\end{equation*}
	hvor $ \delta_{km} $ kaldes Kroneckers delta. Eksempelvis multipliceres $ 5\times 5 $ matricen $ \mat{I}_{2,4} $ (altså 1 i anden række, fjerde søjle) med sig selv, fås følgende:
	\begin{align*}
		\mat{I}_{2,4}\mat{I}_{2,4} &\Rightarrow j=m=2, \ k=n=4 \Rightarrow k\neq m \Rightarrow \delta_{km}=0\\
		&= 0 \mat{I}_{2,4} = \mat{0}
	\end{align*}
	Men hvis matricerne $ \mat{I}_{2,4} $ og $ \mat{I}_{4,4} $ multipliceres fås:
	\begin{align*}
		\mat{I}_{2,4}\mat{I}_{4,4} &\Rightarrow j=2, \ k=m=n=4 \Rightarrow k = m \Rightarrow \delta_{km}=1\\
		&= 1 \mat{I}_{2,4} = \mat{I}_{2,4}
	\end{align*}
	Samme resultat kan også opnås ved almindelig matrixmultiplikation.
\end{document}

