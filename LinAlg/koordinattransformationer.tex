\documentclass[LinAlgNoter.tex]{subfiles} % HUSK FOR FANDEN AT REDIGERE DENNE LINJE

% Hvis ikke dokumenterne (hoved & under) er i samme mappe, skal den relative stig bruges.

\begin{document}
	\section{Koordinattransformationer}
	Givet to baser $ \bas{A} =(\V{a}_1, \dots , \V{a}_n)$ (den gamle) og $ \bas{B}=(\V{b}_1, \dots , \V{b}_n) $ (den nye) i $ V $ kan en koordinattransformationsmatrix opstilles, der omdanner en vektor $ \V{v} $ fra én base til en anden:
	\begin{equation}
		\ele{\bas{B}}{\V{v}}= \trans{\bas{B}}{\bas{A}}\ele{\bas{A}}{\V{v}} 
	\end{equation}
	Transformationen skal læses bagfra - der omdannes fra $ \bas{A} $ til $ \bas{B} $. Yderligere gør denne notation, at to ens baser altid står ved siden af hinanden. Så hvis to forskellige baser står ved siden af hinanden, så er der gjort noget galt.
	
	For denne matrix gælder:
	\begin{itemize}
		\item $ \trans{\bas{B}}{\bas{A}}\inverse = \trans{\bas{A}}{\bas{B}} $ og derfor:
		\item $ \ele{\bas{A}}{\V{v}}= \trans{\bas{B}}{\bas{A}}\inverse\ele{\bas{B}}{\V{v}} =\trans{\bas{A}}{\bas{B}}\ele{\bas{B}}{\V{v}} $
		\item $ \trans{\bas{A'}}{\bas{A}} = \transf{\bas{A'}}{\id_U}{\bas{A}} $
	\end{itemize}
	
	\subsection{Lineære afbildninger og matricer}
	For $ U $ og $ V $, der er endeligdimensionale vektorrum, hvor $ \bas{A} = (\V{a}_1, \dots,\V{a}_n) $ er en basis i $ U $, $ \bas{B} = (\V{b}_1, \dots,\V{b}_m) $ er en basis i $ V $, og med en given lineær afbildning $ \morf{f}{U}{V} $, definerer vi afbildningen $ \morffFF[\alpha]{n}{m} $, hvor følgende $ m\times n $-matrix er tilknyttet:
	\begin{equation*}
		\transf{\bas{B}}{f}{\bas{A}}=\begin{pmatrix}
			\ele{\bas{B}}{f(\V{a}_1)} & \dots & \ele{\bas{B}}{f(\V{a}_n)}
		\end{pmatrix}
	\end{equation*}
	Denne afbildning har inputs i $ \bas{A} $ og output i $ \bas{B} $, og hvis $ \ele{\bas{A}}{\V{x}} $ betegner $ \V{x} $ i basen $ \bas{A} $, og $ \ele{\bas{B}}{\V{y}}=\ele{\bas{B}}{f(\V{x})} $ betegner $ \V{y}=f(\V{x}) $ i basen $ \bas{B} $, giver dette anledning til den generelle søjleregel:
	\subsubsection{den generelle søjleregel}
	hvis $ \ele{\bas{A}}{\V{x}} $ og $ \ele{\bas{B}}{\V{y}}=\ele{\bas{B}}{f(\V{x})} $ er koordinatsøjlerne for henholdsvis $ \V{x} $ og $ \V{y}=f(\V{x}) $ i deres respektive baser, er
	\begin{equation}
		\ele{\bas{B}}{f(\V{x})} = \transf{\bas{B}}{f}{\bas{A}} \ele{\bas{A}}{\V{x}}
	\end{equation}
	Hvor $ m \times n $-matricen $ \transf{\bas{B}}{f}{\bas{A}} $ er givet ved
	\begin{equation}
		\transf{\bas{B}}{f}{\bas{A}}=\begin{pmatrix}
			\ele{\bas{B}}{f(\V{a}_1)} & \dots & \ele{\bas{B}}{f(\V{a}_n)}
		\end{pmatrix}
	\end{equation}
	For sammensatte afbildninger gælder: for $ \morf{f}{W}{V} $ og $ \morf{g}{U}{W} $, med baserne $ \bas{A} $ i U, $ \bas{B} $ i V og $ \bas{C} $ i W, hvor $ f $ repræsenteres af $ \transf{\bas{B}}{f}{\bas{C}} $ og $ g $ repræsenteres af $ \transf{\bas{C}}{g}{\bas{A}} $, er den sammensatte afbildning $ \morf{f \circ g}{U}{V} $ givet ved $\transf{\bas{B}}{f\circ g}{\bas{A}} = \transf{\bas{B}}{f}{\bas{C}}\transf{\bas{C}}{g}{\bas{A}}$.
	
	\subsubsection{Opskrift for at finde koordinattransformationsmatricer:}
	
	Lad $ \bas{A} =(\V{a}_1, \dots , \V{a}_n)$ være den gamle basis og  $ \bas{B}=(\V{b}_1, \dots , \V{b}_n) $ være den nye.
	\begin{enumerate}
		\item Dan $ n\times 2n $-blokmatricen $ (\mat{B} | \mat{A}) = (\V{b}_1 \dots  \V{b}_n | \V{a}_1 \dots \V{a}_n)$, der har de ordnede basiselementer for $ \bas{B} $ og $ \bas{A} $ som søjler.
		\item Omdan denne blokmatrix til en reduceret trappematrix af formen $ (\mat{E}|\mat{C}) $. Da er $ \mat{C}=\trans{\bas{B}}{\bas{A}} $
	\end{enumerate}
	
	\subsubsection{Opskrift på at finde $ \transf{\bas{B}}{f}{\bas{A}} $ for $ \morffFF{n}{m} $:}
	
	Antag at $ \bas{A} = (\V{a}_1, \dots,\V{a}_n) $ er en basis for $ \Set{F}^n $ og $ \bas{B} = (\V{b}_1, \dots,\V{b}_m) $ er en basis for $ \Set{F}^m $.
	\begin{enumerate}
		\item Dan $ m\times (n+m) $ blokmatricen $ (\ \V{b}_1,\dots, \V{b}_m \ |\  f(\V{a}_1), \dots , f(\V{a}_n)\ ) $
		\item Omdan denne til en reduceret trappematrix på formen $ (\mat{E}|\mat{B}) $. Da er $ \mat{B} = \transf{\bas{B}}{f}{\bas{A}} $ matrixrepræsentationen af $ f $ med hensyn til baserne $ \bas{A} $ og $ \bas{B} $.
	\end{enumerate}
	\subsubsection*{Koordinattransformationer for lineære afbildninger}
	Lad U og V være endeligdimensionale vektorrum. Lad $ \bas{A} $ og $ \bas{B} $ være de gamle baser for henholdsvis U og V, og lad $ \bas{A}' $ og $ \bas{B}' $ være de nye baser for henholdsvis U og V. $ \trans{\bas{A}'}{\bas{A}} $ og $ \trans{\bas{B}'}{\bas{B}} $ betegner koordianttransformationsmatricerne for de respektive vektorrum og baser.
	
	Hvis den lineære afbildning $ \morf{f}{U}{V} $ er repræsenteret ved matricen $ \transf{\bas{B}}{f}{\bas{A}} $ og $ \transf{\bas{B}'}{f}{\bas{A}'} $ i henholdsvis de gamle og nye baser, da er
	\begin{equation}
		\transf{\bas{B}'}{f}{\bas{A}'} = \trans{\bas{B}'}{\bas{B}} \transf{\bas{B}}{f}{\bas{A}} \trans{\bas{A}}{\bas{A}'},
	\end{equation}
	hvor $ \trans{\bas{A}}{\bas{A}'} = \trans{\bas{A}'}{\bas{A}}\inverse $. Specielt for en afbildning $ \morf{f}{V}{V} $, hvor $ \bas{A} $ og $ \bas{A}' $ er den gamle og nye base, gælder
	\begin{equation}
		\transf{\bas{A}'}{f}{\bas{A}'}= \trans{\bas{A}'}{\bas{A}}\transf{\bas{A}}{f}{\bas{A}}\trans{\bas{A}'}{\bas{A}}\inverse
	\end{equation}
\end{document}