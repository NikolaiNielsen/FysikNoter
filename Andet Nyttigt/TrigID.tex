% Nikolai Nielsens "Fysiske Fag" preamble
\documentclass[a4paper,10pt]{article} 	% A4 papir, 10pt størrelse
\usepackage[danish]{babel}
\usepackage{Nikolai} 					% Min hjemmelavede pakke
\usepackage[dvipsnames]{xcolor}

% Margen
\usepackage[margin=1in]{geometry}

% Max antal kolonner i en matrix. Default er 10
%\setcounter{MaxMatrixCols}{20}

% Hvor dybt skal kapitler labeles?
%\setcounter{secnumdepth}{4}	
%\setcounter{tocdepth}{4}


% Hvilket nummer skal der startes med i sections? (n-1)
%\setcounter{section}{0}	

% Til nummerering af ligninger. Så der står (afsnit.ligning) og ikke bare (ligning)
\numberwithin{equation}{section}


% Header
%\usepackage{fancyhdr}
%\head{}
%\pagestyle{fancy}

%Titel
\title{}
\author{Nikolai Plambech Nielsen, LPK331. Version 1.0}
\date{}

\begin{document}
	
	\selectlanguage{danish}
	
	\section{Inverse trigonometriske funktioner som argument til almindelige trigonometriske funktioner}
	
	\begin{table}[H]
		\centering
		\begin{tabular}{>{$\displaystyle} l<{$}|>{$\displaystyle} l<{$}|>{$\displaystyle} l<{$}}
			\sin(\arcsin(x)) = x & \cos(\arcsin(x)) = \sqrt{1-x^2} & \tan(\arcsin(x)) = \frac{x}{\sqrt{1-x^2}} \\ \hline
			\sin(\arccos(x)) = \sqrt{1-x^2} & \cos(\arccos(x))= x & \tan(\arccos(x)) = \frac{\sqrt{1-x^2}}{x} \\ \hline
			\sin(\arctan(x)) = \frac{x}{\sqrt{1+x^2}} & \cos(\arctan(x)) = \frac{1}{\sqrt{1+x^2}} & \tan(\arctan(x)) = x
		\end{tabular}
	\end{table}
	
	
\end{document}

