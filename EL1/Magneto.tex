\documentclass[EL1Noter.tex]{subfiles} % HUSK FOR FANDEN AT REDIGERE DENNE LINJE

\begin{document}
	\section{Magnetostatik}
	Magnetostatik kan, lige som elektrostatik, opsummeres i en figur. Figuren for magnetostatik ses herunder:
	
	\begin{figure}[H]
		\centering
		\includegraphics[width=0.5\textwidth]{img/JAB}
		\caption{Opsummering af magnetostatik. Fra bogen side 240}
	\end{figure}		
	
	\subsection{Introduktion}
	Indtil videre har vi regnet med punktpartikler, eller partikelfordelinger, der står stille, og set hvordan disse påvirker en testladning. Nu skal der ses på situationer, hvor kildeladningerne bevæger sig. Nærmere specifikt skal der ses på, når den samlede strøm er konstant. Situationer som denne kaldes for \textbf{magnetostatik}, da de danner magnetfelter, der er stationære i tid.
	
	Et eksempel kan ses, hvis man hænger en ledning op i loftet, så de to ender begge hænger nedad. Hvis ledningens ender da forbindes til et batteri, sådan så der løber strøm fra pluspolen, op ad den ene del af ledningen, forbi ophængningspunktet og ned ad den anden del til minuspolen, så vil de to ledninger \textit{frastøde} hinanden.
	
	Hvis man derimod hænger dem op på sådan en måde, så der løber strøm gennem de to dele af ledningen, parallelt i forhold til hinanden (altså opad, eller nedad for dem begge), da vil de to dele \textit{tiltrække} hinanden.
	
	Der vil dog ikke ske noget, hvis man bare sætter en testladning i nærheden af ledningerne - der bliver ikke dannet noget elektrostatisk felt, da der er lige så mange positive som negative ladninger i ledningen. Godt nok bevæger de negative kildeladninger sig, men ladningsfordelingen bliver konstant 0, og der dannes altså intet elektrisk felt. Hvis partiklen til gengæld er i bevægelse vil denne opleve en kraft fra ledningerne - en magnetisk kraft.
	
	Det magnetfelt der dannes, når der flyder strøm gennem ledningen vil ikke bevæge sig væk fra ledningen, som ved elektriske felter. Magnetfeltet bevæger sig ">rundt om"< ledningen: Hvis man tager sin højre hånd, stikker tommelfingeren i retning af strømmen og krummer fingrene om ledningen, vil fingrenes retning beskrive magnetfeltets retning.
	
	\subsection{Lorentzloven}
	Selve den magnetiske kraft er givet ved Lorentzloven:
	\begin{equation}
		\V{F}_{\text{mag}} = Q (\V{v} \times \V{B})
	\end{equation}
	Hvor $ Q $ er testladningen, $ \V{v} $ er dennes hastighed og $ \V{B} $ er magnetfeltet. Hvis der både er et elektrisk og magnetisk felt, er den samlede kraft på $ Q $ givet ved summen af disse
	\begin{equation}
		\V{F} = Q[\V{E} + (\V{v} \times \V{B})]
	\end{equation}
	Et eksempel på en partikelbane som følge af magnetiske krafter er et homogent magnetisk felt $ \V{B} = -B \Vz $ og en partikel der bevæger sig i $ xy $-planen. Hvis denne bevæger sig mod uret vil kraften pege indad mod origo og partiklen oplever en centripetal kraft, givet ved
	\begin{equation}
		F = QvB = m \frac{v^2}{R}, \quad p=QBR
	\end{equation}
	Denne formel kaldes for \textbf{cyklotronformlen}, og kan bruges til at finde en partikels impulsmoment: send en partikel gennem et homogent magnetisk felt og mål dennes afbøjningsradius.
	
	Hvis partiklen også har en komponent parallelt med $ \V{B} $, altså i $ z $-retningen, vil denne ikke have nogen indflydelse på partiklens afbøjningsradius, og den vil heller ikke opleve nogen kraft i $ z $-retningen. Resultatet af dette er, at partiklen bevæger sig i en helix omkring $ z $-aksen.
	
	Et vigtigt resultat af Lorentzloven er at \textbf{magnetiske felter udfører intet arbejde}. Dette kan ses ved at lade en partikel $ Q $ bevæge sig et stykke vej $ \uuud \V{l} = \V{v} \uuud t $. Arbejdet er da
	\begin{equation}
		\uuud W = \V{F} \D \uuud \V{l} = Q(\V{v} \times \V{B}) \D \V{v} \uuud t = 0
	\end{equation}
	Hvilket følger af at $ \V{v}\times \V{B} $ og $ \V{v} $ er ortogonale, og dermed at $ (\V{v} \times \V{B}) \D \V{v} = 0 $. Magnetiske felter kan altså kun afbøje partikler og dermed ændre deres retning - de kan ikke øge eller sænke hastigheden af en partikel. Nogle gange kan det se ud som om, at magnetiske felter udfører et arbejde, men i sandheden sker dette ikke. De kan godt omdirigere krafter, hvilket for det til at se ud som om det magnetiske felt udfører et arbejde.
	
	En analogi er normalkraften i mekanik. Denne er altid vinkelret på overfladen, og dermed på en eventuel bevægelse. Altså udfører denne heller intet arbejde. Hvis der dog ses på en kasse på et skråplan, der påvirkes med en horisontal kraft, vil normalkraften (der har en vertikal komponent) ">omdirigere"< den resulterende kraft, således at kassen bevæger sig op ad skråplanen. Her er det ikke normalkraften der udfører et arbejde, men den eksterne, horisontale kraft, der gør dette.
	
	\subsubsection{Strømme}
	En strøm i en ledning er den mængde ladning per tidsenhed, der passerer gennem et givent punkt. Her gælder at negative ladninger der bevæger sig mod venstre, giver den samme strøm som positive ladninger, der bevæger sig mod højre (såfremt hastighed og ladningsmængde er ens i de to tilfælde). Normalt ses strøm som bevægende sig fra pluspolen af en strømkilde mod minuspolen. Egentlig er det elektronerne, der er negativt ladede, der bevæger sig fra minus mod plus. Resultatet er det samme, og man kan se det som positivt ladede ">huller"<, der bevæger sig fra plus til minus.
	
	Strøm måles i coulumb per sekund, kaldet ampere: 1 A = 1 C/s. En linjeladning $ \gamma $ der bevæger sig gennem en ledning med en hastighed $ \V{v} $ udgør en ladning
	\begin{equation}
		\V{I} = \lambda \V{v}
	\end{equation}
	Det ses, at strøm er en vektorial størrelse. Normalt opgives den ikke ved en vektor, da strømmens retning angives af ledningens retning $ \uuud \V{l} $, hvori selve strømmen flyder. Ladningsdensiteten $ \lambda $ hentyder kun til den mængde frie ladning, der rent faktisk bevæger sig i ledningen. Kraften på et ledningssegment er da givet ved
	\begin{equation}
		\V{F} = \int (\V{v} \times \V{B})  \ud q = \int (\V{v} \times \V{B}) \lambda \ud l = \int (\V{I} \times \V{B}) \ud l
	\end{equation}
	Og da strømmen og ledningen peger i samme retning, og at strømmen normalt er konstant, kan dette skrives ved
	\begin{equation}
		\V{F} = \int I (\uuud\V{l} \times \V{B}) = I \int (\uuud\V{l} \times \V{B}), \ \text{konstant strøm}
	\end{equation}
	
	Når en strøm løver over en flade, beskrives dette ved \textbf{overfladestrøm} $ \V{K} $. Denne kan ses som et infinitisemalt bånd med bredden $ \uuud l_{\perp} $, der løber parallelt med strømmen $ \uuud \V{I} $. Overfladestrømmen er da
	\begin{equation}
		\V{K} = \diff[\uuud]{\V{I}}{l_{\perp}}
	\end{equation}
	Med ord er $ \V{K} $ strømmen per enhedsbredde vinkelret på strømmen. Hvis der er en overfladeladning $ \sigma $, der bevæger sig med hastigheden $ \V{v} $ er $ \V{K} $ givet ved
	\begin{equation}
		\V{K} = \sigma \V{v}
	\end{equation}
	Kraften på overfladeladningen er 
	
	\begin{equation}
		\V{F} = \int (\V{v} \times \V{B}) \sigma \ud a = \int (\V{K} \times \V{B}) \ud a
	\end{equation}
	Lige som  at elektriske felter oplever en diskontinuitet ved overfladeladninger, så oplever magnetiske felter en diskontinuitet ved overfladestrømme. Dermed skal $ \V{B} $ i formlen ovenfor være \textbf{gennemsnittet} af feltet over og under overfladen - igen, præcis som elektriske felter.
	
	Når der løber strøm gennem en tredimensional region, kaldes det for en \textbf{volumenstrøm} $ \V{J} $, og er defineret ved et infinitisemalt rør med tværsnitsareal $ \uuud a_{\perp} $, der løber parallelt med strømmen $ \uuud \V{I} $. Da er
	\begin{equation}
		\V{J} = \diff[\ud]{\V{I}}{a_{\perp}}
	\end{equation}
	med ord er dette strømmen per arealenhed, der står vinkelret på strømmen. Hvis en volumenladning $ \rho $ bevæger sig med hastigheden $ \V{v} $ er $ \V{J} $ givet ved
	\begin{equation}
		\V{J} = \rho \V{v}
	\end{equation}
	Ligesom analogerne til linjeladning og overfladeladning. Kraften på en volumenstrøm er da
	\begin{equation}
		\V{F} = \int (\V{v} \times \V{B}) \rho \ud \tau = \int (\V{J} \times \V{B}) \ud \tau
	\end{equation}
	
	Ud fra definitionen af volumenstrømmen, kan den samlede strøm, der løber gennem en overflade $ \mathcal{S} $ gives ved at integrere:
	\begin{equation}
		I = \int_{\mathcal{S}} J \ud a_{\perp} = \int_{\mathcal{S}} \V{J} \D \ud\V{a}
	\end{equation}
	Den samlede strøm der forlader en volumen $ \mathcal{V} $ er givet ved
	\begin{equation}
	\oint_{\mathcal{S}} \V{J} \D \ud \V{a} = \int_{\mathcal{V}} (\grad \D \V{J}) \ud \tau
	\end{equation}
	Men da ladning er bevaret, må dette komme på bekostning af ladningen inde i volumenet:
	\begin{equation}
		\int_{\mathcal{V}} (\grad \D \V{J}) \ud \tau = -\diff[\ud]{}{t} \int_{\mathcal{V}} \rho \ud \tau = -\int_{\mathcal{V}} \diff{\rho}{t} \ud \tau
	\end{equation}
	Siden dette gælder for alle volumener, konkluderes det at
	\begin{equation}
		\grad \D \V{J} = -\diff{\rho}{t}
	\end{equation}
	Denne formel kaldes for \textbf{kontinuitetsligningen}.
	
	\subsubsection{Jævn strøm}
	Magnetostatik kræver en jævn strøm, og med dette menes en kontinuert strøm af ladning, der har været i gang for evigt og uden at der ophobes ladning nogen steder. Der findes da ingen perfekt jævn strøm, på samme måde som der egentlig heller ikke findes nogen perfekt stationær ladning. Dermed er både elektrostatik og magnetostatik egentlig beskrivelser ef teoretiske systemer. Til gengæld er de passende approksimationer på mange fysiske systemer. Magnetostatik er en fin approksimation såfremt fluktuationerne i strømmen ikke er alt for hurtige. I bogen nævner Griffiths blandt andet vekselstrømmen i hjemmene, der har en frekvens på 60 Hz (i USA. Danmarks el-net bruger 50 Hz), er i mange henseender velapproksimeret ved magnetostatik.
	
	Der lægges blandt andet mærke til, at en enkelt punkladning aldrig vil kunne udgøre en jævn strøm, idet ladningsfordelingen hele tiden fluktuerer. Til ét tidspunkt er partiklen i ét punkt, og til det næste er den væk.
	
	I magnetostatik er $ \rho $ altså konstant, hvilket gør at kontinuitetsligningen bliver
	\begin{equation}
		\grad \D \V{J} = 0
	\end{equation}
	
	
	
	\subsection{Biot-Savart loven}
	Magnetfeltet af en jævn linjeladning er givet ved Biot-Savart loven:
	\begin{equation}
		\V{B}(\V{r}) = \frac{\mu_0}{4 \pi} \int \frac{\V{I}\times\usr}{\sr^2} \ud l' = \frac{\mu_0}{4 \pi} I \int \frac{\ud \V{l}' \times \usr}{\sr^2}
	\end{equation}
	Hvor der integreres langs strømmens retning. $ \mu_0 $ kaldes \textbf{vakuumpermeabiliteten} og har værdien
	\begin{equation}
		\mu_0 = 4 \pi \D 10^{-7} \ \e{N} / \e{A}
	\end{equation}
	Det er en eksakt værdi for at få enheden Ampere til at passe, og for at få $ \V{B} $ til at have enheden \textbf{tesla}:
	\begin{equation}
		1\, \e{T} = 1 \,\e{N}/(\e{A} \D \e{m})
	\end{equation}
	For henholdsvis en \textbf{overfladestrøm} og \textbf{volumenstrøm} er Biot-Savart loven
	\begin{equation}
		\V{B}(\V{r}) = \frac{\mu_0}{4 \pi} \int \frac{\V{K}(\V{r}') \times \usr}{\sr^2} \ud a', \qquad \V{B}(\V{r}) = \frac{\mu_0}{4 \pi} \int \frac{\V{J}(\V{r}') \times \usr}{\sr^2} \ud \tau'
	\end{equation}
	Igen er der \textbf{ingen} Biot-Savart lov for en punktpartikel, da denne aldrig kan udgøre en jævn strøm. Magnetfelter overholder også \textbf{superpositionsprincippet}, og det samlede vektorfelt er altså vektorsummen af de enkelte magnetiske felter.
	
	\subsubsection{Eksempler og anvendelser}
	Et eksempel på anvendelsen af Biot-Savart loven er en \textbf{lang lige ledning, der bærer en jævn strøm} $ I $. Magnetfeltet i et punkt $ P $, en afstand $ s $ fra ledningen er givet ved
	\begin{equation}
		B = \frac{\mu_0 I}{4 \pi s} (\sin \theta_2 - \sin \theta_1)
	\end{equation}
	Hvor $ \theta_1 $ er startvinklen og $ \theta_2 $ er slutvinklen. Disses placering ses på figuren herunder
	
	\begin{figure}[H]
		\centering
		\includegraphics[width=0.5\textwidth]{img/ekskap5}
		\caption{Fra bogen, side 217.}
	\end{figure}
	
	For en uendelig ledning er $ \theta_1 = -\pi/2$ og $ \theta_2 = \pi /2 $ og magnetfeltet bliver
	\begin{equation}
		B = \frac{\mu_0 I}{2 \pi s}
	\end{equation}
	Her er feltet invers proportionelt med afstanden $ s $ - lige som det elektrostatiske tilfælde.
	
	\textbf{Kraften mellem to uendeligt lange, parallelle ledninger}, der har strømmen $ I_1 $ og $ I_2 $ (med samme fortegn) findes da som følger. Magnetfeltet på den anden ledning er
	\begin{equation}
		B = \frac{\mu_0 I_1}{2\pi s}
	\end{equation}
	Kraften er her tiltrækkende og givet ved Lorentz lov:
	\begin{equation}
		F = I_2 \frac{\mu_0 I_1}{2 \pi s} \int \ud l
	\end{equation}
	Integralet divergerer mod uendeligt, så den samlede kraft er også uendelig, men kraften per enhedslænge er
	\begin{equation}
		f = \frac{\mu_0}{2 \pi} \frac{I_1 I_2}{s}
	\end{equation}
	
	Et andet eksempel er magnetfeltet i et punkt $ P $, en afstand $ z $ over midten af en \textbf{cirkulær løkke} med radius $ R $ og jævn strøm $ I $. Magnetfeltet $ \ud \V{B} $ fra linjestykket $ \ud \V{l}' $ står vinkelret på separationsvektoren mellem $ \ud \V{l}' $ og $ P $. Denne er da skrå i forhold til vertikalen, og det vil $ \ud \V{B} $ også være (denne peger skråt opad). Når der integreres rundt i cirklen vil $ \ud \V{B} $ da tegne en kegle med spids i punktet $ P $. Grundet denne symmetri vil alle horisontale komponenter udslukke hinanden, hvormed der kun er en vertikal tilbage. Denne er givet ved
	\begin{equation}
		B(z) = \frac{\mu_0}{4 \pi} I \int \frac{\ud l'}{\sr^2} \cos \theta = \frac{\mu_0 I}{2} \frac{R^2}{(R^2+z^2)^{3/2}}
	\end{equation}
	
	\subsection{Divergens og rotation for \texorpdfstring{$ \V{B} $}{B}, samt Ampères lov}
	\label{sec:Ampere}
	Fra Biot-Savart loven kan både divergens og rotation for $ \V{B} $ udregnes. Den generelle form af Biot-Savart loven, den for volumenstrømme, er givet ved
	\begin{equation}
		\V{B}(\V{r})=\frac{\mu_0}{4 \pi} \int \frac{\V{J}(\V{r}') \times \usr}{\sr^2} \ud \tau'
	\end{equation}
	Ved at tage divergensen af dette (med hensyn til de almindelige koordinater, ikke de mærkede), og bruge produktregel nr. 6, samt at $ \grad \times (\usr /\sr^2) = 0 $ fås følgende resultat:
	\begin{equation}
		\grad \D \V{B} = 0
	\end{equation}
	Og divergensen af et magnetfelt er altså altid 0.
	
	Ved at tage rotationen af magnetfeltet og bruge produktregel nr. 8, samt det faktum at $ \grad \D (\usr / \sr^2) = 4 \pi \delta^3(\bsr) $ fås
	\begin{equation}
		\grad \times \V{B} = \mu_0 \, \V{J}(\V{r})
	\end{equation}
	Begge disse resultater gælder for \textbf{alle} magnetfelter. Formlen for magnetfeltets rotation kaldes for \\\textbf{Ampères lov} (i differentialform). I integralform lyder den
	\begin{equation}
		\oint \V{B} \D \ud \V{l} = \mu_0 \, I_{\text{enc}}
	\end{equation}
	For at finde den rigtige vej at integrere rundt, bruges højrehåndsreglen: Hvis dine fingre på højre hånd indikerer integrationsvejen, så vil din tommel indikere retningen af positiv strøm.
	
	Lige som ved Gauss' lov kan Ampères lov med fordel bruges i situationer med symmetri. I dette tilfælde er symmetrierne som følger:
	\begin{itemize}
		\item Uendelige, lige ledninger: Brug en cirkulær Ampereløkke, der bevæger sig rundt, vinkelret på ledningen. (eksempel 5.7 i bogen)
		\item Uendelige planer: Brug en rektangulær Ampereløkke, der står vinkelret på både planen og strømret-ningen (eksempel 5.8 i bogen)
		\item Uendelig solenoider (en cylinder med ledning vundet tæt sammen på overfladen): Brug en rektangulær Ampereløkke normalt på solenoidens krumme overflade, med strømretningen vinkelret gennem løkken (såfremt ledningerne er vundet tæt nok sammen, kan de godt ses som værende vinkelrette i forhold til solenoiden. Se eksempel 5.9)
		\item Toroider, hvor ledningen er vundet tæt rundt i en ">cirkel"< (tænk på det som en krum solenoide, hvor enderne sættes sammen, så det ligner en donut med ledninger om): Brug en cirkulær Ampereløkke rundt om aksen på toroiden (altså samme plan som toroiden/donutten). (Se eksempel 5.10) 
	\end{itemize}
	Resultaterne af disse prototyper er som følger:
	
	\subsubsection*{Uendelig, lige ledning}
	For en \textbf{uendelig, lige ledning}, med en jævn strøm $ I $, er det magnetiske felt cirkumfærentielt, og størrelsen i en afstand $ s $ er
	\begin{equation}
		B = \frac{\mu_0 I}{2\pi s}
	\end{equation}
	Dette findes, igen, ved at tegne en cirkulær Ampereløkke, der løber rundt om ledningen, og står vinkelret på denne.
	
	\subsubsection*{Uendelige plan}
	For en \textbf{uendelig plan} (eksempelvis $ z $-planen), med et homogen overfladestrøm, eksempelvis $ \V{K} = K \Vx $, der løber i $ xy $-planen. Her vil $ \V{B} $ kun have en $ y $-komponent givet ved
	\begin{equation}
		\V{B} = \begin{cases}
		\displaystyle
			+(\mu_0 / 2) K \Vy, & \text{for}\  z<0, \\
			\displaystyle
			-(\mu_0 / 2) K \Vy, & \text{for}\  z>0.
		\end{cases}
	\end{equation}
	Dette kan ses ved at bruge højrehåndsreglen ($ +x $, $ +z $ $ \to $ $ -y $, og $ +x $, $ -z $ $ \to $ $ +y $). For at udregne denne bruges en rektangulær Ampereløkke, der er parallelt med $ yz $-planen, og hvis top/bund er henholdsvis over/under $ xy $-planen, således at strømmen løber vinkelret gennem Ampereløkken. Her afhænger feltet ikke af afstanden fra planen - lige som analogen i elektrostatik.
	
	\subsubsection*{Uendelig solenoide}
	For en \textbf{uendelig solenoide} med $ n $ tæt vundne bindinger af ledning per længdeenhed på den krumme overflade af en cylinder med radius $ R $, der bærer en jævn strøm $ I $ gennem ledningen. Såfremt ledningen er vundet tæt nok, kan hver binding betragtes som værende cirkulær. Ellers kan dette også visualiseres med en homogen overfladestrøm $ K = nI $. Her er magnetfeltet parallelt med cylinderens akse, men kun inden i cylinderen er denne forskellig fra 0. Resultatet er
	\begin{equation}
		\V{B} = \begin{cases}
		\displaystyle
			\mu_0 n I \Vz, & \text{inden i solenoiden,}\\
			\displaystyle
			0, & \text{uden for solenoiden.}
		\end{cases}
	\end{equation}
	Her bruges en rektangulær Ampereløkke, der står normalt på den krumme overflade, med ">bunden"< inden i solenoiden og ">toppen"< uden for solenoiden. Her løber strømmen igen vinkelret gennem løkken
	
	\subsubsection*{Toroide}
	For en \textbf{toroide}, med en tæt vunden ledning (så hver binding kan regnes som en lukket løkke), som foretager $ N $ vindinger, hvor der løber en jævn strøm $ I $ gennem ledningen, vil have et magnetfelt der er sonalt, som altså løber i en cirkel rundt om toroidens akse. 
	
	Dette kan også ses som en solenoide, der krummer således at de to ender sættes sammen, og den danner en ">donut"<. I den almindelige solenoide løber magnetfeltet parallelt med omdrejningsaksen. Denne bøjes, så den bliver til en cirkel, og magnetfeltet ">følger med"<. Faktisk behøver toroiden ikke at være cirkulært i tværsnittet. Den kunne også være firkantet, så længe den bare har den samme form hele vejen rundt (se eksempel 5.10 for figurer).
	
	Magnetfeltet af toroiden er
	\begin{equation}
		\V{B} = \begin{cases} \displaystyle
			\frac{\mu_0 N I}{2 \pi s} \Vp, & \text{for punkter inden i spolen,}\\
			0, & \text{for punkter uden for spolen.}
		\end{cases} 
	\end{equation}
	Denne udregnes ved brug af en cirkulær Ampereløkke med radius $ s $, der ligger om toroidens akse (som altså ">følger"< toroidens form).
	
	
	
	
	\subsubsection{Sammenligning mellem magnetostatik og elektrostatik}
	Divergensen og rotationen for elektrostatiske felter er givet ved
	\begin{equation*}
		\begin{cases}\displaystyle
			\grad \D \V{E} = \frac{1}{\epsilon_0} \rho, &\text{(Gauss' lov)}, \\
			\displaystyle
			\grad \times \V{E} = 0,  &\text{(uden navn)}.
		\end{cases}
	\end{equation*}
	Dette er \textbf{Maxwells ligninger} for elektrostatik. Sammen med randbetingelserne at $ \V{E} \to 0 $ for $ r \to \infty $ beskriver disse feltet, såfremt $ \rho $ er opgivet. De indeholder den samme information som Coulumbs lov og superpositionsprincippet.
	
	For magnetostatik er det samme
	\begin{equation}
		\begin{cases}
		\displaystyle
			\grad \D \V{B} = 0, & \text{(uden navn)}, \\
		\displaystyle
			\grad \times \V{B} = \mu_0 \V{J}, & \text{(Ampères lov)}.
		\end{cases}
	\end{equation}
	Dette er Maxwells ligninger for magnetostatik. Sammen med den lignende randbetingelse at $ \V{B} \to 0 $ for $ r \to \infty $ beskriver disse ligninger det magnetiske felt. Disse ligninger er ækvivalente med Biot-Savart loven og superpositionsprincippet - lige som ved Elektrostatik.
	
	
	Maxwells ligninger og kraftloven
	\begin{equation}
		\V{F} = Q (\V{E}+ \V{v} \times \V{B})
	\end{equation}
	er den mest simple/elegante formulering af både elektrostatik og magnetostatik.
	
	Det ses, at elektriske feltelinjer \textit{divergerer} væk fra en (positiv) ladning: starter i positive ladninger, og slutter ved negative. Magnetiske feltlinjer \textit{roterer} om strømme, og hverken starter eller slutter. De laver enten lukkede løkker eller går ud mod uendeligt. Hvis de \textit{ikke} gjorde dette, ville divergensen for magnetfelter være forskellige fra 0. Dette vil også sige, at der er ikke nogen magnetiske punktladninger, altså ingen magnetiske monopoler.
	
	I normale situationer vil det elektriske felt klart dominere det magnetiske felt. Det er først ved relativistiske hastigheder (for både kilde- og testladninger), at det magnetiske felt begynder at have en meget stor indflydelse. Dette ses ved størrelsen af de konstante, der indgår i Maxwells ligninger.
	
	Grunden til, vi så overhovedet lægger mærke til magnetiske kræfter er, at i stedet for at skrue hastigheden op til latterlige proportioner, gør vi det med strømmen. Dette ville normalt lave et enormt elektrisk felt - hvis ikke vi sørger for at holde ledningen neutral, ved at have en lige så stor ladning, modsat ladning i hvile. Dette sørger for at udslukke det genererede elektriske felt, og er netop det, der sker i helt almindelige ledninger.
	
	
	
	
	\subsection{Magnetisk vektorpotentiale}
	I elektrostatik tillod $ \grad \times \V{E} = 0$ os at formulere det elektriske felt som et skalarpotentiale $ \V{E} = - \grad V $. Dette kan dog ikke gøres, da $ \grad \times \V{B} \neq 0$. Til gengæld tillader $ \grad \D \V{B} = 0 $ os at formulere det magnetiske felt som et \textbf{vektorpotentiale}:
	\begin{equation}
		\V{B} = \curl{A}
	\end{equation}
	For det elektriske potentiale kan der altid lægges et felt hvor gradienten er 0 (altså en konstant), uden at ændre på $ \V{E} $. På samme måde kan, for det magnetiske potentiale, lægges ethvert felt til, hvor rotationen er 0 (altså gradienten af enhver skalar), uden at ændre på $ \V{B} $. Dette kan bruges til altid at gøre $ \V{A} $ divergensløst.
	
	Eksempelvis kan det originale potentiale $ \V{A}_0 $ have én eller anden divergens $ \diverg{A}_0 $. Der kan da lægges gradienten af en skalar $ \grad \lambda $ til, der dermed er rotationsløs. Den samlede divergens for feltet bliver da
	\begin{equation}
		\diverg{A} = \diverg{A}_0 + \grad^2 \lambda
	\end{equation}
	For at få vektorpotentialet divergensløst må
	\begin{equation}
		\grad^2 \lambda = -\diverg{A}_0
	\end{equation}
	Dette er et tilfælde af Poissons ligning, der er matematisk identisk med Poissons ligning for det elektrostatiske potentiale $ V $, bortset fra at $ V \to \lambda $ og $ \rho/\epsilon_0 \to \diverg{A}_0 $. Dermed er løsningen også af identisk form:
	\begin{equation}
		\lambda = \frac{1}{4\pi} \int \frac{\diverg{A}_0}{\sr} \ud \tau'
	\end{equation}
	Såfremt gradienten til denne skalar, lægges til vektorpotentialet $ \V{A}_0 $, vil det resulterende potentiale altså være divergensfrit, og stadig give det samme magnetiske felt.
	
	Ved at bruge Ampères lov, muligheden for at gøre $ \V{A} $ divergensløst, samt andenafledte regel nr. 11, fås:
	\begin{equation}
		\curl{B} = \grad \times (\curl{A}) = \grad(\diverg{A}) - \grad^2 \V{A} = \mu_0 \V{J}
	\end{equation}
	Der giver
	\begin{equation}
		\grad^2 \V{A} = -\mu_0 \V{J}
	\end{equation}
	Hvilket jo er Poissons ligning! Eller, rettere tre gange Poissons ligning: én for hver kartesisk koordinat. Under randbetingelsen af, at $ \V{J}, \V{K}, \V{I} \to 0$ ved uendeligt, bliver løsningen for henholdsvis volumen-, overfalde- og linjestrømme:
	\begin{equation}
		\V{A}(\V{r}) = \frac{\mu_0}{4 \pi} \int \frac{\V{J}(\V{r}')}{\sr} \ud \tau', \quad \V{A}(\V{r}) = \frac{\mu_0}{4 \pi} \int \frac{\V{K}(\V{r}')}{\sr} \ud a', \quad \V{A}(\V{r}) = \frac{\mu_0\, I}{4 \pi} \int \frac{1}{\sr} \ud \V{l}' \label{eq:ANem}
	\end{equation}
	Normalt set vil $ \V{A} $ have samme retning som strømmen. Dette er altid tilfældet, hvis der kun løber strøm i én retning. Man kan altid lægge en konstant vektor til $ \V{A} $, uden at det ændrer på $ \V{B} $, hvilket svarer til at ændre referencepunktet for $ V $.
	
	Hvis dette \textit{ikke} er tilfældet, må der selvfølgelig andre metoder til, for at finde $ \V{A} $. En øvelse, der overlades til læseren, selvfølgelig... Just kidding. Der er en lignende metode til Ampères lov i integralform, der kan bruges til at finde $ \V{A} $:
	
	For at finde \textbf{vektorpotentialet} af en \textbf{uendelig solenoide}, med $ n $ vindinger per længdeenhed, radius $ R $ og strøm $ I $, kan det udnyttes at
	\begin{equation}
		\oint \V{A} \D \ud \V{l} = \int (\curl{A}) \D \ud \V{a} = \int \V{B} \D \ud \V{a} = \Phi
	\end{equation}
	hvor $ \Phi $ er fluxen af $ \V{B} $ gennem løkken der anvendes. Dette svarer netop til Améres lov i integralform med $ \V{B} \to \V{A} $ og $ \mu_0 I_{\text{end}} \to \Phi $. Dermed kan der bruges symmetrier til nemt at udregne $ \V{A} $. I dette tilfælde bruges der to cirkulære Ampereløkker: én inden i solenoiden og én uden for solenoiden, begge med radius $ s $. Da fås
	\begin{equation}
		\V{A} = \begin{cases}
		\displaystyle
		\frac{\mu_0 n I}{2} s \Vp, & \text{for } s<R \vspace{0.2 cm} \\
		\displaystyle
		\frac{\mu_0 n I}{2} \frac{R^2}{s} \Vp, & \text{for } s>R
		\end{cases}
	\end{equation}
	Det ses ved udregning at $ \curl{A} = \V{B} $ og $ \diverg{A} = 0 $.
	
	Et eksempel på brug af \eqref{eq:ANem} er en \textbf{sfærisk skal, der roterer} med vinkelhastighed $ \Vg{\omega} $, har en homogen overfladestrømsdensitet $ \sigma $, og radius $ R $. For at finde vektorpotentialet i punktet $ \V{r} $ vælges der at bruge kartesiske koordinater, med $ \V{r} $ liggende langs $ z $-aksen, således at rotationsaksen befinder sig i en vinkel $ \psi $ i forhold til $ z $-aksen. Her vælges, at den ligger i $ xz $-planen. Det fås at
	\begin{equation}
		\V{A}(\V{r}) = \begin{cases}
		\displaystyle \frac{\mu_0 R \sigma}{3} (\Vg{\omega} \times \V{r}), & \text{for punkter \textit{inden} i sfæren},  \vspace{0.2 cm}\\
		\displaystyle \frac{\mu_0 R^4 \sigma}{3 r^3} (\Vg{\omega} \times \V{r}), & \text{for punkter \textit{uden} for sfæren.} 
		\end{cases}
	\end{equation}
	Eller i sfæriske koordinater:
	\begin{equation}
		\V{A}(\V{r}) = \begin{cases}
			\displaystyle \frac{\mu_0 R \omega \sigma}{3} r \sin \theta \Vp, & (r\leq R), \vspace{0.2 cm}\\
			\displaystyle \frac{\mu_0 R^4 \omega \sigma}{3} \frac{\sin \theta}{r^2} \Vp, & (r \geq R). 
		\end{cases}
	\end{equation}
	Det ses også, at det magnetiske felt inden i kugleskallen er homogent: $ \V{B} = \frac{2}{3} \mu_0 \sigma R \Vg{\omega} $.
	
	\subsubsection{Magnetostatiske randbetingelser: overfladestrømme}
	På samme måde, som elektrostatiske felter er diskontinuerte over overfladeladninger, er magnetostatiske felter diskontinuerte over \textit{overfladestrømme}. Her er det dog den tangentielle komponent af $ B $-feltet (tangentielt til overfladen og ortogonalt på strømmen), der er diskontinuert. De normale og parallelle komponenter af $ B $-feltet er begge kontinuerte. Størrelsen af diskontinuiteten er
	\begin{equation}
		B_{\text{tangent, over}}- B_{\text{tangent, under}} = \mu_0 K
	\end{equation}
	Og alle tre randbetingelser kan beskrives ved følgende ligning:
	\begin{equation}
		\V{B}_{\text{over}} - \V{B}_{\text{under}} = \mu_0 (\V{K} \times \U{n})
	\end{equation}
	hvor $ \U{n} $ er enhedsvektornormalen til overfladen, der pejer ">opad"<.
	
	Vektorpotentialet, lige som skalarpotentialet for elektrostatiske felter, er dog kontinuert over en overfladestrøm (såfremt divergensen er 0), og dermed er
	\begin{equation}
		\V{B}_{\text{over}} = \V{B}_{\text{under}}
	\end{equation}
	og den afledte af $ \V{A} $ giver diskontinuiteten:
	\begin{equation}
		\diff{\V{A}_{\text{over}}}{n} - \diff{\V{A}_{\text{under}}}{n} = -\mu_0 \V{K}
	\end{equation}
	
	
	
	\subsubsection{Multipolekspansion af vektorpotentialet}
	Lige som man kan lave en multipolekspansion for elektriske potentialer, der giver en approksimeret værdi for punkter langt væk kilden, kan det samme gøres for magnetiske felter. Med samme metoder fås at
	\begin{equation}
		\V{A}(\V{r}) = \frac{\mu_0 I}{4 \pi} \sum_{n=0}^{\infty} \frac{1}{r^{n+1}} \oint (r')^n P_n(\cos \theta') \ud \V{l}'
	\end{equation}
	hvor igen $ P_n $ er det $ n $-te Legendre polynomium, og $ \theta' $ er vinklen mellem $ \V{r} $ og $ \V{r}' $. Dette kan skrives ud til
	\begin{equation}
		\V{A}(\V{r}) = \frac{\mu_0 I}{4 \pi} \bb{\frac{1}{r} \oint \ud \V{l}' + \frac{1}{r^2} \oint r' \cos \theta' \ud \V{l}' + \frac{1}{r^3} \oint (r')^2 \pp{\frac{3}{2} \cos^2 \theta' - \frac{1}{2}} \ud \V{l}' + \dots}
	\end{equation}
	Her er monopolleddet ($ n=0 $) altid nul, da $ \oint \ud \V{l}'=0 $, idet det er den vektoriale forskydning rundt i en lukket løkke. Igen en indikation, at der ikke er nogen magnetiske monopoler.
	
	Idet dette er tilfældet, er det dominerende led i ekspansionen dipolleddet (såfremt dette ikke også forsvinder). Dette er givet ved
	\begin{equation}
		\V{A}_{\text{dip}}(\V{r}) = \frac{\mu_0 I}{4 \pi r^2} \oint r' \cos \theta' \ud \V{l}' = \frac{\mu_0 I}{4 \pi r^2} \oint (\Vr \D \V{r}') \ud \V{l}'
	\end{equation}
	Integralet kan omskrives til
	\begin{equation}
		\oint (\Vr \D \V{r}') \ud \V{l}' = -\Vr \times \int \ud \V{a}'
	\end{equation}
	Dipolleddet bliver da
	\begin{equation}
		\V{A}_{\text{dip}}(\V{r}) = \frac{\mu_0}{4 \pi} \frac{\V{m} \times \V{r}}{r^2}, \quad \V{m} = I \int \ud \V{a}
	\end{equation}
	hvor $ \V{m} $ kaldes for det \textbf{magnetiske dipolmoment}. $ \V{a} = \U{n} a$ er løkkens ">vektoriale"< areal. Hvis løkken er flad er $ \V{a} $ det almindelige areal. 
	
	Til forskel fra det elektriske dipolmoment, er det magnetiske altid uafhængig af de valgte koordinater (dette ses også, da det samlede monopolmoment altid er 0 for magnetiske dipolmomenter - hvilket også er kravet for at det elektriske dipolmoment er koordinatuafhængigt).
	
	Idet dipolledet (normalt, med mindre $ \V{m}=0 $) er det dominerende er det en god approksimation for det magnetiske felt, langt væk fra strømmen. Der vil stadig være højereordensled, men disse giver mindre og mindre bidrag, jo højere orden de er.
	
	I teorien er det muligt at danne en matematisk, magnetisk dipol, hvis potentiale kun består af dipolleddet. Det gøres ved at have en infinitisemalt lille løkke om origo, og for så at holde dipolmomentet endeligt, skal strømmen gå mod uendelig, med $ m=Ia $ som konstant. Så igen, kun i teorien.
	
	Potentialet af en ren dipol, der sidder i origo og pejer i $ +z $-retningen er givet ved
	\begin{equation}
		\V{A}_{\text{dip}}(\V{r}) = \frac{\mu_0}{4 \pi} \frac{m \sin \theta}{r^2} \Vp
	\end{equation}
	Og magnetfeltet er da
	\begin{equation}
		\V{B}_{\text{dip}}(\V{r}) = \frac{\mu_0 m}{4 \pi r^3} (2 \cos \theta \V{r} + \sin \theta \Vt)
	\end{equation}
	Det ses, at det er nøjagtig den samme struktur som en ren elektrisk dipol! For en fysisk dipol er det dog en noget anderledes historie, når man er tæt på.
	
	I koordinatfri form lyder magnetfeltet
	\begin{equation}
		\V{B}_{\text{dip}} (\V{r}) = \frac{\mu_0}{4 \pi} \frac{I}{r^3} [3(\V{m} \D \Vr)\Vr - \V{m}]
	\end{equation}
\end{document}