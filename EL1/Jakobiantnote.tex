% Nikolai Nielsens "Fysiske Fag" preamble
\documentclass[a4paper,10pt]{article} 	% A4 papir, 10pt størrelse

% Packages:
\usepackage[Danish]{babel}		% Dansk og engelsk
\usepackage[utf8]{inputenc}				% Vi bruger UTF-8 til Æ, Ø & Å
\usepackage{amsmath,amssymb,url,graphicx} % Math-ting, URLs og illustrationer
\usepackage{subfiles}
\usepackage{mathtools, gauss}
\usepackage{enumerate}

\usepackage{bm}					% BoldMath


% Forskellige pakker til illustrationer og figurer. Udkommenter hvis unødvendigt
\usepackage{wrapfig}		% Ombrydning af tekst
\usepackage{caption}
\usepackage{subcaption}
\usepackage{floatrow}		% Float-rækker

% Links i indholdsfortegnelsen. Udkommenter hvis unødvendig
\usepackage{hyperref}
\hypersetup{
	colorlinks,
	citecolor=black,
	filecolor=black,
	linkcolor=black,
	urlcolor=blue,
	breaklinks=true
}

% Commands:
\newcommand{\D}{\cdot}					% Gange
\newcommand{\V}[1]{\mathbf{#1}}				% Vektornotation
\newcommand{\Set}[1]{\mathbb{#1}}		% Sæt, fx reelle tal
\newcommand{\e}[1]{e^{#1}}				% e^x
\newcommand{\DU}[1]{\underline{\underline{#1}}} % dobbelt understregning
\newcommand{\Lim}[2]{\lim\limits_{#1 \rightarrow #2}}	% Nyttig grænseværdi
\newcommand{\Maple}[2]{\\\includegraphics[width=#2cm]{img/maple#1.png}} % Til Matematikopgaver
\newcommand{\diff}[3][\partial]{\frac{#1 #2}{#1 #3}}
\newcommand{\Bf}[1]{\mathbf{#1}} 		% til fed matematikskrift
\newcommand{\pp}[1]{\left( #1 \right)}	% Stor blød parentes
\newcommand{\bb}[1]{\left[ #1 \right]}	% Stor firkantet parentes
\newcommand{\kk}[1]{\left\{ #1 \right\}}% Stor tuborgklamme
\newcommand{\inverse}{^{-1}}			% hvis man tit skal skrive ^{-1}, ie $X\inverse$
\newcommand{\konj}[1]{\overline{#1}}	% Kompleks konjugeret


% 'upright d'. Gør at du kan afslutte integraler med f.eks. \int x \ud x
\DeclareMathOperator{\ud}{\,d\negthickspace\,}

% LinAlg
\usepackage{mathrsfs}
\usepackage{ushort}					% kort understregning
\usepackage{blkarray}				% blockarrays, til angivning af rækker/søjler i en matrix.
\DeclareMathOperator{\sign}{sign}
\DeclareMathOperator{\Pol}{Pol}
\DeclareMathOperator{\Span}{span}
\DeclareMathOperator{\rg}{rg}
\DeclareMathOperator{\id}{id}					% Identitetafbildning - LinAlg
\newcommand{\morf}[3]{#1\colon #2 \to #3} 		% \morf{f}{A}{B} = f\colon A\to B
\newcommand{\morffFF}[3][f]{#1 \colon \Set{F}^{#2} \to \Set{F}^{#3}} % Afbildning i Set(F) - LinAlg
\newcommand{\U}[1]{\ushort{#1}} 				% vektornotation - LinAlg
\newcommand{\mat}[1]{\ushortd{#1}}				% Matrixnotation - LinAlg
\newcommand{\trans}[2]{{}_{#1}\mathsf{T}_{#2}}
\newcommand{\bas}[1]{\mathscr{#1}}
\newcommand{\ele}[2]{{}_{#1}[#2]}
\newcommand{\transf}[3]{{}_{#1}[#2]_{#3}}

% MatF 1 & 2
% Standardenhedsvektorer
\newcommand{\Vi}{\V{i}}
\newcommand{\Vj}{\V{j}}
\newcommand{\Vk}{\V{k}}
\newcommand{\grad}{\nabla}
\newcommand{\ind}[2]{\langle #1 | #2 \rangle}
\newcommand{\norm}[1]{\lVert #1 \rVert}
\DeclareMathOperator{\erf}{erf}
\DeclareMathOperator{\Ln}{Ln}


% EL1
\newcommand{\kc}[1][1]{\frac{#1}{4 \pi \epsilon_0}}
\def\sr{{\mbox{$\resizebox{.09in}{.08in}{\includegraphics[trim= 1em 0 14em 0,clip]{ScriptR}}$}}}
\def\bsr{{\mbox{$\resizebox{.09in}{.08in}{\includegraphics[trim= 1em 0 14em 0,clip]{BoldR}}$}}}
\def\usr{{\mbox{$\hat \bsr$}}}
\newcommand{\unit}[1]{\hat{\V{#1}}}
\newcommand{\Vx}{\unit{x}}
\newcommand{\Vy}{\unit{y}}
\newcommand{\Vz}{\unit{z}}

% Margen
\usepackage[margin=1in]{geometry}

% Max antal kolonner i en matrix. Default er 10
%\setcounter{MaxMatrixCols}{20}

% Hvor dybt skal kapitler labeles?
%\setcounter{secnumdepth}{0}		

% Til nummerering af ligninger. Så der står (afsnit.ligning) og ikke bare (ligning)
\numberwithin{equation}{section}		

% Header
%\usepackage{fancyhdr}
%\lhead{Nikolai Plambech Nielsen, 21-06-95\\Dato:. Klasse 5.C - De Fysiske Fag}
%\pagestyle{fancy}

%Titel
\title{Noter til EM1 på KU (Elektromagnetisme 1)}
\author{af Nikolai Plambech Nielsen, LPK331 \\ Version 1.0}


\begin{document}
	\selectlanguage{danish}
	%Udkommenter ovenstående, hvis du skriver på engelsk.

	\section*{Kort note om integraler og koordinattransformationer}
	Det er til tider gavnligt at skifte fra ét koordinatsystem til et andet - kartesisk til polært, eksempelvis. Man kan dog ikke ">bare"< skifte fra $ \ud x \ud y $ til $ \ud r \ud \theta $. Der skal ganges en ekstra faktor på funktionen i det nye koordinatsystem. Dette kaldes for ">Jakobianten"<, og er en specifik determinant. Dette blev meget, meget kort gennemgået til sidst i MatIntro (afsnit 5.2 og 5.3 i Funktioner af flere variable), men da stoffet fra LinAlg ikke er gennemgået, var det ikke muligt selv at regne Jakobianten.
	
	Den generelle koordinattransformation er givet ved:
	\begin{equation}
	\int f(x_1,x_2,x_3) \ud x_1 \ud x_2 \ud x_3 = \int f[x_1(y),x_2(y),x_2(y)] \D|\V{J}| \ \ud y_1 \ud y_2 \ud y_3
	\end{equation}
	Hvor prikken altså er helt almindelig multiplikation, og ikke et prikprodukt (det er skalarer, så det giver egentlig det samme, hvis skalarene ses som endimensionale vektorer, men whatever). I tilfældet af transformation fra kartesisk til sfærisk er $ x_1,x_2,x_3 = x,y,x $ og $ y_1,y_2,y_3 = r, \theta, \phi $. Den generelle Jakobiant $ \V{J} $ er givet ved
	\begin{equation}
	\V{J} = \begin{bmatrix}
	\diff{x_1(y)}{y_1} & \diff{x_1(y)}{y_2} & \diff{x_1(y)}{y_3} \\[0.5em]
	\diff{x_2(y)}{y_1} & \diff{x_2(y)}{y_2} & \diff{x_2(y)}{y_3} \\[0.5em]
	\diff{x_3(y)}{y_1} & \diff{x_3(y)}{y_2} & \diff{x_3(y)}{y_3} 
	\end{bmatrix}
	\end{equation}
	Hvor $ x_i(y) $ er koordinaten $ x_i $ udtrykt ved $ y $. Eksempelvis er $ x() = r \cos \theta = x $ i polære koordinater. For at gøre det lidt mere overskueligt, kan matricen ses som følger:
	\begin{equation}
	\V{J} = \begin{blockarray}{c c c c }
	& y_1 & y_2 & y_3 \\
	\begin{block}{c [ c c c ]}
	x_1(y) & \partial & \partial & \partial \\[0.5em]
	x_2(y) & \partial & \partial & \partial \\[0.5em]
	x_3(y) & \partial & \partial & \partial \\
	\end{block}
	\end{blockarray}
	\end{equation}
	Hvor ">rækkerne"< altså skal differentieres med hensyn til ">kolonnerne"<. For sfæriske koordinater, hvor $ y = r,\theta,\phi $ er Jakobianten givet ved
	\begin{equation}
	\V{J} = \begin{bmatrix}
	\diff{x}{r} & \diff{x}{\theta} & \diff{x}{\phi} \\[0.5em]
	\diff{y}{r} & \diff{y}{\theta} & \diff{y}{\phi} \\[0.5em]
	\diff{z}{r} & \diff{z}{\theta} & \diff{z}{\phi} 
	\end{bmatrix}
	\end{equation}
	For \textbf{plane polære koordinater} er $ x_1 = x $, $ x_2 = y $, $ y_1 = r $, $ y_2 = \theta $, og Jakobianten er givet ved
	\begin{align*}
	x &= r \cos \theta, \quad y = r \sin \theta\\
	\V{J}_{pol} &= \begin{bmatrix}
	\diff{r \cos \theta}{r} & \diff{r \cos \theta}{\theta} \\[0.5em]
	\diff{r \sin \theta}{r} & \diff{r \sin \theta}{\theta}
	\end{bmatrix} = \begin{bmatrix}
	\cos \theta & - r\sin \theta \\
	\sin \theta & r \cos \theta
	\end{bmatrix} \\
	|\V{J}_{pol}| &= r
	\end{align*}
	For \textbf{sfæriske koordinater} er
	\begin{equation*}
	x = r \sin \theta \cos \phi, \quad y = r \sin \theta \sin \phi, \quad z = r \cos \theta, \quad |\V{J}_{sph}| = r^2 \sin \theta 
	\end{equation*}
	Og for \textbf{cylinderkoordinater}
	\begin{equation*}
	x = r \cos \theta, \quad y = r \sin \theta, \quad z = z, \quad  |\V{J}_{cyl}| = r
	\end{equation*}
	Et fladeintegral for $ f(x,y) $ bliver da
	\begin{equation}
	\iint f(x,y) \ud x \ud y = \iint f[x(r,\theta), y(r,\theta)] \,r \ud r \ud \theta
	\end{equation}
	Og ligeledes for de andre koordinattransformationer.

\end{document}

