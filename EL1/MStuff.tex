\documentclass[EL1Noter.tex]{subfiles}


\begin{document}
	
	\section{Magnetiske felter i materialer}
	
	\subsection{Magnetisering}
	
	\subsubsection{Diamagneter, paramagneter og ferromagneter}
	I materialer er der to former for små strømme: elektroner, der bevæger sig rundt om atomkernerne, og elektroner, der roterer om sig selv. Med makroskopiske briller er begge disse så små, at begge fænomener kan betragtes som magnetiske dipoler. 
	
	Normalt set er alle de forskellige dipoler tilfældigt orienteret, så det samlede dipolmoment er 0, men hvis et eksternt magnetfelt indføres vil dipolerne rotere, og materialet bliver magnetisk polariseret - \textbf{magnetiseret}.
	
	For elektrisk polarisation, hvor de elektriske dipoler altid orienterer sig i retning af det elektriske felt, er dette ikke altid tilfældet for magnetisk polarisation. Nogle materialer bliver magnetiseret parallelt med $ \V{B} $ (paramagneter), mens nogen materialer bliver magnetiseret antiparallelt med $ \V{B} $ (diamagneter). En anden gruppe materialer, kaldet ferromagneter (hvor jern, eller ferrum, er det mest almindelige) beholder deres magnetisering efter det eksterne magnetiske felt fjernes igen.
	
	\subsubsection{Kraftmomenter og kræfter på magnetiske dipoler}
	En magnetisk dipol vil opleve et kraftmoment i et magnetisk felt, på samme måde som en elektrisk dipol i et elektrisk felt. For en rektangulær strømløkke (med sider $ a $ og $ b $) i et homogent magnetfelt $ \V{B} $ er kraftmomentet
	\begin{equation}
		\V{N} = \V{m} \times \V{B}
	\end{equation}
	hvor $ m = I ab$ er det magnetiske dipolmoment af løkken. Dette giver faktisk det eksakte kraftmoment på enhver lokal strømfordeling i et homogent felt. I et inhomogent felt er formlen det eksakte kraftmoment (om centrum) af en perfekt, infinitisemal dipol.
	
	Dette kraftmoment er i en retning, som får dipolen til at rette sig med feltet (paramagnetisme). Grundet kvantemekanik (Paulis udelukkelsesprincip) findes paramangetisme oftest kun i atomer eller molekyler med et ulige antal elektroner, idet elektroner findes i par med modsat spin (og dermed modsat magnetisk dipolmoment). Det er dermed den ekstra elektron, der giver dette kraftmoment. Denne effekt er dog langt fra fuldstændig, da termiske kollisioner arbejder mod denne samlede orientering.
	
	I et homogent felt er den samlede kraft på en strømløkke altid 0:
	\begin{equation}
		\V{F} = I \oint (\ud \V{l} \times \V{B}) = I \pp{\oint \ud \V{l}} \times \V{B} = 0
	\end{equation}
	mens i et inhomogent felt er dette ikke tilfældet. For en infinitisemal løkke, med dipolmoment $ \V{m} $ i et vilkårligt magnetfelt $ \V{B} $ er kraften
	\begin{equation}
		\V{F} = \grad (\V{m}\D \V{B})
	\end{equation}
	Det ses at både formlen for kraftmomentet og kraften om en infinitisemal løkke er på ens form som deres elektriske analoger.
	
	\subsubsection{Effekten af magnetiske felter på atomare kredsløb}
	Elektroner bevæger sig også omkring atomkernen. Dette sker så hurtigt, at det til en god approksimation er en jævn strøm, givet ved
	\begin{equation}
		I = \frac{e}{T} = \frac{e v}{2 \pi R}
	\end{equation}
	hvor $ e $ er elementarladningen, $ T = v/(2\pi R)$ er omløbstiden i et cirkulært kredsløb med hastighed $ v $ og radius $ R $. Dermed er dipolmomentet af dette system
	\begin{equation}
		\V{m} = -\frac{1}{2} ev R\Vz
	\end{equation}
	hvor minustegnet er til for at tage højde for elektronens negative ladning. Denne magnetiske dipol oplever selvfølgelig også et kraftmoment givet ved samme formel som før, men det er ret meget sværere at vippe et helt atomart kredsløb, end en enkelt elektron, så denne effekt er ret lille i kontekst af paramagnetisme.
	
	Til gengæld sker der det, at elektronernes hastighed stiger eller falder, alt efter manetfeltets retning. Dette sker, idet det cirkulære kredsløb normalt kun er grundet elektriske kræfter:
	\begin{equation}
		\kc[1] \frac{e^2}{R^2} = m_e \frac{v^2}{R}
	\end{equation}
	hvor $ v^2/R $ er centripetalaccelerationen. Hvis der er et B-felt til stede vil elektronen også opleve en kraft fra dette: $ -e(\V{v} \times \V{B}) $. Hvis magnetfeltet er vinkelret på kredsløbets plan, fås
	\begin{equation}
		\kc \frac{e^2}{R^2} + e\overline{v} B = m_e \frac{\overline{v}^2}{R}
	\end{equation}
	hvor $ \overline{v} $ er den nye hastighed. I dette tilfælde er den nye hastighed større end tilfældet kun med elektriske kræfter. Såfremt forskellen $ \Delta v = \overline{v}-v $ i hastighed er lille, er den approksimeret ved
	\begin{equation}
		\Delta v = \frac{eRB}{2m_e}
	\end{equation}
	Og en ændring i orbital hastighed giver et ændret dipolmoment:
	\begin{equation}
	\Delta \V{m} = -\frac{1}{2} e(\Delta v ) R \Vz = -\frac{e^2 R^2}{4 m_e} \V{B}
	\end{equation}
	hvor det ses, at ændringen i dipolmoment er i den modsatte retning af $ \V{B} $. Normalt er de atomare kredsløb orienteret tilfældigt, så deres dipolmomenter udslukker hinanden, men ved et eksternt magnetfelt får alle atomerne et ekstra lille dipolmoment, der er i modsat retning af $ \V{B} $, og altså antiparallelt til feltet. Dette er mekanismen bag \textbf{diamagnetisme}. Effekten er til stede i alle materialer, men er typisk set meget svagere end paramagnetisme, så ses oftest i atomer med et lige antal elektroner.
	
	Denne udledning antager dog, at kredsløbet forbliver cirkulært, hvilket måske ikke altid er tilfældet (faktisk er diamagnetisme et kvantemekanisk fænomen, så denne klassiske model er mere hullet end schweizerost), men det giver alligevel en god intuition for, hvad der er et empirisk faktum.
	
	\subsubsection{Magnetisation}
	Lige som polarisationen $ \V{P} $ i et materiale nært et elektrisk felt er givet ved det elektriske dipolmoment per volumenenhed, så defineres magnetisationen  $ \V{M} $ også som
	\begin{equation}
		\V{M} = \text{ magnetisk dipolmoment per volumenenhed}
	\end{equation}
	Magnetisationen kan enten være forskyldt af paramagnetisme eller diamagnetisme - det er i og for sig lige meget, og i de næste afsnit gøres der ingen tanker om \textit{hvordan} magnetisationen opstod - vi regner bare på feltet der dannes fra magnetisationen.
	
	Normalt set er både paramagnetisme og diamagnetisme meget, meget svage i forhold til ferromagnetisme (typisk 4-5 størrelsesordner mindre), hvilket også er grunden til, at man ikke selv mærker nogen effekt tæt ved magnetfelter (omkring 40 T er dog nok til at få en frø - der er diamagnetisk grundet vandinholdet - til at svæve).
	
	
	
	\subsection{Feltet af et magnetiseret objekt}
	\subsubsection{Bundne strømme}
	Det magnetiske vektorpotentiale af en enkelt dipol er givet ved
	\begin{equation}
		\V{A}(\V{r}) = \frac{\mu_0}{4 \pi} \frac{\V{m} \times \usr}{\sr^2}
	\end{equation}
	Og i et materiale med magnetisation, hvor hvert volumenelement $ \ud \tau' $ indeholder et dipolmoment $ \V{M}\ud \tau' $, er det samlede vektorpotentiale givet ved
	\begin{equation}
		\V{A}(\V{r}) = \frac{\mu_0}{4 \pi} \int \frac{\V{M}(\V{r}') \times \usr}{\sr^2} \ud \tau'
 	\end{equation}
 	Dette kan dog, på samme vis som for elektrisk polariserede materialer, omskrives til
 	\begin{equation}
	 	\V{A}(\V{r}) = \frac{\mu_0}{4 \pi} \int_{\mathcal{V}} \frac{\V{J}_b (\V{r})'}{\sr} \ud \tau' + \frac{\mu_0}{4 \pi} \oint_{\mathcal{S}} \frac{\V{K}_b(\V{r}')}{\sr} \ud a', \quad \V{J}_b = \grad \times \V{M}, \quad \V{K}_b = \V{M} \times \U{n}
 	\end{equation}
 	Hvor $ \V{J}_b $ er den bundne volumenstrøm, og $ \V{K}_b $ er den bundne overfladestrøm (helt som den elektriske analog).
 	
 	\paragraph{Et eksempel.} En homogent magnetiseret sfære. Lægges $ \V{M} $ langs $ z $-aksen fås: $ \V{J}_b = 0 $ og $ \V{K}_b = M \sin \theta \Vp $. For en roterende kugleskal med homogen overfladeladning er overfladestrømmen $ \V{K} = \sigma \omega R \sin \theta \Vp $, og det konkluderes da, at dette svarer til en homogent magnetiseret sfære med $ \sigma \Vg{\omega} R \to \V{M} $. Dermed er det magnetiske felt
 	\begin{equation}
	 	\V{B} = \frac{2}{3} \mu_0 \V{M}, \quad \text{inde i sfæren}
 	\end{equation}
 	Og udenfor sfæren er det magnetiske felt ens med en magnetisk dipol i midten, med dipolmoment
 	\begin{equation}
	 	\V{m} = \frac{4}{3} \pi R^3 \V{M}
 	\end{equation}
 	Det ses, at lige som den elektriske analog, er felterne homogene inde i sfærerne, også selvom de to formler er lidt forskellige (forskellige faktorer). Ligeledes svarer de begge (uden for sfæren) til en enkelt dipol.
 	
 	
 	\subsubsection{Det mangetiske feld inde i materialet}
	Lige som i det elektriske tilfælde, er det mikroskopiske magnetiske felt et virvar inde i materialet. Så når der snakkes om magnetiske felter i materialet, er det altid det \textit{makroskopiske} felt, som er et gennemsnit af et lille stykke, stort nok til at indeholde mange atomer, men småt nok til ikke at udvaske forskelle på stor skala. Det er også dette felt som udregnes, hvis formlerne for det forrige afsnit bruges.
	
	
	
	
	\subsection{Hjælpefeltet H}
	\subsubsection{Ampères lov i magnetiserede materialer}
	I magnetiserde materialer er magnetfeltet grundet både den \textit{frie} strøm og den \textit{bundne} strøm (fundet i sidste afsnit). På helt samme vis som for polariserede materialer, kan den samlede strøm skrives som
	\begin{equation}
		\V{J} = \V{J}_f + \V{J}_b
	\end{equation}
	Med dette i tankerne kan Ampères lov skrives ved
	\begin{equation}
		\frac{1}{\mu_0} (\curl{B}) = \V{J} = \V{J}_f+\V{J}_b = \V{J}_f + (\curl{M})
	\end{equation}
	Samles de to rotationer fås
	\begin{equation}
		\curl{H} = \V{J}_f, \quad \V{H} = \pp{\frac{1}{\mu_0} \V{B} - \V{M}}, \label{eq:AmpH}
	\end{equation}
	Hvor $ \V{H} $ ikke har noget navn. Det er bare H-feltet, eller $ \V{H} $. Dette er da Ampères lov i materialer, i differentialform, mens på integralform lyder den
	\begin{equation}
		\oint \V{H} \D \ud \V{l} = I_{f_{\text{enc}}}
	\end{equation}
	$ \V{H} $ er den magnetiske analog til $ \V{D} $, og vi kan finde den bare ved de frie strømme (eller frie ladninger, i $ \V{D} $'s tilfælde). På samme vis som vi kan bruge Gaussflader med $ \V{D} $, hvis der er symmetrier i systemet, kan vi også bruge Ampereløkker med $ \V{H} $, såfremt der er symmetri. De prototyper og vejledninger til Ampereløkker i afsnit \ref{sec:Ampere} kan altså bruges på helt samme måde i disse tilfælde.
	
	\paragraph{Et eksempel.} En lang kobberstang med radius $ R $ bærer en homogen fri strøm $ I $. Find $ H $ inden i og uden for stangen.
	
	Kobber er svagt diamagnetisk, så dipolerne retter sig antiparallelt med feltet. Det giver at den bundne volumenstrøm løber antiparallelt med $ I $, og den bundne overfladestrøm løber parallelt med $ I $.
	
	Idet alle strømme bevæger sig \textit{langs} stangen, må både $ \V{B} $, $ \V{M} $ og $ \V{H} $ løbe tangentielt til stangen - altså rundt om den ($ \Vp $). Med en cirkulær Ampereløkke med radius $ s$ fås
	\begin{equation}
		\V{H} = \begin{cases}
			\displaystyle
			\frac{I}{2 \pi R}s \Vp, & s\leq R \vspace{0.2 cm} \\
			\displaystyle
			\V{H} = \frac{I}{2 \pi s}, & s>R
		\end{cases}
	\end{equation}
	Og $ \V{B} $ kan nemt regnes uden for cylinderen (hvor $ \V{M} $ er 0):
	\begin{equation}
		\V{B} = \mu_0 \V{H} = \frac{\mu_0 I}{2 \pi s} \Vp, \quad s\geq R
	\end{equation}
	Inden i cylinderen kan det ikke regnes, da der ikke vides noget om $ \V{M} $.
	
	\subsubsection{En advarelse om parallellen mellem \texorpdfstring{$ \V{B} $ og $ \V{H} $}{B og H}}
	På samme vis som for $ E $ og $ D $, skal man passe på med at sige at $ \V{B} $ er præcis som $ \mu_0 \V{H} $, blot med $ \V{J}_f $ i stedet for $ \V{J} $, også selvom ligning \eqref{eq:AmpH} ligner Ampéres lov meget. Dette er fordi vi kun kender rotationen af $ \V{H} $, og ikke dens divergens, og man skal kende begge, for at bestemme et vektorfelt. Normalt set er divergensen af $ \V{H} $ nemlig ikke 0:
	\begin{equation}
		\diverg{H} = - \diverg{M} \label{eq:DivergH}
	\end{equation}
	Så kun når divergensen af magnetisationen forsvinder, kan der sættes et ordentligt lighedstegn mellem $ \V{B} $ og $ \mu_0 \V{H} $.
	
	\subsubsection{Randbetingelser}
	Det magnetostatiske randbetingelser kan omskrives så de kun refererer til $ \V{H} $, ved hjælp af \eqref{eq:DivergH}:
	\begin{equation}
		H^{\perp}_{\text{over}} - H^{\perp}_{\text{under}} = -(M^{\perp}_{\text{over}} - M^{\perp}_{\text{under}})
	\end{equation}
	og
	\begin{equation}
		\V{H}^{\parallel}_{\text{over}} - \V{H}^{\parallel}_{\text{under}} = \V{K}_f \times \U{n}
	\end{equation}
	
	
	
	
	\subsection{Lineære og ikkelineære medier}
	\subsubsection{Magnetisk susceptibilitet og permabilitet}
	Ligesom der er lineære dielektriske medier, er der også lineære paramagnetiske og diamagnetiske medier. Faktisk opfører de fleste medier lineært, såfremt det magnetiske felt ikke er alt for kraftigt. Her defineres proportionalitetskonstanten dog ved $ \V{H} $, og ikke $ \V{B} $ (som ellers er tilfældet i det elektriske tilfælde, hvor $ \V{E} $ bruges, og ikke $ \V{D} $). Ligningen lyder
	\begin{equation}
		\V{M} = \chi_m \V{H}
	\end{equation}
	hvor $ \chi_m $ kaldes for den \textbf{magnetiske susceptibilitet}. Denne er dimensionsløs, samt positiv for paramagnetiske materialer og negativ for diamagnetiske materialer. Typisk er den omkring størrelsesordenen $ 10^{-5} $. En tabel over nogle materialer ses herunder:
	\begin{table}[H]
		\centering
		\begin{tabular}{*{2}{l >{$} r <{$}}}
			\hline
			Materiale (dia) & \text{Susceptibilitet} & Materiale (para) & \text{Susceptibilitet} \\
			\hline
			Bismut 		& -1.6 \D 10^{-4} & Oxygen 		& 1.9 \D 10^{-6} \\
			Guld 		& -3.4 \D 10^{-5} & Natrium		& 8.5 \D 10^{-6} \\
			Sølv 		& -2.4 \D 10^{-5} & Aluminium 	& 2.1 \D 10^{-5} \\
			Kobber 		& -9.7 \D 10^{-6} & Wolfram		& 7.8 \D 10^{-5} \\
			Vand 		& -9.0 \D 10^{-6} & Platin 		& 2.8 \D 10^{-4} \\
			Kuldioxid 	& -1.2 \D 10^{-8} & Flydende oxygen (-200$\degree  $ C) & 3.9 \D 10^{-3} \\
			Hydrogen 	& -2.2 \D 10^{-9} & Gadolinium 	& 4.8 \D 10^{-1} \\
			\hline
		\end{tabular}
		\label{tab:chi_m}
		\caption{Tabel over magnetiske susceptibiliteter. Værdierne er for 1 atm og $ 20\degree $ C, med mindre andet er skrevet. Fra \textit{Handbook of Chemistry and Physics}, 67. udgave}
	\end{table}
	Materialer af denne type kaldes, selvfølgelige, lineære medier. $ \V{B} $ kan her skrives som
	\begin{equation}
		\V{B} = \mu_0 (\V{H}+\V{M}) = \mu_0 (1+\chi_m) \V{H}
	\end{equation}
	eller
	\begin{equation}
		\V{B} = \mu \V{H}, \quad \mu = \mu_0 \mu_r= \mu_0(1+\chi_m)
	\end{equation}
	Hvor $ \mu $ kaldes \textbf{permeabiliteten} af materialet. I vakuum er $ \chi_m=0 $ og permeabiliteten er her lig $ \mu_0 $, hvilket er grunden til det hedder vakuumpermeabiliteten. $ \mu_r =(1+\chi_m)=\mu/\mu_0$ kaldes for den \textbf{relative permeabilitet}.
	
	I lineære medier, hvor både $ \V{H} $ og $ \V{M} $ er proportionelle med $ \V{B} $ forsvinder divergensen af $ \V{H} $ desværre stadig ikke altid: I grænsen mellem materialer med forskellig permeabilitet kan divergensen nemlig være uendelig. Hvis der tegnes en Gaussisk dåse omkring en grænse fås at $ \oint \V{M} \D \ud \V{a} \neq 0 $, og dermed siger divergenssætningen, at $ \diverg{M} $ ikke kan være 0 over det hele i dette område. 
	
	Til gengæld, er den bundne volumenstrøm i et homogent lineært materiale proportionelt med den frie volumenstrøm:
	\begin{equation}
		\V{J}_b = \curl{M} = \grad \times (\chi_m \V{H}) = \chi_m \V{J}_f
	\end{equation}
	Det vil altså sige, at med mindre der rent faktisk løber en strøm gennem materialet, vil al bunden strøm være overfladestrøm.
	
	\paragraph{Et eksempel.}
	En uendelig solenoide med $ n $ vindinger per enhedslængde og strøm $ I $ løbende gennem, er fyldt med et lineært materiale, med susceptibilitet $ \chi_m $. Find det magnetiske felt inde i solenoiden.
	
	$ \V{B} $ skyldes både frie og bundne strømme, så den kan ikke regnes endnu. til gengæld kan $ \V{H} $ regnes ved hjælp af Ampères lov i materialer. Da fås
	\begin{equation}
		\V{H} = nI\Vz
	\end{equation}
	Da fås $ \V{B} $ til
	\begin{equation}
		\V{B} = \mu_0 (1 + \chi_m) n I \Vz
	\end{equation}
	Hvis materialet er paramagnetisk, er feltet en smule kraftigere end i en leder, mens hvis det er diamagnetisk er det en smule svagere. Dette ses også ved den bundne overfladestrøm
	\begin{equation}
		\V{K}_f = \chi_m(\V{H} \times \U{n}) = \chi_m n I \Vp
	\end{equation}
	Denne er i samme retning som I for paramagnetisme, men modsat for diamagnetisme.
	
	
	
	
\end{document}