\documentclass[EL1Noter.tex]{subfiles} % HUSK FOR FANDEN AT REDIGERE DENNE LINJE

\begin{document}
	\section{Elektriske felter i materialer}
	\subsection{Polarisering}
	Indtil videre er ét af de to vigtigste typer af materialer inden for elektromagnetisme blevet beskrevet - ledere. Det andet er isolatorer eller dielektrikum. I ledere er der en nærmest uendelig mængde tilgængelige elektroner, som kan bevæge sig frit mellem materialet. Dette er ikke tilfældet for isolatorer. Her er elektronerne bundet til materialet, og kan kun bevæge sig i et begrænset omfang.
	
	Men de kan bevæge sig, og hvis et dielektrikum bliver udsat for et elektrisk felt, kan der ske to forskellige ting, enten ">strækkes"< eller ">roteres"<.
	
	\subsubsection{Inducerede dipoler}
	Når et neutralt atom placeres i et elektrisk felt vil den positive kerne bevæge sig med feltet, mens den negative elektronsky vil bevæge sig mod feltet. Lige så snart de gør dette, vil dette skabe et nyt elektrisk felt, da atomet nu er en dipol. Hvis det elektriske felt er kraftigt nok, kan feltet rive atomet fra hinanden, hvormed det bliver ioniseret. Såfremt det elektriske felt ikke er alt for kraftigt vil afstanden, som elektronskyen bevæger sig, være så lille førend et ækvilibrium er nået, at den approksimativt vil beholde sin sfæriske form. Denne inducerede dipol har et dipolmoment, der peger i samme retning som de eksterne elektriske felt, og styrken af dipolen er proportional med det elektriske felt (igen, hvis E-feltet ikke er alt for kraftigt):
	\begin{equation}
		\V{p} = \alpha \V{E} \label{eq:inducerdipol}
	\end{equation}
	her er $ \alpha $ proportionalitetskonstanten, og kaldes for den \textbf{atomare polarisabilitet} (polariseringsevne). Nogle værdier for $ \alpha $ er tabuleret herunder
	\begin{table}[H]
		\centering
		\begin{tabular}{*{10}{c}}
			H & He & Li & Be & C & Ne & Na & Ar & K & Cs \\
			\hline
			0.667 & 0.205 & 24.3 & 5.60 & 1.76 & 0.396 & 24.1 & 1.64 & 43.4 & 59.6 \\
		\end{tabular}
		\label{tab:Polarisabilitet}
		\caption{Atomisk polarisabilitet, $ \alpha/4\pi\epsilon_0 $ i enheder af $ 10^{-30} \text{m}^3 $, fra \textit{Handbook of Chemistry and Physics}, 78. udgave.}
	\end{table}
	En primitiv model af et atom, hvis kerne har en ladning $ q $ og hvor elektronskyen er homogent ladet med $ -q $, og har radius $ a $, vil have en atomar polarisabilitet givet ved
	\begin{equation}
		\alpha = 4\pi \epsilon_0 a^3 = 3\epsilon_0 v
	\end{equation}
	hvor $ v $ er atomets volumen. Dette resultat er som regel inden for en faktor 4 af den egentlige værdi for simple atomer.
	
	For molekyler, der jo ikke er sfærisk symmetriske, vil polarisationen afhænge af det elektriske felts retning. For atomer som $ CO_2 $, der danner en ret linje, skal dipolmomentet splittes op i parallelle og retvinklede komponenter:
	\begin{equation}
		\V{p} = \alpha_{\perp} \V{E}_{\perp} + \alpha_{\parallel} \V{E}_{\parallel}
	\end{equation}
	I sådanne tilfælde er det muligt, at den inducerede dipol ikke peger i samme retning som $ \V{E} $. For andre atomer, der er fuldstændige asymmetriske, erstattes \eqref{eq:inducerdipol} med dennes mest generelle form:
	\begin{align}
		p_x &= \alpha_{xx}E_x+\alpha_{xy} E_y + \alpha_{xz}  E_z \nonumber \\
		p_y &= \alpha_{yx}E_x+\alpha_{yy} E_y + \alpha_{yz}  E_z \\
		p_z &= \alpha_{zx}E_x+\alpha_{zy} E_y + \alpha_{zz}  E_z \nonumber
	\end{align}
	hvor de ni $ \alpha_{ij} $ kaldes for \textbf{polarisabilitetstensoren} for molekylet. Deres værdi afhænger af orienteringen af koordinatsystemet, der bruges, men det kan altid orienteres, således at alle konstante, der \textit{ikke} er på diagonalen, forsvinder.
	
	\subsubsection{Orientation af polære molekyler}
	Til forskel fra neutrale atomer, har nogle molekyler permanente dipolmomenter. Et eksempel er vand, der er bøjet i en vinkel på $ 105\degree $, og har en negativ pol ved oxygen-atomet (i ">spidsen"<), samt to positive poler ved hydrogen-atomerne (i ">enderne"<). Dennes dipolmoment er på $ 6.1 \D 10^{-30} \e{C}\D \e{m} $, hvilket også er grunden til, at den er et godt opløsningsmiddel. Molekyler af denne art kaldes \textbf{polære molekyler}.
	
	Når et polært molekyle udsættes for et \textbf{homogent} elektrisk felt, vil kraften på den positive og negative ende af atomet være lige store, således at de udligner hinanden. Molekylet oplever dog et kraftmoment, givet ved
	\begin{equation}
		\V{N} = \V{p} \times \V{E}  \label{eq:PolarTorque}
	\end{equation}
	der får molekylet til at pege i retning af feltet - præcist som ved normale dipoler.
	
	Hvis feltet er \textbf{inhomogent} vil kræfterne ikke udlignes, og der vil være en resulterende kraft på molekylet, ud over kraftmomentet. Denne kraft er givet ved
	\begin{equation}
		\V{F} = (\V{p} \D \grad ) \V{E}
	\end{equation}
	Idet molekyler er så små, som de er, skal feltet dog ændres meget, over en meget kort afstand, førend dette har nogen effekt. 
	
	For en perfekt dipol, med en infinitisemal længde, giver \eqref{eq:PolarTorque} formlen for kraftmomentet omkring centrum af dipolen, selv i et inhomogent felt. Omkring ethvert andet punkt er kraftmomentet $ \V{N} = (\V{p} \times \V{E}) + (\V{r}+\V{F}) $.
	
	\subsubsection{Polarisering}
	Når et dielektrisk materiale placeres i et elektrisk felt, vil der induceres elektriske dipoler i materialet. Være dette enkelte neutrale atomer eller hele molekyler, afhænger af materialet. Hvis det er rene neutrale atomer, eller ikke-polære molekyler, vil hver af disse blive til en lille dipol, der peger i retning af feltet (dette gælder dog ikke nødvendigvis for asymmetriske molekyler). Hvis materialet består af polære molekyler, der har en permanent dipol, vil hver af disse opleve et kraftmoment, som roterer dem i retning af feltet (termiske fluktuationer gør dog, at denne proces aldrig er helt fuldstændig).

	I begge tilfælde bliver der induceret en masse små dipoler, der alle peger i samme retning som feltet, og materialet siges at være \textbf{polariseret}. Et mål for dette er \textbf{polarisationen}
	\begin{equation}
		\V{P} = \text{dipolmoment per volumenenhed}
	\end{equation}	
	Og det samlede dipolmoment er da $ v\V{P} $, hvor $ v $ er volumenet af materialet.
	
	
	\subsection{Feltet af et polariseret objekt}
	\subsubsection{Bundne ladninger}
	For et polariseret materiale, hvor $ \V{P} $ er opgivet, er potentialet givet ved
	\begin{equation}
		V(\V{r}) = \kc[1] \int_{\mathcal{V}} \frac{\usr \D \V{P}(\V{r}')}{\sr^2} \ud \tau '
	\end{equation}
	Hvor integralet er fået ved at omskrive potentialet for en enkelt dipol, til potentialet for en kontinuert volumenfordeling af dipoler. Dette kan omskrives ved lidt smart matematik til 
	\begin{equation}
		V(\V{r}) = \kc[1] \oint_{\mathcal{S}} \frac{\sigma_b}{\sr} \ud a' + \kc[1]  \int_{\mathcal{V}} \frac{\rho_b}{\sr} \ud \tau', \quad \sigma_b = \V{P} \D \U{n}, \quad \rho_b = -\grad \D \V{P}
	\end{equation}
	hvor $ \sigma_b $ kaldes den bundne overfladeladning og $ \rho_b $ kaldes den bundne volumenladning. Volumenintegralet skal integreres over den givne volumen, mens overfladeintegralet integreres over volumenets overflade. Ladningen kaldese bundet, idet den ikke er fri, og dermed ikke kan bevæge sig rundt mellem atomerne i materialet.
	
	Dette vil sige, at potentialet af et polariseret materiale er det samme, som feltet dannet af en volumenladning $ \rho_b $ og en overfladeladning $ \sigma_b $.
	
	\textbf{Et eksempel} på dette er feltet produceret af en \textbf{homogent polariseret sfære}, med radius $ R $. Her er $ \sigma_b = P \cos \theta $ og $ \rho_b = 0 $, da $ \V{P} $ er homogen. Potentialet er fundet i eksempel 3.9, og ved differentiation fås at
	\begin{equation}
		\V{E} = -\frac{1}{3 \epsilon_0} \V{P}, \quad r<R
	\end{equation}
	og
	\begin{equation}
		V = \kc[1] \frac{\V{p} \D \Vr}{r^2}, \quad r\geq R
	\end{equation}
	
	
	
	\subsubsection{Feltet inde i et dielektrikum}
	I udledelsen af polarisering inde i materialer, er der ikke gjort nogen stor forskel på matematiske og fysiske dipoler. Dipolerne i materialerne er selvfølgelig fysiske, idet de har en ikke-infinitisemal udstrækning, men i udledningen af potentialet af et polariseret materiale, blev der taget udgangspunkt i potentialet for en matematisk dipol.
	
	Uden for materialet gør manglen på distinktion mellem de to typer ingenting, idet separationsafstanden $ \sr $ er mange, mange gange større end de enkelte dipoler, og dipolledet dominerer da enormt.
	
	Inde i materialet er det dog en anden historie. I det helt mikroskopiske varierer feltet enormt, idet tæt på en elektron er dennes felt enormt, mens øjeblikket efter er elektronen et andet sted henne. Når der snakkes om det elektriske felt i et materiale, snakkes der om det \textbf{makroskopise felt}, der er gennemsnittet af feltet i et område, der er stort nok til at indeholde tusinder af dipoler, men småt nok til fluktuationer på stor skala ikke udvaskes. Dette er fuldstændig analogt til behandling af densitet i objekter - kigges der tæt nok på atomerne, stiger densiteten voldsomt i nærheden af atomkernerne, mens den falder til 0 lige ude for disse.
	
	Når der bruges formlerne i dette afsnit er det dog det elektriske felt fra disse makroskopiske områder, der opnås. Udledningen af dette kan ses i afsnit 4.2.3, der har samme navn som afsnittet du netop nu læser.
	
	
	
	
	
	\subsection{Elektrisk forskydning}
	\subsubsection{Gauss's lov i nærheden af dielektrikum}
	Ved et polariseret dielektrikum er det samlede E-felt givet ved summen af det eksterne E-felt, samt E-feltet fra det polariserede dielektrikum. I selve dielektriket er den samlede ladningsdensitet givet ved
	\begin{equation}
		\rho = \rho_b+\rho_f
	\end{equation}
	hvor $ b $ hentyder til bundne ladninger, og $ f $ til ">frie"< ladninger - ladninger der \textit{ikke} er forårsaget af polarisering. Dette kan eksempelvis være elektroner i en leder eller ioner i dielektriket. Gauss's lov for dette system lyder da:
	\begin{equation}
		\epsilon_0 \grad \D \V{E} = \rho = \rho_b+\rho_f = -\grad \D \V{P} + \rho_f
	\end{equation}
	Hvor $ \V{E} $ er det \textbf{samlede} elektriske felt for systemet - både det eksterne og dét, skabt af det polariserede dielektrikum.
	 
	I formlen kan de to gradienter slås sammen, og ">differentienterne"< (det der skal tages gradienten af), kaldes da for den elektriske forskydning $ \V{D} $:
	\begin{equation}
		\grad \D \V{D} = \rho_f, \quad \V{D} = \epsilon_0 \V{E} + \V{P}
	\end{equation}
	Dette er Gauss's lov for dielektrikum, i differentialform. I integralform bliver dette
	\begin{equation}
		\oint \V{D} \D \ud \V{a} = Q_{f_{\text{enc}}}	
	\end{equation}
	Disse to formler har ingen reference til $ \rho_b $, og de bundne ladninger, hvilket gør dem rigtig smarte til udregning af det elektriske felt i nærheden af dielektrikum. 
	
	Det ses dog, at $ \sigma_b $ ikke optræder nogen steder i denne udledning. Dette er fordi Gauss's lov ikke fungerer direkte på overfladeladninger. Alle andre steder er logikken ganske fin at bruge. Ydermere er en mere realistisk model, at dielektrikets sider har en endelig tykkelse, hvor polariseringen går mod 0, hvilket gør at $ \rho_b $ er kontinuert over hele dielektriket, og Gauss's lov kan bruges over det hele. Lige meget hvad, så indeholder integralformen ikke denne defekt, og den kan med god samvittighed bruges.
	
	\subsubsection{En advarsel om parallellen mellem \texorpdfstring{$ \V{E} $ og $ \V{D} $}{E og D}}
	Gauss's lov for dielektrikum ligner den almindelige lov utroligt meget, bare med $ \rho $ udskiftet med $ \rho_f $ og $ \epsilon_0 $ udskiftet med $ \V{D} $, så man kan fristes til at tro, at $ \V{D} $ er helt analogt til $ \V{E} $. Dette er dog ikke tilfældet, og man kan ikke \textit{bare} løse dielektrikumproblemer ved at se helt bort fra den bundle ladning $ \rho_b $, løse det som et normalt problem, og udskifte $ \V{D} $ med $ \V{E} $. Grunden til dette er blandt andet, at der ikke er nogen ">Coulombs lov"< for $ \V{D} $, og fordi $ \grad \times \V{D} $ ikke altid er lig 0: $ \grad \times \V{D} = \grad \times \V{P} $. Og $ \grad\times \V{P} $ er bestemt ikke altid 0.
	
	Til gengæld, hvis der er symmetri i problemet, kan Gauss's lov for $ \V{D} $ bruges helt uden problemer, da dette automatisk sørger for, at $ \grad \times \V{P} = 0 $. Men hvis \textbf{ikke} dette er tilfældet, må anden tankegang til, og man kan ikke bare antage, at $ \V{D} $ kan bestemmes kun ud fra den frie ladning.
	
	\subsubsection{Randbetingelser}
	Randbetingelserne fra afsnit \ref{seq:E-graenser} kan omskrives, så det beskrives ved $ \V{D} $:
	\begin{equation}
		D_{\text{over}}^{\perp} - D_{\text{under}}^{\perp} = \sigma_f, \quad \V{D}_{\text{over}}^{\parallel} - \V{D}_{\text{under}}^{\parallel} = \V{P}_{\text{over}}^{\parallel} - \V{P}_{\text{under}}^{\parallel}
	\end{equation}
	
	
	
	
	\subsection{Lineære dielektrikum}
	\subsubsection{Susceptibilitet, permitivitet og den dielektriske konstant}
	Normalt er polarisationen $ \V{P} $ af et materiale opstået grundet et eksternt elektrisk felt, der får de elektriske dipoler i materialet (atomer/molekyler) til at orientere sig parallelt med feltet. I mange materialer er polarisationen lineært proportional med det elektriske felt, såfremt denne ikke er alt for stor:
	\begin{equation}
		\V{P} = \epsilon_0 \chi_e \V{E}_{\text{Alle}}
	\end{equation} 
	Hvor $ \chi_e $ kaldes den \textbf{elektriske susceptibilitet} (elektriske tilbøjelighed) af mediet. Faktoren $ \epsilon_0 $ er til for at gøre denne dimensionsløs.	Materialer af denne type, hvor polarisationen er lineært proportional med det elektriske felt, kaldes for lineære dielektrikum.
	
	$ \V{E} $ i denne ligning er \textbf{hele} det elektriske felt, både det eksterne og det fra polarisationen selv. Det vil sige, at hvis et lineært dielektrikum placeres i et eksternt E-felt $ \V{E}_0 $ vil materialet blive polariseret, og danne et nyt elektrisk felt, der igen polariserer dielektriket mere, og skaber et større elektrisk felt, und so weiter - en uendelig serie af polarisering og elektriske felter (heldigvis er det en konvergent serie, ellers ville verden ikke rigtig fungere. Så hurra for det!).
	
	Det nemmeste er oftest at starte med den elektriske forskydning $ \V{D} $. For lineære medier fås:
	\begin{equation}
		\V{D} = \epsilon_0 \V{E} + \V{P} = \epsilon_0 (1+\chi_e) \V{E} = \epsilon \V{E}, \quad \epsilon = \epsilon_0 (1+\chi_e) \label{eq:EasyD}
	\end{equation}
	hvor $ \epsilon $ kaldes for materialets \textbf{permitivitet}. $ \V{E} $ er i denne ligning dog kun det \textbf{eksterne} elektriske felt. I et vakuum er der intet materiale at polarisere, så $ \chi_0 =0$ og $ \epsilon=\epsilon_0 $, hvilket også er grunden til, at $ \epsilon_0 $ kaldes for vakuumpermitiviteten. Hvis faktoren af $ \epsilon_0 $ fjernes fra permitiviteten, kaldes den dimensionsløse rest for den \textbf{relative permitivitet} eller \textbf{dielektriske konstant}:
	\begin{equation}
		\epsilon_r = 1+\chi_e = \frac{\epsilon}{\epsilon_0}
	\end{equation}
	Herunder er en tabel over nogle almindelige materialer og deres dielektriske konstant:
	
	
	\begin{table}[H]
		\centering
		\begin{tabular}{ll|ll}
			Materiale					& Dielektriske konstant	& Materiale 					& Dielektriske konstant \\
			\hline
			Vakuum 						& 1 					& Benzene 						& 2.28 \\
			Helium 						& 1.000065 				& Diamant 						& 5.7 \\
			Neon 						& 1.00013 				& Salt 							& 5.9 \\
			Hydrogen 					& 1.00025 				& Silicium 						& 11.8 \\
			Argon 						& 1.00052 				& Methanol 						& 33.0 \\
			Luft (tør) 					& 1.00054 				& Vand 							& 80.1 \\
			Nitrogen 					& 1.00055 				& Ice (-30$ \degree $ C) 		& 99 \\
			Vanddamp (100$ \degree $ C)	& 1.00587 				& KTaNbO$ _3 $ (0$ \degree $ C)	& 34,000			
		\end{tabular}
		\label{tab:DielektrikKonstant}
		\caption{Tabel over dielektriske konstanter, ved 1 atm, 20$ \degree $ C (medmindre andet er opgivet). Fra \textit{Handbook of Chemistry and Physics}, 78. udgave.}
	\end{table}
	En lader kan i nogen henseende ses som et dielektrikum i grænsen hvor $ \chi_e \to \infty $, hvilket kan være en god måde at tjekke sine løsninger på.
	
	I lineære materialer er rotationen af $ \V{D} $ stadig ikke altid 0. Dette ses, ved at integrere om en lukket kurve, der går gennem grænsen mellem to lineære dielektriske materialer. Linjeintegralet for $ \V{E} $ er her 0, da $ \V{E} $ er rotationsfrit (følger fra Stokes' sætning), men fordi $ \V{P} $ kan have forskellig værdi over og under grænsen, og dermed er kurveintegralet heller ikke 0. Det giver (igen med Stokes' sætning) at rotationen ikke kan være 0 overalt, inden for den lukkede kurve. Faktisk er rotationen uendelig ved grænsen.
	
	Til gengæld, hvis rummet er fyldt med et homogent, lineært dielektrisk materiale, vil rotationen være 0 (egentlig skal materialet bare være homogent, der hvor $ \V{E} $ er forskellig fra 0, da steder hvor $ \V{E}=0 $ ikke er polariserende). Her gælder at $ \grad \D\V{D} = \rho_f $ og $ \grad \times \V{D} = 0 $. Dermed kan $ \V{D} $ findes blot ved den frie ladning:
	\begin{equation}
		\V{D} = \epsilon_0 \V{E}_{\text{vac}}
	\end{equation}
	hvor $ \V{E}_{\text{vac}} $ er det felt, som den frie ladning ville lave i et vakuum. Da gives det elektriske felt ved
	\begin{equation}
		\V{E} = \frac{1}{\epsilon} \V{D} = \frac{1}{\epsilon_r} \V{E}_{\text{vac}}
	\end{equation}
	
	Dette vil altså sige, at feltet i sådanne områder (hvor dielektriket et homogent, ud over steder hvor $ \V{E}=0 $), vil dette reduceres med en faktor $ \epsilon\inverse_r $. Eksempelvis er ladningen fra en punktpartikel $ q $, der sidder inde i et stort dielektrikum, givet ved
	\begin{equation}
		\V{E} = \frac{1}{4\pi\epsilon} \frac{q}{r^2}\Vr
	\end{equation}
	(læg mærke til at det \textit{ikke} er $ \epsilon_0 $, men bare $ \epsilon $). Det dielektriske materiale ">skærmer"< altså for det elektriske felt, ved at der induceres dipoler, som modvirker det elektriske felt.
	
	Et eksempel er en parallel pladekapacitor, der er fyldt med et isolerende materiale med den relative permitivitet $ \epsilon_r $. Her er $ \V{E}=\frac{1}{\epsilon_r} \V{E}_{\text{vac}} $, hvilket gør at også $ V $ reduceres. Dog vil kapacitansen $ C=Q/V $ øges med faktoren $ \epsilon_0$, og den samlede kapacitans er nu
	\begin{equation}
		C = \epsilon_r C_{\text{vac}} = \epsilon_r \epsilon_0 \frac{A}{d}
	\end{equation}
	
	I en krystal er det dog ikke altid helt lige så nemt. Her er det oftest nemmere at polarisere materialet i nogle retninger. Her er den elektriske susceptibilitet udskiftet med en susceptibilitetstensor:
	\begin{align}
		P_x = \epsilon_0 (\chi_{e_{xx}} E_x + \chi_{e_{xy}} E_y + \chi_{e_{xz}} E_z) \nonumber \\
		P_x = \epsilon_0 (\chi_{e_{yx}} E_x + \chi_{e_{yy}} E_y + \chi_{e_{yz}} E_z)  \\
		P_x = \epsilon_0 (\chi_{e_{zx}} E_x + \chi_{e_{zy}} E_y + \chi_{e_{zz}} E_z) \nonumber
	\end{align}
	Lige som ved polarisabilitetstensoren for inducerede dipoler i dielektrikum. Materialer, der \textit{ikke} har en tensor, men blot en konstant, kaldes for \textbf{isotrope}, og susceptibiliteten er altså uafhængig af rotation (rotationel symmetri).
	
	
	
	\subsubsection{Randbetingelsesproblemer med lineære dielektrikum}
	I et homogent, lineært dielektrikum er den bundne volumenladning $ \rho_b $ proportional med den frie volumenladning $ \rho_f $ ved
	\begin{equation}
		\rho_b = -\grad \D \V{P} = -\frac{\chi_e}{1+\chi_e} \rho_f
	\end{equation}
	Og hvis der ikke er nogen fri ladning i materialet, er $ \rho = 0 $, hvormed al ladning må sidde på overfladen. Det byder da, at potentialet overholder Laplaces ligning, og metoderne fra afsnittet ">Specielle teknikker"< kan bruges. Her kan randbetingelserne omskrives, så de kun refererer til den frie ladning:
	\begin{equation}
		\epsilon_{\text{over}} E_{\text{over}}^{\perp} - \epsilon_{\text{under}} E_{\text{under}}^{\perp} = \sigma_f 
	\end{equation}
	Og potentialet:
	\begin{equation}
		\epsilon_{\text{over}} \diff{V_{\text{over}}}{n} - \epsilon_{\text{under}}  \diff{V_{\text{under}}}{n} = (\epsilon_{\text{over}}-\epsilon_{\text{under}}) \diff{V}{n} = - \sigma_f
	\end{equation}
	
	Et eksempel er en homogen sfære af et lineært dielektrikum, der placeres i et homogent elektrisk felt $ \V{E}_0 $. Her ses det (efter en masse udregninger), at elektriske felt inde i sfæren også er homogent, og givet ved
	\begin{equation}
		\V{E} = \frac{3}{\epsilon_r + 2} \V{E}_0
	\end{equation}
	I tilfældet $ \epsilon_r = 1+\chi_e \to \infty $ bliver $ \V{E} = 0 $. Dette er præcis som forventet, hvis dielektriket udskiftes med en leder).
	
	Et andet eksempel er en ladning $ q $ i en afstand $ d $ over origo ($ z $-retning), hvor hele rummet under $ z=0 $ er fyldt med et homogent, lineært dielektrikum, med susceptibilitet $ \chi_e $. Her er den inducerede overfladeladning givet ved
	\begin{equation}
		\sigma_b = -\frac{1}{2\pi} \frac{\chi_e}{\chi_e+2} \frac{qd}{(r^2+d^2)^{3/2}}
	\end{equation}
	Og den samlede inducerede ladning er
	\begin{equation}
		q_b = -\frac{\chi_e}{\chi_e+2} q
	\end{equation}
	hivilket er præcis det samme, bortset fra faktoren $ \chi_e/(\chi_e+2) $, som for en uendelig leder i $ xy $-planen. Igen, hvis $ \chi_e \to \infty$ går den ekstra faktor mod 1, og de oprindelige udtryk fås igen. Kraften er her givet ved
	\begin{equation}
		\V{F} = -\kc[1] \frac{\chi_e}{\chi_e+2} \frac{q^2}{4d^2} \Vz
	\end{equation}
	
	
	\subsubsection{Energi i dielektriske systemer}
	For at oplade en almindelig pladekapacitor skal der bruges et arbejde givet ved $ W = 1/2 CV^2 $. For en pladekapacitor med et lineært dielektrikum i stedet for vakuum, er $ C = \epsilon_r C_{\text{vac}} $, og arbejdet da $ 1/2 \epsilon_0 C_{\text{vac}} V^2 $.
	
	Det generelle udtryk for energien i et elektrostatisk system skal da også ganges med en faktor $ \epsilon_r $:
	\begin{equation}
		W = \frac{\epsilon_0}{2} \int \epsilon_r E^2 \ud \tau = \frac{1}{2} \int \V{D}  \D \V{E} \ud \tau \label{eq:energidielektrik}
	\end{equation}
	Begge formler er stadig rigtige, men situationerne de beskriver er lidt forskellige fra hinanden. I den generelle (hvor integranten er $ E^2 $), er det den \textbf{samlede} ladning (altså både fri og bunden ladning), der bringes ind fra uendeligt langt væk, og ">limes fast"< i deres position, mens i den specifikke for dielektrikum er det kun den frie ladning der bringes ind, og dielektriket der reagerer på dette, ved at inducere dipoler.
	
	I den første situation tages der dog ikke højde for det arbejde, der går til at rotere og strække de dielektriske molekyler, mens i den anden situation ses der helt bort fra både det ">bundne"< arbejde og ">fjederenergien"< der går til at rotere og strække molekylerne. Dette kan gøres, idet den totale energi er givet ved
	\begin{equation}
		W_{\text{tot}} = W_{\text{fri}}+W_{\text{bunden}}+W_{\text{fjeder}}
	\end{equation}
	hvor de sidste to er lige store og modsatrettede, så den specifikke formel, hvor der kun regnes på den frie ladning, giver altså den totale energi i systemet (hvis der ses bort fra, hvordan dielektriket kom til).
	
	Dermed skal der altså normalt bruges \eqref{eq:energidielektrik} i situationer med dielektrikum, da man ellers kommer til at regne på en helt anden situation.
	
	
	
	
	\subsubsection{Kræfter på dielektrikum i en pladekapacitor}
	Et dielektrikum er, på samme måde som en leder, tiltrukket ind i et elektrisk felt. Til gengæld er udregning for dielektrikum oftest ret svære. Eksempelvis på en pladekapacitor er der faktisk et ">kantfelt"<, der \textit{ikke} er homogent (i modsætning til den ">indre"< del af kapacitoren) langs kanterne. Dette felt er svært at beregne, men man kan faktisk komme helt uden om det.
	
	Hvis en plade af dielektrikum indsættes et stykke ved inde i kapacitoren, og $ W $ betegner systemets energi (der afhænger af, hvor langt inde dielektriket er), da vil energien der skal til for at trække dielektriket en infinitisemal afstand ud af kapacitoren være givet ved
	\begin{equation}
		\uuud W = F_{\text{mig}} \uuud x
	\end{equation}
	hvor $ F_{\text{mig}} $ er den kraft jeg påvirker pladen med, som er modsat den kraft som kapacitoren trækker dielektriket ind med: $ F_{\text{mig}} = -F $. Da er kraften på dielektriket givet ved
	\begin{equation}
		F =-\diff[\ud]{W}{x}
	\end{equation}
	Energien i kapacitoren er $ W = 1/2\ C V^2 $ og kapacitansen er
	\begin{equation}
		C = \frac{\epsilon_0 w}{d} (\epsilon_r l - \chi_e x)
	\end{equation}
	hvor $ w $ er bredden af pladen, $ l $ er længden af pladen og $ x $ er afstanden fra den ene ende af kapacitoren til dielektriket (altså hvor stor afstand der er tilbage, førend kapacitoren er helt fyldt med dielektrikum). Hvis det antages at $ Q $ er konstant, så $ W =1/2\ Q^2/C $, da er kraften givet ved
	\begin{align}
		F &= -\diff[\ud]{W}{x} = \frac{1}{2} \frac{Q^2}{C^2} \diff[\ud]{C}{x} = \frac{1}{2} V^2 \diff[\ud]{C}{x}, \quad \diff[\ud]{C}{x} = -\frac{\epsilon_0 \chi_e w}{d} \nonumber \\
		&= -\frac{\epsilon_0 \chi_e w}{2d} V^2
	\end{align}
	
	
	
\end{document}