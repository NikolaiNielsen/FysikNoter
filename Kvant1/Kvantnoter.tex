% Nikolai Nielsens "Fysiske Fag" preamble
\documentclass[a4paper,10pt]{article} 	% A4 papir, 10pt størrelse
\usepackage[danish]{babel}
\usepackage{Nikolai} 					% Min hjemmelavede pakke
\renewcommand{\konj}{^*}
\usepackage[dvipsnames]{xcolor}
\usepackage[thinlines]{easytable}
% Margen
\usepackage[margin=1in]{geometry}

\newtheorem{theo}{Teorem}
\newtheorem{ax}{Aksiom}

% Max antal kolonner i en matrix. Default er 10 Poop
%\setcounter{MaxMatrixCols}{20}

% Hvor dybt skal kapitler labeles?
%\setcounter{secnumdepth}{4}	
%\setcounter{tocdepth}{4}


% Hvilket nummer skal der startes med i sections? (n-1)
%\setcounter{section}{0}	

% Til nummerering af ligninger. Så der står (afsnit.ligning) og ikke bare (ligning)
\numberwithin{equation}{section}
\date{}

% Header
%\usepackage{fancyhdr}
%\head{Nikolai Plambech Nielsen, 21-06-95\\Dato:. Klasse 5.C - De Fysiske Fag}
%\pagestyle{fancy}

%Titel
\title{Noter til KM1 og KM2 på KU (Kvantemekanik 1 og 2)}
\author{Nikolai Plambech Nielsen, LPK331. Version 1.2}

\begin{document}
	
	\selectlanguage{danish}
	
	\maketitle
	\tableofcontents
	
	\section*{Introduktion, Kvant1 noter}
	Her er mit notesæt til Kvant1. Vi har brugt bogen ">Introduction to Quantum Mechanics"< af David J Griffiths (jeg har selv 2. udgave, Cambridge Press), samt diverse noter, skrevet af Anders Sørensen. Og så også lige en note om Diracnotation af Andrzej Jarosz. Pensum løber over de første 4 kapitler, men gennemgangen af 4. kapitel er struktureret anderledes end i bogen. Dette er Anders' valg, og jeg følger da dette. Det 4. kapitel er også skrevet ret hastigt, her 2 dage inden eksamen, så hvis det ikke er helt så godt, så må I undskylde. Hvis det er fantastisk, så hurra da for det. Jeg håber ikke det sker igen. God arbejdslyst!
	
	\section*{Introduktion, Kvant2 formelsamling}
	Jeg havde egentlig ikke tænkt mig at skrive noter til Kvant2, men en del inde i kurset så jeg, at der ikke var så mange formler vi brugte, men jeg gad ikke lede gennem bogen hele tiden, så jeg besluttede mig for at skrive en lille formelsamling til kurset. Pensum er kapitel 6,9,7,8 (i den rækkefølge), men jeg har ikke skrevet noget til kapitel 8 (WKB-approksimationen) idet der ikke stilles nogen opgaver til dette i eksamensættene, og jeg har valgt at prioritere AnalMek-oplæsningen lidt højere. Du må \textit{meget} gerne skrive dette afsnit. Bare download .tex-filerne, skriv løs og upload på psi.nbi.dk
	
	\newpage
	\subfile{omslag}
	\newpage
	\part{Kvant 1}
	\subfile{Boelgefunkt}
	\newpage
	\subfile{tidsuafhaeng}
	\newpage
	\subfile{formalisme}
	\newpage
	\subfile{threedee}
	\newpage
	\part{Kvant 2}
	\subfile{kvant2}
	\newpage
	\part{Appendiks}
	\subfile{ekstra}
	
	
\end{document}

