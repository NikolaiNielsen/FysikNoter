\documentclass[Kvantnoter.tex]{subfiles}

\begin{document}
	
	
	\section{Den tidsuafhængige Schrödingerligning}
	\subsection{Stationære tilstande}
	For at løse Schrödinger ligningen antages det først i dette kapitel (og i det meste af bogen), at potentialet $ V(x,t) $ er \textit{uafhængigt af} $ t $. I dette tilfælde, kan ligningen løses ved separation af de variable. Juhu! Altså:
	\begin{equation}
		i\hbar \diff{\Psi(x,t)}{t} = -\frac{\hbar^2}{2m} \diff{^2 \Psi(x,t)}{x^2} + V(x)\Psi(x,t) \quad \Rightarrow \quad \Psi(x,t) = \psi(x)\varphi(t).
	\end{equation}
	Som sædvanligt vil dette kun give et undersæt af de mulige løsninger til ligningen, men de har en masse gode egenskaber, og den generelle løsning til ligningen kan som oftest konstrueres fra de separable løsninger. Efter den normale behandling af separation af de variable (indsæt den antagede løsningsform, og divider igennem med den), fås
	\begin{equation}
		i \hbar \frac{1}{\varphi} \diff[\ud]{\varphi}{t} = - \frac{\hbar^2}{2m} \frac{1}{\psi} \diff[\ud]{^2 \psi}{x} + V
	\end{equation}
	Læg mærke til symbolerne og de variable. Det er almindeligt afledte nu, og på venstre side står der kun noget, der afhænger af tid, og på højre noget, der kun afhænger af position. For at dette må være en løsning, må begge disse da være konstante! Denne konstant vælger vi at kalde $ E $, da dette er ret smart (kan du gætte hvorfor?). De to ligninger lyder da (efter lidt omrokering)
	\begin{equation}
		\diff[\ud]{\varphi}{t} = -\frac{iE}{\hbar} \varphi, \quad - \frac{\hbar^2}{2m} \diff[\ud]{^2 \psi}{x^2} + V\psi = E \psi
	\end{equation}
	Det ses da, at hvis $ V $ havde afhængt både af $ x $ og $ t $, ville denne ikke generelt kunne separeres. Den første ligning har løsningen $ C \exp(-iEt/\hbar) $, men konstanten $ C $ kan lige så godt absorberes i $ \psi $, idet det er produktet af de to funktioner, der er betydende (plus, så kommer den bare med i normaliseringen til sidst). Dermed:
	\begin{equation}
		\varphi(t) = e^{-iEt/\hbar}
	\end{equation}
	Den anden ligning kaldes for den \textbf{Tidsuafhænginge Schrödingerligning}, og vi kommer ingen løsning nærmere, med mindre potentialet $ V(x) $ specificeres. Separable løsninger har tre vigtige egenskaber:
	\paragraph{De er stationære stadier.} Selvom bølgefunktionen, $ \Psi(x,t) = \psi(x) e^{-iEt/\hbar}$, afhænger af tid, så gør sandsynlighedstætheden det ikke:
	\begin{equation}
		|\Psi(x,t)|^2 = \psi\konj e^{iEt/\hbar} \psi \konj e^{-iEt/\hbar} = |\psi(x)|^2.
	\end{equation}
	De tidslige dele går nemlig ud med hinanden, når absolutkvadratet tages! Det samme gør sig faktisk gældende for udregning af forventningsværdien af enhver dynamisk variabel $ Q $, idet:
	\begin{equation}
		\brac{Q(x,p)} = \infint \psi\konj Q\pp{x, -i\hbar\diff[\ud]{}{x}} \psi \ud x.
	\end{equation}
	Her går de tidslige komponenter nemlig også ud, idet $ Q $ ikke afhænger af tiden, og dermed ikke ændrer på disse. Dette betyder da, at forventningsværdien altid er konstant i tid! Dermed arbejdes der som oftest kun med $ \psi $, fordi $ \varphi $ for det meste er lige meget. Den har altid den samme form, og det er kun når den samlede bølgefunktion skal opgives, at det er nødvendigt at smide den med. Det ses da, at $ \brac{x} $ er konstant, og dermed fra Ehrenfests teorem, må $ \brac{p} = 0 $, altid.
	
	I klassisk mekanik, er den samlede mekaniske energi (kinetisk plus potentiel) givet ved Hamiltonen:
	\begin{equation}
		H(x,p) = \frac{p^2}{2m} + V(x).
	\end{equation}
	Og det ses, at hvis man substituerer impulsen med dens kvantemekaniske modpart, fås
	\begin{equation}
		\op{H} =  - \frac{\hbar^2}{2 m} \diff{^2}{x^2} + V(x),
	\end{equation}
	Og dermed kan den tidsuafhængige Schrödingerligning skrives kompakt ved
	\begin{equation}
		\op{H}\psi = E\psi.
	\end{equation}
	Ved dette, kan standardafvigelsen i $ H $ udregnes gennem dennes forventningsværdier:
	\begin{equation}
		\brac{H} = E, \quad \brac{H^2} = E^2 \quad \Rightarrow \sigma_H^2 = \brac{H^2}- \brac{H}^2 = E^2-E^2 = 0.
	\end{equation}
	Dette sker kun, hvis alle individer i en given population har den samme værdi for $ H $. Dette vil altså at enhver måling af den samlede energi, giver $ E $. Se nu, hvor smart det var, at kalde separationskonstanten for $ E $!.
	
	
	\paragraph{De er tilstande med bestemt total energi.}
	I klassisk mekanik, kan den totale mekaniske energi (kinetisk plus potentiel) beskrives ved hamiltonen:
	\begin{equation}
		H(p,x) = \frac{p^2}{2m} + V(x).
	\end{equation}
	Og ved substitution med den kvantemekaniske ækvivalent til impulsen, fås \textit{Hamiltonoperatoren}:
	\begin{equation}
		\op{H} = - \frac{\hbar^2}{2m} \diff{^2}{x^2} + V(x).
	\end{equation}
	Og med dette kan den tidsuafhængige Schrödingerligning skrives på den kompakte form
	\begin{equation}
		\op{H}\psi = E\psi
	\end{equation}
	Variansen i $ H $ kan udregnes ved dennes forventningsværdier:
	\begin{equation}
		\brac{H} = E, \quad \brac{H^2} = E^2, \quad \Rightarrow \quad \sigma_H^2 = \brac{H^2}- \brac{H}^2 = E^2 - E^2 = 0
	\end{equation}
	Dette betyder, at alle individer af populationen har den samme værdi! I dette tilfælde vil det sige, at enhver måling på en partikel i tilstanden $ \psi $, vil give energien $ E $. Se nu, hvor smart det var at kalde separationskonstanten for $ E $!
	
	\paragraph{Den generelle løsning er en lineær kombination af separable løsninger.}
	Det viser sig, at der for alle de potentialer vi møder, er uendeligt mange løsninger til den tidsuafhængige Schrödingerligning, $ \psi_1 $, $ \psi_2 $, etc, hver med sin energi $ E_1 $, $ E_2 $, etc. Ydermere viser det sig, at enhver løsning til den tidsafhængige (altså \textit{med} tid) kan opskrives som en linearkombination af disse løsninger (idet de er komplette, som vi kommer mere ind på i kapitel 3. Det er lige som med $ \sin $ og $ \cos $ for Fourierrækker). Først konstrueres $ \Psi(x,0) $ ud fra løsningerne til den tids\textit{uafhængige} Schrödingerligning:
	\begin{equation}
		\Psi(x,0) = \sum_{n = 1}^{\infty} c_n \psi_n(x)
	\end{equation}
	Og da smides den karakteristiske tidsfaktor bare på hvert led, for at få tidsudviklingen med:
	\begin{equation}
		\Psi(x,t) = \sum_{n = 1}^{\infty} c_n \psi_n(x) e^{-iE_n t / \hbar} = \sum_{n = 1}^{\infty} c_n \Psi_n(x,t)
	\end{equation}
	Spørgsmålet er bare at finde de konstante $ c_n $.
	Det ses, at $ \Psi_n $ alle er stationære tilstande, idet tidsudviklingen går ud med sig selv, når absolutkvadratet regnes ($ |\exp{-iE_n t / \hbar}|^2 = 1 $). Dette betyder dog \textit{ikke}, at det samme gør sig gældende for de generelle løsninger. For når du har en linearkombination af flere stationære tilstande, går hver deres karakteristiske tidsfaktorer ikke ud med \textit{hinanden}. De interfererer i tid, hvilket er hvad, der giver tidsudviklingen i den generelle løsning.
	
	\subsubsection{Resultater fra problem 2.1 og 2.2}
	I problem 2.1 (og 2.2) skal 3 (1) nyttige teoremer bevises. Jeg vil kort beskrive resultaterne fra disse:
	\begin{enumerate}
		\item \textbf{For normaliserbare løsninger, må separationskonstanten $ E $ nødvendigvis være reel.}
		\item \textbf{Den tidsuafhængige bølgefunktion $ \psi(x) $ kan altid regnes for reel.} Dette betyder \textit{ikke}, at alle løsninger til den tidsuafhængige Schrödingerligning er reelle, men at alle komplekse bølgefunktioner kan skrives som en superposition af reelle bølgefunktioner. Dermed kan man lige så godt bare arbejde med de reelle.
		\item \textbf{Hvis $ V(x) $ er en lige funktion, kan $ \psi(x) $ altid regnes for at være enten lige eller ulige} (lige og ulige funktioner kan konstrueres ud fra $ \psi(x) $ og $ \psi(-x) $ som beskrevet i afsnittet om lige og ulige funktioner).
		\item \textbf{Energien $ E $ skal overstige den mindste værdi af potentialet $ V $ for at løsningen til Schrödingerligningen er normaliserbar}. Dette gælder dog \textit{ikke} for ubundne tilstande, da disse ikke er normalt normaliserbare.
	\end{enumerate}
	
	
	
	\subsection{Den uendelig potentialbrønd}
	\textbf{I dette og de resten af afsnittene af dette kapitel, gider jeg ikke at skrive ">den tidsuafhængie Schrödingerligning"<, så jeg skriver bare Schrödingerligningen. Hvis jeg mener den tids\textit{afhængige} liging, så skriver jeg det eksplicit.}
	
	Den uendelige potentialbrønd er et potential, der har formen
	\begin{equation}
		V(x) = \begin{cases}
		0, & 0\leq x \leq a \\
		\infty, & \text{ellers.}
		\end{cases}
	\end{equation}
	Inden for brønden (mellem 0 og $ a $) er partiklen fri, men alle andre steder, er potentialet uendeligt, og en uendelig kraft holder da partiklen inden for brønden. I steder, hvor potentialet er uendeligt er bølgefunktionen da 0. Inde i brønden lyder Schrödingerligningen
	\begin{equation}\label{eq:InfBrond}
		-\frac{\hbar^2}{2m} \diff[\ud]{^2 \psi}{x^2} = E\psi, \quad \diff[\ud]{^2 \psi}{x^2} = -k\psi, \quad k= \frac{\sqrt{2mE}}{\hbar}.
	\end{equation}
	hvor energien $ E $ antages for større end, eller lig 0, per problem 2.2. Dette er den klassiske simple harmoniske oscillator, og løsningen er
	\begin{equation}
		\psi(x) = A \sin kx + B \cos kx.
	\end{equation}
	For at finde $ A $ og $ B $, må der indføres randbetingelser. Normalt skal bølgefunktionen både være kontinuert og differentiabel (den kan differentieres, og dennes afledte er kontinuert) i alle punkter. Dog gælder den sidste randbetingelse ikke for uendelige potentialer (mere om det i afsnit \ref{seq:delta}). Idet bølgefunktionen er 0 uden for brønden, må der gælde følgende
	\begin{equation}
		\psi(0) = \psi(a) = 0
	\end{equation}
	Den første randbetingelse giver at $ B = 0$, og den anden giver enten $ A = 0 $, eller $ \sin ka = 0 $. Vi vælger $ \sin ka = 0 $, da vi ellers ender med den unormaliserbare løsning $ \psi = 0 $. Dermed fås
	\begin{equation}
		ka = n\pi, \quad \Leftrightarrow k_n = \frac{n\pi}{a}, \quad n \in \Set{N}
	\end{equation}
	hvor $ n $ altså løber over alle positive heltal. De negative heltal giver bare negative løsninger, og vi kan lige så godt tage dette fortegn med i $ A $, og $ k = 0 $ giver også 0-løsningen. Fra ligning \eqref{eq:InfBrond} fås da, at partiklen kun kan have bestemte energier $ E_n $:
	\begin{equation}
		E_n = \frac{\hbar^2 k_n^2}{2m} = \frac{n^2 \pi^2 \hbar^2}{2 m a^2}
	\end{equation}
	For at finde den sidste ubekendte, $ A $, normaliseres bølgefunktionen, og man får
	\begin{equation}
		\int_{0}^{a} |A|^2 \sin^2(kx) = |A|^2 \frac{a}{2} = 1, \quad \Rightarrow \quad A = \sqrt{\frac{2}{a}},
	\end{equation}
	hvor den positive reelle rod vælges. Dermed bliver den samlede løsning til den uendelige potentialbrønd:
	\begin{equation}
		\psi_n(x) = \sqrt{\frac{2}{a}} \sin \pp{\frac{n\pi}{a} x}.
	\end{equation}
	Det ses da, at ligningen har uendeligt mange løsninger (én for hvert positivt heltal $ n $). Den første tilstand, $ n=1 $, kaldes da for \textbf{grundtilstanden}. Disse løsninger har en række smarte egenskaber
	\begin{enumerate}
		\item De er skiftevis lige og ulige, med hensyn til midten af brønden ($ \psi_1 $ er ulige, $ \psi_2 $ er lige, etc).
		\item De har $ n-1 $ knudepunkter (punkter $ x_0 $ hvor $ \Psi_n(x_0,t) = 0 $ for alle $ t $), hvor man ikke tæller de trivielle knudepunkter i enden med.
		\item De er ortonormale:
		\begin{equation}
			\int_{0}^{a} \psi_m(x)\konj \psi_n(x) \ud x = \delta_{mn},
		\end{equation}
		hvor $ \delta_{mn} $ er Kroneckerdeltaet. Alle $ \psi_m $ er reelle, så konjugeringen er ikke strengt nødvendigt, men det er god øvelse, altid at have det i baghovedet, at man egentlig skal konjugere denne.
		\item De udgør et komplet sæt, idet enhver velopførende funktion $ f(x) $ kan opskrives som en linearkombination af disse:
		\begin{equation}
			f(x) = \sum_{n=1}^{\infty} c_n \psi_n(x)
		\end{equation}
		Og dette er jo bare Fourierrækken for $ f(x) $! For at finde $ f(x) $ bruger man følgende formel (som Griffiths kalder for \textbf{Fouriers trick}):
		\begin{equation}
			c_n = \int_{0}^{a} \psi_n(x)\konj f(x) \ud x.
		\end{equation}
	\end{enumerate}
	Disse fire egenskaber er dog ikke enestående for den uendelige potentialbrønd. Den første opstår når potentialet er symmetrisk. Den anden gælder altid. Ortogonalitet ligeså. At løsningerne udgør et komplet sæt er som oftest sandt (i hvert fald for alle de potentialer, vi kommer til at støde ind i).
	
	Med alt dette i baghovedet er de stationære tilstande for den uendelige potentialbrønd
	\begin{equation}
		\Psi_n (x,t) = \sqrt{\frac{2}{a}} \sin \pp{\frac{n\pi}{a} x} e^{-iE_n t / \hbar}, \quad E_n = \frac{n^2 \pi^2 \hbar^2}{2 m a^2},
	\end{equation}	
	og den generelle løsning er en linearkombination af disse
	\begin{equation}
		\Psi (x,t) = \sum_{n = 1}^{\infty} c_n \Psi_n(x,t).
	\end{equation}
	I praksis får man en bølgefunktion til tiden $ t=0 $, og da er
	\begin{equation}
		\Psi(x,0) = \sum_{n = 1}^{\infty} c_n \psi_n (x), \quad c_n = \sqrt{\frac{2}{a}} \int_0^a \sin \pp{\frac{n\pi}{a} x}\Psi(x,0) \ud x.
	\end{equation}
	\textbf{Absolutkvadratet på koefficienterne, $ |c_n|^2 $ fortæller, hvad sandsynligheden er, for at en måling på partiklen giver resultatet $ E_n $.} Og fra normaliseringen af bølgefunktionen følger det også, at summen af disse giver én:
	\begin{equation}
		\sum_{n=1}^{\infty} |c_n|^2 = 1.
	\end{equation} 
	Forventningsværdien for energien (Hamiltonoperatoren) er
	\begin{equation}
		\brac{\op{H}} = \sum_{n=1}^{\infty} |c_n|^2 E_n,
	\end{equation}
	og det faktum, at forventningsværdien er tidsuafhængig, viser da energibevarelse i kvantemekanik.
	
	
	
	\subsection{Den harmonisk oscillator}
	\textbf{I dette og de resten af afsnittene af dette kapitel, gider jeg ikke at skrive ">den tidsuafhængie Schrödingerligning"<, så jeg skriver bare Schrödingerligningen. Hvis jeg mener den tids\textit{afhængige} liging, så skriver jeg det eksplicit.}
	
	Den klassiske harmoniske oscillator (uden friktion) er givet ved den samme ligning som den uendelige potentialbrønd:
	\begin{equation}
		F = m\diff[\ud]{^2 x}{t^2} = -kx, \quad k > 0
	\end{equation}
	Ved at bruge, at $ F $ er et konservativt kraftfelt, fås at potentialet er
	\begin{equation}
		V(x) = \frac{1}{2} k x^2
	\end{equation}
	I praksis er der selvfølgelig ingen perfekt harmonisk oscillator, men faktum er, at for små afvigelser fra et minimum i potentialet, er bevægelsen af en partikel tilnærmelsesvist harmonisk oscillerende (jeg undskylder meget for denne sætning). Hvis man Taylorekspanderer potentialet om minimummet $ x_0 $ fås
	\begin{equation}
		V(x) = V(x_0) + V'(x)(x-x_0) + \frac{1}{2} V''(x_0) (x-x_0)^2 + \dots.
	\end{equation}
	Idet man kan trække konstante fra potentialet uden problemer, og at $ V'(x_0) = 0 $, er potentialet approksimativt
	\begin{equation}
		V(x) \approx \frac{1}{2} V''(x_0) (x-x_0)^2
	\end{equation}
	hvor der ses bort fra led af højere orden, da vi ikke bevæger os langt fra minimummet. Dette er da en simpel harmonisk oscillator om $ x=x_0 $, med fjederkonstanten $ k = V''(x_0) $. I kvantemekanik skrives potentialet oftest med den klassiske frekvens $ \omega = \sqrt{k/m} $, og potentialet er da
	\begin{equation}
		V(x) = \frac{1}{2} m\omega^2 x^2
	\end{equation}
	Med dette bliver Schrödingerligningen
	\begin{equation}
		-\frac{\hbar^2}{2m} \diff[\ud]{^2 \psi}{x^2} + \frac{1}{2} m \omega^2 x^2 \psi = E\psi.
	\end{equation}
	Denne løses oftest på to måder: polynomiumsserier (Power series) eller algebraisk. I disse noget beskrives kun den algebraiske metode, da den er klart nemmere, involverer \textbf{hæve/sænkeoperatorer}, og fordi vi ikke har brugt resultaterne fra den anden metode i kurset endnu.
	
	Ideen med denne metode er at faktorisere Hamiltonoperatoren $ \op{H} $. Dette gøres ved at opskrive Schrödingerligningen på en lidt anden form, og introducere to nye operatorer. Først ligningen:
	\begin{equation}
		\frac{1}{2m} [\op{p}^2+(m\omega \op{x})^2] \psi = E\psi.
	\end{equation}
	(I den følgende udledning bruger jeg altid $ \op{x} $, selvom denne også bare er lig $ x $, for at illustrere, at det er operatorer vi arbejder med) Hvor Hamiltonoperatoren er givet ved
	\begin{equation}
		\op{H} = \frac{1}{2m} [\op{p}^2+(m\omega \op{x})^2].
	\end{equation}
	Hvis $ \op{p} $ og $ x $ bare var tal, ville dette være nemt nok, da disse kommuterer. Men det gør operatorer desværre ikke (normalt, i hvert fald). Derfor indføres ">kommutatoren"< af to operatorer $ \op{A} $ og $ \op{B} $:
	\begin{equation}
		[\op{A},\op{B}] = \op{A}\op{B} - \op{B} \op{A},
	\end{equation}
	der er et mål for, hvor ">dårligt de to operatorer kommuterer"<. For $ \op{x} $ og $ \op{p} $ fås det, som kaldes for den \textbf{kanoniske kommutator}:
	\begin{equation}
		[\op{x},\op{p}] = i\hbar.
	\end{equation}
	For at udregne kommutatoren af to operatorer opskriver man kommutatoren og lader en testfunktion $ f(x) $ virke på den. Da udregner man resultatet, og smider funktionen væk til sidst, når et pænt (eller, i hvert fald så pænt som muligt) resultat opnås.
	
	Men nok om kommutatorer for $ \op{x} $ og $ \op{p} $. Nu skal de to relevante operatorer indføres
	\begin{equation}
		\op{a}_{\pm} = \frac{1}{\sqrt{2\hbar m \omega}} (\mp i \op{p}+m\omega \op{x}).
	\end{equation}
	Produktet $ \saenk{a}\haev{a}$ er
	\begin{equation}
		\saenk{a}\haev{a} = \frac{1}{2 \hbar m \omega} [\op{p}^2 + (m\omega \op{x})^2 - im\omega (\op{x}\op{p}-\op{p}\op{x})]
	\end{equation}
	Den sidste parentes i parentesen er den kanoniske kommutator, og produktet kan da skrives ved denne:
	\begin{align}
		\saenk{a}\haev{a} &= \frac{1}{2 \hbar m \omega} [p^2+(m\omega \op{x})^2] + \frac{1}{2i\hbar} [\op{x},\op{p}] \\
		&= \frac{1}{\hbar \omega} \op{H} + \frac{1}{2}
	\end{align}
	På lige vis fås produktet $ \haev{a} \saenk{a} $
	\begin{equation}
		\haev{a}\saenk{a} = \frac{1}{\hbar \omega} \op{H} - \frac{1}{2}
	\end{equation}
	Dette giver da kommutatoren $ [\saenk{a},\haev{a}] = \saenk{a}\haev{a}-\haev{a}\saenk{a} 1 $, og $ \op{H} $ kan skrives ved
	\begin{equation}
		\op{H} = \hbar \omega \pp{\saenk{a}\haev{a}-\frac{1}{2}} =  \hbar \omega \pp{\haev{a}\saenk{a}+\frac{1}{2}}.
	\end{equation}
	og Schrödingerligningen er da
	\begin{equation}
		\op{H} \psi = \hbar \omega \pp{\op{a}_{\pm}\op{a}_{\mp}\pm \frac{1}{2}} \psi = E\psi.
	\end{equation}
	Så nu kan Schrödingerligningen opskrives ved \textit{to} operatorer, i stedet for kun én. Woop de fucking do! Men vent, for vi har jo ikke navngivet $ \haev{a} $ og $ \saenk{a} $ endnu! Vi kalder dem nemlig for henholdsvis \textit{hæve}- og \textit{sænke}-operatorerne (I bet you didn't see that coming). De har dette navn, fordi det viser sig, at hvis $ \op{H}\psi = E\psi $, så er $ \op{H}\haev{a}\psi = (E+\hbar \omega) \psi$ og $ \op{H}\saenk{a} \psi = (E-\hbar \omega) \psi $. De to operatorer \textbf{hæver} og \textbf{sænker} altså energiniveauet af en given løsning! Men vi er jo stadig ikke tættere på nogen løsning. Nu ved vi bare at der er flere end én...
	
	
	Det er dog sådan, at hvis man bruger sænkeoperatoren nok gange, må man nødvendigvis nå en negativ energi, hvilket ikke kan lade sig gøre i følge problem 2.2. Dermed må der være en \textbf{grundtilstand} med
	\begin{equation}
		\saenk{a} \psi_0 = 0. \quad \text{(læg mærke til nul-indekseringen)}
	\end{equation}
	(det kunne også være, at dens kvadratintegral var uendeligt, men normalt er det ikke sådan). Ud fra dette fås en differentialligning og medfølgende løsning
	\begin{equation}
		\diff[\ud]{\psi_0}{x} = - \frac{m \omega}{\hbar} x \psi_0, \quad \Rightarrow \quad \psi_0 (x) = A e^{-m\omega x^2/2 \hbar}.
	\end{equation}
	Og $ A $ fås ved normalisering til $ A^2 = \sqrt{m\omega/\pi \hbar} $. Da er grundtilstanden:
	\begin{equation}
		\psi_0 (x) = \pp{\frac{m\omega}{\hbar \pi}}^{1/4} e^{-m\omega x^2 / 2 \hbar}.
	\end{equation}
	Denne tilstand har energien $ E_0 = \frac{1}{2}\hbar\omega $. Hvilket man får fra Schrödingerligningen og at $ \saenk{a}\psi_0 = 0 $. Hermed er det bare at bruge hæveoperatoren for at få alle de næste tilstande.
	\begin{equation}
		\psi_n = A_n (\haev{a})^n \psi_0, \quad A_n = \frac{1}{\sqrt{n!}}, \quad E_n = \pp{n+\frac{1}{2}} \hbar \omega.
	\end{equation}
	Normaliseringsfaktoren $ A_n $ kan fås ud fra den egenskab at $ \op{a}_{\pm} \psi_n \propto \psi_{n\pm 1} $ og at $ \haev{a} $ er den \textbf{hermitisk konjugerede} til $ \saenk{a} $ (og omvendt):
	\begin{equation}
		\infint f\konj (\op{a}_{\pm} g) \ud x = \infint (\op{a}_{\mp} f)\konj g \ud x
	\end{equation}
	Ud fra dette, fås også følgende nyttige egenskaber
	\begin{equation}
		\haev{a} \psi_n = \sqrt{n+1}\, \psi_{n+1}, \quad \saenk{a} \psi_n = \sqrt{n} \,\psi_{n-1}.
	\end{equation}
	Ydermere er løsningerne til den harmoniske oscillator ortonormale. Dette vil sige, at vi kan bruge Fouriers trick, og at $ |c_n|^2 $ igen er sandsynligheden for at partiklen måles til at have energien $ E_n $:
	\begin{equation}
		\Psi (x,0) = \sum_{n=0}^{\infty} c_n \psi_n = \sum_{n=0}^{\infty} c_n \frac{1}{\sqrt{n!}} (\haev{a})^n \psi_0, \quad c_n = \infint \psi_n(x)\konj \Psi(x,0) \ud x
	\end{equation}
	Og igen skal den karakteristiske tidsfaktor bare smækkes på hvert led, for at få løsningen til den tidsafhængige Schrödingerligning.
	
	Der er yderligere et par smarte resultater fra disse hæve/sænkeoperatorer. Man kan nemlig udtrykke $ \op{x} $ og $ \op{p} $ ved dem:
	\begin{align}
		\op{x} &= \sqrt{\frac{\hbar}{2m\omega}} (\haev{a}+\saenk{a}),\\
		\op{p} &= i \sqrt{\frac{\hbar m \omega}{2}} (\haev{a}-\saenk{a}),\\
		\op{x}^2 &= \frac{\hbar}{2m\omega} \bb{\haev{a}^2 + \haev{a} \saenk{a} + \saenk{a} \haev{a} + \saenk{a}^2}, \\
		\op{p}^2 &= \frac{-\hbar m \omega}{2} \bb{\haev{a}^2 - \haev{a} \saenk{a} - \saenk{a} \haev{a} + \saenk{a}^2}.
	\end{align}
	Dette gør det meget nemt at udregne forventningsværdier, idet man bare kan udnytte hæve/sænkeoperatorerne og ortonormaliteten af de stationære tilstande til at evaluere integralerne.
	
	
	
	
	\subsection{Den fri partikel}
	\textbf{I dette og de resten af afsnittene af dette kapitel, gider jeg ikke at skrive ">den tidsuafhængie Schrödingerligning"<, så jeg skriver bare Schrödingerligningen. Hvis jeg mener den tids\textit{afhængige} liging, så skriver jeg det eksplicit.}
	
	For den frie partikel er $ V(x) = 0 $ over det hele, og Schrödingerligningen er da den samme som for den uendelige potentialbrønd.
	\begin{equation}
		-\frac{\hbar^2}{2m} \diff[\ud]{^2 \psi}{x^2} = E\psi, \quad \diff[\ud]{^2 \psi}{x^2} = -k\psi, \quad k= \frac{\sqrt{2mE}}{\hbar} > 0.
	\end{equation}
	Det er dog mere vanligt at skrive løsningen op på eksponentiel form, i stedet for trigonometrisk form:
	\begin{equation}
		\psi(x) = A e^{ikx} + B e^{-ikx}.
	\end{equation}
	Og med tidsfaktoren på
	\begin{equation}
		\Psi(x,t) = A e^{ik(x-\hbar k t/2m)} + B e^{-ik(x+\hbar k t/2m)}.
	\end{equation}
	I tilfældet af en fri partikel, er der ingen randbetingelser, der putter restriktioner på den tilladte energi, og denne udgør da et kontinuert spektrum.
	
	Løsningen svarer til superpositionen af én bølge der bevæger sig i positiv $ x $-retning, og én der bevæger sig i negativ $ x $-retning. Hvis man lader $ k $ også være negativ, kan man skrive den som
	\begin{equation}
		\Psi_k (x,t) = A e^{i(kx-\hbar k^2 t / 2m)}, \quad k \equiv \pm \frac{\sqrt{2mE}}{\hbar},
	\end{equation}
	hvor $ k > 0 $ er en bølge der bevæger sig mod højre, og $ k < 0 $ er en bølge, der bevæger sig mod venstre.
	
	Disse løsninger har ét problem: de er ikke normaliserbare. Til gengæld kan superpositionen af dem stadig normaliseres (nogle gange). I dette tilfælde er det dog ikke en sum over diskrete værdier af $ k $, men rettere et integral:
	\begin{equation}
		\Psi(x,t) = \frac{1}{\sqrt{2\pi}} \infint \phi(k) e^{i(kx-\hbar k^2 t / 2m)}\ud k,
	\end{equation}
	hvor $ (1/\sqrt{2\pi})\phi(k)\ud k $ svarer til $ c_n $ for de diskrete tilfælde. Dette integral kan normaliseres for nogle værdier af $ \phi(k) $.
	
	Som normalt, får man givet $ \Psi(x,0) $, og vi skal så \textit{finde} $ \Psi(x,t) $. Da skal $ \phi(k) $ findes. $ \Psi(x,0) $ er da givet ved
	\begin{equation}
		\Psi(x,0) = \frac{1}{\sqrt{2\pi}} \infint \phi(k) e^{ikx}\ud k,
	\end{equation}
	hvilket jo bare er den Fouriertransformerede af $ \phi(k) $! Dermed kan den inverse Fouriertransformation bruges til at finde $ \phi(k) $:
	\begin{equation}
		\phi(k) = \frac{1}{\sqrt{2 \pi}} \infint \Psi(x,0)e^{-ikx} \ud x.
	\end{equation}
	så er det jo klaret! Men hvad \textit{er} $ \phi(k) $ egentlig? Det er en funktion, der beskriver spredningen i $ k $ for bølgen. Og da $ k $ er relateret til impulsen gennem bølgelængden $ \lambda $ og de Broglieformlen, svarer $ \phi(k) $ også til \textit{spredningen i impuls}.
	
	Hvis man ser på hastigheden af de enkelte bølger $ \Psi_k $ (koefficienten af $ x $ over koefficienten af $ t $) får man
	\begin{equation}
		v_{\text{kvantum}} = \frac{\hbar |k|}{2m} = \sqrt{\frac{E}{2m}},
	\end{equation}
	mens den klassiske hastighed for en fri partikel ($ E = mv^2 /2 $) er
	\begin{equation}
		v_{\text{klassisk}} = \sqrt{\frac{2E}{m}} = 2 v_q
	\end{equation}
	hvilket er jo er ret spøjst. Til gengæld, fordi $ \Psi(x,k) $ er opbygget af en linearkombination af en masse bølger, alle med forskellige værdier af $ k $, giver dette anledning til et interferensmønster, og man kalder den samlede bølgefunktion for en \textbf{bølgepakke}.
	
	Den samlede bølgefunktion er givet ved
	\begin{equation}
		\Psi(x,t) = \frac{1}{\sqrt{2\pi}} \infint \phi(k) e^{i(kx-\omega t)}\ud k,
	\end{equation}
	med $ \omega = \hbar k^2 / 2m $. Med lidt fancy matematik og nogle koordinatskifte, får man, at den samlede bølgepakke bevæger sig med en bestemt hastighed (kaldet for \textbf{gruppehastigheden}). Denne er givet ved
	\begin{equation}
		v_{\text{gruppe}} = \diff[\ud]{\omega}{k}, 
	\end{equation}
	(denne differentialkvotient skal evalueres i punktet $ k_0 $, der er det ">typiske"< bølgetal. Normalt tales der om en skarpt peaket funktion $ \phi(k) $ med centrum i $ k_0 $, ellers giver udtrykket bølgepakke heller ikke meget mening, grundet spredningen i impuls) og de enkelte bølger bevæger sig med \textbf{fasehastigheden}
	\begin{equation}
		v_{\text{fase}} = \frac{\omega}{k}
	\end{equation}
	I dette tilfælde fås
	\begin{equation}
		v_{\text{fase}} = \frac{\hbar k}{2 m} = v_{\text{kvantum}}, \quad v_{\text{gruppe}} = \frac{\hbar k}{m} = 2 v_{\text{fase}} = v_{\text{klassisk}}.
	\end{equation}
	Og dermed ses det, at bølgepakken, som beskriver partiklen, netop bevæger sig med den forventede hastighed.
	
	
	
	\subsection{Deltafunktionspotentialet}
	\label{seq:scatter}
	\textbf{I dette og de resten af afsnittene af dette kapitel, gider jeg ikke at skrive ">den tidsuafhængie Schrödingerligning"<, så jeg skriver bare Schrödingerligningen. Hvis jeg mener den tids\textit{afhængige} liging, så skriver jeg det eksplicit.}
	
	\subsubsection{Bundne og ubundne tilstande}
	Indtil videre er der to typer af løsninger til Schrödingerligningen: normaliserbare løsninger, der er en sum over den diskrete variabel $ n $ (uendelig brønd, harmonisk oscillator), og ikkenormaliserbare løsninger, der er et integral over den kontinuerte variabel $ k $ (fri partikel).
	
	Dette svarer også til to forskellige typer problemer i klassisk mekanik. Hvis vi har en partikel med energien $ E $, og potentialet $ V $ overstiger partiklens energi på hver sin side af partiklen, så vil partiklen oscillere mellem disse to punkter (kaldet vendepunkter). Dette kaldes for en \textbf{bunden tilstand}. Hvis $ E $ derimod overstiger $ V $ på den ene side (eller begge) af partiklen, så vil denne komme ind fra uendelighed, og vende tilbage til uendelighed (enten den samme, eller modsatte, alt efter om der er ét eller nul vendepunkter). Dette kaldes for \textbf{ubundne tilstande}.
	
	I kvantemekanik svarer de normaliserbare løsninger over $ n $ til bundne tilstande, mens de unormaliserbare løsninger over $ k $ svarer til ubundne tilstande. Nogle potentialer, som den harmoniske oscillator, tillader kun bundne tilstande, mens andre, som den frie partikel, tillader kun ubundne tilstande. Og andre igen (som deltafunktionspotentialet og den endelige potentialbrønd), tillader begge, alt efter partiklens energi.
	
	I kvantemekanik kan partikler yderemere godt overstige et lokalt maksimum i potentialet (et fænomen, der kaldes tunnelering), hvilket betyder, at det kun er potentialet i uendeligt, der er vigtigt:
	
	\begin{align}
		\begin{cases}
			\begin{array}{llll}
			E < [V(-\infty) &\text{og} & V(+\infty)] \Rightarrow & \text{bunden tilstand}, \\
			E > [V(-\infty) &\text{eller} & V(+\infty)] \Rightarrow & \text{ubunden tilstand}. 
			\end{array}
		\end{cases}
	\end{align}
	Og idet alle ">virkelige"< potentialer går mod 0 i uendeligt fås
	\begin{equation}
		\begin{cases}
			E < 0 \Rightarrow & \text{bunden tilstand}, \\
			E > 0 \Rightarrow & \text{ubunden tilstand}.
		\end{cases}
	\end{equation}
	
	\subsubsection{Deltafunktionsbrønden}
	\label{seq:delta}
	Forestil dig en potentialbrønd af formen
	\begin{equation}
		V(x) = -\alpha \delta(x), \quad \alpha > 0
	\end{equation}
	Altså en uendeligt dyb brønd i origo. Her er Schrödingerligningen
	\begin{equation}
		-\frac{\hbar^2}{2m} \diff[\ud]{^2 \psi}{x^2} - \alpha \delta(x) \psi = E\psi,
	\end{equation}
	og den tillader både bundne og ubundne tilstande ($ E<0 $ og $ E> 0 $ henholdsvis). 
	
	\paragraph{Bundne tilstande.} For \textbf{bundne} tilstande, med $ E < 0 $, er der altid én løsning, givet ved
	\begin{equation}
		\psi(x) = \frac{\sqrt{m\alpha}}{\hbar} e^{-m\alpha |x|/\hbar^2}, \quad E = -\frac{m\alpha^2}{2 \hbar^2}
	\end{equation}
	
	Denne fås således: Schrödingerligningen i området $ x  < 0 $ lyder:
	\begin{equation}
		\diff[\ud]{^2 \psi}{x^2} = \kappa^2 \psi, \quad \kappa \equiv \frac{\sqrt{-2mE}}{\hbar},
	\end{equation}
	hvor $ \kappa $ er positiv og reel, idet $ E $ er negativ og reel. Da er løsningen summen af to eksponentialfunktioner: $ \psi(x) = A e^{-\kappa x} + B e^{\kappa x} $,	hvor $ A $ nødvendigvis må være 0, idet dette led går mod uendeligt for $ x \to - \infty $. Da
	\begin{equation}
		\psi(x) = B e^{\kappa x}, \quad x < 0.
	\end{equation}
	Og for $ x > 0  $ fås per samme argument
	\begin{equation}
		\psi(x) = F e^{-\kappa x}, \quad x > 0.
	\end{equation}
	Disse skal da bare sættes sammen med de normale randbetingelser: $ \psi $ skal altid være kontinuert, og $ \ud \psi/\ud x $ skal være kontinuert, ud over i punkter, hvor potentialet er uendeligt. Den første randbetingelse giver at $ F = B $, men hvad med den anden? Den giver nødvendigvis at differentialkvotienten i $ x = 0 $ er diskontinuert, men vi ved ikke med hvor meget, og for den sags skyld, hvilken værdi $ \kappa $ har.
	
	For at finde ud af dette (og hvorfor differentialkvotienten ikke er kontinuert i uendelige potentialer), tager man og integrerer Schrödingerligningen fra $ -\epsilon $ til $ +\epsilon $, og tager grænsen hvor $ \epsilon \to 0 $:
	\begin{equation}
		- \frac{\hbar}{2m} \int_{-\epsilon}^{+ \epsilon} \diff[\ud]{^2 \psi}{x^2} \ud x + \int_{-\epsilon}^{+ \epsilon} V(x) \psi(x) \ud x = E \int_{-\epsilon}^{+ \epsilon} \psi(x) \ud x.
	\end{equation}
	I grænsen $ \epsilon \to 0 $ er første led bare nogle konstante ganget med $ \ud \psi / \ud x $. Højresiden er 0, idet det er et integral fra 0 til 0. Dermed fås
	\begin{equation}
		\Delta \pp{\diff[\ud]{\psi}{x}} \equiv \diff{\psi}{x} \bigg\vert_{+\epsilon} - \diff{\psi}{x} \bigg\vert_{-\epsilon} = \frac{2 m}{\hbar} \lim\limits_{\epsilon \to 0} \int_{-\epsilon}^{+\epsilon} V(x) \psi(x) \ud x.
	\end{equation}
	Typisk er højresiden 0 (igen fordi det er et integral fra 0 til 0), og differentialkvotienten er kontinuert. Men når potentialet er uendeligt i punktet, er dette ikke tilfældet. Specielt for deltafunktionsbrønden er $ \psi(x)V(x) = -\alpha \psi(0) $ og integralet er
	\begin{equation}
		\Delta \pp{\diff[\ud]{\psi}{x}} = - \frac{2 m \alpha}{\hbar} \psi(0).
	\end{equation}
	De to differentialkvotienter er
	\begin{equation}
		\diff{\psi}{x} \bigg\vert_{+\epsilon} = - B \kappa, \quad \diff{\psi}{x} \bigg\vert_{-\epsilon} = + B \kappa
	\end{equation}
	Sættes dette ind, får man værdien for $ \kappa $
	\begin{equation}
		-2B\kappa = -\frac{2m\alpha}{\hbar^2}, \quad \Leftrightarrow \quad \kappa = \frac{m \alpha}{\hbar^2}
	\end{equation}
	Dermed er energien
	\begin{equation}
		E = -\frac{\hbar^2 \kappa^2}{2m} = -\frac{m \alpha^2}{2 \hbar^2}.
	\end{equation}
	Og ved normalisering fås $ B = \sqrt{\kappa} = \sqrt{m\alpha}/\hbar $, hvor man vælger den positive, reelle rod. Dermed har deltafunktionsbrønden altid én bunden tilstand:
	\begin{equation}
		\psi(x) = \frac{\sqrt{m\alpha}}{\hbar} e^{-m\alpha |x|/\hbar^2}, \quad E = -\frac{m\alpha^2}{2 \hbar^2}
	\end{equation}
	
	\paragraph{Ubundne tilstande.} Nu til de ubundne tilstande, med $ E > 0 $. Da er Schrödingerligningen
	\begin{equation}
		\diff[\ud]{^2\psi}{x^2} = -k^2 \psi, \quad k \equiv \frac{\sqrt{2mE}}{\hbar},
	\end{equation}
	hvor $ k $ er positiv og reel. Denne gang er løsningen:
	\begin{align}
		\psi(x) &= A \exp(ikx) + B \exp(-ikx), \quad x < 0, \\
		\psi(x) &= F \exp(ikx) + G \exp(-ikx), \quad x > 0.
	\end{align}
	Det er ikke helt nok til at løse vores problem, idet vi har lidt for mange ubekendte. Men lad os lige se på, hvad de fire amplituder er:
	
	I normale spredningseksperimenter sender man en bølge ind fra én side (venstre, eksempelvis). Dette giver da, at $ A $ svarer til den indsendte bølge, der kommer \textit{fra} venstre, ind mod brønden. $ B $ er da den del af bølgen, der bliver reflekteret i brønden, og bevæger sig \textit{mod} venstre. $ F $ er den del af bølgen, der transmitteres gennem brønden. $ G $ vil da være amplituden af en \textit{anden} bølge, der kommer \textit{fra} højre, og går ind mod brønden. Men da vi sender bølger ind fra \textit{venstre}, og ikke højre, kan vi lige så godt sætte $ G = 0 $.
	
	Hvis man vil have bølger ind fra højre, skifter man bare om på betydningen af $ A $ og $ G $, og på betydningen af $ B $ og $ F $.
	
	Med dette bliver randbetingelserne for $ x = 0 $:
	\begin{equation}
		F = A+B, \quad ik(F-A+B) = -\frac{2m\alpha}{\hbar^2} (A+B),
	\end{equation}
	og ved isolering fås
	\begin{equation}
		B = \frac{i\beta}{1-i\beta} A, \quad F = \frac{1}{1-i\beta}A, \quad \beta \equiv \frac{m\alpha}{\hbar^2 k}.
	\end{equation}
	Den relative sandsynlighed af, at en partikel bliver reflekteret tilbage fra brønden er, som i EM2:
	\begin{equation}
		R \equiv \frac{|B|^2}{|A|^2} = \frac{\beta^2}{1+\beta^2},
	\end{equation}
	hvor $ R $ kaldes for refleksionskoefficienten. Og ligeledes for transmission:
	\begin{equation}
		T \equiv \frac{|F|^2}{|A|^2} = \frac{1}{1+\beta^2},
	\end{equation}
	hvor $ T $ kaldes for transmissionskoefficienten. Det ses, at sammenlagt giver disse 1: $ R+T = 1 $. Og idet de er funktioner af $ \beta $, er de også funktioner af energien $ E $:
	\begin{equation}
		R = \frac{1}{1+(2\hbar^2E/m\alpha^2)}, \quad T = \frac{1}{1+(m\alpha^2/2\hbar^2 E)}.
	\end{equation}
	Det ses, at jo højere energi, jo større er sandsynligheden for, at bølgen bliver transmitteret.

	Der er dog stadig det problem, at disse bølger ikke er normaliserbare. Men som i den fri partikel, så kan man bruge en bølgepakke, hvormed denne har en række forskellige værdier for $ E $. Dermed skal $ R $ og $ T $ tolkes som ">cirkasandsynligheden"< for partikler med energier i nærheden af $ E $.
	
	En sidste note er, at refleksions- og transmissionskoefficienterne ikke afhænger af fortegnet på $ \alpha $, idet de afhænger af $ \alpha^2 $. Dermed kan man lige så godt have en ">deltafunktionsbarriere"<, med et \textit{negativt} $ \alpha $. Resultatet er, at der stadig er en sandsynlighed for, at bølgepakken, og dermed partiklen, bevæger sig gennem barrieren, hvilket ikke er muligt klassisk, idet ingen energi er stor nok til at overstige et uendeligt potentiale. Dette er netop fænomenet \textbf{tunnelering}. 
	
	\subsection{Den endelige potentialbrønd}
	\textbf{I dette og de resten af afsnittene af dette kapitel, gider jeg ikke at skrive ">den tidsuafhængie Schrödingerligning"<, så jeg skriver bare Schrödingerligningen. Hvis jeg mener den tids\textit{afhængige} liging, så skriver jeg det eksplicit.}
	
	Det sidste potential, der behandles i dette kapitel, er den endelige potentialbrønd:
	\begin{equation}
		V(x) = \begin{cases}
			-V_0, & \text{for } -a < x < a, \\
			0	& \text{for } |x| > a. 
		\end{cases}
	\end{equation}
	hvor $ V_0 $ er en positiv konstant. Dette er også et eksempel på et potential, der tillader både bundne ($ E < 0 $) og ubundne ($ E > 0 $) tilstande. Først kigges der på de bundne:
	
	\paragraph{Bundne tilstande.} For bundne tilstande er $ E < 0 $, og i $ x < -a $ lyder Schrödingerligningen
	\begin{equation}
		\diff[\ud]{^2 \psi}{x^2} = \kappa^2 \psi, \quad \kappa \equiv \frac{\sqrt{-2mE}}{\hbar},
	\end{equation}
	med $ \kappa $ reel og positiv. Igen er løsningen her
	\begin{equation}
		\psi(x) = B e^{\kappa x}, \quad x < -a.
	\end{equation}
	Ligeledes for $ x > a $ fås
	\begin{equation}
		\psi(x) = F e^{-\kappa x}, \quad x > a. 
	\end{equation}
	I midten, hvor $ V(x) = -V_0 $ lyder Schrödingerligningen
	\begin{equation}
		\diff[\ud]{^2 \psi}{x^2} = -\ell^2 \psi, \quad \ell \equiv \frac{\sqrt{2m(E+V_0)}}{\hbar},
	\end{equation}
	hvor, per problem 2.2, $ E > -V_0 $. Dermed er $ \ell $ positiv og reel. Løsningen er
	\begin{equation}
		\psi(x) = C  \sin \ell x + D \cos \ell x, \quad -a < x < a.
	\end{equation}
	Idet potentialet er symmetrisk, så kan vi vælge enten lige eller ulige løsninger. Fordelen er her, at man da kun skal finde randbetingelserne på den ene side. For \textbf{lige} funktioner fås
	\begin{equation}
		\psi(x) = \begin{cases}
			F e^{-\kappa x}, & x > a, \\
			D \cos \ell x, & 0 < x < a, \\
			\psi(-x), & x < 0.
		\end{cases}
	\end{equation}
	Og ved de to randbetingelser (kontinuitet af $ \psi $ og $ d\psi/dx $ i punktet $ x = a $. Divider den ene med den anden) fås følgende ligning:
	\begin{equation}
		\kappa = \ell \tan (\ell a)
	\end{equation}
	og med et koordinatskifte fås
	\begin{equation}
		\tan z = \sqrt{\frac{z_0^2}{z^2} - 1}, \quad z \equiv \ell a, \quad z_0 \equiv \frac{a}{\hbar} \sqrt{2mV_0},
	\end{equation}
	Der er en transcendental ligning i $ z $ (og da også $ E $, idet $ \ell $ afhænger af $ E $), som funktion af $ z_0 $, der er et mål for ">størrelsen"< af brønden. At den er transcendental betyder bare, at den ikke kan løses algebraisk; kun numerisk. 
	
	For \textbf{ulige} funktioner fås
	\begin{equation}
		\psi(x) = \begin{cases}
			F e^{-\kappa x}, & x > a, \\
			C \sin \ell x, & 0 < x < a, \\
			-\psi(-x), & x < 0.
		\end{cases}
	\end{equation}
	Her giver de samme randbetingelser
	\begin{equation}
		\kappa = - \ell \cot(\ell a)
	\end{equation}
	eller
	\begin{equation}
		\cot z = - \sqrt{\frac{z_0^2}{z^2}-1}
	\end{equation}
	med samme definitioner på $ z $ og $ z_0 $. Der er to interessante tilfælde:
	\begin{enumerate}
		\item \textbf{Bred, dyb brønd}. Hvis $ z/z_0 << 1 $ ligger løsningerne lige før $ z_n = n\pi/2 $, for ulige $ n $, hvilket giver
		\begin{equation}
			E_n + V_0 \approx \frac{n^2 \pi^2 \hbar^2}{2m(2a)^2},
		\end{equation}
		hvilket netop er energierne (for ulige $ n $, de andre er fra de ulige funktioner) for den uendelige brønd af bredde $ 2a $. Dette giver da god mening, for jo dybere brønden er, jo nærmere approksimerer den, den uendelige brønd. Det skal dog bemærkes, at der kun er et endeligt antal bundne tilstande (førend $ E>0 $, og de bliver ubundne).
		\item \textbf{Lav, lille brønd.} Som $ z_0 $ bliver mindre kommer der færre og færre bundne tilstande, indtil $ z_0 < \pi/2 $, hvor den sidste \textit{ulige} bundne tilstand forsvinder, og kun en enkelt lige bunden tilstand er tilbage. Der vil dog altid være mindst én bunden tilstand, lige meget størrelsen af $ V_0 $ og $ a $.
	\end{enumerate}
	
	
	\paragraph{Ubundne tilstande.} For ubundne tilstande, med $ E> 0 $, fås i området $ x < -a $:
	\begin{equation}
		\psi(x) = A e^{ikx} + B e^{-ikx}, \quad k \equiv \frac{\sqrt{2mE}}{\hbar}, \quad x <-a,
	\end{equation}
	i området $ x> a $, hvis vi antager ingen indkommen bølge her:
	\begin{equation}
		\psi(x) = F e^{ikx}, \quad x > a,
	\end{equation}
	og til sidst, i midten får vi som før:
	\begin{equation}
		\psi(x) = C \sin \ell x + D \cos \ell x, \quad \ell \equiv \frac{\sqrt{2m(E+V_0)}}{\hbar}, \quad -a < x < a.
	\end{equation}
	Igen er $ A $ den indkomne amplitude, $ B $ den reflekterede og $ F $ den transmitterede. Der er fire randbetingelser: kontinuitet i $ \pm a $ for $ \psi $ og $ \ud\psi/\ud x $. Ud fra disse fås
	\begin{equation}
		B = i \frac{\sin (2\ell a)}{2kl} (\ell^2 - k^2) F, \quad F = \frac{e^{-2ika} A}{\cos (2 \ell a) - i \frac{k^2+\ell^2}{2k\ell}} \sin (2 \ell a).
	\end{equation}
	Dette giver
	\begin{equation}
		T\inverse = 1+ \frac{V_0^2}{4 E(E+V_0)} \sin^2 \pp{\frac{2a}{\hbar} \sqrt{2m(E+V_0)}}
	\end{equation}
	$ T $ er 1, hver gang sinusleddet er 0, hvilket sker for
	\begin{equation}
		\frac{2a}{\hbar} \sqrt{2m(E+V_0)} = n\pi, \quad n\in \Set{N}.
	\end{equation}
	Dette svarer til energierne:
	\begin{equation}
		E_n + V_0 = \frac{n^2 \pi^2 \hbar^2}{2 m (2a)^2},
	\end{equation}
	hvilket igen er præcis energierne for den uendelige brønd. Så for disse energier bliver brønden altså ">gennemsigtig"<. Og for $ E \to 0 $ går $ T \to 0 $, hvilket også er forventet, idet jo lavere energi, jo mindre af bølgen ">slipper igennem"<. 
	
\end{document}