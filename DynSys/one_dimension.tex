\documentclass[dynsysnotes.tex]{subfiles}

\begin{document}
	\section{One dimensional systems}
	In this section we deal with systems of the form
	\begin{equation}\label{key}
		\dot{x} = f(x).
	\end{equation}
	Useful quantities that are easily understood in one dimensional systems (but of course apply to systems of many dimensions) include fixed points, linear stability analysis, potentials and bifurcations. 
	
	\subsection{Fixed point and stability analysis}
	Fixed points $ x^* $ occur whenever the function $ f(x) $ has a root. These points are called fixed since the solution to the equation is constant in time. A simple example is the system $ \dot{x} = x $, which has a fixed point at $ x^*=0 $. One represents the system graphically like this:
	
	
	However not all fixed points are created equally. In the example above, we see that the flow is negative when $ x $ is negative, whilst it is positive when $ x $ is positive. This then means that any small perturbation from the fixed point will always increase (in magnitude) with time. We call these points \textit{unstable}, and draw them as unfilled circles.
	
	On the other hand, the system $ \dot{x} = -x $ has a \textit{stable} fixed point at $ x^* = 0 $, since all perturbations will be (exponentially) damped. These points are drawn as filled circles.
	
	In general, the derivative of $ f(x) $, with respect to $ x $ determines the stability of the point. Often, \textbf{linear stability analysis} enough. This states that $ x^* $ is a stable (unstable) fixed point, if $ f'(x^*) < 0 $ ($ f'(x^*) > 0 $). If, however, $ f'(x^*) = 0 $ then linear stability analysis will not be enough to determine the stability of the point, and one needs to look at the higher order derivatives of $ f(x) $.
	
	For example, if $ f''(x^*) > 0 $ ($ f''(x^*) < 0 $), the system has minimum (maximum), and the system is left (right) half stable. A left half stable fixed point is stable to perturbations pushing it to the left, but unstable to perturbations pushing it to the right. An example of this is $ \dot{x} = x^2 $, which has a left stable fixed point at $ x^* = 0 $:
	
	If then both the first and second derivative at the fixed point is zero, then the third derivative can be used, where the point is stable if the third derivative is less than zero, and unstable if it is positive. Of course one can go as far as one likes with this, and utilize higher and higher derivatives.
	
	
	\subsection{Potentials}
	Another way to view the one dimensional system is through the use of potentials. Here we define 
	\begin{equation}\label{key}
		f(x) = - \diff[d]{V(x)}{x}
	\end{equation}
	Fixed points in the system are then stationary points for the potential. In this formulation, one can imagine a ball rolling down a hill (though without inertia, as if flowing through molasses). Minima correspond to stable fixes points and maxima to unstable points (and saddle points to half stable points, of course).
	
	
	\subsection{Bifurcations}
	A bifurcation is a point in parameter space, at which the stability or number of fixed points changes. In this section a number of bifurcation types appear:
	\begin{itemize}
		\item Saddle-node bifurcations
		\item Transcritical bifurcations
		\item Pitchfork bifurcations (both subcritical and supercritical)
	\end{itemize}
	When introducing these types of bifurcations one often looks at normal forms, which are the simplest system that exhibit the behaviour in question. Another way of looking at this, is that the Taylor expansion for a system follows the normal form (or at least a scaled version of it) in the vicinity of the bifurcation.

	A saddle-node bifurcation is a point in which two fixed points appear out of nothing. The normal form for this is
	\begin{equation}\label{key}
		\dot{x} = r + x^2
	\end{equation}
	This system has a saddle-node bifurcation at $ r_c = 0 $, at which a left half stable point appears at $ x^*=0 $. If $ r $ is decreased further, this point separates into a stable point at $ x^*=-\sqrt{-r} $ and an unstable point at $ x^*=\sqrt{-r} $. This can be drawn in a \textbf{bifurcation diagram}, where one plots the fixed points as a function of the parameter (in this case $ r $):
	
	In a bifurcation diagram, stable \textit{branches} (lines indicating stable fixed points) are drawn up fully, whilst unstable branches are drawn dashed.
	
	A transcritical bifurcation is one in which no fixed points are created or destroyed, but in stead changes its stability. The normal form for these kinds of bifurcations is
	\begin{equation}\label{key}
		\dot{x} = rx - x^2,
	\end{equation}
	which always has a fixed point at $ x^*=0 $ (and one at $ x^*= r $). The bifurcation occurs at $ r_c = 0 $, with the fixed point being stable (unstable) for $ r<r_c $ ($ r>r_c $) and right half stable at $ r=r_c $. Its bifurcation diagram looks like so:
	
	
	A pitchfork bifurcation is one where a single fixed point turns into three fixed points, reminiscent of the shape of a pitchfork. The supercritical pitchfork starts out as a stable point, which turns into an unstable fixed point at $ r=r_c $, and forms two additional stable points for $ r>r_c $. Its normal form is
	\begin{equation}\label{key}
		\dot{x} = rx - x^3, \quad r_c=0, \quad x^*=0 \vee x^* = \pm \sqrt{r},
	\end{equation}
	and its bifurcation diagram looks like so:
	
	For a subcritical pitchfork bifurcation the pitchfork is inverted. The normal form and fixed points are
	\begin{equation}\label{key}
		\dot{x} = rx+x^3, \quad r_c=0, \quad x^*=0 \vee x^*= \pm\sqrt{-r},
	\end{equation}
	with the following bifurcation diagram:
	
	
	
	
	
	
	
\end{document}