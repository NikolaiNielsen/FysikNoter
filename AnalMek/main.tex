% Nikolai Nielsens "Fysiske Fag" preamble
\documentclass[a4paper,10pt]{report} 	% A4 papir, 10pt størrelse

\usepackage{Nikolai} 					% Min hjemmelavede pakke
\usepackage[dvipsnames]{xcolor}
\usepackage[Danish]{babel}

% Margen
\usepackage[margin=1in]{geometry}

% Max antal kolonner i en matrix. Default er 10
%\setcounter{MaxMatrixCols}{20}

% Hvor dybt skal kapitler labeles?
%\setcounter{secnumdepth}{4}	
%\setcounter{tocdepth}{4}


% Hvilket nummer skal der startes med i sections? (n-1)
%\setcounter{section}{0}	

% Til nummerering af ligninger. Så der står (afsnit.ligning) og ikke bare (ligning)
\numberwithin{equation}{section}


% Header
%\usepackage{fancyhdr}
%\head{Nikolai Plambech Nielsen, 21-06-95\\Dato:. Klasse 5.C - De Fysiske Fag}
%\pagestyle{fancy}

%Titel
\title{Noter til Analytisk Mekanik 2017}
\author{Nikolai Plambech Nielsen, LPK331. Version 1.0}
\date{}

\begin{document}
	
	\selectlanguage{danish}
	
	
	\maketitle
	\tableofcontents
	\newpage
	\section*{Eksamensspørgsmål}
	Eksamensspørgsmålene er som følger
	\begin{enumerate}
		\item \textbf{Newtons love og Kepler-problemet} [Første indgang til bevarelseslove, konservative kræfter, $ 1/r $ potentialet, planetbaner, spredning, tværsnit]
		\item \textbf{Accelererede koordinatsystemer} [Rotation og translation, bevægelse på den roterende jord]
		\item \textbf{Lagrangedynamik} [Holonomiske constraints, generaliserede koordinater, d'Alemberts Princip, Langrange-ligningerne, generaliserede kræfter]
		\item \textbf{Variationsregning} [Lagrange-ligningerne fra et variationsprincip (Hamiltons Princip), Lagrange-multiplikatorer og constraint-kræfter]
	\end{enumerate}
	\newpage
	\subfile{BasicPrinciples}
	
\end{document}

