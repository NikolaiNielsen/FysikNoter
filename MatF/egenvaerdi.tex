\documentclass[MatFNoter.tex]{subfiles} % HUSK FOR FANDEN AT REDIGERE DENNE LINJE

% Hvis ikke dokumenterne (hoved & under) er i samme mappe, skal den relative stig bruges.



\begin{document}
	
	\section{Egenfunktionsmetoden til Differentialligninger}
	\label{sec:egenfunkt}
	Denne metode til løsning af differentialligninger beskæftiger sig med inhomogene ligninger på formen
	\begin{equation}
		\mathcal{L} y(x) = f(x)
	\end{equation}
	Hvor $ \mathcal{L} $ er en lineær differentialoperator. Eksempelvis fås Legendreligningen:
	\begin{equation*}
		(1-x^2)y''-2xy'+\lambda y = f(x), \quad \mathcal{L} = (1-x^2) \diff[d]{^2}{x^2} - 2x \diff[d]{}{x} + \lambda
	\end{equation*}
	Hvis ikke $ f(x) $ er simpel eller kendt, så løses ligningen ved at opbygge en superposition, normalt af en uendelig serie af funktioner $ \{y_i(x)\} $. Vi opbygger sættet af funktioner, der opfylder følgende ligning
	\begin{equation}
		\mathcal{L} y_i(x) = \lambda_i y_i(x)
	\end{equation}
	Hvor $ \lambda_i $ er en konstant og kaldes \textbf{egenværdien} til \textbf{egenfunktionen} $ y_i(x) $.
	
	Der kan også diskuteres en mere generel form af ovenstående ligning, hvor der inkluderes en "vægtfunktion":
	\begin{equation}
			\mathcal{L}y_i(x) = \lambda_i \rho(x) y_i(x)
	\end{equation}
	hvor $ \rho(x) $ er vægtfunktionen. Normalt set er denne 1, hvormed forrige ligning genopstår. Generelt set er den dikteret af valget af koordinatsystem, og der er nogle restriktioner på funktionen:
	\begin{itemize}
		\item Den skal være reel
		\item Den må ikke skifte fortegn i intervallet $ [a;b] $, hvor $ a $ og $ b $ er grænserne for differentialligningen
		\item Den skal være ens for alle $ \lambda_i $.
	\end{itemize} 
	
	
	
		
	\subsection{Indre produkt for funktioner og Hilbertrum}
	Til behandlingen af disse typer differentialligninger, skal indre produkter for funktioner, samt uendeligtdimensionelle funktionsrum diskuteres. Ses enhver velopførende funktion i intervallet $ a\leq x \leq b $ opbygget af en uendelig serie lineært uafhængige basisfunktioner, kan den opskrives som:
	\begin{equation}
		f(x) = \sum_{n=0}^{\infty} c_n y_n(x)
	\end{equation}
	Hvor $ c_n $ er skalaren og $ y_n(x) $ er basisfunktionen. Med velopførende menes at den opfylder Dirichletbetingelserne opstillet i næste kapitel (Fourierserier, Kap \ref{sec:FourierSerie}).
	
	Til dette funktionsrum defineres det \textbf{indre produkt} som:
	\begin{equation}
		\ind{f}{g} = \int_{a}^{b} f^*(x) g(x) \rho(x) \ud x
	\end{equation}
	Hvis dette integral er 0 er funktionerne \textbf{ortogonale}. \textbf{Normen} (længden el. størrelsen) af en funktion er defineret som
	\begin{equation}
		\norm{f}= \ind{f}{f}^{1/2} = \bb{\int_{a}^{b} f^*(x) f(x) \rho(x) \ud x}^{1/2} = \bb{\int_{a}^{b} \lvert f(x) \rvert^2 \rho(x) \ud x}^{1/2}
	\end{equation}
	En \textbf{normaliseret} funktion defineres som $ \hat{f} = f/\norm{f} $, lige som ved vektorer. Normen af denne er da 1.
	
	Et uendeligdimensionalt funktionsrum, hvor det indre produkt er defineret, kaldes for et \textbf{Hilbertrum}. Til disse rum defineres ofte en ortonormalbasis $ \hat{\psi}_n(x), n=0,1,2,\dots $:
	\begin{equation}
		\ind{\hat{\psi}_i}{\hat{\psi}_j} = \int_{a}^{b} \hat{\psi}_i^* (x) \hat{\psi}_j \rho(x) \ud x = \delta_{ij}
	\end{equation}
	hvor $ \delta_{ij} $ er \textbf{Kroneckers delta} (fra LinAlg): $ \delta_{ij} = 1 $ hvis $ i=j $ og 0 ellers.
	
	I en ortonormalbasis er funktionen givet ved
	\begin{equation}
		f(x) = \sum_{n=0}^{\infty} c_n \hat{\psi}_n(x)
	\end{equation}
	Og $ c_n $ er givet ved
	\begin{equation}
		c_n = \ind{\hat{\psi}_n}{f} = \int_{a}^{b} \hat{\psi}_i^* (x) f(x) \rho(x) \ud x 
	\end{equation}
	For Hilbertrum gælder både Schwarz uligheden og trekantuligheden:
	\begin{equation}
		\lvert \ind{f}{g} \rvert \leq \norm{f} \D \norm{g}, \quad \norm{f+g} \leq \norm{f} + \norm{g}
	\end{equation}
	Er der ydermere opgivet en ortonormalbasis gælder Bessel's ulighed:
	\begin{equation}
		\ind{f}{f} \geq \sum_{n} |c_n|^2
	\end{equation}
	Ligheden holder hvis der summeres over alle basiselementer, mens uligheden holder hvis nogle værdier af $ n $ udelades.
	
	\subsection{Adjungerede og Hermitiske operatorer}
	Den \textbf{adjungerede} til den lineære differentialoperator $ \mathcal{L} $ betegnes $ \mathcal{L}^{\dagger} $, og er defineret ved
	\begin{equation}
		\int_{a}^{b} f^*(x) [\mathcal{L} g(x)] \ud x = \int_{a}^{b} [\mathcal{L}^{\dagger} f(x)]^* g(x) \ud x + \text{randbetingelser}
	\end{equation}
	Hvor randbetingelserne evalueres i grænserne af intervallet $ [a;b] $. Den adjungerede til enhver lineær differentialoperator kan findes ved $ n $ gange partiel integration (hvor $ n $ er den største grad af differentiation i operatoren).
	
	Operatoren siges at være selvadjungeret, hvis $ \mathcal{L}^{\dagger} = \mathcal{L} $ (som ved vektorer). Hvis randbetingelserne forsvinder - enten grundet egenskaber ved funktionerne $ f $ og $ g $, eller ved operatoren $ \mathcal{L} $, kaldes operatoren for \textbf{Hermitisk} over intervallet $ a\leq x \leq b $. I så fald fås
	\begin{equation}
		\int_{a}^{b} f^*(x) [\mathcal{L} g(x)] \ud x = \int_{a}^{b} [\mathcal{L} f(x)]^* g(x) \ud x
	\end{equation}
	
	\subsubsection*{Egenskaber ved Hermitiske operatorer}
	\begin{itemize}
		\item Egenværdierne til Hermitiske operatorer er altid reelle
		\item Egenfunktioner hørende til forskellige egenværdier, for en Hermitisk operator, er ortogonale
		\item Egenfunktionerne til en Hermitisk operator danner en komplet basis over det relevante interval
		\item Der kan altid konstrueres reelle egenfunktioner fra lineære kombinationer af komplekse egenfunktioner (mindst én af $ y_i^* + y_i $ og $ i(y_i^*-y_i) $ er en reel egenfunktion, forskellig fra 0, svarende til dennes egenværdi)
	\end{itemize}
	
	\subsection{Sturm-Liouvilleligninger}
	En vigtig type differentialligninger, hvor differentialoperatoren er Hermitisk, er Sturm-Liouvilleligninger, der har formen
	\begin{equation}
		p(x) \diff[d]{^2 y}{x^2} + r(x) \diff[d]{y}{x}+q(x) y + \lambda \rho(x) y= 0, \quad r(x) = \diff[d]{p(x)}{x}
	\end{equation}
	eller
	\begin{equation}
		(py')'+qy+\lambda \rho y = 0
	\end{equation}
	Hvor $ p $, $ q $ og $ r $ er reelle funktioner. Denne kan omskrives til formen givet i starten af kapitlet:
	\begin{align}
		\mathcal{L}y = \lambda \rho(x) y, \quad \mathcal{L} &= -\bb{p(x) \diff[d]{^2}{x^2}+ r(x) \diff[d]{}{x} + q(x)} \\
		&= -\bb{\diff[d]{}{x}\pp{p(x) \diff[d]{}{x}}+q(x)}
	\end{align}
	
	Sturm-Liouvilleoperatoren er Hermitisk over intervallet $ [a;b] $ såfremt der for enhver to egenfunktioner $ y_i $ og $ y_j $ gælder
	\begin{equation}
		[y_i^* \, p \,  y'_j]_{x=a} = [y_i^* \, p \,  y'_j]_{x=b}, \quad [y_i^* \, p \,  y'_j]_{x=a}^{x=b} = 0
	\end{equation}
	Eksempelvis er disse betingelser mødt, hvis $ y(a) = y(b) = 0 $; $ y(a) = y'(b) = 0 $ eller $ p(a)=p(b) = 0 $. Grundet det sidste eksempel har mange Sturm-Liouvilleligninger et ``naturligt'' interval, hvor operatoren er Hermitisk, uanset funktionsværdierne til $ x=a $ og $ x=b $.
	
	\subsubsection*{Transformation til Sturm-Liouvilleform}
	Enhver andenordens differentialligning, der kan skrives på formen
	\begin{equation}
		p(x)y''+r(x) y' + q(x) y + \lambda\rho(x) y = 0
	\end{equation}
	(hvor $ r = p' $ ikke nødvendigvis er sandt), kan omskrives til Sturm-Liouvilleform, ved at gange igennem med faktoren:
	\begin{equation}
		F(x) = \exp \kk{\int^{x} \frac{r(u)-p'(u)}{p(u)} \ud u}
	\end{equation}
	Det ses, at hvis $ r=p' $ fås $ F(x) = e^0 = 1 $ og den generelle Sturm-Liouville optræder.
	
	En række differentialligningers transformation til Sturm-Liouvilleform er opgivet i følgende tabel\footnote{Tabellen er taget fra EMM og navnene på differentialliningerne er bare groft oversat. Hvis der er fejl så ret dem endelig}. Stjernen ved Besselligningen betegner, at et skift af variabel, $ x \to  x/\alpha $ er nødvendigt for at få den konventionelle normalisering, men er ikke nødvendig for transformationen.
	\begin{table}[H]
		\centering
		\begin{tabular}{lcccc}
			Ligning                   &        $ p(x) $         &      $ q(x) $      & $ \lambda $  &       $ \rho(x) $        \\ \hline
			Hypergeometrisk           & $ x^c (1-x)^{a+b-c+1} $ &         0          &   $ -ab $    & $ x^{c-1} (1-x)^{a+b-c}$ \\
			Legendre                  &        $ 1-x^2 $        &         0          &  $ \ell(\ell+1) $  &            1             \\
			Associeret Legendre       &        $ 1-x^2 $        & $ -m^2 / (1-x^2) $ &  $ \ell(\ell+1) $  &            1             \\
			Chebyshev                 &    $ (1-x^2)^{1/2} $    &         0          &  $ \nu^2 $   &    $ (1-x^2)^{-1/2} $    \\
			Konfluent hypergeometrisk &     $ x^c e^{-x} $      &         0          &    $ -a $    &    $ x^{c-1}e^{-x} $     \\
			Bessel*                   &          $ x $          &    $ -\nu^2/x $    & $ \alpha^2 $ &          $ x $           \\
			Laguerre                  &       $ x\e{-x} $       &         0          &   $ \nu $    &        $ \e{-x} $        \\
			Associerede Laguerre      &   $ x^{m+1} \e{-x} $    &         0          &   $ \nu $    &      $ x^m \e{-x} $      \\
			Hermite                   &      $ \e{-x^2} $       &         0          &   $ 2\nu $   &       $ \e{-x^2} $       \\
			Simpel harmonisk          &            1            &         0          & $ \omega^2 $ &            1
		\end{tabular} 
	\end{table} 
	Fra $ p(x) $ i tabellen kan det naturlige interval $ [a;b] $, hvor $ p(a)=p(b) $ aflæses. For Legendre, den associerede Legendre og Chebyshevligningerne er intervallet $ [-1;1] $. For Laguerre og dens associerede er intervallet $ [0;\infty] $. For Hermiten er intervallet $ [-\infty;\infty] $. For den simple harmoniske er intervallet $ [x_0;x_0+2\pi] $ (følger af randbetingelsen for at SL-operatoren er Hermitisk).
\end{document}

