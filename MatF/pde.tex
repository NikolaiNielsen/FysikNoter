\documentclass[MatFNoter.tex]{subfiles} % HUSK FOR FANDEN AT REDIGERE DENNE LINJE

% Hvis ikke dokumenterne (hoved & under) er i samme mappe, skal den relative stig bruges.



\begin{document}
	
\section{Partielle differentialligninger}
En partiel differentialligning (Partial Differential Equation, eller PDE på engelsk) er en differentialligning, hvor der indgår partielle differentialkvotienter i ligningen - altså en differentialligning af flere variabler. En stor del af vigtige differentialligninger i fysikken er både lineære (så teknikkerne fra kapitel \ref{sec:egenfunkt} kan bruges), og af anden orden. 

til partielle differentialligninger bruges $ u $ som benævnelse for funktionen, i stedet for $ y $ for ordinære differentialligninger (diff-ligninger af én variabel).

Som eksempler på PDE ses følgende
\subsection*{Bølgeligningen}
\begin{equation}
	\grad^2 u = \frac{1}{c^2}\diff{^2u}{t^2}
\end{equation}
Funktionen i bølgeligning beskriver et system i tid, der oplever en forskydning fra et ækvilibrium, og en genoprettende kraft, der forsøger at få systemet tilbage i ækvilibrium. Eksempler er guitarstrenge (1 rumlig dimension), trommeskind (2 rumlige dimensioner) og elektromagnetiske bølger (1-3 rumlige dimensioner).

\subsection*{Diffusionsligningen}
\begin{equation}
	\kappa \grad^2 u = \diff{u}{t}
\end{equation}
Diffusionsligningen beskriver temperaturen i en varmeledende region, hvor der ikke er nogen resevoirer, der tilfører eller fjerner varme fra systemet, eller diffusionen af kemikalier, der har koncentrationen $ u(\V{r},t) $. $ \kappa $ kaldes for diffusiviteten af materialet.

Diffusionsligningen kan generaliseres til følgende:
\begin{equation}
	k \grad^2 u + f(\V{r},t) = s\rho \diff{u}{t}
\end{equation}
Hvor $ f(\V{r},t) $ repræsenterer en varierende densitet af varmekilder i legemet, men bruges ofte ikke i fysiske sammenhænge. Både $ k $, $ s $ og $ \rho $ kan afhænge af positionen $ \V{r} $, hvormed første led bliver til $ \grad\D (k\grad u) $.

\subsection*{Laplaces ligning}
\begin{equation}
	\grad^2 u = 0
\end{equation}
Laplaces ligning beskriver blandt andet \textit{steady-state} temperaturen af et system, der ikke indeholder nogen varmekilder (altså systemet efter lang tid). Ligningen beskriver også det gravitationelle potentiale i et felt, hvor der ikke er masse, eller lignende med elektriske felter uden ladninger. Den ses også generelt for divergens- og rotationsfrie felter.

\subsection*{Poissons ligning}
\begin{equation}
	\grad^2 u = \rho (\V{r})
\end{equation}
Poissons ligning beskriver samme fysiske situationer som Laplaces ligning - men hvor der er varmekilder, masse eller elektrisk ladning. $ \rho(\V{r}) $ kaldes for kildedensiteten.

\subsection*{Schrödingers ligning}
\begin{equation}
	-\frac{\hbar^2}{2m} \grad^2 u + V(\V{r})u = i\hbar \diff{u}{t}
\end{equation}
Schrödingers ligning er basis for den kvantemekaniske bølgefunktion af en ikkerelativistisk partikel $ u(\V{r},t) $. $ \hbar $ er den reducerede Planck konstant og $ m $ er partiklens masse.



\subsection{Generel løsningsform for 2 variable (parametrisering)}
Vi søger generelt løsninger af formen:
\begin{equation}
	u(x,y) = f(p(x,y))
\end{equation}
De partielle differentialkvotienter for løsningen bliver:
\begin{align}
	\diff{u}{x} &= \diff[d]{f(p)}{p} \diff{p}{x} \\
	\diff{u}{y} &= \diff[d]{f(p)}{p} \diff{p}{y}
\end{align}
Ved at sætte disse lig hinanden, kan alle referencer til $ f(p) $ forsvinde:
\begin{equation}
	\diff{p}{y} \diff{u}{x} = \diff{p}{x} \diff{u}{y}
\end{equation}
Det betyder, at enhver funktion af typen $ f(p) $ løser en given differentialligning - såfremt parameteriseringen $ p(x,y) $ er korrekt. Det er denne metode, der generelt bruges til løsning af ligninger med 2 variable.



\subsection{Løsning af 1. ordens PDE}
Den mest generelle førsteordens partielle differentialligning er
\begin{equation*}
	A(x,y) \diff{u}{x} + B(x,y) \diff{u}{y} + C(x,y)u = R(x,y)
\end{equation*}
hvor $ A,B,C,R $ er funktioner af $ x $ og $ y $ (eller konstante). Til start arbejdes der kun med ligninger, hvor $ C=R=0 $. Sættes $ u=f(p) $ ind i denne reducerede ligning fås:
\begin{equation*}
	\bb{A(x,y) \diff{p}{x} + B(x,y)\diff{p}{y}} \diff[d]{f(p)}{p} = 0
\end{equation*}
Her skal den sidste faktor være lig 0, hvilket giver at $ p = konstant $. Dette opnås ved at integrere følgende ligning, og sætte en konstant lig $ p $
\begin{equation}
	\frac{dx}{A(x,y)} = \frac{dy}{B(x,y)}
\end{equation}
Det er ikke nødvendigvis den første integrationskonstant, der skal sættes lig $ p $. Der er ret stor frihed for at vælge $ p $, så udtrykket for denne bliver pænt (for eksempel uden brøk-potenser eller lignende).

\subsubsection*{\texorpdfstring{Ligninger med $ C(x,y)\neq 0 $}{Ligninger med C(x,y) forskellig fra 0}}
Er $ C \neq 0 $ i differentialligningen, ledes der efter løsninger af formen
\begin{equation}
	u = h(x,y) f(p)
\end{equation}
hvor $ h(x,y) $ er en funktion (lige meget hvor simpel - tjek med en konstant først), der løser ligningen hele ligningen. $ f(p) $ er her stadig løsningen til ligningen $ A \diff{u}{x} + B\diff{u}{y} = 0 $.

Fremgangsmetoden er da at løse ligningen som normalt, hvor $ C = 0 $, og derefter finde et simpelt udtryk, der løser ligningen, hvor $ C \neq 0 $.



\subsection{Inhomogene ligninger og problemer}
Der gøres forskel på inhomogene og homogene problemer og ligninger. Hvis $ u(x,y) $ er en løsning til en given PDE, med eller uden randbetingelser, siges \textbf{ligningen} at være \textbf{homogen}, hvis $ \lambda u(x,y) $, hvor $ \lambda $ er en given konstant, også er en løsning (dette er gældende, hvis $ R=0 $). \textbf{Problemet} siges at være \textbf{homogent}, hvis $ \lambda u(x,y) $ også opfylder randbetingelserne for problemet. Randbetingelser af denne type kaldes også for \textbf{homogene randbetingelser}.

Generelt, til løsning af inhomogene ligninger eller problemer, ledes der efter en løsning på formen
\begin{equation}
	u(x,y) = v(x,y) + w(x,y)
\end{equation}
Hvor $ v(x,y) $ er en tilfældig løsning til problemet
\begin{equation}
	A(x,y)\diff{u(x,y)}{x} + B(x,y)\diff{u(x,y)}{y} + C(x,y)u(x,y) = R(x,y)
\end{equation}
der også opfylder randbetingelserne. Mens $ w(x,y) $ er den generelle løsning til
\begin{equation}
	A(x,y)\diff{u(x,y)}{x} + B(x,y)\diff{u(x,y)}{y} + C(x,y)u(x,y) = 0
\end{equation}
hvor $ w(x,y) = 0 $ i randbetingelserne. Dermed opfylder den samlede løsning $ u(x,y) $ enhver randbetingelse for problemet. Dette princip kan også udvides, således at der inkluderes flere funktioner - måske $ n+1 $, hvor $ n $ er antallet af randbetingelser. Én løsning til det homogene problem, der er 0 i alle randbetingelser, mens hver af de andre løsninger løser én randbetingelse hver især. Ydermere fungerer dette princip også til PDE'er af højere orden. 

\subsection{Løsning af 2. ordens PDE}
Den mest generelle, lineære, andenordens, partielle differentialligning, af to variable, er:
\begin{equation}
	A \diff{^2 u}{x^2} + B \diff{^2 u}{x \partial y} + C \diff{^2 u}{y^2} + D\diff{u}{x} + E\diff{u}{y} + Fu = R(x,y),
\end{equation}
hvor $ A,B,C,D,E,F,R $ er funktioner af $ x $ og $ y $. Disse ligninger deles ofte ind i tre klasser:
\begin{itemize}
	\item Hyperbolsk, hvis $ B^2 > 4AC $
	\item Parabolsk, hvis $ B^2 = 4AC $
	\item Elliptisk, hvis $ B^2 < 4AC $
\end{itemize}
Hvis $ A,B,C $ ikke er konstante, men funktioner af $ x $ og $ y $, kan ligningen skifte klasse i forskellige dele af planen.	I dette afsnit beskæftiger vi os kun med ligninger, hvor $ A,B,C $ er konstante, og hvor $ D=E=F=R(x,y) = 0 $. Dermed er det kun ligninger på formen
\begin{equation}
	A \diff{^2 u}{x^2} + B \diff{^2 u}{x \partial y} + C \diff{^2 u}{y^2} = 0. \label{eq:andenordensnem}
\end{equation}
Igen ledes der efter løsninger på formen $ u = f(p(x,y)) $, hvor $ p = ax+by $. Differentieres dette udtryk for $ u $ to gange, med alle kombinationer af variable, og substitueres ind i ligningen, fås:
\begin{equation*}
	(Aa^2+Bab+Cb^2) \diff[d]{f(p)}{p^2} = 0
\end{equation*}
Hvis den anden faktor er lig 0, er løsningen $ u = kx+ly+m $, mens hvis parentesen sættes lig 0, kan den løses som en andengradsligning, hvor $ b/a $ er variablen. Løsningen er
\begin{equation*}
	b/a = \frac{-B \pm \sqrt{B^2-4AC}}{2C}
\end{equation*}
Så $ A $ og $ C $ har altså skiftet plads, i forhold til den normale andengradsløsning! Hvis de to løsninger kaldes $ \lambda_1 $ og $ \lambda_2 $ fås
\begin{equation}
	p_1 = x+\lambda_1 y, \quad p_2 = x+\lambda_2 y
\end{equation}
Og \textbf{den generelle løsning til ligning} \eqref{eq:andenordensnem} bliver
\begin{equation}
	u(x,y) =  f(p_1) + g(p_2)
\end{equation}
hvor $ f $ og $ g $ er arbitrære funktioner af henholdsvis $ p_1 $ og $ p_2 $.

\subsubsection{Hyperbolske og elliptiske ligninger}
I ligninger, der er enten hyperbolske eller elliptiske, hvor $ A,B,C \in \Set{R} $ fås løsninger, hvor $ p_1 = x+\alpha y $ og $ p_2 = x+i\beta y $, hvor $ \alpha,\beta \in \Set{R} $.

\subsubsection{Parabolske ligninger}
I parabolske ligninger er dobbeltroden givet ved $ -B/2C $, og løsninger af formen
\begin{equation*}
	u(x,y) = f(x-(B/2C)y) +  h(x,y)\D g(x-(B/2c)y),
\end{equation*}
søges, hvor $ h(x,y) $ er en simpel løsning til den oprindelige ligning. Normalt sættes $ h(x,y)= x $, og løsningen bliver

\begin{equation}
	u(x,y) = f(x-(B/2C)y) +  xg(x-(B/2c)y),
\end{equation}

\subsection{Tabel over løsninger}
Løsninger på de ovenstående ligninger kan opsummeres i følgende tabel, hvor $ P,Q,R $ er funktioner af $ x $ og $ y $, og $ A,B,C $ er konstante.
\begin{table}[H]
	\centering
	\begin{tabular}{p{3cm} l p{7cm}}
		Navn                                                 & Generel form                                    & Typisk løsning ved parametrisering                                                                                               \\
		\hline
		\rule[-2ex]{0pt}{5.5ex}
		 Første orden              & $ P \diff{u}{x} + Q \diff{u}{y} = 0 $                                     & $ u=f(p) $, hvor $ p= \int \frac{\ud x}{P} - \int \frac{\ud y}{Q} $                                                              \\
		\rule[-2ex]{0pt}{5.5ex}
		 Første orden              & $  P \diff{u}{x} + Q \diff{u}{y} + Ru = 0 $                               & $ u=h(x,y)f(p) $, hvor $ p $ er som ovenstående, og $ h $ er en tilfældig løsning til ligningen                                  \\
		\rule[-2ex]{0pt}{5.5ex}
		 Anden orden               & $  A \diff{^2u}{x^2} + B \diff{^2u}{x\partial y} + C \diff{^2u}{y^2}= 0 $ & $ f(x+\lambda_1 y)+g(x+\lambda_2 y) $, hvor $ \lambda_i $ løser $ A+B\lambda^2+C\lambda^2 = 0$                                   \\
		\rule[-2ex]{0pt}{5.5ex}
		Anden orden, med $B^2=4AC$ & $  A \diff{^2u}{x^2} + B \diff{^2u}{x\partial y} + C \diff{^2u}{y^2}= 0 $ & $ f(x+\lambda_1 y)+xg(x+\lambda_1 y) $                                                                                           \\
		\rule[-2ex]{0pt}{5.5ex}
		   Laplace                 & $ \grad^2 u = 0  $                                                        & 2D: $ f(x+iy)+g(x-iy) $                                                                                                          \\
		\rule[-2ex]{0pt}{5.5ex}
		   Poisson                 & $ \grad^2 u = \rho(\V{r}) $                                               & 2D: $ f(x+iy)+g(x-iy)+h(x,y) $, hvor $ h $ er en tilfældig løsning til ligningen                                                 \\
		\rule[-2ex]{0pt}{5.5ex}
		    Bølge                  & $ \grad^2 u = \frac{1}{c^2} \diff{^2 u}{t^2} $                            & 1D: $ f(x-ct)+g(x+ct) $. 3D: $ f(\hat{\V{n}}\D \V{r}-ct)+g(\hat{\V{n}}\D \V{r}+ct) $                                             \\
		\rule[-2ex]{0pt}{5.5ex}
		  Diffusion                & $ \kappa \grad^2 u = \diff{u}{t} $                                        & 1D: $ \frac{\alpha}{2\kappa}x^2 + gx + \alpha t + c, \propto [\erf(\zeta)-\erf(\zeta_0)] $, hvor $ \zeta = x/2(\kappa t)^{1/2} $
	\end{tabular} 
\end{table}

\section{Separation af de variable for PDE (kartesiske koordinater)}
Til denne metode for løsning af PDE, ledes der efter løsninger på formen
\begin{equation}
	u(x,y,z,t) = X(x)Y(y)Z(z)T(t)
\end{equation}
Hvor $ X,Y,Z,T $ hver især afhænger af deres respektive små bogstaver. Denne type løsning kaldes \textit{separabel} i $ x,y,z,t $, men løsninger kan sagtens være separable i færre dimensioner (for eksempel kan en ligning i 4 dimensioner sagtens kun være separabel i $ x $ og $ z $).

I løsninger af denne type, må der gerne indgå parametre af forskellig art i flere af koordinatfunktionerne. Restriktionen ligger i, at $ X $ ikke må afhænge af hverken $ y,z,t $, og lignende for de andre funktioner.

Et eksempel på denne metode er den tredimensionelle bølgeligning:
\begin{equation}
	\diff{^2 u}{x^2}+\diff{^2 u}{y^2}+\diff{^2 u}{z^2}= \frac{1}{c^2}\diff{^2 u}{t^2}
\end{equation}
Hvis løsningstypen indsættes, og der divideres igennem med $ u=XYZT $ fås følgende:
\begin{equation}
	\frac{X''}{X}+\frac{Y''}{Y}+\frac{Z''}{Z}=\frac{1}{c^2}\frac{T''}{T}
\end{equation}
hvor dobbeltmærke her er dobbelt differentiation med hensyn til deres respektive variabel ($ X'' $ er differentieret med hensyn til $ x $ to gange). Her ses det, at hvert led afhænger af én bestemt variabel, men alligevel er der et lighedstegn mellem dem. Dermed må hvert led være lig med en konstant. Her sættes
\begin{equation}
	\frac{X''}{X} = -l^2, \quad \frac{Y''}{Y} = -m^2, \quad \frac{Z''}{Z} = -n^2, \quad \frac{1}{c^2}\frac{T''}{T} = -\mu^2
\end{equation}
Hvormed 4 ordinære differentialligninger opnås. De fire konstante kaldes \textit{separationskonstanter}.

Løsningen til den første ligning er eksempelvis:
\begin{equation}
	X(x) = A \exp (ilx) + B \exp(-ilx) = A' \cos (lx) + B' \sin(lx)
\end{equation}
hvor $ A,B,A',B' $ er konstante, der kan findes ved randbetingelser.


\subsection{Superposition af separerede løsninger}
Hvis differentialligningen er lineær, kan superpositionsprincippet bruges. Eksempelvis fås Laplaces ligning i to dimensioner, hvor løsningen er af formen
\begin{equation}
	u_{\lambda_j}(x,y) = X_{\lambda_j}(x)Y_{\lambda_j}(y)
\end{equation}
hvor $ \lambda_j $ er separationskonstanten. Her er superpositionen:
\begin{equation}
	u(x,y) = \sum_{i} a_i X_{\lambda_i}(x)Y_{\lambda_i}(y)
\end{equation}
også en løsning til ligningen, såfremt $ \lambda_i $ er en tilladt værdi af separationskonstanten (dette bestemmes af randbetingelserne). Fordelen ved en superposition af denne type, er at konstanterne $ a_i $ kan vælges, så summen overholder alle randbetingelser.

Dette betyder også, at løsningerne kan vælges således, at de hver især løser en enkelt randbetingelse, mens den samlede superposition opfylder alle randbetingelser (lige som ved parametrisering).

Som eksempler på disse metoder anbefales klart side 426-432 i EMM, da disse tydeligt viser brugen af metoden.




\subsection{Laplaces ligninger i polære koordinater}
Nogle gange kan det være mere gavnligt at løse en differentialligning i et andet koordinatsystem, således, at der kan bruges symmetrier fuldt ud. Først opstilles Laplace-operatoren i plane ($ u(\rho,\phi) $) og sfæriske ($ u(r,\theta,\phi) $) polarkoordinater:
\begin{align}
	\grad^2 &= \frac{1}{\rho} \diff{}{\rho} \pp{\rho \diff{}{\rho}}+ \frac{1}{\rho^2} \diff{^2}{\phi^2} \\
	\grad^2 &= \frac{1}{r^2} \diff{}{r} \pp{r^2 \diff{}{r}} + \frac{1}{r^2 \sin \theta} \diff{}{\theta} \pp{\sin \theta \diff{}{\theta}} + \frac{1}{r^2 \sin \theta} \diff{^2}{\phi^2}
\end{align}

Laplaces ligning er givet ved
\begin{equation}
	\grad^2 u(\V{r}) = 0
\end{equation}


\subsubsection{Plane polarkoordinater}
I plane polarkoordinater ledes der efter løsninger på formen
\begin{equation}
	u(\rho,\phi) = P(\rho) \Phi(\phi)
\end{equation}
Sættes dette ind i Laplaces ligning, og ganges der med $ \frac{\rho^2}{u}=\frac{\rho^2}{P\Phi} $ fås
\begin{equation}
	\frac{\rho}{P} \diff{}{\rho} \pp{\rho \diff{P}{\rho}} + \frac{1}{\Phi} \diff{^2 \Phi}{\Phi^2} = 0
\end{equation}
Igen optræder to ordinære ligninger:
\begin{equation}
	\frac{\rho}{P} \diff{}{\rho} \pp{\rho \diff{P}{\rho}} = n^2, \quad \frac{1}{\Phi} \diff{^2 \Phi}{\Phi^2} = -n^2
\end{equation}
hvor $ n $ er et eller andet generelt komplekst tal. Løsningerne til disse ligninger, hvor $ n\neq 0 $ er
\begin{equation}
	P(\rho)=C\rho^n+D\rho^{-n},	\quad \Phi(\phi) = A \exp (in\phi) + B \exp (-in\phi) = A \cos n\phi + B \sin n\phi
\end{equation}
$ n $ skal her være et positivt heltal, for ellers har løsningen flere værdier (ikke ">single valued"<). Første ligning fås far kapitel 6.5.1 i EMM (side 244-246. Det er MatF2 stof). For $ n = 0 $ fås
\begin{equation}
	P(\rho) = C \ln \rho + D, \quad \Phi(\phi) = A \phi + B
\end{equation}
Her skal $ A = 0 $, for at funktionen kun har enkelte værdier. superpositionen af disse løsninger kan skrives som følger (hvor $ B $ og $ D $ slås sammen)
\begin{equation}
	u(\rho,\phi) = C_0 \ln \rho +D_0 + \sum_{n=1}^{\infty} (A_n \cos n\phi + B_n \sin n\phi)(C_n \rho^n + D_n \rho^{-n})
\end{equation}
hvor $ n $ er positive heltal, og $ C_0 = D_n = 0$ for $n>0 $, hvis ligningen løses i et område, der inkluderer origo (da løsningen ellers ikke er endelig).





\subsubsection{Sfæriske koordinater}
I sfæriske polarkoordinater ledes der efter løsninger på formen
\begin{equation}
	u(r,\theta,\phi) = R(r) \Theta(\theta) \Phi(\phi)
\end{equation}
Den generelle separable løsning har formen
\begin{equation}
	u(r,\theta,\phi) = (A r^{\ell} + B r^{-(\ell+1)})(C \cos m\phi+D\sin m\phi)[E P^m_{\ell}(\cos \theta)+F Q^m_{\ell}(\cos \theta)]
\end{equation}
hvor $ m $ og $ \ell $ er helttal, med $ 0 \leq |m| \leq \ell $. $ P $ og $ Q $ er de associerede Legendrefunktioner (som er tabulerede og ikke pensum i dette kursus). Skal løsningen være endelig på den polære akse, så skal $ F=0 $ for alle $ m,\ell $, da $ Q(\cos \theta) $ divergerer ved $ \pm 1 $. Dette er som oftest tilfældet, og vinkeldelen bliver da
\begin{equation}
	\Theta(\theta)\Phi(\phi) = P^m_{\ell} (\cos\theta) (C \cos m\phi+D\sin m\phi)
\end{equation}
hvor $ \ell,m $ stadig er heltal, med $ -\ell \leq m \leq \ell $. Disse kaldes for kuglefunktioner (se afsnit \ref{sec:kugel})
%
%
% Udledning af løsning herunder. Jeg gider virkelig ikke skrive den færdig.....
%
%
%Ved at indsætte dette i Laplaces ligning, dividere med $ u=R\Theta\Phi $, gange med $ r^2 $, og skille det radiale ($ r $) og  vinkeldelen ($ \theta $ og $ \phi $) fra hinanden fås:
%\begin{equation}
%	R\inverse \diff[\ud]{}{r} \pp{r^2 \diff[\ud]{R}{r}} = \lambda, \quad (\Theta \sin \theta)\inverse \diff[\ud]{}{\theta} \pp{\sin \theta \diff[\ud]{\Theta}{\theta}} + (\Phi \sin^2 \theta) \diff[\ud]{^2 \Phi}{\phi^2} = -\lambda
%\end{equation}
%Den radiale ligning kan løses ved at substituere $ r=\exp t $ og $ R(r) = S(t) $, og følgende løsning for $ R(r) $ fås
%\begin{equation}
%	R(r) = A r^{\ell} + B r^{-(\ell+1)}
%\end{equation}
%hvor produktet af de to eksponenter skal være lig den negative separationskonstant, altså $ \ell(\ell+1)=\lambda $. Substitueres dette ind i den anden ligning, og ganges der med $ \sin^2 \theta $ fås en ny ligning med separationskonstanten $ m^2 $

\section{Kuglefunktioner}
\label{sec:kugel}
Til hjælp for funktioner af typen
\begin{equation}
\Theta(\theta)\Phi(\phi) = P^m_{\ell} (\cos\theta) (C \cos m\phi+D\sin m\phi)
\end{equation}
som er vinkeldelen af Laplace's ligning i polære koordinater, er \textbf{kuglefunktioner} defineret. Kuglefunktionen $ Y^m_{\ell} $ er givet ved
\begin{equation}
	Y^m_{\ell} (\theta,\phi)= (-1)^m \bb{\frac{2\ell + 1}{4\pi} \frac{(\ell-m)!}{(\ell+m)!}}^{1/2} P^m_{\ell} (\cos \theta) \exp (im\phi)
\end{equation}
Og
\begin{equation}
	Y^{-m}_{\ell} (\theta,\phi) = (-1)^m [Y^m_{\ell}(\theta,\phi)]^*
\end{equation}
Hvor stjernen betyder den kompleks konjugerede.

Sættet af kuglefunktioner er alle indbyrdes ortogonale, udgør et ortonormalt sæt, og udspænder derfor et Hilbertrum. Det vil altså sige, at enhver funktion af $ \theta $ og $ \phi $ kan opbygges som en sum af disse
\begin{equation}
	f(\theta,\phi) = \sum_{\ell=0}^{\infty} \sum_{m=-\ell}^{\ell} a_{\ell m} Y^m_{\ell} (\theta,\phi), \quad a_{\ell m} = \int_{-1}^{1} \int_{0}^{2 \pi} [Y^m_{\ell} (\theta,\phi)]^* f(\theta,\phi) \ud \phi \ud(\cos \theta)
\end{equation}
Præcis lige som Fourierserien. Dette er også et eksempel på egenskaberne ved løsninger til Sturm-Liouville-problemer.

Til kuglefunktioner hører også \textit{kuglefunktionsadditionsteoremet}, der siger
\begin{equation}
	P_\ell (\cos \gamma) = \frac{4\pi}{2\ell+1} \sum_{m=-\ell}^{\ell} Y^m_{\ell} (\theta, \phi) [Y^m_{\ell} (\theta', \phi')]^*
\end{equation}
hvor $ (\theta,\phi) $ og $ (\theta',\phi') $ er to forskellige vinkelsæt, separeret af vinklen $ \gamma $. Der er følgende relation mellem disse vinkler:
\begin{equation}
	\cos \gamma = \cos \theta \cos \theta' + \sin \theta \sin \theta' \cos(\phi-\phi')
\end{equation}

\end{document}

