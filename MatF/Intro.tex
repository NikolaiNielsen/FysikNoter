\documentclass[MatFNoter.tex]{subfiles} % HUSK FOR FANDEN AT REDIGERE DENNE LINJE

% Hvis ikke dokumenterne (hoved & under) er i samme mappe, skal den relative stig bruges.



\begin{document}
	\section{Introduktion og notation}
	I disse noter bruges følgende notation for vektorer:
	\begin{equation*}
		\vec{v}=\V{v}, \quad \hat{i} = \V{i}, \dots
	\end{equation*}
	hvor $ \V{i} $ er standardenhedsvektoren i x-retningen, med koordinater $ (1,0,0) $, ligeledes for $ \V{j} $ og $ \V{k} $ (y og z)
	
	Vektorproduktet (krydsproduktet) af to vektorer i 3 dimensioner er givet ved en $ 3\times 3 $ determinant:
	\begin{align}
		\V{A}&=(A_x,A_y,A_z), \quad \V{B}=(B_x,B_y,B_z) \nonumber\\
		\V{A}\times \V{B} &= \begin{vmatrix}
		\V{i} & \V{j} & \V{k} \\
		A_x & A_y & A_z \\
		B_x & B_y & B_z
		\end{vmatrix} \\
		%&= \pp{A_y B_z - A_z B_y} \V{i}+ \pp{A_z B_x - A_x B_z} \V{j} + \pp{A_x B_y - A_y B_x} \V{k} \nonumber\\
		|\V{A}\times\V{B}|&= |\V{A}| \ |\V{B}| \ \sin\theta 
	\end{align}
	hvor $ \theta $ er vinklen mellem de to vektorer. 
	
	For vektorer kun med $ x $- og $ y $-komponenter bliver produktet
	\begin{equation}
		\V{A}\times\V{B} = \begin{vmatrix}
		\V{i} & \V{j} & \V{k} \\
		A_x & A_y & 0 \\
		B_x & B_y & 0
		\end{vmatrix} = (A_x B_y - A_y B_x) \V{k}
	\end{equation}
	
	\subsection{Felter generelt}
	Felter er en konstruktion, hvor der til ethvert punkt knyttes en eller anden værdi (en skalar eller en vektor). Disse felter kan også variere i tiden, men hvis de ikke gør dette, kaldes de for \textbf{stationære}.
	\subsubsection{Skalarfelt}
	Med et skalarfelt menes, at der til ethvert punkt i rummet $ (x,y,z) $ knyttes en værdi $ \beta $. Dette kan da opfattes som en almindelig funktion af flere variable:
	
		
	\begin{equation*}
		\beta = \beta(x,y,z,t)
	\end{equation*}
	I rummet kan der tegne flader med en konstant værdi af $ \beta $, kaldet ækviskalarflader (niveauflader fra MatIntro). Disse betegnes:
	
	
	\begin{equation*}
		\beta(x,y,z) = \beta_0 = \text{konstant}
	\end{equation*}
	Hermed kan $ z $ findes som en funktion af $ x $, $ y $ og $ \beta_0 $: $ z=z(x,y,\beta_0) $.
	
	En måde at plotte 3d-flader i et 2d-koordinatsystem, er ved at tegne konturlinjer, hvor én værdi (eksempelvis $ z $ eller $ \beta_0 $) indtegnes ved hver konturlinje.
	
	
	Eksempler på skalarfelter, der ikke er stationære, er temperatur eller tryk, hvor disse ændres både forskellige steder på jordkloden, samt i højden over jordoverfladen, og tiden.
	
	\subsubsection*{Skalering}
	I stedet for at beskrive ethvert punkt  med SI-enheder, kan det nogle gange være gavnligt, at gøre det i en anden målestok, der er knyttet til fænomenet man nu beskriver. Eksempelvis kan $ x $ og $ y $ til højden af et bjerg beskrives ved radius $ R $ af bjerget. Her skalerer man så feltet, så akseindelingerne bliver:
	
	
	\begin{equation}
	x^*=\frac{x}{R}, \ y^*=\frac{y}{R}
	\end{equation}
	Dette har fordelen at være enhedsløst, idet både $ x $, $ y $ og $ R $ har samme enhed (som oftest meter), hvilket kan  forenkle nogle udtryk. Eksempelvis bliver et idealiseret, isoleret bjerg til:
	
	
	\begin{equation*}
		h=\frac{h_0}{1+\frac{x^2+y^2}{R^2}} \to h = \frac{h_0}{1+x^{*2}+y^{*2}}
	\end{equation*}
	Skaleres højden så $ h^*=h/h_0 $, bliver formlen endnu enklere:
	
	\begin{equation*}
		h^*=\frac{1}{1+x^{*2}+y^{*2}}
	\end{equation*}
	
	\subsubsection{Vektorfelt}
	Til et vektorfelt knyttes der, som navnet antyder, en entydig vektor, til ethvert punkt i rummet. Disse vektorer afhænger altså af positionen (og tiden, hvis feltet ikke er stationært), og der kan ikke ``bare'' rykkes rundt på dem. Vektorfeltet kan da opfattes som en funktion:
	\begin{equation*}
		\V{A}=\V{A}(x,y,z,t)
	\end{equation*}
	
	Som oftest snakkes der om enten \textit{strømfelter} eller \textit{kraftfelter}, hvor den første beskriver strømningen i et medium (vindhastighed, vandstrømning og lignende), mens den anden beskriver kraften i ethvert punkt (elektriske/magnetiske felter, tyngdefelter).
	
	
	\subsection{Taylor-approksimation for to variable}
	Taylor-approksimationen for én variabel kendes, og er givet ved summen:
	\begin{equation}
		T_nf(x) = \sum_{k=0}^n{\frac{f^{(k)}(a)}{k!}(x-a)^k}
	\end{equation}
	Hvor $ a $ er udviklingspunktet, $ n $ er graden og $ f^{(k)} $ er den $ k $'te afledte ($ f^0(a)=f(a) $)
	
	For to variable bliver andenordensapproksimationen ($ n=2 $) om punktet ($ x_0,y_0 $) til
	\begin{align}
		T_n f(x,y) = &f(x_0,y_0)+ \pp{\diff{f}{x}}_{x_0,y_0} (x-x_0)+\pp{\diff{f}{y}}_{x_0,y_0} (y-y_0)+ \\
		& \frac{1}{2} \pp{\diff{^2 f}{x^2}}_{x_0,y_0} (x-x_0)^2+\frac{1}{2} \pp{\diff{^2 f}{y^2}}_{x_0,y_0} (y-y_0)^2+ \nonumber\\
		& \pp{\diff{^2 f}{x\partial y}}_{x_0,y_0} (x-x_0)(y-y_0) \nonumber
	\end{align}
	Læg mærke til det sidste led, hvor der \textit{ikke} indgår 1/2! Dette er fordi der er to led med samme form, fordi $ \diff{^2 f}{x \partial y} = \diff{^2 f}{y \partial x} $.
	
	Ud fra dette, kan det ses, hvordan Taylor-approksimationen fungerer for 3 variable (i hvert fald for $ n=1 $: der kommer 4 led, i stedet for 3. Ét med hver variabel, plus funktionsværdien i udviklingspunktet)
	
	
	
	\subsection{Strømlinjer og feltlinjer} 
	Feltlinjer er linjer hvor feltvektorerne altid tangerer linjen. Det er kun muligt at lave kontinuerlige feltlinjer hvis der er tale om et kontinuerligt felt (og der dermed er en gradvis ændring i vektorværdierne). Strømlinjer er specielbetegnelsen, når der er tale om et strømhastighedsfelt.
	
	I et \textbf{todimensionalt}, \textbf{stationært} strømhastighedsfelt, hvor strømhastig-hedsvektoren er givet ved
	\begin{equation}
		\V{v}=v_x(x,y) \Vi + v_y (x,y) \Vj
	\end{equation}
	hvor $ v_x $ og $ v_y $ er vektorkomponenterne i $ x $- og $ y $-retningen, og som afhænger af rumkoordinaterne. For en strømlinje gælder at
	\begin{equation}
		v_x dy = v_y dx
	\end{equation}
	Der integreres henholdsvis med hensyn til $ y $ og $ x $, hvormed et udtryk for strøm-linjerne kan findes.
\end{document}

